\documentclass[notes,11pt, aspectratio=169]{beamer}

\usepackage{pgfpages}
% These slides also contain speaker notes. You can print just the slides,
% just the notes, or both, depending on the setting below. Comment out the want
% you want.
\setbeameroption{hide notes} % Only slide
%\setbeameroption{show only notes} % Only notes
%\setbeameroption{show notes on second screen=right} % Both

%\usepackage[scaled=1.0]{helvet}
\usepackage{array}

\usepackage{tikz}
\usepackage{verbatim}
\setbeamertemplate{note page}{\pagecolor{gray!5}\insertnote}
\usetikzlibrary{positioning}
\usetikzlibrary{snakes}
\usetikzlibrary{calc}
\usetikzlibrary{arrows}
\usetikzlibrary{decorations.markings}
\usetikzlibrary{shapes.misc}
\usetikzlibrary{matrix,shapes,arrows,fit,tikzmark}
\usepackage{amsmath}
\usepackage{mathpazo}
\usepackage{hyperref}
\usepackage{lipsum}
\usepackage{multimedia}
\usepackage{graphicx}
\usepackage{multirow}
\usepackage{graphicx}
\usepackage{dcolumn}
\usepackage{bbm}
\newcolumntype{d}[0]{D{.}{.}{5}}

\usepackage{changepage}
\usepackage{appendixnumberbeamer}
\newcommand{\beginbackup}{
   \newcounter{framenumbervorappendix}
   \setcounter{framenumbervorappendix}{\value{framenumber}}
   \setbeamertemplate{footline}
   {
     \leavevmode%
     \hline
     box{%
       \begin{beamercolorbox}[wd=\paperwidth,ht=2.25ex,dp=1ex,right]{footlinecolor}%
%         \insertframenumber  \hspace*{2ex} 
       \end{beamercolorbox}}%
     \vskip0pt%
   }
 }
\newcommand{\backupend}{
   \addtocounter{framenumbervorappendix}{-\value{framenumber}}
   \addtocounter{framenumber}{\value{framenumbervorappendix}} 
}


\usepackage{graphicx}
\usepackage[space]{grffile}
\usepackage{booktabs}

% These are my colors -- there are many like them, but these ones are mine.
\definecolor{blue}{RGB}{0,114,178}
\definecolor{red}{RGB}{213,94,0}
\definecolor{yellow}{RGB}{240,228,66}
\definecolor{green}{RGB}{0,158,115}

\hypersetup{
  colorlinks=false,
  linkbordercolor = {white},
  linkcolor = {blue}
}


%% I use a beige off white for my background
\definecolor{MyBackground}{RGB}{255,253,218}

%% Uncomment this if you want to change the background color to something else
%\setbeamercolor{background canvas}{bg=MyBackground}

%% Change the bg color to adjust your transition slide background color!
\newenvironment{transitionframe}{
  \setbeamercolor{background canvas}{bg=white}
  \begin{frame}}{
    \end{frame}
}

\setbeamercolor{frametitle}{fg=blue}
\setbeamercolor{title}{fg=black}
\setbeamertemplate{footline}[frame number]
\setbeamertemplate{navigation symbols}{} 
\setbeamertemplate{itemize items}{-}
\setbeamercolor{itemize item}{fg=blue}
\setbeamercolor{itemize subitem}{fg=blue}
\setbeamercolor{enumerate item}{fg=blue}
\setbeamercolor{enumerate subitem}{fg=blue}
\setbeamercolor{button}{bg=MyBackground,fg=blue,}



% If you like road maps, rather than having clutter at the top, have a roadmap show up at the end of each section 
% (and after your introduction)
% Uncomment this is if you want the roadmap!
% \AtBeginSection[]
% {
%    \begin{frame}
%        \frametitle{Roadmap of Talk}
%        \tableofcontents[currentsection]
%    \end{frame}
% }
\setbeamercolor{section in toc}{fg=blue}
\setbeamercolor{subsection in toc}{fg=red}
\setbeamersize{text margin left=1em,text margin right=1em} 

\newenvironment{wideitemize}{\itemize\addtolength{\itemsep}{10pt}}{\enditemize}
\newenvironment{wideenumerate}{\enumerate\addtolength{\itemsep}{10pt}}{\endenumerate}

\usepackage{environ}
\NewEnviron{videoframe}[1]{
  \begin{frame}
    \vspace{-8pt}
    \begin{columns}[onlytextwidth, T] % align columns
      \begin{column}{.58\textwidth}
        \begin{minipage}[t][\textheight][t]
          {\dimexpr\textwidth}
          \vspace{8pt}
          \hspace{4pt} {\Large \sc \textcolor{blue}{#1}}
          \vspace{8pt}
          
          \BODY
        \end{minipage}
      \end{column}%
      \hfill%
      \begin{column}{.42\textwidth}
        \colorbox{green!20}{\begin{minipage}[t][1.2\textheight][t]
            {\dimexpr\textwidth}
            Face goes here
          \end{minipage}}
      \end{column}%
    \end{columns}
  \end{frame}
}

\title[]{\textcolor{blue}{Introduction \\ PhD Industrial Organization}}
\author[PGP]{}
\institute[FRBNY]{\small{\begin{tabular}{c c c}
Nicholas Vreugdenhil \\
\end{tabular}}}
\date{} 

\begin{document}

%%% TIKZ STUFF
\tikzset{   
        every picture/.style={remember picture,baseline},
        every node/.style={anchor=base,align=center,outer sep=1.5pt},
        every path/.style={thick},
        }
\newcommand\marktopleft[1]{%
    \tikz[overlay,remember picture] 
        \node (marker-#1-a) at (-.3em,.3em) {};%
}
\newcommand\markbottomright[2]{%
    \tikz[overlay,remember picture] 
        \node (marker-#1-b) at (0em,0em) {};%
}
\tikzstyle{every picture}+=[remember picture] 
\tikzstyle{mybox} =[draw=black, very thick, rectangle, inner sep=10pt, inner ysep=20pt]
\tikzstyle{fancytitle} =[draw=black,fill=red, text=white]
%%%% END TIKZ STUFF

% Title Slide
\begin{frame}
\maketitle
  \centering
\end{frame}

% INTRO

\begin{frame}{Plan for today}
  \begin{wideenumerate}
    \item What is industrial organization?
    \item Discuss syllabus
    \item Introduce yourself
  \end{wideenumerate}
\end{frame}

\begin{frame}{Plan for today}
	\begin{wideenumerate}
		\item \textbf{What is industrial organization?}
		\item Discuss syllabus
		 \item Introduce yourself
	\end{wideenumerate}
\end{frame}

\begin{frame}{What is industrial organization (IO)?}
\begin{wideitemize}
	\item \textbf{IO is the study of firm and consumer behavior in markets between (and including) the polar opposites of perfect competition and monopoly.}
	\item Why is this useful? Some examples...
	\begin{wideenumerate}
		\vspace{11pt}
		\item Designing regulation and thinking about counterfactual policies:
				\begin{wideitemize}
					\item Hinges on the details of how firms and consumer behave. E.g. merger policy
				\end{wideitemize}
		\item Firm strategy
				\begin{wideitemize}
					\item e.g. How to set prices? How to design online marketplaces?{\tiny }
				\end{wideitemize}
		\item Getting a job (?)
		\begin{wideitemize}
			\item Estimating a model in your research can be a good way to differentiate yourself from other candidates
			\item Candidates with IO skills are in short supply and high demand in both academia and industry.
			\item Emphasis: models in this course are a complement - not a substitute - to detailed micro-data and other approaches (like the causal inference toolbox)
		\end{wideitemize}
	\end{wideenumerate}
\end{wideitemize}
\end{frame}

\begin{frame}{What are the aims of this course?}
	\begin{wideitemize}
		\item This is the first in a two course sequence in Empirical IO
		\item \textbf{Main aim:} Get you up to speed with the core \textbf{methods} from `New' Empirical IO 
		\begin{wideitemize}
			\item This toolbox started to be developed in the late '80s. Development continues up to the present day.
			\item Limitation of the course: will not go into great depth about `traditional' IO applications of these methods
		\end{wideitemize}
		\item \textbf{Second aim:} get you started on research and get you to understand where the frontier of knowledge lies.
	\end{wideitemize}
\end{frame}

\begin{frame}{History of IO}
	\begin{wideitemize}
		\item \textbf{History}: in the 1980s IO was dominated by game theoretic methods to think about competition, oligopoly, firm decision making. 
		\begin{wideitemize}
			\item A key reason for developing the empirical methods discussed in this course is applying these theories to data, testing them, and using the models to do, for example, policy evaluation
		\end{wideitemize}
		\item \textbf{Recent developments}: applying these methods outside `traditional' IO topics
		\begin{wideitemize}
			\item `Traditional' IO topics: mergers, competition policy
			\item Recent applications: health, education, energy, environmental, trade, urban economics, market design,...
		\end{wideitemize}
	\end{wideitemize}
\end{frame}

\begin{frame}{Plan for today}
	\begin{wideenumerate}
		\item What is industrial organization?
		\item \textbf{Discuss syllabus}
		    \item Introduce yourself
	\end{wideenumerate}
\end{frame}

\begin{frame}{Main topics in this course}
	\begin{wideenumerate}
		\item Demand estimation
		\item Single agent dynamics
		\item Dynamic games/oligopoly
		\item Paper presentations (focusing on recent influential IO papers across many subfields including Education, Healthcare, Environmental, etc...)
	\end{wideenumerate}
\end{frame}

\begin{frame}{Textbook and readings}
	\begin{wideitemize}
		\item There is no textbook for this course. But, the lectures will focus on the first four chapters of the new volume of the \textit{Handbook of Industrial Organization (Volume 4)}
		\item Key papers I will discuss are on the reading list in the syllabus (and the most important have a *)
		\item I encourage you to read the papers (particularly those with a *), and I \textit{expec}t you to read the papers for the student paper presentations before the lecture.
	\end{wideitemize}
\end{frame}

\begin{frame}{Assessment and grading}
	\begin{wideitemize}
		\item Second year PhD course, so the priority is research and noone will ever look at your grades...
		\item ...but this is how your final grade will be determined:
	\begin{wideitemize}
		\item \underline{\textbf{50\%}} Two homework assignments
		\item \underline{\textbf{40\%}} Half-hour paper presentations
		\item \underline{\textbf{10\%}} Engagement in class discussion
	\end{wideitemize}
	\end{wideitemize}
\end{frame}

\begin{frame}{ASU Sync}
	\begin{wideitemize}
		\item Link is on the syllabus
		\item Email me if you intend to use it so I can make sure that ASU Sync is connected
	\end{wideitemize}
\end{frame}

\begin{frame}{Preliminaries/preparation (do these early in the semester)}
\begin{wideitemize}
	\item \underline{Computational resources}: apply for access to the ASU Sol computer system and familiarize yourself in how to use it (you could do this by attending a regularly scheduled tutorial by the ASU computing people). 
	\item \underline{Version control}: learn how to use Git, and sign up for Github.
	\item \underline{Programming language}: settle on a programming language to use in your work. 
	\begin{wideitemize}
		\item Important: I have put a .pdf on canvas about `best practices' in programming that includes advice on unit testing, profiling, structuring your code
	\end{wideitemize}
\end{wideitemize}
\end{frame}

\begin{frame}{Background}
	\begin{wideitemize}
		\item \underline{Theory} You are well prepared for this course from your first year PhD coursework. 
		\item For this course, I expect that you have some understanding of IO at the level of an undergraduate IO course.
			\begin{wideitemize}
		\item e.g. Cabral ``Introduction to Industrial Organization'
			\end{wideitemize}
		\item \underline{Empirics} We will pick up where Alvin Murphy's Econometrics II course left off (review lecture 11 about discrete choice models)
			\begin{wideitemize}
		\item e.g. Train ``Discrete Choice Methods with Simulation'' is another good book to review
			\end{wideitemize}
	\end{wideitemize}
\end{frame}


\begin{frame}{Other things to consider}
	\begin{wideitemize}
		\item Applied micro hiring this year: come to job talks
		\item Attend applied micro seminar
		\item Start thinking about your third-year paper. 
		\begin{wideitemize}
			\item Important: as well as thinking about general topics and questions, focus on \underline{obtaining detailed micro-data}.
		\end{wideitemize}
	\end{wideitemize}
\end{frame}

\begin{frame}{Plan for today}
	\begin{wideenumerate}
		\item What is industrial organization?
		\item Discuss syllabus
		\item \textbf{Introduce yourself}
	\end{wideenumerate}
\end{frame}


\begin{comment}
\begin{frame}{Terminology: identification}{See haile.pdf on Canvas}
	\begin{wideitemize}
		\item \textbf{Intuitive idea:} whether/how the things we observe are capable of revealing the answers to the questions we care about.
		\item Formal idea:
		\begin{wideitemize}
			\item structure $S$: a data generating process (`DGP')
			\begin{wideitemize}
				\item DGP: set of relationships between observable and latent variables that generates a joint distribution of observables
			\end{wideitemize}
			\item $\mathfrak{S}$: set of all structures, denote $S_0 \in \mathfrak{S}$ the true structure
			\item Hypothesis: nonempty subset of $\mathfrak{S}$
			\item  
		\end{wideitemize}
	\end{wideitemize}
\end{frame}
\end{comment}

\end{document}
