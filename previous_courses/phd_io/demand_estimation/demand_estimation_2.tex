\documentclass[notes,11pt, aspectratio=169]{beamer}

\usepackage{pgfpages}
% These slides also contain speaker notes. You can print just the slides,
% just the notes, or both, depending on the setting below. Comment out the want
% you want.
\setbeameroption{hide notes} % Only slide
%\setbeameroption{show only notes} % Only notes
%\setbeameroption{show notes on second screen=right} % Both

%\usepackage[scaled=1.0]{helvet}
\usepackage{array}

\usepackage{tikz}
\usepackage{verbatim}
\setbeamertemplate{note page}{\pagecolor{gray!5}\insertnote}
\usetikzlibrary{positioning}
\usetikzlibrary{snakes}
\usetikzlibrary{calc}
\usetikzlibrary{arrows}
\usetikzlibrary{decorations.markings}
\usetikzlibrary{shapes.misc}
\usetikzlibrary{matrix,shapes,arrows,fit,tikzmark}
\usepackage{amsmath}
\usepackage{mathpazo}
\usepackage{hyperref}
\usepackage{lipsum}
\usepackage{multimedia}
\usepackage{graphicx}
\usepackage{multirow}
\usepackage{graphicx}
\usepackage{dcolumn}
\usepackage{bbm}
\newcolumntype{d}[0]{D{.}{.}{5}}

\usepackage{changepage}
\usepackage{appendixnumberbeamer}
\newcommand{\beginbackup}{
   \newcounter{framenumbervorappendix}
   \setcounter{framenumbervorappendix}{\value{framenumber}}
   \setbeamertemplate{footline}
   {
     \leavevmode%
     \hline
     box{%
       \begin{beamercolorbox}[wd=\paperwidth,ht=2.25ex,dp=1ex,right]{footlinecolor}%
%         \insertframenumber  \hspace*{2ex} 
       \end{beamercolorbox}}%
     \vskip0pt%
   }
 }
\newcommand{\backupend}{
   \addtocounter{framenumbervorappendix}{-\value{framenumber}}
   \addtocounter{framenumber}{\value{framenumbervorappendix}} 
}


\usepackage{graphicx}
\usepackage[space]{grffile}
\usepackage{booktabs}

% These are my colors -- there are many like them, but these ones are mine.
\definecolor{blue}{RGB}{0,114,178}
\definecolor{red}{RGB}{213,94,0}
\definecolor{yellow}{RGB}{240,228,66}
\definecolor{green}{RGB}{0,158,115}

\hypersetup{
  colorlinks=false,
  linkbordercolor = {white},
  linkcolor = {blue}
}


%% I use a beige off white for my background
\definecolor{MyBackground}{RGB}{255,253,218}

%% Uncomment this if you want to change the background color to something else
%\setbeamercolor{background canvas}{bg=MyBackground}

%% Change the bg color to adjust your transition slide background color!
\newenvironment{transitionframe}{
  \setbeamercolor{background canvas}{bg=white}
  \begin{frame}}{
    \end{frame}
}

\setbeamercolor{frametitle}{fg=blue}
\setbeamercolor{title}{fg=black}
\setbeamertemplate{footline}[frame number]
\setbeamertemplate{navigation symbols}{} 
\setbeamertemplate{itemize items}{-}
\setbeamercolor{itemize item}{fg=blue}
\setbeamercolor{itemize subitem}{fg=blue}
\setbeamercolor{enumerate item}{fg=blue}
\setbeamercolor{enumerate subitem}{fg=blue}
\setbeamercolor{button}{bg=MyBackground,fg=blue,}



% If you like road maps, rather than having clutter at the top, have a roadmap show up at the end of each section 
% (and after your introduction)
% Uncomment this is if you want the roadmap!
% \AtBeginSection[]
% {
%    \begin{frame}
%        \frametitle{Roadmap of Talk}
%        \tableofcontents[currentsection]
%    \end{frame}
% }
\setbeamercolor{section in toc}{fg=blue}
\setbeamercolor{subsection in toc}{fg=red}
\setbeamersize{text margin left=1em,text margin right=1em} 

\newenvironment{wideitemize}{\itemize\addtolength{\itemsep}{10pt}}{\enditemize}
\newenvironment{wideenumerate}{\enumerate\addtolength{\itemsep}{10pt}}{\endenumerate}

\usepackage{environ}
\NewEnviron{videoframe}[1]{
  \begin{frame}
    \vspace{-8pt}
    \begin{columns}[onlytextwidth, T] % align columns
      \begin{column}{.58\textwidth}
        \begin{minipage}[t][\textheight][t]
          {\dimexpr\textwidth}
          \vspace{8pt}
          \hspace{4pt} {\Large \sc \textcolor{blue}{#1}}
          \vspace{8pt}
          
          \BODY
        \end{minipage}
      \end{column}%
      \hfill%
      \begin{column}{.42\textwidth}
        \colorbox{green!20}{\begin{minipage}[t][1.2\textheight][t]
            {\dimexpr\textwidth}
            Face goes here
          \end{minipage}}
      \end{column}%
    \end{columns}
  \end{frame}
}

\title[]{\textcolor{blue}{Demand Estimation 2 \\ PhD Industrial Organization}}
\author[PGP]{}
\institute[FRBNY]{\small{\begin{tabular}{c c c}
Nicholas Vreugdenhil \\
\end{tabular}}}
\date{} 

\begin{document}

%%% TIKZ STUFF
\tikzset{   
        every picture/.style={remember picture,baseline},
        every node/.style={anchor=base,align=center,outer sep=1.5pt},
        every path/.style={thick},
        }
\newcommand\marktopleft[1]{%
    \tikz[overlay,remember picture] 
        \node (marker-#1-a) at (-.3em,.3em) {};%
}
\newcommand\markbottomright[2]{%
    \tikz[overlay,remember picture] 
        \node (marker-#1-b) at (0em,0em) {};%
}
\tikzstyle{every picture}+=[remember picture] 
\tikzstyle{mybox} =[draw=black, very thick, rectangle, inner sep=10pt, inner ysep=20pt]
\tikzstyle{fancytitle} =[draw=black,fill=red, text=white]
%%%% END TIKZ STUFF

% Title Slide
\begin{frame}
\maketitle
  \centering
\end{frame}

% INTRO

\begin{frame}{Plan}
\begin{wideenumerate}
\item BLP setup
\item Price elasticity/substitution patterns
\item Estimation: overview and typical data
\item Identification: what if we had micro-data?
%\item Identification: $\Sigma$ 
%\item Estimation algorithm
%\item Instrumental variables
%\item Extensions
%\item Applications
\end{wideenumerate}
\end{frame}

\begin{frame}{Plan}
	\begin{wideenumerate}
		\item \textbf{ BLP setup }
		\item Price elasticity/substitution patterns
		\item Estimation: overview and typical data
		\item Identification: what if we had micro-data?
		%\item Identification: $\Sigma$ 
		%\item Estimation algorithm
		%\item Instrumental variables
		%\item Extensions
		%\item Applications
	\end{wideenumerate}
\end{frame}


\begin{frame}{Discrete choice demand models: general setup of `BLP'}{Berry, Levinsohn, and Pakes (1995)}
	\begin{align*}
		u_{ijt} = x_{jt} \beta_{it} + \alpha_{it} p_{jt} + \xi_{jt} + \epsilon_{ijt}
	\end{align*}
	\begin{wideitemize}
		\item \underline{What about income?}:
		\item In the above equation, $p_{jt}$ should really be $y_i-p_{jt}$ where $y_i$ is income.
		\item Leaving it out has no impact for choices however, and just simplifies exposition. Why? \pause
		\begin{wideitemize}
			\item Income enters linearly to all options, and only relative differences in utilities matter for choice probabilities 
			\item If $y_i-p_{jt}$ entered non-linearly then it would affect the choice probabilities and should be included
		\end{wideitemize}
	\end{wideitemize}
\end{frame}

\begin{frame}{Discrete choice demand models: general setup of `BLP'}{Berry, Levinsohn, and Pakes (1995)}
	\begin{wideitemize}
		\item \underline{Even more notation}: 
		\begin{wideitemize}
			\item Recall $L$ number of demographic vars, $K$ number of product characteristics
		\end{wideitemize}
		\item Define:
		\begin{wideitemize}
			\item The \textbf{mean utility} of product j in market t: $\delta_{jt} = x_{jt} \beta_0 + \alpha_0 p_{jt} + \xi_{jt}$
			\item $\Gamma$: $(K+1) \times L$ matrix with coefficients of demographic variables
			\item $\Sigma$: $(K+1) \times (K+1)$ diagonal matrix with diagonal $(\alpha_{\nu},\beta_{\nu}^{(1)},..., \beta_{\nu}^{(K)} )$
			\item $\nu_{it} = ( \nu^{(0)}_{it}, ..., \nu^{(K)}_{it} )^T$
			\item $\mu_{ijt} = (x_{jt}, p_{jt}) \cdot (\Gamma D_{it} + \Sigma \nu_{it})$
		\end{wideitemize}
		\item Then we can rewrite our utility equation as:
	\end{wideitemize}
	\begin{align*}
		u_{ijt} = \underbrace{\delta_{jt}}_{\text{mean utility}} + \underbrace{\mu_{ijt}}_{\text{interaction between  consumer tastes + product characteristics}}+ \underbrace{\epsilon_{ijt}}_{ \text{idiosyncratic error}}
	\end{align*}
\end{frame}

\begin{frame}{Discrete choice demand models: general setup of `BLP'}{Berry, Levinsohn, and Pakes (1995)}
	\begin{wideitemize}
		\item \underline{Review of where we are}: we just characterized a very flexible model of consumer utility. 
		\item Assuming i.i.d. extreme value errors $\epsilon_{ijt}$ the probability consumer $i$ chooses product $j$ is:
		\begin{align*}
			\frac{ \exp(\delta_{jt} + \mu_{ijt}) }{1 + \sum_{k=1}^J \exp(\delta_{kt} + \mu_{ikt})}
		\end{align*}
		\item And \textbf{demand} (the share of consumers who purchase good $j$ in market $t$) is:
		\begin{align*}
			s_{jt} = \sigma_j ( \boldsymbol{\delta}_t,  \boldsymbol{x}_t,  \boldsymbol{p}_t;\Gamma, \Sigma  ) = \int \frac{ \exp(\delta_{jt} + \mu_{ijt}) }{1 + \sum_{k=1}^J \exp(\delta_{kt} + \mu_{ikt})} dF(D_{it}, v_{it})
		\end{align*}
		\item Here:
		\begin{wideitemize}
			\item $\boldsymbol{\delta}_t,  \boldsymbol{x}_t,  \boldsymbol{p}_t$ are vectors of mean utilities, observed product characteristics, prices, in market $t$
			\item $F$ is the joint distribution of observed demographics and unobserved tastes
		\end{wideitemize}
	\end{wideitemize}
\end{frame}

\begin{frame}{Plan}
	\begin{wideenumerate}
		\item BLP setup
		\item \textbf{ Price elasticity/substitution patterns}
		\item Estimation: overview and typical data
		\item Identification: what if we had micro-data?
		%\item Identification: $\Sigma$ 
		%\item Estimation algorithm
		%\item Instrumental variables
		%\item Extensions
		%\item Applications
	\end{wideenumerate}
\end{frame}

\begin{frame}{Price elasticity/substitution patterns}
\begin{wideitemize}
\item \textbf{Question}: is all the complexity in the previous section necessary (in terms of heterogeneous consumers etc)? 
\begin{wideitemize}
\item What would a simpler model (for example, with homogeneous consumers) fail to capture?
\end{wideitemize}
\item \textbf{Answer}: (Typically) it is! 
\begin{wideitemize}
\item Key implication of a demand model: substitution patterns between goods/price elasticity
\item I will now argue that allowing for flexible consumer heterogeneity is \textbf{necessary to get the model to generate realistic substitution patterns}.
\end{wideitemize}
\end{wideitemize}
\end{frame}

\begin{frame}{Price elasticity/substitution patterns: implications of homogeneous consumer model}
\begin{wideitemize}
\item \textbf{Thought experiment}: switch off consumer heterogeneity.
\begin{wideitemize}
\item E.g. accomplish this by setting $\Gamma=0$ and $\Sigma=0$. So, $\mu_{ijt}=0$.
\end{wideitemize}
\item Then, just a Logit model: 
\begin{align*}
s_{jt} = \frac{\exp(\delta_{jt})}{1 + \sum_{k=1}^J \exp (\delta_{kt})}
\end{align*}
\item Price elasticities:
\[  \eta_{jkt} = \frac{\partial s_{jt}}{ \partial p_{kt} } \frac{p_{kt}} {s_{jt}} =  \left\{
\begin{array}{ll}
\alpha_0 p_{jt} (1-s_{jt}) & \text{if } $j=k$ \\
-\alpha_0 p_{kt} s_{kt} & \text{otherwise} \\
\end{array} 
\right. \]
\end{wideitemize}
\end{frame}

\begin{frame}{Price elasticity/substitution patterns: implications of homogeneous consumer model}
\begin{wideitemize}
\item Price elasticities:
\[  \eta_{jkt} = \frac{\partial s_{jt}}{ \partial p_{kt} } \frac{p_{kt}} {s_{jt}} =  \left\{
\begin{array}{ll}
\alpha_0 p_{jt} (1-s_{jt}) & \text{if } $j=k$ \\
-\alpha_0 p_{kt} s_{kt} & \text{otherwise} \\
\end{array} 
\right. \]
\item \textbf{Implication 1}: 
\begin{wideitemize}
\item Typically, $\alpha_0 (1-s_{jt}) \approx \alpha_0 $ since there are many products and market share of each product is small.
\item So, own price-elasticities (j=k) are proportional to price.
\item What does the model imply about prices vs demand elasticity?
\begin{wideitemize}
	\item This demand model implies that the \textbf{lower the price, the more inelastic is demand}
	\item Further implication: under typical pricing models $\rightarrow$ higher markup for these lower priced goods
	\item Question: do you think that these implications are reasonable predictions for the model to make?
\end{wideitemize}
\end{wideitemize}
\end{wideitemize}
\end{frame}

\begin{frame}{Price elasticity/substitution patterns: implications of homogeneous consumer model}
\begin{wideitemize}
\item Price elasticities:
\[  \eta_{jkt} = \frac{\partial s_{jt}}{ \partial p_{kt} } \frac{p_{kt}} {s_{jt}} =  \left\{
\begin{array}{ll}
\alpha_0 p_{jt} (1-s_{jt}) & \text{if } $j=k$ \\
-\alpha_0 p_{kt} s_{kt} & \text{otherwise} \\
\end{array} 
\right. \]
\item \textbf{Implication 2}: 
\begin{wideitemize}
\item Consider an increase in the price of product $k$. Concretely, think about the market for cars. The price of a BMW goes up. Do you think consumers will substitute towards a Mercedes or a Honda Civic? \pause
\begin{wideitemize}
\item Usually, we'd expect consumers to substitute towards similar products (i.e. the Mercedes)
\end{wideitemize}
\item But, the homogeneous consumer model predicts the following \textbf{diversion ratio}:
\begin{align*}
\frac{\partial s_{jt}}{ \partial p_{kt} } / \frac{\partial s_{kt}}{ \partial p_{kt} } = s_{jt} / (1-s_{kt})
\end{align*}
\item Here, substitution is proportional to market share, not how close the products are in terms of their characteristics. 
\begin{wideitemize}
\item Idea: as $p_k$ increases, consumers who no longer choose $k$ choose other options at the same frequency as the `average' consumer (i.e. in proportion to their market share).
\end{wideitemize}
\end{wideitemize}
\end{wideitemize}
\end{frame}

\begin{frame}{Price elasticity/substitution patterns}
\begin{wideitemize}
\item Price elasticities in the full BLP model (which heterogeneous consumers):
\[  \eta_{jkt} = \frac{\partial s_{jt}}{ \partial p_{kt} } \frac{p_{kt}} {s_{jt}} =  \left\{
\begin{array}{ll}
\frac{p_{jt}}{s_{jt}} \int \alpha_{it} s_{ijt} (1-s_{ijt}) dF(D_{it}, \nu_{it}) & \text{if } $j=k$ \\
-\frac{p_{kt}}{s_{jt}} \int \alpha_{it} s_{ijt} s_{ikt} dF(D_{it}, \nu_{it}) & \text{otherwise} \\
\end{array} 
\right. \]
\item Notation: $s_{ijt}$: probability that $i$ purchases $j$ in market $t$
\item \textbf{Observation 1}: 
\item Each consumer has a different price sensitivity, which is averaged to a product-specific mean price sensitivity using the individual probabilities of purchase as weights.
\item This relaxes `implication 1' from before. I.e. model could generate that low-price products have more elastic demand
\end{wideitemize}
\end{frame}

\begin{frame}{Price elasticity/substitution patterns}
\begin{wideitemize}
\item Price elasticities in the full BLP model (i.e. including heterogeneous consumers):
\[  \eta_{jkt} = \frac{\partial s_{jt}}{ \partial p_{kt} } \frac{p_{kt}} {s_{jt}} =  \left\{
\begin{array}{ll}
\frac{p_{jt}}{s_{jt}} \int \alpha_{it} s_{ijt} (1-s_{ijt}) dF(D_{it}, \nu_{it}) & \text{if } $j=k$ \\
-\frac{p_{kt}}{s_{jt}} \int \alpha_{it} s_{ijt} s_{ikt} dF(D_{it}, \nu_{it}) & \text{otherwise} \\
\end{array} 
\right. \]
\item Notation: $s_{ijt}$: probability that $i$ purchases $j$ in market $t$
\item \textbf{Observation 2}: 
\item Model generates flexible cross-product substitution patterns.
\begin{itemize}
\item How? Correlation in $\mu_{ijt}$ and $\mu_{ikt}$ induces correlation between $s_{ijt}$ and $s_{ikt}$, which then determines substitution patterns.
\end{itemize}
\item Note: alternatively, may be able to generate realistic substitution patterns with a nested logit (e.g. put the luxury cars in the same nest)
\begin{itemize}
\item ...but this requires a-priori decisions about how to segment the market. 
\end{itemize}
\end{wideitemize}
\end{frame}


\begin{frame}{Plan}
	\begin{wideenumerate}
		\item BLP setup
		\item Price elasticity/substitution patterns
		\item  \textbf{ Estimation: overview and typical data }
		\item Identification: what if we had micro-data?
		%\item Identification: $\Sigma$ 
		%\item Estimation algorithm
		%\item Instrumental variables
		%\item Extensions
		%\item Applications
	\end{wideenumerate}
\end{frame}

\begin{frame}{Estimation: setup of the problem}
	\begin{wideitemize}
		\item What are the parameters we need to estimate? \pause
		\item \underline{Linear parameters:}
		\begin{wideitemize}
			\item Parameters from the mean utility equation: $(\alpha_0, \beta_0)$
		\end{wideitemize}
		\item \underline{Nonlinear parameters}
		\begin{wideitemize}
			\item $\Gamma$: coefficients on (observed) demographics 
			\item $\Sigma$:  idiosyncratic ``taste for characteristics''
		\end{wideitemize}
		\item So, full parameter vector to estimate: $\theta = (\alpha_0, \beta_0, \Gamma, \Sigma)$.
	\end{wideitemize}
\end{frame}

\begin{frame}{Estimation: setup of the problem}
	\begin{wideitemize}
		\item Common assumptions in empirical work:
		\begin{wideitemize}
			\item We will also make these assumptions from now on. 
			\item They simplify the model, but are not necessary, see Chapter 1 of the Handbook for ways to relax these assumptions.
			\item (Always good to know common assumptions people make in empirical work, especially when they simplify the model!)
		\end{wideitemize}
		\item 1. Distribution of ``taste for characteristics'' $\nu_{it} =  ( \nu^{(0)}_{it}, ..., \nu^{(K)}_{it} )$ is independent of the distribution of demographics $D_{it}$.
		\begin{wideitemize}
			\item Then, $F(D_{it}, \nu_{it}) = F_D(D_{it})F_{\nu}(\nu_{it})$.
		\end{wideitemize}
		\item 2. Each $v_{it}^{(k)}$ is independent across $k=0,...,K$ and distributed standard normal.
	\end{wideitemize}
\end{frame}

\begin{frame}{Data: typically, the data have three types of variables}
	\begin{wideitemize}
		\item \underline{1. Quantities of the J products purchased in market t.}
		\begin{wideitemize}
			\item These are aggregations of individual consumer choices. 
			\item As we will see, only aggregate data (ie. data on total quantities at the market level) is required for identification. However, information from micro-data (i.e. data on individual choices) can be incorporated. 
			\item Remember: we are implicit assuming that definition of a market is narrow enough that consumers in the market face the same prices, characteristics, and demand shocks.
			\item Can convert aggregate quantities to market shares if we know the total market size $I_t$: $s_{jt} = q_{jt} / I_t$.
		\end{wideitemize}
	\end{wideitemize}
\end{frame}

\begin{frame}{Data: typically, the data have three types of variables}
	\begin{wideitemize}
		\item \underline{2. Prices $p_{jt}$ and ``observed'' product characteristics $x_{jt}$ of each of $J$ products in market $t$.}
		\item \underline{3. Information on consumer demographics.}
		\begin{wideitemize}
			\item In micro-data, will observe $D_{ilt}$ i.e. demographics for each $i$
			\begin{wideitemize}
				\item e.g. survey data on car purchases
			\end{wideitemize}
			\item In other applications, more aggregated data
			\begin{wideitemize}
				\item e.g. \textit{distribution} of demographics $F_t(D)$ 
				\item could obtain such data from e.g. the Current Population Survey in different cities in the US)
			\end{wideitemize}
			\item In other applications, data at a granularity somewhere between the two above cases. 
			\begin{wideitemize}
				\item e.g. average age of consumers who purchase product $j$
			\end{wideitemize}
		\end{wideitemize}
	\end{wideitemize}
\end{frame}

\begin{frame}{Plan}
	\begin{wideenumerate}
		\item BLP setup
		\item Price elasticity/substitution patterns
		\item Estimation: overview and typical data
		\item  \textbf{ Identification: what if we had micro-data?}
		%\item Identification: $\Sigma$ 
		%\item Estimation algorithm
		%\item Instrumental variables
		%\item Extensions
		%\item Applications
	\end{wideenumerate}
\end{frame}

\begin{comment}
\begin{frame}{Identification: review}
	\begin{wideitemize}
		\item \underline{What does `identification' mean?}
		\item Very vague definition: `whether the things we observe are capable of revealing the answers to  the questions we care about'?
		\item Slightly less vague (but still not precise) (`point identification'): `under the assumptions of the model, can the data distinguish the true parameter(s) $\theta_0$ from other parameters $\theta \neq \theta_0$?'
		\item Precise econometric definition: see haile.pdf on Canvas, or `The Identification Zoo' (Lewbel)
		\item Important: identification is \textbf{completely separate} from statistical precision, estimators etc
		\begin{wideitemize}
			\item Often the thought experiment for where the parameters are `identified' is: if we had unlimited data, could we distinguish the true parameters?
		\end{wideitemize}
	\end{wideitemize}
\end{frame}
\end{comment}

\begin{frame}{Identification}
	\begin{wideitemize}
		\item \underline{What variation in the data can identify the parameters?}
		\begin{wideitemize}
			\item  Precise econometric definition of identification: see haile.pdf on Canvas, or `The Identification Zoo' (Lewbel)
		\end{wideitemize}
		\item \textbf{Thought experiment}: what if we:
		\begin{wideitemize}
			 \item 1. have micro-data on individual consumers
			 \item 2. observe a single market
			 \item 3. switch off $\Sigma=0$ (i.e. ignore any idiosyncratic ``taste for characteristics'', implies heterogeneity is only driven by observed demographics)
		\end{wideitemize}
			\item Later, we will build on this intuition to discuss what to do if we had more aggregated market-level data with random taste shocks etc...
	\end{wideitemize}
\end{frame}


\begin{frame}{Identification using individual-level data}
	\begin{wideitemize}
		\item \underline{Data:}
		\item $\{y_{ij}, D_i\}_{i=1,...,I}$ where $y_{ij}=1$ for $j=0,1,...,J$ if consumer $i$ chooses product $j$ and $\sum_j y_{ij}=1$.
		\item All consumers are from the same market (same prices, same product characteristics both observed $\textbf{x}$ and unobserved $\xi$)
		\item \underline{Comment:}
		\item Estimating demand might seem hopeless here: we only see one market, so how are we supposed to get how quantities vary with prices if there is no price variation in the data?
		\item But, we will now see that it is in fact possible.
	\end{wideitemize}
\end{frame}

\begin{frame}{Identification using individual-level data}
	\begin{wideitemize}
		\item (Conditional indirect) utility from product $j$ (dropping $t$ subscript and incorporating price $p_j$ as a `characteristic' in $x_j$ to simplify exposition):
		\begin{align*}
			u_{ij} = \underbrace{x_j \beta_0 + \xi_j}_{\delta_j} + \sum_{k,l} \beta_d^{(l,k)} D_{il} x_{jk} + \epsilon_{ij}
		\end{align*}
		\item \underline{Comment:} if we didn't have the (unobserved) demand shock $\xi_j$ then we could estimate all the parameters of the model at once by maximum likelihood.
	\end{wideitemize}
\end{frame}

\begin{frame}{Identification using individual-level data}
	\begin{wideitemize}
		\item (Conditional indirect) utility from product $j$ (dropping $t$ subscript and incorporating price $p_j$ as a `characteristic' in $x_j$ to simplify exposition):
		\begin{align*}
			u_{ij} = \underbrace{x_j \beta_0 + \xi_j}_{\delta_j} + \sum_{k,l} \beta_d^{(l,k)} D_{il} x_{jk} + \epsilon_{ij}
		\end{align*}
		\item Instead, use a \textbf{two-step procedure}:
		\item 1. Include a product-specific intercept to capture $\delta_j=x_j \beta_0 + \xi_j$ (i.e. estimate $\tilde{\theta}= (\delta_1,...,\delta_J, \Gamma)$ using maximum likelihood )
		\item 2. Estimate $\beta_0$ by `projecting' estimated $\delta$'s on the $x$'s.
		\begin{wideitemize}
			\item If assume $E(\xi_j|x_j)=0$ then can use (weighted) least squares
			\item If concerned x's are correlated with $\xi$ can use $E(\xi_j|Z_j)=0$ where $Z$ are a vector of exogeneous variables (discuss more in a few slides...)
		\end{wideitemize}
	\end{wideitemize}
\end{frame}

\begin{frame}{Identification using individual-level data: step 1}
	\begin{wideitemize}
		\item Estimate the $\delta$ and $\Gamma$ parameters by maximum likelihood with utility:
		\begin{align*}
			u_{ij} = \delta_j + \sum_{k,l} \beta_d^{(l,k)} D_{il} x_{jk} + \epsilon_{ij}
		\end{align*}
		\item \underline{Identifying $\delta_j$:} 
		\item Take FOC of likelihood $\rightarrow$ can show that intercepts $\delta_j$ are found by setting observed market shares equal to the ones predicted by model. That is, if for a fixed $\Gamma$, set:
		\begin{align*}
			\hat{s}_j = \hat{\sigma}(\hat{\delta}_1,..., \hat{\delta}_J)
		\end{align*}
		\item Under some general technical conditions can invert this relationship (the `Berry inversion'):
		\begin{align*}
			\hat{\delta}_j = \hat{\sigma}_j^{-1}(\hat{s}_1,..., \hat{s}_J)
		\end{align*}
		\item Asymptotically as $I \rightarrow \infty$:
		\begin{align*}
			\delta_j = \sigma_j^{-1}(s_1,..., s_J)
		\end{align*}
	\end{wideitemize}
\end{frame}

\begin{frame}{Identification using individual-level data: step 1}
	\begin{wideitemize}
		\item To be continued...
	\end{wideitemize}
\end{frame}


\end{document}

