\documentclass[notes,11pt, aspectratio=169]{beamer}

\usepackage{pgfpages}
% These slides also contain speaker notes. You can print just the slides,
% just the notes, or both, depending on the setting below. Comment out the want
% you want.
\setbeameroption{hide notes} % Only slide
%\setbeameroption{show only notes} % Only notes
%\setbeameroption{show notes on second screen=right} % Both

%\usepackage[scaled=1.0]{helvet}
\usepackage{array}

\usepackage{tikz}
\usepackage{verbatim}
\setbeamertemplate{note page}{\pagecolor{gray!5}\insertnote}
\usetikzlibrary{positioning}
\usetikzlibrary{snakes}
\usetikzlibrary{calc}
\usetikzlibrary{arrows}
\usetikzlibrary{decorations.markings}
\usetikzlibrary{shapes.misc}
\usetikzlibrary{matrix,shapes,arrows,fit,tikzmark}
\usepackage{amsmath}
\usepackage{mathpazo}
\usepackage{hyperref}
\usepackage{lipsum}
\usepackage{multimedia}
\usepackage{graphicx}
\usepackage{multirow}
\usepackage{graphicx}
\usepackage{dcolumn}
\usepackage{bbm}
\newcolumntype{d}[0]{D{.}{.}{5}}

\usepackage{changepage}
\usepackage{appendixnumberbeamer}
\newcommand{\beginbackup}{
   \newcounter{framenumbervorappendix}
   \setcounter{framenumbervorappendix}{\value{framenumber}}
   \setbeamertemplate{footline}
   {
     \leavevmode%
     \hline
     box{%
       \begin{beamercolorbox}[wd=\paperwidth,ht=2.25ex,dp=1ex,right]{footlinecolor}%
%         \insertframenumber  \hspace*{2ex} 
       \end{beamercolorbox}}%
     \vskip0pt%
   }
 }
\newcommand{\backupend}{
   \addtocounter{framenumbervorappendix}{-\value{framenumber}}
   \addtocounter{framenumber}{\value{framenumbervorappendix}} 
}


\usepackage{graphicx}
\usepackage[space]{grffile}
\usepackage{booktabs}

% These are my colors -- there are many like them, but these ones are mine.
\definecolor{blue}{RGB}{0,114,178}
\definecolor{red}{RGB}{213,94,0}
\definecolor{yellow}{RGB}{240,228,66}
\definecolor{green}{RGB}{0,158,115}

\hypersetup{
  colorlinks=false,
  linkbordercolor = {white},
  linkcolor = {blue}
}


%% I use a beige off white for my background
\definecolor{MyBackground}{RGB}{255,253,218}

%% Uncomment this if you want to change the background color to something else
%\setbeamercolor{background canvas}{bg=MyBackground}

%% Change the bg color to adjust your transition slide background color!
\newenvironment{transitionframe}{
  \setbeamercolor{background canvas}{bg=white}
  \begin{frame}}{
    \end{frame}
}

\setbeamercolor{frametitle}{fg=blue}
\setbeamercolor{title}{fg=black}
\setbeamertemplate{footline}[frame number]
\setbeamertemplate{navigation symbols}{} 
\setbeamertemplate{itemize items}{-}
\setbeamercolor{itemize item}{fg=blue}
\setbeamercolor{itemize subitem}{fg=blue}
\setbeamercolor{enumerate item}{fg=blue}
\setbeamercolor{enumerate subitem}{fg=blue}
\setbeamercolor{button}{bg=MyBackground,fg=blue,}



% If you like road maps, rather than having clutter at the top, have a roadmap show up at the end of each section 
% (and after your introduction)
% Uncomment this is if you want the roadmap!
% \AtBeginSection[]
% {
%    \begin{frame}
%        \frametitle{Roadmap of Talk}
%        \tableofcontents[currentsection]
%    \end{frame}
% }
\setbeamercolor{section in toc}{fg=blue}
\setbeamercolor{subsection in toc}{fg=red}
\setbeamersize{text margin left=1em,text margin right=1em} 

\newenvironment{wideitemize}{\itemize\addtolength{\itemsep}{10pt}}{\enditemize}
\newenvironment{wideenumerate}{\enumerate\addtolength{\itemsep}{10pt}}{\endenumerate}

\usepackage{environ}
\NewEnviron{videoframe}[1]{
  \begin{frame}
    \vspace{-8pt}
    \begin{columns}[onlytextwidth, T] % align columns
      \begin{column}{.58\textwidth}
        \begin{minipage}[t][\textheight][t]
          {\dimexpr\textwidth}
          \vspace{8pt}
          \hspace{4pt} {\Large \sc \textcolor{blue}{#1}}
          \vspace{8pt}
          
          \BODY
        \end{minipage}
      \end{column}%
      \hfill%
      \begin{column}{.42\textwidth}
        \colorbox{green!20}{\begin{minipage}[t][1.2\textheight][t]
            {\dimexpr\textwidth}
            Face goes here
          \end{minipage}}
      \end{column}%
    \end{columns}
  \end{frame}
}

\title[]{\textcolor{blue}{Demand Estimation 1 \\ PhD Industrial Organization}}
\author[PGP]{}
\institute[FRBNY]{\small{\begin{tabular}{c c c}
Nicholas Vreugdenhil \\
\end{tabular}}}
\date{} 

\begin{document}

%%% TIKZ STUFF
\tikzset{   
        every picture/.style={remember picture,baseline},
        every node/.style={anchor=base,align=center,outer sep=1.5pt},
        every path/.style={thick},
        }
\newcommand\marktopleft[1]{%
    \tikz[overlay,remember picture] 
        \node (marker-#1-a) at (-.3em,.3em) {};%
}
\newcommand\markbottomright[2]{%
    \tikz[overlay,remember picture] 
        \node (marker-#1-b) at (0em,0em) {};%
}
\tikzstyle{every picture}+=[remember picture] 
\tikzstyle{mybox} =[draw=black, very thick, rectangle, inner sep=10pt, inner ysep=20pt]
\tikzstyle{fancytitle} =[draw=black,fill=red, text=white]
%%%% END TIKZ STUFF

% Title Slide
\begin{frame}
\maketitle
  \centering
\end{frame}

% INTRO

\begin{comment}
\begin{frame}{Plan for today}
  \begin{wideenumerate}
    \item Why is estimating demand useful?
    \item General setup of discrete choice demand models
    \item Homogenous case
    \item Solving the full model
    \item Instruments 
    \item Estimation algorithm
    \item Extensions
  \end{wideenumerate}
\end{frame}
\end{comment}

\begin{frame}{Why is estimating demand useful?}
	  \begin{wideitemize}
		\item Quantifying market power (think: inverse elasticity rule)
		\item Effects of a merger on prices
		\item Value of new goods
		\item Any question about consumer welfare
		\item Numerous other applications: e.g. school choice models, health insurance models, etc
		\item As a component of a larger model with a supply-side
	\end{wideitemize}
\end{frame}

\begin{frame}{Discrete choice demand models}
	\begin{wideitemize}
		\item One of the key IO methods: \textbf{estimating demand in differentiated-product markets}.
		\item We will discuss the `BLP' model: Berry, Levinsohn, and Pakes ``Automobile Prices in Market Equilibrium'' (1995)
		\begin{wideitemize}
			\item Classic paper, and extremely influential.
			\item Essentially, a `mixed logit' model 
		\end{wideitemize}
	\end{wideitemize}
	\begin{figure}
		\includegraphics[scale=0.3]{civic.jpeg}
	\end{figure}
\end{frame}

\begin{frame}{Discrete choice demand models}
	\begin{wideitemize}
		\item Many people consider the treatment in Nevo (2000) and Nevo (2001) to be more accessible if you are seeing the model for the first time. 
		\begin{wideitemize}
			\item I will closely follow the notation and content from HIO4 Chapter 2 (also written by Nevo along with Gandhi)
			\item HIO4 Chapter 1 also covers demand estimation, but more from the perspective of thinking deeply about identification. In the interests of time, I will focus a little less on this chapter.
		\end{wideitemize}
		\item As well as being an extremely useful method in its own right, understanding the BLP model will help you to learn skills more broadly useful in structural  modeling work (e.g. computational simulation, dealing with endogeneity, etc).
	\end{wideitemize}
\end{frame}

\begin{frame}{Plan for today}
	\begin{wideenumerate}
		\item General setup of the BLP model
		\item Price elasticity/substitution patterns
	\end{wideenumerate}
\end{frame}

\begin{frame}{Plan for today}
	\begin{wideenumerate}
		\item \textbf{General setup of the BLP model}
		\item Price elasticity/substitution patterns
	\end{wideenumerate}
\end{frame}

\begin{frame}{Discrete choice demand models: general setup of `BLP' (`mixed logit')}{Berry, Levinsohn, and Pakes (1995)}
\begin{align*}
	u_{ijt} = x_{jt} \beta_{it} + \alpha_{it} p_{jt} + \xi_{jt} + \epsilon_{ijt}
\end{align*}
\begin{wideitemize}
	\item i: denotes consumer; t: denotes market; j: denotes product
	\item $u_{ijt}$: utility of consumer i for product j in market t
	\item $x_{jt}$: vector of observed product characteristics
	\item $p_{jt}$: price of product $j$ in market $t$
	\item $\xi_{jt}$: unobserved demand shock/product characteristics (observed by consumers and firms but not the econometrician)
	\item $\epsilon_{ijt}$: idiosyncratic taste shock (typically i.i.d. across (i,j,t) and drawn from type-1 extreme value distribution with scale parameter normalized to 1)
\end{wideitemize}
\end{frame}

\begin{frame}{Discrete choice demand models: general setup of `BLP'}{Berry, Levinsohn, and Pakes (1995)}
	\begin{align*}
		u_{ijt} = x_{jt} \beta_{it} + \alpha_{it} p_{jt} + \xi_{jt} + \epsilon_{ijt}
	\end{align*}
	\begin{wideitemize}
		\item Why is it useful to write the error term in this particular way ($\xi_{jt}+ \epsilon_{ijt}$)?
		\item Later, we will see writing the error term like this is helpful when estimating the model using market-level data.
		\begin{wideitemize}
			\item Idea: $\epsilon_{ijt}$ do not impact pricing, but firms observe $\xi_{jt}$ for all products $j$ when setting prices.
			\item Therefore, $\xi_{jt}$ will typically be the econometric error term (+ prices will be correlated with it). 
			\item Notice that we are being very explicit about the problem of price endogeneity in the model.
			\item This will allow us later to be explicit about how we solve it (spoiler: we will use instrumental variables).
		\end{wideitemize}
	\end{wideitemize}
\end{frame}

\begin{frame}{Discrete choice demand models: general setup of `BLP'}{Berry, Levinsohn, and Pakes (1995)}
	\begin{align*}
		u_{ijt} = x_{jt} \beta_{it} + \alpha_{it} p_{jt} + \xi_{jt} + \epsilon_{ijt}
	\end{align*}
	\begin{wideitemize}
		\item \underline{More on $\xi_{jt}$}:
		\item These unobservables do \underline{not} vary within a market. But what are the boundaries of a `market'?
		\begin{wideitemize}
			\item Geography: city? state? country?
			\item Time: week? month? day?
			\item Answers depend on the specific industry details and data availability - needs to be thought about carefully.
		\end{wideitemize}
	\end{wideitemize}
\end{frame}

\begin{frame}{Discrete choice demand models: general setup of `BLP'}{Berry, Levinsohn, and Pakes (1995)}
	\begin{align*}
		u_{ijt} = x_{jt} \beta_{it} + \alpha_{it} p_{jt} + \xi_{jt} + \epsilon_{ijt}
	\end{align*}
	\begin{wideitemize}
		\item \underline{More on $\alpha_{it}$ (price sensitivity)}:
		\begin{align*}
			\alpha_{it} = \alpha_0 + \sum_{l=1}^L \alpha_l D_{ilt} + \alpha_{\nu} \nu_{it}^{(0)}
		\end{align*}
		\item $D_{ilt}$: $L$ `demographic' variables (e.g. income, age, family size). This is either observed or the distribution is known. 
		\begin{wideitemize}
			\item E.g. get distribution of income, education, etc from Current Population Survey in different cities in the US.
		\end{wideitemize}
		\item $\nu_{it}^{(0)}$: random variable
		\begin{wideitemize}
			\item Captures variation in preferences beyond what the demographic variables can explain (for cars, e.g. dog ownership is typically not observed but potentially a factor in car purchase)
			%\item Later, add some additional parametric assumptions on the distributions of $\nu_{it}^{(0)}$ and $D_{ilt}$. 
		\end{wideitemize}
	\end{wideitemize}
\end{frame}

\begin{frame}{Discrete choice demand models: general setup of `BLP'}{Berry, Levinsohn, and Pakes (1995)}
	\begin{align*}
		u_{ijt} = x_{jt} \beta_{it} + \alpha_{it} p_{jt} + \xi_{jt} + \epsilon_{ijt}
	\end{align*}
	\begin{wideitemize}
		\item \underline{More on $\beta_{it}$ (product characteristics sensitivity)}:
		\item Denote $\beta_{it}^{(k)}$ the weight consumer $i$ places on characteristic $x_{jt}^{(k)}$.
		\begin{align*}
			\beta_{it}^{(k)} = \beta_0^{(k)} + \sum_{l=1}^L \beta_d^{(l,k)} D_{ilt} + \beta_{\nu}^{(k)} \nu_{it}^{(k)}
		\end{align*}
		\item $\beta_0^{(k)}$: parameter common to all consumers
		\item $D_{ilt}$: $L$ `demographic' variables (e.g. income, age, family size)
		\item $\nu_{it}^{(k)}$: random variable
	\end{wideitemize}
\end{frame}

\begin{frame}{Discrete choice demand models: general setup of `BLP'}{Berry, Levinsohn, and Pakes (1995)}
	\begin{align*}
		u_{ijt} = x_{jt} \beta_{it} + \alpha_{it} p_{jt} + \xi_{jt} + \epsilon_{ijt}
	\end{align*}
	\begin{wideitemize}
		\item \underline{Outside good j=0}:
		\item Consumers have the option to not purchase any of the $J$ products.
		\item Indirect utility:
		\begin{align*}
			u_{i0t} = \epsilon_{i0t}
		\end{align*}
		\item In above, non-idiosyncratic components are normalized to 0.
		\item So, utility of the other $J$ goods should be interpreted as relative to this outside good.
	\end{wideitemize}
\end{frame}

\begin{comment}
\begin{frame}{Discrete choice demand models: general setup of `BLP'}{Berry, Levinsohn, and Pakes (1995)}
	\begin{align*}
		u_{ijt} = x_{jt} \beta_{it} + \alpha_{it} p_{jt} + \xi_{jt} + \epsilon_{ijt}
	\end{align*}
	\begin{wideitemize}
		\item \underline{What about income?}:
		\item In the above equation, $p_{jt}$ should really be $y_i-p_{jt}$ where $y_i$ is income.
		\item Leaving it out has no impact for choices however, and just simplifies exposition. Why? \pause
		\begin{wideitemize}
			\item Income enters linearly to all options, and only relative differences in utilities matter for choice probabilities 
			\item If $y_i-p_{jt}$ entered non-linearly then it would affect the choice probabilities and should be included
		\end{wideitemize}
	\end{wideitemize}
\end{frame}

\begin{frame}{Discrete choice demand models: general setup of `BLP'}{Berry, Levinsohn, and Pakes (1995)}
	\begin{wideitemize}
		\item \underline{Even more notation}: 
		\begin{wideitemize}
			\item Recall $L$ number of demographic vars, $K$ number of product characteristics
		\end{wideitemize}
		\item Define:
		\begin{wideitemize}
		 	\item The \textbf{mean utility} of product j in market t: $\delta_{jt} = x_{jt} \beta_0 + \alpha_0 p_{jt} + \xi_{jt}$
			\item $\Gamma$: $(K+1) \times L$ matrix with coefficients of demographic variables
			\item $\Sigma$: $(K+1) \times (K+1)$ diagonal matrix with diagonal $(\alpha_{\nu},\beta_{\nu}^{(1)},..., \beta_{\nu}^{(K)} )$
			\item $\nu_{it} = ( \nu^{(0)}_{it}, ..., \nu^{(K)}_{it} )^T$
			\item $\mu_{ijt} = (x_{jt}, p_{jt}) \cdot (\Gamma D_{it} + \Sigma \nu_{it})$
		\end{wideitemize}
		\item Then we can rewrite our utility equation as:
	\end{wideitemize}
	\begin{align*}
			u_{ijt} = \underbrace{\delta_{jt}}_{\text{mean utility}} + \underbrace{\mu_{ijt}}_{\text{interaction between  consumer tastes + product characteristics}}+ \underbrace{\epsilon_{ijt}}_{ \text{idiosyncratic error}}
	\end{align*}
\end{frame}

\begin{frame}{Discrete choice demand models: general setup of `BLP'}{Berry, Levinsohn, and Pakes (1995)}
	\begin{wideitemize}
		\item \underline{Review of where we are}: we just characterized a very flexible model of consumer utility. 
		\item Assuming i.i.d. extreme value errors $\epsilon_{ijt}$ the probability consumer $i$ chooses product $j$ is:
		\begin{align*}
			\frac{ \exp(\delta_{jt} + \mu_{ijt}) }{1 + \sum_{k=1}^J \exp(\delta_{kt} + \mu_{ikt})}
		\end{align*}
		\item And \textbf{demand} (the share of consumers who purchase good $j$ in market $t$) is:
		\begin{align*}
			s_{jt} = \sigma_j ( \boldsymbol{\delta}_t,  \boldsymbol{x}_t,  \boldsymbol{p}_t;\Gamma, \Sigma  ) = \int \frac{ \exp(\delta_{jt} + \mu_{ijt}) }{1 + \sum_{k=1}^J \exp(\delta_{kt} + \mu_{ikt})} dF(D_{it}, v_{it})
		\end{align*}
		\item Here:
		\begin{wideitemize}
			\item $\boldsymbol{\delta}_t,  \boldsymbol{x}_t,  \boldsymbol{p}_t$ are vectors of mean utilities, observed product characteristics, prices, in market $t$
			\item $F$ is the joint distribution of observed demographics and unobserved tastes
		\end{wideitemize}
	\end{wideitemize}
\end{frame}


\begin{frame}{Plan for today}
	\begin{wideenumerate}
		\item General setup of the BLP model
		\item \textbf{Price elasticity/substitution patterns}
	\end{wideenumerate}
\end{frame}

\begin{frame}{BLP: price elasticity/substitution patterns}
	\begin{wideitemize}
		\item \textbf{Question}: is all the complexity in the previous section necessary (in terms of heterogeneous consumers etc)? 
		\begin{wideitemize}
			\item What would a simpler model (for example, with homogeneous consumers) fail to capture?
		\end{wideitemize}
		\item \textbf{Answer}: (Typically) it is! 
		\begin{wideitemize}
			\item Key implication of a demand model: substitution patterns between goods/price elasticity
			\item I will now argue that allowing for flexible consumer heterogeneity is \textbf{necessary to get the model to generate realistic substitution patterns}.
		\end{wideitemize}
	\end{wideitemize}
\end{frame}

\begin{frame}{BLP: price elasticity/substitution patterns: implications of homogeneous consumer model}
	\begin{wideitemize}
		\item \textbf{Thought experiment}: switch off consumer heterogeneity.
		\begin{wideitemize}
			\item E.g. accomplish this by setting $\Gamma=0$ and $\Sigma=0$. So, $\mu_{ijt}=0$.
		\end{wideitemize}
		\item Then, just a Logit model: 
		\begin{align*}
			s_{jt} = \frac{\exp(\delta_{jt})}{1 + \sum_{k=1}^J \exp (\delta_{kt})}
		\end{align*}
		\item Price elasticities:
	\[  \eta_{jkt} = \frac{\partial s_{jt}}{ \partial p_{kt} } \frac{p_{kt}} {s_{jt}} =  \left\{
	\begin{array}{ll}
		\alpha_0 p_{jt} (1-s_{jt}) & \text{if } $j=k$ \\
		-\alpha_0 p_{kt} s_{kt} & \text{otherwise} \\
	\end{array} 
	\right. \]
	\end{wideitemize}
\end{frame}

\begin{frame}{BLP: price elasticity/substitution patterns: implications of homogeneous consumer model}
	\begin{wideitemize}
		\item Price elasticities:
		\[  \eta_{jkt} = \frac{\partial s_{jt}}{ \partial p_{kt} } \frac{p_{kt}} {s_{jt}} =  \left\{
		\begin{array}{ll}
			\alpha_0 p_{jt} (1-s_{jt}) & \text{if } $j=k$ \\
			-\alpha_0 p_{kt} s_{kt} & \text{otherwise} \\
		\end{array} 
		\right. \]
		\item \textbf{Implication 1}: 
			\begin{wideitemize}
				\item Typically, $\alpha_0 (1-s_{jt}) \approx \alpha_0 $ since there are many products and market share of each product is small.
				\item So, own price-elasticities (j=k) are proportional to price.
				\item This demand model implies that the \textbf{lower the price, the more inelastic is demand}
				\item Further implication: under typical pricing models $\rightarrow$ higher markup for these lower priced goods
				\item Question: do you think that these implications are reasonable predictions for the model to make?
			\end{wideitemize}
	\end{wideitemize}
\end{frame}

\begin{frame}{BLP: price elasticity/substitution patterns: implications of homogeneous consumer model}
	\begin{wideitemize}
		\item Price elasticities:
		\[  \eta_{jkt} = \frac{\partial s_{jt}}{ \partial p_{kt} } \frac{p_{kt}} {s_{jt}} =  \left\{
		\begin{array}{ll}
			\alpha_0 p_{jt} (1-s_{jt}) & \text{if } $j=k$ \\
			-\alpha_0 p_{kt} s_{kt} & \text{otherwise} \\
		\end{array} 
		\right. \]
		\item \textbf{Implication 2}: 
		\begin{wideitemize}
			\item Consider an increase in the price of product $k$. Concretely, think about the market for cars. The price of a BMW goes up. Do you think consumers will substitute towards a Mercedes or a Honda Civic? \pause
			\begin{wideitemize}
				\item Usually, we'd expect consumers to substitute towards similar products (i.e. the Mercedes)
			\end{wideitemize}
			\item But, the homogeneous consumer model predicts the following \textbf{diversion ratio}:
			\begin{align*}
				\frac{\partial s_{jt}}{ \partial p_{kt} } / \frac{\partial s_{kt}}{ \partial p_{kt} } = s_{jt} / (1-s_{kt})
			\end{align*}
			\item Here, substitution is proportional to market share, not how close the products are in terms of their characteristics. 
							\begin{wideitemize}
			\item Idea: as $p_k$ increases, consumers who no longer choose $k$ choose other options at the same frequency as the `average' consumer (i.e. in proportion to their market share).
							\end{wideitemize}
		\end{wideitemize}
	\end{wideitemize}
\end{frame}

\begin{frame}{BLP: price elasticity/substitution patterns}
	\begin{wideitemize}
		\item Price elasticities in the full BLP model (which heterogeneous consumers):
		\[  \eta_{jkt} = \frac{\partial s_{jt}}{ \partial p_{kt} } \frac{p_{kt}} {s_{jt}} =  \left\{
		\begin{array}{ll}
			-\frac{p_{jt}}{s_{jt}} \int \alpha_{it} s_{ijt} (1-s_{ijt}) dF(D_{it}, \nu_{it}) & \text{if } $j=k$ \\
			\frac{p_{kt}}{s_{jt}} \int \alpha_{it} s_{ijt} s_{ikt} dF(D_{it}, \nu_{it}) & \text{otherwise} \\
		\end{array} 
		\right. \]
	\item Notation: $s_{ijt}$: probability that $i$ purchases $j$ in market $t$
	\item \textbf{Observation 1}: 
	\item Each consumer has a different price sensitivity, which is averaged to a product-specific mean price sensitivity using the individual probabilities of purchase as weights.
	\item This relaxes `implication 1' from before. I.e. model could generate that low-price products have more elastic demand
	\end{wideitemize}
\end{frame}

\begin{frame}{BLP: price elasticity/substitution patterns}
	\begin{wideitemize}
		\item Price elasticities in the full BLP model (i.e. including heterogeneous consumers):
		\[  \eta_{jkt} = \frac{\partial s_{jt}}{ \partial p_{kt} } \frac{p_{kt}} {s_{jt}} =  \left\{
		\begin{array}{ll}
			-\frac{p_{jt}}{s_{jt}} \int \alpha_{it} s_{ijt} (1-s_{ijt}) dF(D_{it}, \nu_{it}) & \text{if } $j=k$ \\
			\frac{p_{kt}}{s_{jt}} \int \alpha_{it} s_{ijt} s_{ikt} dF(D_{it}, \nu_{it}) & \text{otherwise} \\
		\end{array} 
		\right. \]
		\item Notation: $s_{ijt}$: probability that $i$ purchases $j$ in market $t$
		\item \textbf{Observation 2}: 
		\item Model generates flexible cross-product substitution patterns.
		\begin{itemize}
			\item How? Correlation in $\mu_{ijt}$ and $\mu_{ikt}$ induces correlation between $s_{ijt}$ and $s_{ikt}$, which then determines substitution patterns.
		\end{itemize}
		\item Note: alternatively, may be able to generate realistic substitution patterns with a nested logit (e.g. put the luxury cars in the same nest)
		\begin{itemize}
			\item ...but this requires a-priori decisions about how to segment the market. 
		\end{itemize}
	\end{wideitemize}
\end{frame}
	\end{comment}

\end{document}
