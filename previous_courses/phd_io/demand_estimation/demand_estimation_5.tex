\documentclass[notes,11pt, aspectratio=169]{beamer}

\usepackage{pgfpages}
% These slides also contain speaker notes. You can print just the slides,
% just the notes, or both, depending on the setting below. Comment out the want
% you want.
\setbeameroption{hide notes} % Only slide
%\setbeameroption{show only notes} % Only notes
%\setbeameroption{show notes on second screen=right} % Both

%\usepackage[scaled=1.0]{helvet}
\usepackage{array}

\usepackage{tikz}
\usepackage{verbatim}
\setbeamertemplate{note page}{\pagecolor{gray!5}\insertnote}
\usetikzlibrary{positioning}
\usetikzlibrary{snakes}
\usetikzlibrary{calc}
\usetikzlibrary{arrows}
\usetikzlibrary{decorations.markings}
\usetikzlibrary{shapes.misc}
\usetikzlibrary{matrix,shapes,arrows,fit,tikzmark}
\usepackage{amsmath}
\usepackage{mathpazo}
\usepackage{hyperref}
\usepackage{lipsum}
\usepackage{multimedia}
\usepackage{graphicx}
\usepackage{multirow}
\usepackage{graphicx}
\usepackage{dcolumn}
\usepackage{bbm}
\newcolumntype{d}[0]{D{.}{.}{5}}

\usepackage{changepage}
\usepackage{appendixnumberbeamer}
\newcommand{\beginbackup}{
   \newcounter{framenumbervorappendix}
   \setcounter{framenumbervorappendix}{\value{framenumber}}
   \setbeamertemplate{footline}
   {
     \leavevmode%
     \hline
     box{%
       \begin{beamercolorbox}[wd=\paperwidth,ht=2.25ex,dp=1ex,right]{footlinecolor}%
%         \insertframenumber  \hspace*{2ex} 
       \end{beamercolorbox}}%
     \vskip0pt%
   }
 }
\newcommand{\backupend}{
   \addtocounter{framenumbervorappendix}{-\value{framenumber}}
   \addtocounter{framenumber}{\value{framenumbervorappendix}} 
}


\usepackage{graphicx}
\usepackage[space]{grffile}
\usepackage{booktabs}

% These are my colors -- there are many like them, but these ones are mine.
\definecolor{blue}{RGB}{0,114,178}
\definecolor{red}{RGB}{213,94,0}
\definecolor{yellow}{RGB}{240,228,66}
\definecolor{green}{RGB}{0,158,115}

\hypersetup{
  colorlinks=false,
  linkbordercolor = {white},
  linkcolor = {blue}
}


%% I use a beige off white for my background
\definecolor{MyBackground}{RGB}{255,253,218}

%% Uncomment this if you want to change the background color to something else
%\setbeamercolor{background canvas}{bg=MyBackground}

%% Change the bg color to adjust your transition slide background color!
\newenvironment{transitionframe}{
  \setbeamercolor{background canvas}{bg=white}
  \begin{frame}}{
    \end{frame}
}

\setbeamercolor{frametitle}{fg=blue}
\setbeamercolor{title}{fg=black}
\setbeamertemplate{footline}[frame number]
\setbeamertemplate{navigation symbols}{} 
\setbeamertemplate{itemize items}{-}
\setbeamercolor{itemize item}{fg=blue}
\setbeamercolor{itemize subitem}{fg=blue}
\setbeamercolor{enumerate item}{fg=blue}
\setbeamercolor{enumerate subitem}{fg=blue}
\setbeamercolor{button}{bg=MyBackground,fg=blue,}



% If you like road maps, rather than having clutter at the top, have a roadmap show up at the end of each section 
% (and after your introduction)
% Uncomment this is if you want the roadmap!
% \AtBeginSection[]
% {
%    \begin{frame}
%        \frametitle{Roadmap of Talk}
%        \tableofcontents[currentsection]
%    \end{frame}
% }
\setbeamercolor{section in toc}{fg=blue}
\setbeamercolor{subsection in toc}{fg=red}
\setbeamersize{text margin left=1em,text margin right=1em} 

\newenvironment{wideitemize}{\itemize\addtolength{\itemsep}{10pt}}{\enditemize}
\newenvironment{wideenumerate}{\enumerate\addtolength{\itemsep}{10pt}}{\endenumerate}

\usepackage{environ}
\NewEnviron{videoframe}[1]{
  \begin{frame}
    \vspace{-8pt}
    \begin{columns}[onlytextwidth, T] % align columns
      \begin{column}{.58\textwidth}
        \begin{minipage}[t][\textheight][t]
          {\dimexpr\textwidth}
          \vspace{8pt}
          \hspace{4pt} {\Large \sc \textcolor{blue}{#1}}
          \vspace{8pt}
          
          \BODY
        \end{minipage}
      \end{column}%
      \hfill%
      \begin{column}{.42\textwidth}
        \colorbox{green!20}{\begin{minipage}[t][1.2\textheight][t]
            {\dimexpr\textwidth}
            Face goes here
          \end{minipage}}
      \end{column}%
    \end{columns}
  \end{frame}
}

\title[]{\textcolor{blue}{Demand Estimation 5 \\ PhD Industrial Organization}}
\author[PGP]{}
\institute[FRBNY]{\small{\begin{tabular}{c c c}
Nicholas Vreugdenhil \\
\end{tabular}}}
\date{} 

\begin{document}

%%% TIKZ STUFF
\tikzset{   
        every picture/.style={remember picture,baseline},
        every node/.style={anchor=base,align=center,outer sep=1.5pt},
        every path/.style={thick},
        }
\newcommand\marktopleft[1]{%
    \tikz[overlay,remember picture] 
        \node (marker-#1-a) at (-.3em,.3em) {};%
}
\newcommand\markbottomright[2]{%
    \tikz[overlay,remember picture] 
        \node (marker-#1-b) at (0em,0em) {};%
}
\tikzstyle{every picture}+=[remember picture] 
\tikzstyle{mybox} =[draw=black, very thick, rectangle, inner sep=10pt, inner ysep=20pt]
\tikzstyle{fancytitle} =[draw=black,fill=red, text=white]
%%%% END TIKZ STUFF

% Title Slide
\begin{frame}
\maketitle
  \centering
\end{frame}

% INTRO
\begin{frame}{Plan}
	\begin{wideenumerate}
		\item Distinguishing models of competition
		\item Supply-side moments
		\item Welfare from new products: theory
		\item Welfare from new products: application
		\item Apple-cinnamon cheerios war
		\item Main takeaways of demand estimation
	\end{wideenumerate}
\end{frame}

\begin{frame}{Plan}
	\begin{wideenumerate}
		\item \textbf{Distinguishing models of competition}
		\item Supply-side moments
		\item Welfare from new products: theory
		\item Welfare from new products: application
		\item Apple-cinnamon cheerios war
		\item Main takeaways of demand estimation
	\end{wideenumerate}
\end{frame}

\begin{frame}{Application: distinguishing between models of competition}
	\begin{wideitemize}
		\item \textbf{Overall strategy:}
		\item 1. Estimate demand
		\item 2. Use estimates + pricing rules implied by different models of firm conduct to get price-cost margins (PCM)
		\begin{wideitemize}
			\item Challenge: costs not observed
		\end{wideitemize}
		\item 3. Compare predicted PCM from different models of conduct to true PCM
		\begin{wideitemize}
			\item See which model of firm conduct best matches the data.
		\end{wideitemize}
		\item \textbf{Main finding:}
		\item Nash-Bertrand pricing best matches observed PCM
	\end{wideitemize}
\end{frame}

\begin{frame}{Application: distinguishing between models of competition}
\begin{wideitemize}
\item \textbf{Demand side:}
\item Utility model is exactly what we have seen:
\begin{align*}
u_{ijt} = x_{jt} \beta_i + \alpha_i p_{jt} + \xi_{jt} + \epsilon_{ijt}
\end{align*}
\item \textbf{Data:}
\item Use top 25 cereal brands
\item Scanner data $\rightarrow$ get market shares, prices, etc. 
\begin{wideitemize}
\item Aggregate to MSA-quarter level = 1124 markets.
\end{wideitemize}
\item Advertising data
\item Cereal box characteristics (nutritional information); subjective characteristics (`mushy')
\item Demographics from CPS
\end{wideitemize}
\end{frame}

\begin{frame}{Application: distinguishing between models of competition}
\begin{wideitemize}
\item \textbf{Supply side:}
\item Profits of firm $f$:
\begin{align*}
\pi_f = \sum_{j \in \mathcal{J}_f} \left[ (p_j - mc_j) q_j(\textbf{p}) - FC_j \right]
\end{align*}
\item Where:
\begin{wideitemize}
\item $\pi_f$: profits of firm $f$
\item $p_j$: price of product $j$; $\textbf{p}$: vector of prices
\item $mc_j$: marginal cost
\item $FC_j$: fixed cost
\item $q_j$: quantity of product $j$ (depends on \underline{all} prices)
\item $\mathcal{J}_f$: set of products that firm $f$ maximizes profit over
\end{wideitemize}
\end{wideitemize}
\end{frame}

\begin{frame}{Application: distinguishing between models of competition}
\begin{wideitemize}
\item \textbf{Supply side:}
\item Define \textbf{conduct structure} as JxJ matrix:
\[ H_{jk} = \begin{cases} 
1 & \text{ if } \exists $f$ \text{ where } \{j,k\} \subset  \mathcal{J}_f \\
0 & otherwise
\end{cases}
\]
\item Elements of H either 0 or 1
\item If value of element = 1: then product $j$ and $k$ are priced \textit{as if jointly owned}
\item \underline{Examples}:
\begin{wideitemize}
\item Single product firm pricing: identity matrix
\item Joint profit maximization: matrix of 1's
\end{wideitemize}
\end{wideitemize}
\end{frame}

\begin{frame}{Application: distinguishing between models of competition}
\begin{wideitemize}
\item \textbf{Supply side:}
\item Define $\Omega_{jk} = -\partial q_k / \partial p_j \cdot H_{jk}$ 
\begin{wideitemize}
\item Recall: j is index, k is column
\end{wideitemize}
\item First order condition of firms' profit maximization problem (bold denotes vectors):
\begin{align*}
\textbf{q} (\textbf{p}) - \Omega (\textbf{p} - \textbf{mc}) = 0
\end{align*}
\item Implies pricing equation:
\begin{align*}
\textbf{p} - \textbf{mc} = \Omega^{-1} \textbf{q}(\textbf{p})
\end{align*}
\item \underline{Important:} above equation implies that given conduct structure + estimates of demand substitution $\Omega$ $\rightarrow$ can measure price-cost margins \textit{without observing cost data}.
\end{wideitemize}
\end{frame}


\begin{frame}{Application: distinguishing between models of competition: estimation}
\begin{wideitemize}
\item Only estimate demand
\item Use BLP algorithm
\item Instruments: Hausman instruments  (prices in other markets)
\end{wideitemize}
\end{frame}

\begin{frame}
\begin{figure}
\includegraphics[scale=0.16]{nevo_1.jpeg}
\end{figure}
\end{frame}

\begin{frame}
\begin{figure}
\includegraphics[scale=0.16]{nevo_4.jpeg}
\end{figure}
\end{frame}

\begin{frame}
\begin{figure}
\includegraphics[scale=0.3]{nevo_3.jpeg}
\end{figure}
\end{frame}

\begin{frame}{Application: distinguishing between models of competition: mergers}
\begin{wideitemize}
\item Common use of the framework in this paper can also be used to determine the effects of a merger. How would you do this? \pause
\item \underline{Answer:}
\item 1. Using pre-merger data estimate demand and recover marginal costs by inverting the pricing equation:
\begin{align*}
\textbf{mc} = 	\textbf{p} -  \Omega^{-1} \textbf{q}(\textbf{p})
\end{align*}
\item 2. Change the conduct structure $H$ so that the merging firms jointly maximize the profits of their products
\end{wideitemize}
\end{frame}

\begin{frame}{Plan}
	\begin{wideenumerate}
		\item \textbf{Supply-side moments}
		\item Welfare from new products: theory
		\item Welfare from new products: application
		\item Apple-cinnamon cheerios war
		\item Main takeaways of demand estimation
	\end{wideenumerate}
\end{frame}


\begin{frame}{Supply-side moments}
	\begin{wideitemize}
		\item Previously, we used our demand model + different assumptions about how firms set prices to `test' models of firm conduct.
		\item In a related idea, we could alternatively:
		\begin{wideitemize}
			\item make assumptions about how firms choose prices
			\item utilize data on the supply side to help identify the demand model
		\end{wideitemize}
	\end{wideitemize}
\end{frame}

\begin{frame}{Supply-side moments}
	\begin{wideitemize}
		\item Assume marginal cost is given by:
		\begin{align*}
			mc_{jt} = w_{jt} \mathbold{\gamma} + \omega_{jt}
		\end{align*}
		\item Where:
		\begin{wideitemize}
			\item $w_{jt}$: vector of observed characteristics of product $j$
			\item $\omega_{jt}$: unobserved component
			\item $ \mathbold{\gamma} $: parameters to be estimated
		\end{wideitemize}
	\end{wideitemize}
\end{frame}

\begin{frame}{Supply-side moments}
	\begin{wideitemize}
		\item Also assume Nash-Bertrand pricing model and combine with cost parametrization on previous slide (using the notation from the previous lecture)
		\begin{align*}
			\textbf{p}_t = w_t  \mathbold{\gamma} + \Omega^{-1} \textbf{q}(\textbf{p}_t) + \mathbold{\omega}_t
		\end{align*}
		\item Can form supply-side moments: $E(\omega_{jt} | \textbf{Z}_t) = 0$.
		\item Here: $\textbf{Z}_t$: vector of IVs that include product characteristics, and cost shifters
		\item Note that above equation is informative about both supply parameters $ \mathbold{\gamma}$ and the demand parameters (which affect $\Omega$)
		\item Can include these additional moments in the GMM step
	\end{wideitemize}
\end{frame}

\begin{frame}{Plan}
	\begin{wideenumerate}
		\item Supply-side moments
		\item \textbf{Welfare from new products: theory}
		\item Welfare from new products: application
		\item Apple-cinnamon cheerios war
		\item Main takeaways of demand estimation
	\end{wideenumerate}
\end{frame}

\begin{frame}{Consumer welfare}
	\begin{wideitemize}
		\item Define the \textbf{consumer surplus from a logit model}
		\item Expected utility prior to observing the i.i.d. logit draws from $ \{1, 2, ..., J\}$ choice alternatives: (the \textbf{`inclusive value'})
		\begin{align*}
			\omega_{it} = E_{ \{ \epsilon_{i0t},...,\epsilon_{iJt} \}} \max_j \{ \delta_{jt} + \mu_{ijt} + \epsilon_{ijt}\} = \ln \left ( \sum_{j } \exp \{ \delta_{jt} + \mu_{ijt} \} \right )
		\end{align*}
		\item This formula is also known as the `log-sum' formula.
		\item If utility is linear in price, inclusive value can be converted into dollars by dividing by the price coefficient.
		\item You have (probably) seen this value before, since it comes up when computing a nested logit
	\end{wideitemize}
\end{frame}

\begin{frame}{Consumer welfare}
	\begin{wideitemize}
		\item Typically, two cases where we compute welfare.
		\item \underline{Case 1:} observe quantities and prices and want to summarize them into a welfare measure.
		\begin{wideitemize}
			\item Key issue: before we normalized the utility of the outside option to zero.
			\item This is fine for estimating choice probabilities (why?)
			\item But, issues occur if want to compute inclusive value over time (or across markets)
			\item This is because we would be \textit{implicitly assuming that utility from the outside good is constant over time}.
		\end{wideitemize}
	\end{wideitemize}
\end{frame}

\begin{frame}{Consumer welfare}
	\begin{wideitemize}
		\item Typically, two cases where we compute welfare.
		\item \underline{Case 1:} observe quantities and prices and want to summarize them into a welfare measure.
		\begin{wideitemize}
			\item \underline{Example:} What if we see the share of the `inside' products increasing over time. 
			\item \underline{Question:} What could cause this? \pause
			\item Answer: price of inside goods decreased (or quality $\xi_{jt}$ increased) OR the outside option got worse. Different welfare implications, but assuming outside good is 0 rules out latter.
			\item Partial solution: Nevo (2003) compute welfare over time for when  $\xi_{jt}$ changes / outside option constant vs $\xi_{jt}$  fixed /outside option flexible. Report both extreme cases.
		\end{wideitemize}
	\end{wideitemize}
\end{frame}

\begin{frame}{Consumer welfare}
	\begin{wideitemize}
		\item \underline{Case 2:} Use the model to compute a welfare gain from a counterfactual outcome.
		\item Assume we observe one market over time (denoted by t). 
		\item Can show that the change in welfare from introducing a product in period $t$ (which comes from the change in the inclusive value) is:
		\begin{align*}
			\textbf{Logit:     }&\ln \left( \frac{1}{s_{0t}} \right) - \ln \left( \frac{1}{s_{0{t-1}}} \right) \\
			\textbf{Mixed logit:     }& \int \ln \left( \frac{1}{s_{i0t}} \right)dF(D_{it}, \nu_{it}) - \int \ln \left( \frac{1}{s_{0{t-1}}} \right)dF(D_{it-1}, \nu_{it-1})  \\
		\end{align*}
		\item So, logit model: welfare directly related to share of the outside good.
		\item Mixed logit: same idea, but difference depends on heterogeneity of choosing the outside option.
	\end{wideitemize}
\end{frame}

\begin{frame}{Consumer welfare}
	\begin{wideitemize}
		\item \underline{Case 2:} Use the model to compute a welfare gain from a counterfactual outcome.
		\item Problem with computing welfare from new goods: \textbf{red-bus blue-bus problem}.
		\item \underline{Thought experiment}:
		\item Market where consumers choose how to commute. 
		\begin{wideitemize}
			\item Choices: Car, Red Bus
			\item Assume half consumers choose Car, half choose Red Bus
		\end{wideitemize}
		\item Assume we artificially introduce a new option: \textbf{the Blue Bus}
		\begin{wideitemize}
			\item Artificial because we also \textbf{assume consumers are color-blind}
			\item (Also assume price, frequency of service etc are not impacted)
			\item Now, half consumers choose a car, and the rest are split between the two buses.
			\item Clearly, \textbf{consumer welfare has not changed}.
		\end{wideitemize}
	\end{wideitemize}
\end{frame}

\begin{frame}{Consumer welfare}
	\begin{wideitemize}
		\item \underline{Case 2:} Use the model to compute a welfare gain from a counterfactual outcome.
		\item \textbf{What if we now use our logit model to estimate the welfare effects of introducing a Blue Bus? }
		\begin{wideitemize}
			\item (Suppose we only observe data pre-introduction of the Blue Bus.) \pause
			\item \underline{Pre-introduction:}
			\item Assume car is outside good, normalize to zero: $\delta_{car}=0$. Then, also, $\delta_{\text{red-bus}}=0$ since $s_{car}=s_{\text{red-bus}}=0$.
			\item Inclusive value is: $\ln(e^{0} + e^{0}) = \ln(2)$.
			\item  \underline{Post-introduction:}
			\item $\delta_{\text{blue-bus}}=\delta_{\text{red-bus}}=0$ (since same bus)
			\item So, $s_{car}=s_{\text{blue-bus}}=s_{\text{red-bus}}=1/3$. Inclusive value is: $\ln(3)$, a \textbf{welfare increase}!
		\end{wideitemize}
	\end{wideitemize}
\end{frame}

\begin{frame}{Consumer welfare}
	\begin{wideitemize}
		\item \underline{Case 2:} Use the model to compute a welfare gain from a counterfactual outcome.
		\item Where did we go wrong with the previous analysis? \pause
		\item \underline{Main issue:} we are getting an extra logit draw when we introduce the Blue Bus.
		\item \underline{Possible solution:} if we observe the market post-introduction of the new product (i.e. the Blue Bus).
		\item  \underline{Post-introduction:}
		\item $s_{car}=0.5$, $s_{\text{blue-bus}}=s_{\text{red-bus}}=0.25$
		\item Implies: $\delta_{\text{blue-bus}}=\delta_{\text{red-bus}}=\ln(0.5)$
		\item Inclusive value: $\ln(e^0+2*e^{ln(0.5)}) = \ln(2).$ Note that this is the correct answer.
	\end{wideitemize}
\end{frame}

\begin{frame}{Consumer welfare}
	\begin{wideitemize}
		\item \underline{Main takeaways from the above `red-bus blue-bus' exercise:}
		\item 1. Introducing new products also introduces with extra logit draws, which can bias welfare computations
		\item 2. Observing post-introduction market shares can `correct' for this bias. 
		\item Note that the above is true as well in Mixed Logit and other models:
		\begin{wideitemize}
			\item Berry and Pakes (2007): ``the fact that the contraction fits the shares exactly means that the extra gain from the logit errors is offset by lower $\delta$’s, and this roughly counteracts the problems generated for welfare measurement by the model with tastes for products.”
		\end{wideitemize}
	\end{wideitemize}
\end{frame}

\begin{frame}{Plan}
	\begin{wideenumerate}
		\item Supply-side moments
		\item Welfare from new products: theory
		\item \textbf{Welfare from new products: application}
		\item Apple-cinnamon cheerios war
		\item Main takeaways of demand estimation
	\end{wideenumerate}
\end{frame}

\begin{frame}{Application of welfare from new products: the minivan (Petrin, 2002)}
	\begin{columns}
		\begin{column}{0.5\textwidth}
		\begin{wideitemize}
			\item \textbf{Question:} what are the welfare benefits of the minivan?
			\item \underline{Minivan}:
			\item Box-like vehicle first introduced in 1984
			\item Increased height, decreased floor, making entry easier and allowing more movement in the vehicle
			\item Extremely popular with families in the 80s, 90s, etc
		\end{wideitemize}
		\end{column}
		\begin{column}{0.5\textwidth}
			\includegraphics[scale=0.2]{minivan.jpeg}
		\end{column}
	\end{columns}
\end{frame}

\begin{frame}{Application of welfare from new products: the minivan (Petrin, 2002)}
		\begin{wideitemize}
		\item \underline{Motivation}: new products, and product improvements, are a key source of economic growth. But how should we actually value these innovations?
		\item  \underline{Main takeaways} (beyond valuing the minivan):
		\item  1. Logit model performs extremely poorly compared to mixed logit
		\item 2. Use of micro-moments
		\item 3. Dealing with the `red-bus blue-bus' problem
\end{wideitemize}
\end{frame}

\begin{frame}{Application of welfare from new products: the minivan (Petrin, 2002)}
	\begin{wideitemize}
		\item (Conditional, indirect) utility:
		\begin{align*}
			u_{ijt} = x_{jt} \beta_{it} + \alpha_{it} \ln (y_i-p_{jt}) + \xi_{jt} + \epsilon_{ijt}
		\end{align*}
		\item Price coefficient allowed to vary by income
		\item Other notes: `minivan' enters \textit{as a characteristic}, also includes a supply side
		\item \underline{Data}
		\item Observes US market 1981-1993 (for 916 vehicles)
		\item Prices, quantities (from Automotive Data Book)
		\item Product characteristics: fuel efficiency, vehicle dimensions, etc
		\item Important: also has CEX auto supplement, links demographics $\leftrightarrow$ new vehicle purchases
	\end{wideitemize}
\end{frame}

\begin{frame}{Application of welfare from new products: the minivan (Petrin, 2002)}
	\begin{wideitemize}
		\item Main difference to BLP: \underline{includes micro-moments} using his data from CEX
		\begin{wideitemize}
			\item E[i buys new vehicle $\vert$ low income]
			\item E[i buys new vehicle $\vert$ mid income]
			\item E[i buys new vehicle $\vert$ high income]
			\item E[family size of i $\vert$ i purchase minivan]
			\item E[family size of i $\vert$ i purchase station wagon]
			\item etc...
		\end{wideitemize}
		\item Include these as additional moments in the GMM part of the BLP estimator
		\item Benefit: more info on heterogeneity $\rightarrow$ can better capture substitution patterns
		\item Advice: add micro-moments where possible
		\item (Note: also uses product characteristics as instruments)
	\end{wideitemize}
\end{frame}

\begin{frame}{Application of welfare from new products: the minivan (Petrin, 2002)}
\begin{figure}
	\includegraphics[scale=0.25]{table_4_1.jpeg}
\end{figure}
\end{frame}

\begin{frame}{Application of welfare from new products: the minivan (Petrin, 2002)}
	\begin{figure}
		\includegraphics[scale=0.2]{table_4_2.jpeg}
	\end{figure}
\end{frame}

\begin{frame}{Application of welfare from new products: the minivan (Petrin, 2002)}
	\begin{wideitemize}
		\item Counterfactual exercise: eliminate minivan
		\item Compute change in consumer welfare (`compensating variation')
		\item (Note: prices allowed to readjust too using the supply-side)
	\end{wideitemize}
\end{frame}

\begin{frame}{Application of welfare from new products: the minivan (Petrin, 2002)}
	\begin{figure}
		\includegraphics[scale=0.4]{table_8_1.jpeg}
	\end{figure}
\end{frame}

\begin{frame}{Application of welfare from new products: the minivan (Petrin, 2002)}
	\begin{wideitemize}
		\item CV in logit model is biased upwards (due to large logit draws - i.e. consumers in the model with extreme tastes for minivans)
		\item Petrin claims: adding the micro-data reduces reliance of the model on the logit draws, and so reduces the overall welfare gain.
		\item Additionally: directly observes the counterfactual new product choice (so market shares with the new product are correct - we mentioned earlier this was useful in reducing the red-bus blue-bus problem).
	\end{wideitemize}
\end{frame}

\begin{frame}{Plan}
	\begin{wideenumerate}
		\item Supply-side moments
		\item Welfare from new products: theory
		\item Welfare from new products: application
		\item  \textbf{Apple-cinnamon cheerios war}
		\item Main takeaways of demand estimation
	\end{wideenumerate}
\end{frame}

\begin{frame}{Apple-cinnamon cheerios war}
	\begin{columns}
		\begin{column}{0.65\textwidth}
			\begin{wideitemize}
				\item Hausman paper in ``The Economics of New Goods'' (1997)
				\item Values welfare contribution of Apple Cinnamon Cheerios at 60 million dollars per year (in mid 1990s)
				\item Bresnahan (1997) disagrees with this computation. Criticises identifying assumptions, assumptions about competition, etc...  
				\item Not time to go into it in detail, but on reading list and highly recommended 
				\item Quote from Bresnahan (1997): \textit{`I have never met an economist who strayed that
				far from reality by ideology -- only by arrogance.'}
			\end{wideitemize}
		\end{column}
		\begin{column}{0.35\textwidth}
			\includegraphics[scale=0.2]{cheerios.png}
		\end{column}
	\end{columns}
\end{frame}

\begin{frame}{Plan}
	\begin{wideenumerate}
		\item Supply-side moments
		\item Welfare from new products: theory
		\item Welfare from new products: application
		\item Apple-cinnamon cheerios war
		\item  \textbf{Main takeaways of demand estimation}
	\end{wideenumerate}
\end{frame}

\begin{frame}{Main takeaways of demand estimation part of this course}
	\begin{wideitemize}
		\item We learned the Mixed Logit (`BLP') model, a model of \textbf{demand for differentiated products}.
		\item \textbf{Main components}: product characteristics, observed demographics, unobserved tastes for characteristics, unobserved demand shocks
		\item Captures \textbf{more consumer heterogeneity} than standard logit. An example where this really matters is getting substitution patterns right.
		\item We saw how to estimate it using the BLP method, common data that are useful or required, and common computational problems
		\item We studied common instruments (and discussed identification - e.g. recall \textbf{dual role} of instruments)
		\item We saw some applications: measuring conduct, valuing new goods (there are *many* more - often used whenever you need demand in a model)
	\end{wideitemize}
\end{frame}

\begin{frame}{Main takeaways of demand estimation part of this course}
	\begin{wideitemize}
		\item \underline{Computational note}: I will get you to code up a simple version of BLP in the homework. 
		\item However, if you actually use it, I highly recommend the PyBLP package (Conlon and Gortmaker). This will save potentially years of computation/coding time!
	\end{wideitemize}
\end{frame}

\end{document}

