\documentclass[notes,11pt, aspectratio=169]{beamer}

\usepackage{pgfpages}
% These slides also contain speaker notes. You can print just the slides,
% just the notes, or both, depending on the setting below. Comment out the want
% you want.
\setbeameroption{hide notes} % Only slide
%\setbeameroption{show only notes} % Only notes
%\setbeameroption{show notes on second screen=right} % Both

%\usepackage[scaled=1.0]{helvet}
\usepackage{array}

\usepackage{tikz}
\usepackage{verbatim}
\setbeamertemplate{note page}{\pagecolor{gray!5}\insertnote}
\usetikzlibrary{positioning}
\usetikzlibrary{snakes}
\usetikzlibrary{calc}
\usetikzlibrary{arrows}
\usetikzlibrary{decorations.markings}
\usetikzlibrary{shapes.misc}
\usetikzlibrary{matrix,shapes,arrows,fit,tikzmark}
\usepackage{amsmath}
\usepackage{mathpazo}
\usepackage{hyperref}
\usepackage{lipsum}
\usepackage{multimedia}
\usepackage{graphicx}
\usepackage{multirow}
\usepackage{graphicx}
\usepackage{dcolumn}
\usepackage{bbm}
\newcolumntype{d}[0]{D{.}{.}{5}}

\usepackage{changepage}
\usepackage{appendixnumberbeamer}
\newcommand{\beginbackup}{
   \newcounter{framenumbervorappendix}
   \setcounter{framenumbervorappendix}{\value{framenumber}}
   \setbeamertemplate{footline}
   {
     \leavevmode%
     \hline
     box{%
       \begin{beamercolorbox}[wd=\paperwidth,ht=2.25ex,dp=1ex,right]{footlinecolor}%
%         \insertframenumber  \hspace*{2ex} 
       \end{beamercolorbox}}%
     \vskip0pt%
   }
 }
\newcommand{\backupend}{
   \addtocounter{framenumbervorappendix}{-\value{framenumber}}
   \addtocounter{framenumber}{\value{framenumbervorappendix}} 
}


\usepackage{graphicx}
\usepackage[space]{grffile}
\usepackage{booktabs}

% These are my colors -- there are many like them, but these ones are mine.
\definecolor{blue}{RGB}{0,114,178}
\definecolor{red}{RGB}{213,94,0}
\definecolor{yellow}{RGB}{240,228,66}
\definecolor{green}{RGB}{0,158,115}

\hypersetup{
  colorlinks=false,
  linkbordercolor = {white},
  linkcolor = {blue}
}


%% I use a beige off white for my background
\definecolor{MyBackground}{RGB}{255,253,218}

%% Uncomment this if you want to change the background color to something else
%\setbeamercolor{background canvas}{bg=MyBackground}

%% Change the bg color to adjust your transition slide background color!
\newenvironment{transitionframe}{
  \setbeamercolor{background canvas}{bg=white}
  \begin{frame}}{
    \end{frame}
}

\setbeamercolor{frametitle}{fg=blue}
\setbeamercolor{title}{fg=black}
\setbeamertemplate{footline}[frame number]
\setbeamertemplate{navigation symbols}{} 
\setbeamertemplate{itemize items}{-}
\setbeamercolor{itemize item}{fg=blue}
\setbeamercolor{itemize subitem}{fg=blue}
\setbeamercolor{enumerate item}{fg=blue}
\setbeamercolor{enumerate subitem}{fg=blue}
\setbeamercolor{button}{bg=MyBackground,fg=blue,}



% If you like road maps, rather than having clutter at the top, have a roadmap show up at the end of each section 
% (and after your introduction)
% Uncomment this is if you want the roadmap!
% \AtBeginSection[]
% {
%    \begin{frame}
%        \frametitle{Roadmap of Talk}
%        \tableofcontents[currentsection]
%    \end{frame}
% }
\setbeamercolor{section in toc}{fg=blue}
\setbeamercolor{subsection in toc}{fg=red}
\setbeamersize{text margin left=1em,text margin right=1em} 

\newenvironment{wideitemize}{\itemize\addtolength{\itemsep}{10pt}}{\enditemize}
\newenvironment{wideenumerate}{\enumerate\addtolength{\itemsep}{10pt}}{\endenumerate}

\usepackage{environ}
\NewEnviron{videoframe}[1]{
  \begin{frame}
    \vspace{-8pt}
    \begin{columns}[onlytextwidth, T] % align columns
      \begin{column}{.58\textwidth}
        \begin{minipage}[t][\textheight][t]
          {\dimexpr\textwidth}
          \vspace{8pt}
          \hspace{4pt} {\Large \sc \textcolor{blue}{#1}}
          \vspace{8pt}
          
          \BODY
        \end{minipage}
      \end{column}%
      \hfill%
      \begin{column}{.42\textwidth}
        \colorbox{green!20}{\begin{minipage}[t][1.2\textheight][t]
            {\dimexpr\textwidth}
            Face goes here
          \end{minipage}}
      \end{column}%
    \end{columns}
  \end{frame}
}

\title[]{\textcolor{blue}{Demand Estimation 4 \\ PhD Industrial Organization}}
\author[PGP]{}
\institute[FRBNY]{\small{\begin{tabular}{c c c}
Nicholas Vreugdenhil \\
\end{tabular}}}
\date{} 

\begin{document}

%%% TIKZ STUFF
\tikzset{   
        every picture/.style={remember picture,baseline},
        every node/.style={anchor=base,align=center,outer sep=1.5pt},
        every path/.style={thick},
        }
\newcommand\marktopleft[1]{%
    \tikz[overlay,remember picture] 
        \node (marker-#1-a) at (-.3em,.3em) {};%
}
\newcommand\markbottomright[2]{%
    \tikz[overlay,remember picture] 
        \node (marker-#1-b) at (0em,0em) {};%
}
\tikzstyle{every picture}+=[remember picture] 
\tikzstyle{mybox} =[draw=black, very thick, rectangle, inner sep=10pt, inner ysep=20pt]
\tikzstyle{fancytitle} =[draw=black,fill=red, text=white]
%%%% END TIKZ STUFF

% Title Slide
\begin{frame}
\maketitle
  \centering
\end{frame}

% INTRO
\begin{frame}{Plan}
	\begin{wideenumerate}
		%\item Review of the BLP setup
		%\item Price elasticity/substitution patterns
		%\item Estimation: overview and typical data
		\item Estimation algorithm
		\item Instrumental variables
		\item Extensions
		\item Applications
	\end{wideenumerate}
\end{frame}

\begin{frame}{Plan}
	\begin{wideenumerate}
		%\item Review of the BLP setup
		%\item Price elasticity/substitution patterns
		%\item Estimation: overview and typical data
		\item \textbf{Estimation algorithm}
		\item Instrumental variables
		\item Extensions
		\item Applications
	\end{wideenumerate}
\end{frame}

\begin{frame}{Review: setup of the problem}
	\begin{wideitemize}
		\item (Conditional, indirect) utility:
		\begin{align*}
			u_{ijt} = x_{jt} \beta_{it} + \alpha_{it} p_{jt} + \xi_{jt} + \epsilon_{ijt}
		\end{align*}
		\begin{wideitemize}
			\item (Note: first element of $x_{jt}$ is 1 $\rightarrow$ absorbs mean of $ \xi_{jt} $)
		\end{wideitemize}
		\item What are the parameters we need to estimate?
		\item \underline{Linear parameters:}
		\begin{wideitemize}
			\item Parameters from the mean utility equation: $(\alpha_0, \beta_0)$
		\end{wideitemize}
		\item \underline{Nonlinear parameters}
		\begin{wideitemize}
			\item $\Gamma$: coefficients on (observed) demographics 
			\item $\Sigma$:  idiosyncratic ``taste for characteristics''
		\end{wideitemize}
		\item So, full parameter vector to estimate: $\theta = (\alpha_0, \beta_0, \Gamma, \Sigma)$.
	\end{wideitemize}
\end{frame}

\begin{frame}{Review: estimation algorithm: overview}
	\begin{wideitemize}
		\item \underline{Step 1}: For a guess of $\Gamma$ and $\Sigma$, and a vector of mean utilities $\boldsymbol{\delta}_t$, compute model-predicted market shares.
		\item \underline{Step 2}: For a guess of $\Gamma$ and $\Sigma$ do an \textbf{inversion}: find $\boldsymbol{\delta}_t$ where the model-predicted market shares match the empirical market shares $s_t$. 
		\begin{wideitemize}
			\item This step will repeatedly call the function from Step 1.
		\end{wideitemize}
		\item  \underline{Step 3}: Use the computed $\boldsymbol{\delta}_t$ from Step 2 to compute $\xi_{jt}= \delta_{jt}(\Gamma, \Sigma) - x_{jt} \beta_0 - \alpha_0 p_{jt}$. 
		\begin{wideitemize}
			\item Interact with IVs to get the GMM objective function. 
			\item Search over all parameters $\theta$ to minimize objective function using non-linear optimization.
		\end{wideitemize}
	\end{wideitemize}
\end{frame}

\begin{frame}{Estimation algorithm: step 3}
	\begin{wideitemize}
		\item  Denote the mean utilities from step 2: $\delta_{jt}(\Gamma, \Sigma)$
		\item  Compute $\xi_{jt}(\theta) = \delta_{jt}(\Gamma, \Sigma) - x_{jt} \beta_0 - \alpha_0 p_{jt}$
		\begin{wideitemize}
			\item Above equation is why we called $\Gamma, \Sigma$ `nonlinear' variables, and $\beta_0$, $\alpha_0$ the `linear variables' 
		\end{wideitemize}
		\item  Interact with the instrumental variables to get the GMM objective function (denoting W as the GMM weight matrix):
		\begin{align*}
			\xi(\theta)' ZWZ' \xi (\theta)
		\end{align*}
		\item Solve for the parameters using nonlinear optimization.
		\begin{align*}
			\hat{\theta} = arg \min_{\theta} \xi(\theta)' ZWZ' \xi (\theta)
		\end{align*}
		\item \underline{Note}: since this is just a GMM problem, can also get standard errors using standard GMM methods
	\end{wideitemize}
\end{frame}

\begin{frame}{Numerical issues (documented by Knittel and Metaxoglou (2014))}
	\begin{wideitemize}
		\item ( See Conlon and Gortmaker (2020) for latest updates on best practices. )
		\item \underline{1. Objective function is highly nonlinear with many local minima}
		\begin{wideitemize}
			\item Numerical results can be sensitive to starting values or choice of optimizer method
			\item (Partial) solution: test results with different starting values and optimizer methods
			\item (Partial) solution: choose an optimizer that is used for commercial purposes (e.g. Knitro)
			\item (Partial) solution: Conlon and Gortmaker (2020) make some suggestions of free optimizer methods that work well in Scipy (a Python library)
			\item Solution (probably not yet computationally feasible): use a global optimizer like `differential evolution'
		\end{wideitemize}
		\item \underline{2. Need to choose very tight convergence tolerances for the inversion ($< 10^{-12}$)}
		\item \textbf{These are common issues in structural models, so be on the lookout in other contexts.}
	\end{wideitemize}
\end{frame}


\begin{frame}{Alternative algorithm: MPEC (`Mathematical Programming With Equilibrium Constraints')}
	\begin{align}
		\min_{\theta, \xi} &\hspace{22pt}  \xi ' ZWZ' \xi \nonumber  \\
		\text{subject to} &\hspace{22pt} \tilde{\sigma} (\delta(\xi); x, p, \hat{F}, \theta) = s \nonumber
	\end{align}
	\begin{wideitemize}
		\item Notice minimization is over both $\theta$ and $\xi$ here. 
		\item \underline{Advantage of this approach:} \pause No need for inversion step
		\item \underline{Disadvantage of this approach:} \pause Many more parameters to solve for ($\xi$)
		\item Dube et al (2012): claim this approach results in a speedup.
		\begin{wideitemize}
			\item However, can be complicated to program, and some have found it slow for large problems
		\end{wideitemize}
	\end{wideitemize}
\end{frame}

\begin{frame}{Plan}
	\begin{wideenumerate}
		%\item Review of the BLP setup
		%\item Price elasticity/substitution patterns
		%\item Estimation: overview and typical data
		\item Estimation algorithm
		\item \textbf{Instrumental variables}
		\item Extensions
		\item Applications
	\end{wideenumerate}
\end{frame}


\begin{frame}{Instruments}
	\begin{wideitemize}
		\item We used the following moment conditions restriction:
		\begin{align*}
			E(\xi_{jt} | \textbf{Z}_{t}) = 0
		\end{align*}
		\item Here,  $\textbf{Z}_{t}$ is a vector of instruments
		\item Note that above assumption implies a large set of potential IVs $z_{jt} = A_j(\textbf{Z}_t)$ for which the unconditional moment restriction holds:
		\begin{align*}
			E(z_{jt} \xi_{jt}) = 0
		\end{align*}
		\item What instruments should we use for $\textbf{Z}_{t}$?
	\begin{wideitemize}
		\item Recall the dual role for instruments in the model.
	\end{wideitemize}
	\end{wideitemize}
\end{frame}

\begin{frame}{Instruments: BLP instruments}
	\begin{wideitemize}
		\item Use \textbf{characteristics of products in the market}.
		\begin{align*}
			E(\xi_{jt} | \textbf{x}_{t}) = 0
		\end{align*}
		\begin{wideitemize}
			\item $\textbf{x}_{t}$ is the vector of product characteristics in market $t$
			\item (Note: don't include price in characteristics)
			\item `Observed characteristics mean independent of unobserved characteristics'
		\end{wideitemize}
		\item Can form many moment conditions from above assumption. BLP use:
		\begin{wideitemize}
			\item characteristics of own product
			\item sum of characteristics of other products produced by the firm
			\item sum of characteristics of competitiors
		\end{wideitemize}
	\end{wideitemize}
\end{frame}

\begin{frame}{Instruments: BLP instruments}
	\begin{wideitemize}
		\item Power: closeness in characteristic space affects markups which affects price.
		\item Justification for this instrument: \pause \textbf{product characteristics are set before $\xi_{jt}$ known}.
		\begin{wideitemize}
			\item Do you think this is a reasonable assumption in the car market?
		\end{wideitemize}
		\pause
		\item What if firms are forward looking and anticipate  $\xi_{jt}$ when choosing product characteristics?
		\begin{wideitemize}
			\item Possible solution: use panel data.
			\item E.g. Sweeting (2013) assumes $\xi_{jt} = \rho \xi_{jt-1} + u_{jt}$, where $u_{jt}$ unanticipated at time $t-1$
			\item Implies moments: $E(\xi_{jt} - \rho \xi_{jt-1} |x_{t-1}) = 0$.
			\item Comment: many connections here to the dynamic panel / production function literatures.
		\end{wideitemize}
	\end{wideitemize}
\end{frame}

\begin{frame}{Instruments: cost-based / Hausman instruments}
	\begin{wideitemize}
		\item Ideal instrument: cost-shifters
		\item But, cost data are rarely observed in practice.
		\item Hausman (1996) and Nevo (2001) use indirect cost measures: \textbf{prices in other markets}
		\begin{wideitemize}
			\item i.e. $p_{jt'}$ for $t' \neq t$
			\item Validity condition: conditional on $x_t$ and $x_t'$, pricing is independent across markets and $\xi_{jt}$ and $\xi_{jt'}$ are independent.
			\item In words: ``IVs exploit common cost shocks across markets''
		\end{wideitemize}
		\item Problems (example):
		\begin{wideitemize}
			\item Unobserved advertising campaigns
		\end{wideitemize}
	\end{wideitemize}
\end{frame}

\begin{frame}{Instruments: Waldfogel-Fan instruments}
	\begin{wideitemize}
		\item Used in Waldfogel (2003), Fan (2013)
		\item Use \textbf{demographics in other counties where the product is sold}
			\begin{wideitemize}
				\item Fan (2013): newspapers sold in multiple counties, uses demographics in other counties as IVs.
				\item Idea: rely on consumption/preference externalities
				\item E.g. Product offered in multiple counties $\rightarrow$ characteristics of product impacted by the attributes (like demographics) of the other counties.
				\item Validity: conditional on variables in model, $\xi_{jt}$ not correlated across counties (same assumption in Hausman instruments)
				\item Additional concern: set of counties where product is offered is not exogenous
					\end{wideitemize}
	\end{wideitemize}
\end{frame}

\begin{frame}{Plan}
	\begin{wideenumerate}
		%\item Review of the BLP setup
		%\item Price elasticity/substitution patterns
		%\item Estimation: overview and typical data
		\item Estimation algorithm
		\item Instrumental variables
		\item \textbf{Extensions}
		\item Applications
	\end{wideenumerate}
\end{frame}

\begin{frame}{Extensions/other useful data sources}
	\begin{wideitemize}
		\item \underline{Second choice data} 
		\item (e.g. from a survey)
		\item e.g. see Berry, Levinsohn and Pakes: ``Differentiated products demand systems from a
		combination of micro and macro data: the new vehicle market'' (2004)
		\item \underline{Micro-moments} 
	\end{wideitemize}
\end{frame}

\begin{comment}
\begin{frame}{Extensions/other useful data sources}
	\begin{wideitemize}
		\item  \underline{Supply side moments}:
		\item Assume that marginal cost is given by:
			\begin{align*}
				mc_{jt} = w_{jt} \boldsymbol{\gamma} + \omega_{jt}
			\end{align*}
		\item Where:
	\begin{wideitemize}
		\item $w_{jt}$: vector of observed characteristics of product $j$
		\item $\boldsymbol{\gamma}$: vector of parameters
		\item $\omega_{jt}$: unobserved component
	\end{wideitemize}
	\end{wideitemize}
\end{frame}

\begin{frame}{Extensions/other useful data sources}
	\begin{wideitemize}
		\item  \underline{Supply side moments}:
		\item Assuming Nash-Bertrand pricing
	\end{wideitemize}
\end{frame}
\end{comment}

\begin{frame}{Plan}
	\begin{wideenumerate}
		%\item Review of the BLP setup
		%\item Price elasticity/substitution patterns
		%\item Estimation: overview and typical data
		\item Estimation algorithm
		\item Instrumental variables
		\item Extensions
		\item \textbf{Applications}
	\end{wideenumerate}
\end{frame}

\begin{frame}{Application: distinguishing between models of competition}
	\begin{columns}
		\begin{column}{0.65\columnwidth}
				\begin{wideitemize}
				\item Nevo ``Measuring Market Power in the Ready-to-eat
Cereal Industry" (Econometrica, 2001)
				\item RTE  Cereal
				\begin{wideitemize}
					\item Concentrated market
					\item High margins
					\item High advertising-to-sales ratios
					\item Aggressive introduction of new products
				\end{wideitemize}
				\item ``Used as a classic example of a concentrated differentiated-products industry in which price competition is approximately cooperative and rivalry is channeled into advertising and new product innovation''
			\end{wideitemize}
		\end{column}
		\begin{column}{0.35\columnwidth}
			\hspace{-50pt} \includegraphics[scale=0.2]{cereal.png}
		\end{column}
	\end{columns}
\end{frame}

\begin{frame}{Application: distinguishing between models of competition}
	\begin{wideitemize}
		\item \textbf{Research questions}
		\item 1. Is there collusive pricing?
		\item 2. Decompose price-cost margins (PCM) into:
		\begin{wideitemize}
			\item i. Product differentiation
			\item ii. Portfolio effects (firms offer multiple products)
			\item iii. Price collusion
		\end{wideitemize}
			\item \textbf{Main takeaways for this class:}
		\begin{wideitemize}
			\item See demand estimation `in action'
			\item Inclusion of a supply side that allows for horizontal competition
			\item (Application/question is of general interest - but very ``traditional IO'')
		\end{wideitemize}
	\end{wideitemize}
\end{frame}

\begin{frame}{Application: distinguishing between models of competition}
	\begin{wideitemize}
		\item \textbf{Overall strategy:}
		\item 1. Estimate demand
		\item 2. Use estimates + pricing rules implied by different models of firm conduct to get price-cost margins (PCM)
		\begin{wideitemize}
			\item Challenge: costs not observed
		\end{wideitemize}
		\item 3. Compare predicted PCM from different models of conduct to true PCM
				\begin{wideitemize}
			\item See which model of firm conduct best matches the data.
		\end{wideitemize}
		\item \textbf{Main finding:}
		\item Nash-Bertrand pricing best matches observed PCM
	\end{wideitemize}
\end{frame}

\begin{comment}
\begin{frame}{Application: distinguishing between models of competition}
	\begin{wideitemize}
		\item \textbf{Demand side:}
		\item Utility model is exactly what we have seen:
		\begin{align*}
			u_{ijt} = x_{jt} \beta_i + \alpha_i p_{jt} + \xi_{jt} + \epsilon_{ijt}
		\end{align*}
		\item \textbf{Data:}
		\item Use top 25 cereal brands
		\item Scanner data $\rightarrow$ get market shares, prices, etc. 
		\begin{wideitemize}
			\item Aggregate to MSA-quarter level = 1124 markets.
		\end{wideitemize}
		\item Advertising data
		\item Cereal box characteristics (nutritional information); subjective characteristics (`mushy')
		\item Demographics from CPS
	\end{wideitemize}
\end{frame}

\begin{frame}{Application: distinguishing between models of competition}
	\begin{wideitemize}
		\item \textbf{Supply side:}
		\item Profits of firm $f$:
		\begin{align*}
			\pi_f = \sum_{j \in \mathcal{J}_f} \left[ (p_j - mc_j) q_j(\textbf{p}) - FC_j \right]
		\end{align*}
		\item Where:
		\begin{wideitemize}
			\item $\pi_f$: profits of firm $f$
			\item $p_j$: price of product $j$; $\textbf{p}$: vector of prices
			\item $mc_j$: marginal cost
			\item $FC_j$: fixed cost
			\item $q_j$: quantity of product $j$ (depends on \underline{all} prices)
			\item $\mathcal{J}_f$: set of products that firm $f$ maximizes profit over
		\end{wideitemize}
	\end{wideitemize}
\end{frame}

\begin{frame}{Application: distinguishing between models of competition}
	\begin{wideitemize}
		\item \textbf{Supply side:}
		\item Define \textbf{conduct structure} as JxJ matrix:
		\[ H_{jk} = \begin{cases} 
			1 & \text{ if } \exists $f$ \text{ where } \{j,k\} \subset  \mathcal{J}_f \\
			0 & otherwise
		\end{cases}
		\]
		\item Elements of H either 0 or 1
		\item If value of element = 1: then product $j$ and $k$ are priced \textit{as if jointly owned}
		\item \underline{Examples}:
		\begin{wideitemize}
			\item Single product firm pricing: identity matrix
			\item Joint profit maximization: matrix of 1's
		\end{wideitemize}
	\end{wideitemize}
\end{frame}

\begin{frame}{Application: distinguishing between models of competition}
	\begin{wideitemize}
		\item \textbf{Supply side:}
		\item Define $\Omega_{jk} = -\partial q_k / \partial p_j \cdot H_{jk}$ 
		\begin{wideitemize}
			\item Recall: j is index, k is column
		\end{wideitemize}
		\item First order condition of firms' profit maximization problem (bold denotes vectors):
		\begin{align*}
			\textbf{q} (\textbf{p}) - \Omega (\textbf{p} - \textbf{mc}) = 0
		\end{align*}
		\item Implies pricing equation:
		\begin{align*}
			\textbf{p} - \textbf{mc} = \Omega^{-1} \textbf{q}(\textbf{p})
		\end{align*}
		\item \underline{Important:} above equation implies that given conduct structure + estimates of demand substitution $\Omega$ $\rightarrow$ can measure price-cost margins \textit{without observing cost data}.
	\end{wideitemize}
\end{frame}


\begin{frame}{Application: distinguishing between models of competition: estimation}
	\begin{wideitemize}
		\item Only estimate demand
		\item Use BLP algorithm
		\item Instruments: Hausman instruments  (prices in other markets)
	\end{wideitemize}
\end{frame}

\begin{frame}
	\begin{figure}
		\includegraphics[scale=0.16]{nevo_1.jpeg}
	\end{figure}
\end{frame}

\begin{frame}
	\begin{figure}
		\includegraphics[scale=0.16]{nevo_4.jpeg}
	\end{figure}
\end{frame}

\begin{frame}
	\begin{figure}
		\includegraphics[scale=0.3]{nevo_3.jpeg}
	\end{figure}
\end{frame}

\begin{frame}{Application: distinguishing between models of competition: mergers}
	\begin{wideitemize}
		\item Common use of the framework in this paper can also be used to determine the effects of a merger. How would you do this? \pause
		\item \underline{Answer:}
		\item 1. Using pre-merger data estimate demand and recover marginal costs by inverting the pricing equation:
		\begin{align*}
			\textbf{mc} = 	\textbf{p} -  \Omega^{-1} \textbf{q}(\textbf{p})
		\end{align*}
		\item 2. Change the conduct structure $H$ so that the merging firms jointly maximize the profits of their products
	\end{wideitemize}
\end{frame}
\end{comment}

\end{document}

