\documentclass[notes,11pt, aspectratio=169]{beamer}

\usepackage{pgfpages}
% These slides also contain speaker notes. You can print just the slides,
% just the notes, or both, depending on the setting below. Comment out the want
% you want.
\setbeameroption{hide notes} % Only slide
%\setbeameroption{show only notes} % Only notes
%\setbeameroption{show notes on second screen=right} % Both

%\usepackage[scaled=1.0]{helvet}
\usepackage{array}

\usepackage{tikz}
\usepackage{verbatim}
\setbeamertemplate{note page}{\pagecolor{gray!5}\insertnote}
\usetikzlibrary{positioning}
\usetikzlibrary{snakes}
\usetikzlibrary{calc}
\usetikzlibrary{arrows}
\usetikzlibrary{decorations.markings}
\usetikzlibrary{shapes.misc}
\usetikzlibrary{matrix,shapes,arrows,fit,tikzmark}
\usepackage{amsmath}
\usepackage{mathpazo}
\usepackage{hyperref}
\usepackage{lipsum}
\usepackage{multimedia}
\usepackage{graphicx}
\usepackage{multirow}
\usepackage{graphicx}
\usepackage{dcolumn}
\usepackage{bbm}
\newcolumntype{d}[0]{D{.}{.}{5}}

\usepackage{changepage}
\usepackage{appendixnumberbeamer}
\newcommand{\beginbackup}{
   \newcounter{framenumbervorappendix}
   \setcounter{framenumbervorappendix}{\value{framenumber}}
   \setbeamertemplate{footline}
   {
     \leavevmode%
     \hline
     box{%
       \begin{beamercolorbox}[wd=\paperwidth,ht=2.25ex,dp=1ex,right]{footlinecolor}%
%         \insertframenumber  \hspace*{2ex} 
       \end{beamercolorbox}}%
     \vskip0pt%
   }
 }
\newcommand{\backupend}{
   \addtocounter{framenumbervorappendix}{-\value{framenumber}}
   \addtocounter{framenumber}{\value{framenumbervorappendix}} 
}


\usepackage{graphicx}
\usepackage[space]{grffile}
\usepackage{booktabs}

% These are my colors -- there are many like them, but these ones are mine.
\definecolor{blue}{RGB}{0,114,178}
\definecolor{red}{RGB}{213,94,0}
\definecolor{yellow}{RGB}{240,228,66}
\definecolor{green}{RGB}{0,158,115}

\hypersetup{
  colorlinks=false,
  linkbordercolor = {white},
  linkcolor = {blue}
}


%% I use a beige off white for my background
\definecolor{MyBackground}{RGB}{255,253,218}

%% Uncomment this if you want to change the background color to something else
%\setbeamercolor{background canvas}{bg=MyBackground}

%% Change the bg color to adjust your transition slide background color!
\newenvironment{transitionframe}{
  \setbeamercolor{background canvas}{bg=white}
  \begin{frame}}{
    \end{frame}
}

\setbeamercolor{frametitle}{fg=blue}
\setbeamercolor{title}{fg=black}
\setbeamertemplate{footline}[frame number]
\setbeamertemplate{navigation symbols}{} 
\setbeamertemplate{itemize items}{-}
\setbeamercolor{itemize item}{fg=blue}
\setbeamercolor{itemize subitem}{fg=blue}
\setbeamercolor{enumerate item}{fg=blue}
\setbeamercolor{enumerate subitem}{fg=blue}
\setbeamercolor{button}{bg=MyBackground,fg=blue,}



% If you like road maps, rather than having clutter at the top, have a roadmap show up at the end of each section 
% (and after your introduction)
% Uncomment this is if you want the roadmap!
% \AtBeginSection[]
% {
%    \begin{frame}
%        \frametitle{Roadmap of Talk}
%        \tableofcontents[currentsection]
%    \end{frame}
% }
\setbeamercolor{section in toc}{fg=blue}
\setbeamercolor{subsection in toc}{fg=red}
\setbeamersize{text margin left=1em,text margin right=1em} 

\newenvironment{wideitemize}{\itemize\addtolength{\itemsep}{10pt}}{\enditemize}
\newenvironment{wideenumerate}{\enumerate\addtolength{\itemsep}{10pt}}{\endenumerate}

\usepackage{environ}
\NewEnviron{videoframe}[1]{
  \begin{frame}
    \vspace{-8pt}
    \begin{columns}[onlytextwidth, T] % align columns
      \begin{column}{.58\textwidth}
        \begin{minipage}[t][\textheight][t]
          {\dimexpr\textwidth}
          \vspace{8pt}
          \hspace{4pt} {\Large \sc \textcolor{blue}{#1}}
          \vspace{8pt}
          
          \BODY
        \end{minipage}
      \end{column}%
      \hfill%
      \begin{column}{.42\textwidth}
        \colorbox{green!20}{\begin{minipage}[t][1.2\textheight][t]
            {\dimexpr\textwidth}
            Face goes here
          \end{minipage}}
      \end{column}%
    \end{columns}
  \end{frame}
}

\title[]{\textcolor{blue}{Entry and Exit 1 \\ PhD Industrial Organization}}
\author[PGP]{}
\institute[FRBNY]{\small{\begin{tabular}{c c c}
Nicholas Vreugdenhil \\
\end{tabular}}}
\date{} 

\begin{document}

%%% TIKZ STUFF
\tikzset{   
        every picture/.style={remember picture,baseline},
        every node/.style={anchor=base,align=center,outer sep=1.5pt},
        every path/.style={thick},
        }
\newcommand\marktopleft[1]{%
    \tikz[overlay,remember picture] 
        \node (marker-#1-a) at (-.3em,.3em) {};%
}
\newcommand\markbottomright[2]{%
    \tikz[overlay,remember picture] 
        \node (marker-#1-b) at (0em,0em) {};%
}
\tikzstyle{every picture}+=[remember picture] 
\tikzstyle{mybox} =[draw=black, very thick, rectangle, inner sep=10pt, inner ysep=20pt]
\tikzstyle{fancytitle} =[draw=black,fill=red, text=white]
%%%% END TIKZ STUFF

% Title Slide
\begin{frame}
\maketitle
  \centering
\end{frame}

\begin{frame}{Motivation}
	\begin{wideitemize}
		\item So far, we have studied models where the \textbf{market structure is fixed}.
		\begin{wideitemize}
			\item Market structure: number of firms, types of products, firm size etc
			\item Example of what we have implicitly assumed so far: demand estimation took characteristics of products as fixed, supply side: profit maximization given product characteristics 
		\end{wideitemize}
	\end{wideitemize}
\end{frame}

\begin{frame}{Motivation}
	\begin{wideitemize}
		\item Now, we will take a step back and start to think about \textbf{endogenous market structure}.
		\begin{wideitemize}
			\item Fundamental IO question: why exactly does market structure vary across industries?
			\item Clear implications for policy:
			\item \underline{Example 1}: Effects of a merger
			\item \underline{Example 2}: Increasing costs due to environmental regulation
			\item \underline{Example 3}: Bailout of the truck manufacturing industry
		\end{wideitemize}
	\end{wideitemize}
\end{frame}


\begin{frame}{Motivation}
	\begin{wideitemize}
		\item We will start by looking at some simple models of firm entry and exit.
		\item Theoretical model as a starting point:
		\item \textbf{Stage 1}: Potential entrants decide whether to enter
		\item \textbf{Stage 2}: Firms compete given entry
		\item Later on in the course we will study dynamic games which contain strategic interactions between firms, investment, entry, and exit, over time.
	\end{wideitemize}
\end{frame}

\begin{frame}{Motivation}
	\begin{wideitemize}
		\item \textbf{Today:} we will look at some earlier work attempting to estimate entry models.
		\item As we will see, although these paper are creative, influential, and important, they often require some strong assumptions in order to take the model to data.
		\item Next time we will see a new strand of literature that resolves a key assumption of these papers.
		\begin{wideitemize}
			\item Key assumption: uniqueness of equilibrium
		\end{wideitemize}
	\end{wideitemize}
\end{frame}

\begin{frame}{Plan}
	\begin{wideenumerate}
		\item Bresnahan and Reiss (1991)
		\item Mazzeo (2002)
		\item Seim (2006)
	\end{wideenumerate}
\end{frame}

\begin{frame}{Plan}
	\begin{wideenumerate}
		\item \textbf{Bresnahan and Reiss (1991)}
		\item Mazzeo (2002)
		\item Seim (2006)
	\end{wideenumerate}
\end{frame}

\begin{frame}{Bresnahan and Reiss (1991): Research Questions}
	\begin{wideitemize}
		\item 1. How do profits change with the number of firms?
		\item 2. How many firms need to enter before an oligopoly market becomes competitive?
	\end{wideitemize}
\end{frame}

\begin{frame}{Bresnahan and Reiss (1991): Data}
	\begin{wideitemize}
		\item Isolated markets $M$ (202 markets)
		\begin{wideitemize}
			\item Isolated since it is easier to define a `market' and `entry into a market'. I.e. these markets are independent in terms of demand, competition, etc.
			\item Pre-internet, they got this data through the phonebook and also by driving (or maybe getting their RA to drive) around small towns in Western USA (!)
		\end{wideitemize}
		\item Look at retail/professional services e.g. doctors, dentists, plumbers
		\item For each market $m$: observe active firms $n_m$, market size $s_m$, and some exogenous market characteristics that may affect demand/costs $x_m$
	\end{wideitemize}
\end{frame}


\begin{frame}{Bresnahan and Reiss (1991): Model}
	\begin{wideitemize}
		\item Homogeneous firms. N potential entrants.
		\item Equilibrium number of firms $n_m$ satisfies:
		\begin{align*}
			\Pi_m(n_m) \geq 0 \\
			\Pi_m(n_m + 1) < 0 
		\end{align*}
		\item That is, each firm is playing a best respose to the other firms: all active firms stay in the market, all inactive firms stay out.
	\end{wideitemize}
\end{frame}

\begin{frame}{Bresnahan and Reiss (1991): Model}
	\begin{wideitemize}
		\item Parameterize profit as follows:
		\begin{align*}
			\Pi_m(n_m) = V_m(n) - F_m (n)
		\end{align*}
		\item Where $V_m(n)$ is variable profit:
		\begin{align*}
			V_m(n) = s_m v_m (n) = s_m (x_m^D \beta - \alpha (n))
		\end{align*}
		\item $s_m$: market size
		\item $v_m(n)$: variable profit per-capita
		\item $x_m^D$: vector of market characteristics that may affect demand (e.g. per-capita income)
		\item $\beta$: parameter vector
		\item $\alpha(n)$: parameter that captures degree of competition (strictly increasing in n).
	\end{wideitemize}
\end{frame}

\begin{frame}{Bresnahan and Reiss (1991): Model}
	\begin{wideitemize}
		\item Here, $F_m(n)$ is fixed cost:
		\begin{align*}
			F_m (n) = x_m^C \gamma + \delta(n) + \epsilon_m
		\end{align*}
		\item $x_m^C$: vector of observable market characteristics that could affect fixed costs (e.g. rental price)
		\item $\epsilon_m$: unobservable (to the econometrician) market characteristic
		\item $\delta(1), ..., \delta(N)$: parameters (note: a bit odd that fixed cost depends on n) 
	\end{wideitemize}
\end{frame}

\begin{frame}{Bresnahan and Reiss (1991): Model}
	\begin{wideitemize}
		\item Since $\alpha(n)$ and $\delta(n)$ increase with $n$, can show that profit decreases with $n$.
		\item Can also show that the model has a unique equilibrium $n_m$ (given exogenous variables) due to strictly decreasing $\Pi_m(n)$.
	\end{wideitemize}
\end{frame}

\begin{frame}{Bresnahan and Reiss (1991): Estimation}
	\begin{wideitemize}
		\item Assume that unobserved component of entry costs $\epsilon_m$ is independent of market shares and market characteristics, and is distributed $N(0,\sigma)$. 
		\begin{wideitemize}
			\item Also, normalize scale $\sigma=1$.
		\end{wideitemize}
		\item Rearrange equilibrium number of firms condition (i.e $\Pi_m(n_m) \geq 0$ and $\Pi_m(n_m + 1) < 0$) in terms of thresholds for $\epsilon_m$:
		\begin{align*}
			T_m (n+1) < \epsilon_m \leq T_m (n)
		\end{align*}
			\item Where:
		\begin{align*}
			T_m(n) = s_m x_m^D \beta - x_m^C \gamma - \alpha(n) s_m - \delta(n)
		\end{align*}
	\end{wideitemize}
\end{frame}

\begin{frame}{Bresnahan and Reiss (1991): Estimation}
	\begin{wideitemize}
		\item Estimate model using an \underline{ordered probit}:
		\begin{align*}
			Pr(n_m=n \vert s_m, x_m) = \Phi(T_m(n)) -  \Phi(T_m(n+1))
		\end{align*}
		\item Can estimate with maximum likelihood.
	\end{wideitemize}
\end{frame}

\begin{frame}{Bresnahan and Reiss (1991): Results}
	\begin{wideitemize}
			\item Get market size entry thresholds.
			\begin{align*}
				S(n)=\frac{x_m^C \gamma + \delta(n)}{x_m^D \beta - \alpha(n)}
			\end{align*}
			\item Note that these don't depend in the normalization $\sigma=1$.
			\item These are the minimum market size to sustain $n$ firms in the market.
			\item Compute entry threshold ratios e.g. $S_2/S_1 =\frac{S(2)}{2}/\frac{S(1)}{1}$
			\begin{wideitemize}
				\item E.g. as number of firms increases by 1, does market double (ratio $= 1$) or need to more than double (ratio $> 1$)?
			\end{wideitemize}
	\end{wideitemize}
\end{frame}

\begin{frame}{Bresnahan and Reiss (1991)}
	\begin{figure}
		\includegraphics[scale=0.6]{entry_ratios.jpeg}
	\end{figure}
\end{frame}

\begin{frame}{Bresnahan and Reiss (1991): Main Findings}
	\begin{wideitemize}
		\item Monopoly to duopoly (for most of the industries studied) requires more than double the market size
		\begin{wideitemize}
			\item Their data do not allow them to say why. 
			\item One explanation: barriers to entry change with $N$ (cost story). 
			\item Another explanation: markups change with $N$ (e.g. consistent with a Cournot model).
		\end{wideitemize}
		\item When number of firms $>4$: double market size implies double the number of firms. (Consistent with `contestable market hypothesis': if barriers to entry are low then market behaves in a competitive way.)
	\end{wideitemize}
\end{frame}

\begin{frame}{Plan}
	\begin{wideenumerate}
		\item Bresnahan and Reiss (1991)
		\item \textbf{Mazzeo (2002)}
		\item Seim (2006)
	\end{wideenumerate}
\end{frame}

\begin{frame}{Mazzeo (2002)}
	\begin{columns}
		\begin{column}{0.6\textwidth}
			\begin{wideitemize}
				\item \textbf{Question:}
				\item What drives the product-type decisions of firms in oligopoly markets?
				\item \textbf{Approach}
				\item Similar assumptions in the model to Bresnahan and Reiss (1991)
				\begin{wideitemize}
					\item Complete information
					\item No dynamics, no spatial differentiation
				\end{wideitemize}
				\item Endogenizes \textbf{firm product choice}
				\item Data from the motel industry (use local markets along US highway exits)
			\end{wideitemize}
	\end{column}
	\begin{column}{0.4\textwidth}
		\includegraphics[scale=0.25]{exit.jpeg}
	\end{column}
	\end{columns}
\end{frame}

\begin{frame}{Mazzeo (2002): Model}
\begin{wideitemize}
	\item Different types of hotels (H: high-quality, E: economy hotel)
	\item Hotels choose their type and also whether to enter
	\item Profit of an active hotel of type $T \in \{E, H \}$ is:
	\begin{align*}
		\pi_T(n_E, n_H) = s V_T (x, n_E, n_H) - EC_T (x) - \epsilon_T
	\end{align*}
	\item Here, $n_E$ and $n_H$ represent the number of active hotels of low and high quality in the market.
	\item $V_T$: variable profit (per-capita)
	\item $EC_T (x) + \epsilon_T$: entry cost for type $T$ hotels (where $ \epsilon_T$ is unobservable to the researcher).
\end{wideitemize}
\end{frame}

\begin{frame}{Mazzeo (2002): Model}
	\begin{wideitemize}
		\item Paper uses alternative two solution concepts: 
		\item 1. Stackelberg 
		\begin{wideitemize}
			\item Specifically, employs the \textbf{equilibrium selection rule} that firms enter sequentially with high-quality firms moving first
		\end{wideitemize}
		\item 2. A `two-stage game': firms choose whether to enter and their type
		\begin{wideitemize}
			\item We will now talk more about this alternative
		\end{wideitemize}
	\end{wideitemize}
\end{frame}

\begin{frame}{Mazzeo (2002): Model}
	\begin{wideitemize}
		\item In the \textbf{first-stage} the total number of active hotels $n=n_E+n_H$ is determined similarly to the Bresnahan-Reiss model.
		\item That is, hotels continue to enter the market so long as there is some configuration $(n_E, n_H)$ where both low-quality and high-quality hotels make positive profits.
		\begin{align*}
			\Pi(n) = \max_{n_E, n_H: n_E + n_H=n} \min [\pi_E(n_E, n_H), \pi_H (n_E, n_H)]
		\end{align*}
		\item Then, the equilibrium number of hotels in the first-stage is $n^*$ where $\Pi(n^*) \geq 0$ and $\Pi(n^*+1) < 0$)
		\item If $\pi_E$ and $\pi_H$ are strictly decreasing in the number of active firms then $\Pi(n)$ is also strictly decreasing $\rightarrow$ $n^*$ is \textbf{unique}.
	\end{wideitemize}
\end{frame}

\begin{frame}{Mazzeo (2002): Model}
	\begin{wideitemize}
		\item In the \textbf{second-stage} active hotels simultaneously choose their type or quality level.
		\item Here, equilibrium is a pair $(n^*_E, n^*_H)$ such that every firm chooses the type that maximizes its profit given the choices of the other firms.
		\begin{wideitemize}
			\item So, low-quality firms are not better off switching to high-quality etc...
		\end{wideitemize}
		\begin{align*}
			\pi_E(n_E^*, n_H^*) &\geq \pi_H (n^*_E-1, n^*_H+1) \\
			\pi_H(n_E^*, n_H^*) &\geq \pi_E (n^*_E+1, n^*_E-1) 
		\end{align*}
		\item Mazzeo shows that the equilibrium pair given in the above equations is also unique.
	\end{wideitemize}
\end{frame}

\begin{frame}{Mazzeo (2002): Estimation}
	\begin{wideitemize}
		\item Using the equilibrium conditions, possible to obtain a closed-form expression for the region of unobservables $(\epsilon_E, \epsilon_H)$ that generate a particular value of $(n^*_E, n^*_H)$. 
		\item Let $R_E (n_E, n_H;s,x)$ be the region associated with $n_E,n_H$ and $F$ be the CDF of the unobservable variables. Then:
		\begin{align*}
			Pr(n_E^*=n_E, n^*_H=n_H \vert s,x) = \int 1\{(\epsilon_E, \epsilon_H) \in R_E (n_E, n_H;s,x)\} dF(\epsilon_E, \epsilon_H)
		\end{align*}
		\item Can the process similarly to Bresnahan and Reiss (1991): parameterize the payoff function and estimate using observed number of firms in each market using maximum likelihood.
	\end{wideitemize}
\end{frame}

\begin{frame}{Mazzeo (2002): Results (not time to go into these in detail)}
	\begin{wideitemize}
		\item Overall, finds evidence that firms have strong incentives to offer different products to their competitors. 
		\item Specifically,``the negative effect that a competitor has on firm payoffs is up to twice as large
		if that competitor is the same product type''.
	\end{wideitemize}
\end{frame}

\begin{frame}{Plan}
	\begin{wideenumerate}
		\item Bresnahan and Reiss (1991)
		\item Mazzeo (2002)
		\item \textbf{Seim (2006)}
	\end{wideenumerate}
\end{frame}

\begin{frame}{Seim 2006}
	\begin{wideitemize}
		\item  \textbf{Question:} How important is spatial differentiation in explaining market power?
		\item Example: think of two grocery stores located in a city.
		\begin{wideitemize}
			\item Differences in demand between locations
			\item Positioning compared to competitors
			\item Other factors like rent
		\end{wideitemize}
		\item Importantly: relaxes the `isolated market' assumption. 
		\item Application: video-rental industry
	\end{wideitemize}
\end{frame}

\begin{frame}{Seim 2006}
	\begin{figure}
		\includegraphics[scale=0.2]{city.jpeg}
		\caption{New firm locates in position 7.}
	\end{figure}
\end{frame}

\begin{frame}{Seim 2006: \underline{Very} Brief Overview}
\begin{wideitemize}
	\item $N$ potential entrants choose whether to enter a market. If enter they choose their location.
	\item Challenges:
	\begin{wideitemize}
		\item Computation complexity (many choices and configurations of firms)
		\item Multiple equilibria
	\end{wideitemize}
	\item Key assumption: rivals have (some) private information about profitability. Hence, when each firm enters, they do not know for sure where their rivals will enter.
	\item Then, the choice to enter is made on the expected value of profits taken over the probability that other firms will enter other locations.
	\item She shows in the paper this makes it easier to compute an equilibrium. Uniqueness is tricky (she has some simulations for simple cases). 

\end{wideitemize}
\end{frame}

\end{document}

