\documentclass[notes,11pt, aspectratio=169]{beamer}

\usepackage{pgfpages}
% These slides also contain speaker notes. You can print just the slides,
% just the notes, or both, depending on the setting below. Comment out the want
% you want.
\setbeameroption{hide notes} % Only slide
%\setbeameroption{show only notes} % Only notes
%\setbeameroption{show notes on second screen=right} % Both

%\usepackage[scaled=1.0]{helvet}
\usepackage{array}

\usepackage{tikz}
\usepackage{verbatim}
\setbeamertemplate{note page}{\pagecolor{gray!5}\insertnote}
\usetikzlibrary{positioning}
\usetikzlibrary{snakes}
\usetikzlibrary{calc}
\usetikzlibrary{arrows}
\usetikzlibrary{decorations.markings}
\usetikzlibrary{shapes.misc}
\usetikzlibrary{matrix,shapes,arrows,fit,tikzmark}
\usepackage{amsmath}
\usepackage{mathpazo}
\usepackage{hyperref}
\usepackage{lipsum}
\usepackage{multimedia}
\usepackage{graphicx}
\usepackage{multirow}
\usepackage{graphicx}
\usepackage{dcolumn}
\usepackage{bbm}
\newcolumntype{d}[0]{D{.}{.}{5}}

\usepackage{changepage}
\usepackage{appendixnumberbeamer}
\newcommand{\beginbackup}{
   \newcounter{framenumbervorappendix}
   \setcounter{framenumbervorappendix}{\value{framenumber}}
   \setbeamertemplate{footline}
   {
     \leavevmode%
     \hline
     box{%
       \begin{beamercolorbox}[wd=\paperwidth,ht=2.25ex,dp=1ex,right]{footlinecolor}%
%         \insertframenumber  \hspace*{2ex} 
       \end{beamercolorbox}}%
     \vskip0pt%
   }
 }
\newcommand{\backupend}{
   \addtocounter{framenumbervorappendix}{-\value{framenumber}}
   \addtocounter{framenumber}{\value{framenumbervorappendix}} 
}


\usepackage{graphicx}
\usepackage[space]{grffile}
\usepackage{booktabs}

% These are my colors -- there are many like them, but these ones are mine.
\definecolor{blue}{RGB}{0,114,178}
\definecolor{red}{RGB}{213,94,0}
\definecolor{yellow}{RGB}{240,228,66}
\definecolor{green}{RGB}{0,158,115}

\hypersetup{
  colorlinks=false,
  linkbordercolor = {white},
  linkcolor = {blue}
}


%% I use a beige off white for my background
\definecolor{MyBackground}{RGB}{255,253,218}

%% Uncomment this if you want to change the background color to something else
%\setbeamercolor{background canvas}{bg=MyBackground}

%% Change the bg color to adjust your transition slide background color!
\newenvironment{transitionframe}{
  \setbeamercolor{background canvas}{bg=white}
  \begin{frame}}{
    \end{frame}
}

\setbeamercolor{frametitle}{fg=blue}
\setbeamercolor{title}{fg=black}
\setbeamertemplate{footline}[frame number]
\setbeamertemplate{navigation symbols}{} 
\setbeamertemplate{itemize items}{-}
\setbeamercolor{itemize item}{fg=blue}
\setbeamercolor{itemize subitem}{fg=blue}
\setbeamercolor{enumerate item}{fg=blue}
\setbeamercolor{enumerate subitem}{fg=blue}
\setbeamercolor{button}{bg=MyBackground,fg=blue,}



% If you like road maps, rather than having clutter at the top, have a roadmap show up at the end of each section 
% (and after your introduction)
% Uncomment this is if you want the roadmap!
% \AtBeginSection[]
% {
%    \begin{frame}
%        \frametitle{Roadmap of Talk}
%        \tableofcontents[currentsection]
%    \end{frame}
% }
\setbeamercolor{section in toc}{fg=blue}
\setbeamercolor{subsection in toc}{fg=red}
\setbeamersize{text margin left=1em,text margin right=1em} 

\newenvironment{wideitemize}{\itemize\addtolength{\itemsep}{10pt}}{\enditemize}
\newenvironment{wideenumerate}{\enumerate\addtolength{\itemsep}{10pt}}{\endenumerate}

\usepackage{environ}
\NewEnviron{videoframe}[1]{
  \begin{frame}
    \vspace{-8pt}
    \begin{columns}[onlytextwidth, T] % align columns
      \begin{column}{.58\textwidth}
        \begin{minipage}[t][\textheight][t]
          {\dimexpr\textwidth}
          \vspace{8pt}
          \hspace{4pt} {\Large \sc \textcolor{blue}{#1}}
          \vspace{8pt}
          
          \BODY
        \end{minipage}
      \end{column}%
      \hfill%
      \begin{column}{.42\textwidth}
        \colorbox{green!20}{\begin{minipage}[t][1.2\textheight][t]
            {\dimexpr\textwidth}
            Face goes here
          \end{minipage}}
      \end{column}%
    \end{columns}
  \end{frame}
}

\title[]{\textcolor{blue}{Entry and Exit 2 \\ PhD Industrial Organization}}
\author[PGP]{}
\institute[FRBNY]{\small{\begin{tabular}{c c c}
Nicholas Vreugdenhil \\
\end{tabular}}}
\date{} 

\begin{document}

%%% TIKZ STUFF
\tikzset{   
        every picture/.style={remember picture,baseline},
        every node/.style={anchor=base,align=center,outer sep=1.5pt},
        every path/.style={thick},
        }
\newcommand\marktopleft[1]{%
    \tikz[overlay,remember picture] 
        \node (marker-#1-a) at (-.3em,.3em) {};%
}
\newcommand\markbottomright[2]{%
    \tikz[overlay,remember picture] 
        \node (marker-#1-b) at (0em,0em) {};%
}
\tikzstyle{every picture}+=[remember picture] 
\tikzstyle{mybox} =[draw=black, very thick, rectangle, inner sep=10pt, inner ysep=20pt]
\tikzstyle{fancytitle} =[draw=black,fill=red, text=white]
%%%% END TIKZ STUFF

% Title Slide
\begin{frame}
\maketitle
  \centering
\end{frame}

\begin{frame}{Motivation}
	\begin{wideitemize}
		\item Before, we studied three papers about entry.
		\item These papers were extremely creative, but all had to make strong assumptions to allow for point identification.
		\item \textbf{Central challenge:} each paper seeks to estimate parameters of the underlying variable profits, fixed costs etc only using only the observed equilibrium number of firms in each market.
		\begin{wideitemize}
			\item Point identification requires 1:1 mapping between the equilibrium number of firms and each parameter. 
			\item Critical to this is that the model has a \textbf{unique} equilibrium.
		\end{wideitemize}
		\item Today we will relax the assumption of a unique equilibrium and allow for potentially multiple equilibria
	\end{wideitemize}
\end{frame}

\begin{frame}{Plan}
	\begin{wideenumerate}
		\item Partial identification: warm-up
		\item Ciliberto and Tamer (2009)
		\item Wollmann (2018)
	\end{wideenumerate}
\end{frame}

\begin{frame}{Plan}
	\begin{wideenumerate}
		\item \textbf{Partial identification: warm-up}
		\item Ciliberto and Tamer (2009)
		\item Wollmann (2018)
	\end{wideenumerate}
\end{frame}

\begin{frame}{Partial identification: warm-up}
		\begin{wideitemize}
			\item \textbf{Example}: missing data (Tamer, 2010)
			\item  \textbf{Setup}: 
			\item Let $Y$ be a binary 0/1 random variable that is observed only when another binary 0/1 random variable $Z$ is equal to 1.
			\item So, observe: $(Y \vert Z=1)$
			\item \textbf{Question:} What is $P(Y=1)$?
			\item \textbf{Issue:} data alone contain no information about $Y\vert Z=0$. 
		\end{wideitemize}
\end{frame}

\begin{frame}{Partial identification: warm-up}
	\begin{wideitemize}
		\item What if we made \underline{no assumptions} about $Y\vert Z=0$?
		\item Then, there is a \textit{set} of parameters that are consistent with the data.
		\item In this setting, the set is:
		\begin{align*}
			\Phi_I = \{ p \in [0, 1] : p = P(Y=1 \vert Z=1) P(Z=1) + q P(Z=0), \text{ for some } q \in [0,1] \}
		\end{align*}
		\item Another way of writing this is in terms of the following bounds:
		\begin{align*}
			\Phi_I =[\underbrace{ P(Y=1 \vert Z=1) P(Z=1)}_{\text{What if $P(Y=1 \vert Z=0)=0$?}}, \underbrace{ P(Y=1 \vert Z=1) P(Z=1) + P(Z=0)}_{\text{What if $P(Y=1 \vert Z=0) = 1$ ?}}]
		\end{align*} \pause
		\item \textbf{Application:} biased coin flip (tails = 0, heads = 1). 
		\item Know $P(Y=1 \vert Z=1)=0.7$ and $P(Z=1) = 0.9$. What is $P(Y=1)$?
		% \item Answer (without further assumptions): $\Phi_I=[0.6, 0.8]$.
	\end{wideitemize}
\end{frame}

\begin{frame}{Partial identification: what did we learn from the previous exercise?}
	\begin{wideitemize}
		\item \textbf{Idea:} without any assumptions, we might still be able to say something useful about the parameters from the data.
		\item We say the model is \textbf{partially identified} when there is more than one parameter value that is consistent with the data and model.
			\begin{wideitemize}
				\item Similar to point identification, partial identification is completely different to issues with statistical significance or not have a large enough sample size
				\item Often the thought experiment with identification is: if we had unlimited data then what can we say about the parameters?
			\end{wideitemize}
		\item Partially identified models might still be useful! E.g. the previous example could rule out out a value of $P(Y=1)=0.9$. 
		\item We usually say that the bounds are \textbf{informative} if they are tight enough for us to still say something (economically) interesting about the problem.
	\end{wideitemize}
\end{frame}

\begin{frame}{Partial identification: what does all this have to do with entry models?}
	\begin{wideitemize}
		\item We will see that allowing for multiple equilibria in entry models
		\begin{wideitemize}
			\item $\rightarrow$ a \underline{set} of parameters that are consistent with the data 
			\item $\rightarrow$ partial identification and bounds on the parameters
		\end{wideitemize}
	\end{wideitemize}
\end{frame}

\begin{frame}{Plan}
	\begin{wideenumerate}
		\item Partial identification: warm-up
		\item \textbf{Ciliberto and Tamer (2009)}
		\item Wollmann (2018)
	\end{wideenumerate}
\end{frame}

\begin{frame}{Ciliberto and Tamer (2009)}
	\begin{columns}
		\begin{column}{0.6\textwidth}
			\begin{wideitemize}
				\item \textbf{Question:} How can we estimate the payoff functions of plays in complete information, static, discrete choice games \textbf{without} making assumptions about equilibrium selection?
				\item \textbf{Method:} Use `moment inequalities' (I will explain this in more detail later)
				\item \textbf{Application:} Entry into airline markets 
				\item \textbf{Policy counterfactual:} Repealing the Wright Amendment (policy which restricted air service out of Dallas Love airport)
				\item For this paper we will mainly focus on the method rather than the application.
			\end{wideitemize}
		\end{column}
		\begin{column}{0.4\textwidth}
			\includegraphics[scale=0.13]{airline.jpeg}
		\end{column}
	\end{columns}
\end{frame}

\begin{frame}{Ciliberto and Tamer (2009): simple example}
		\begin{wideitemize}
			\item Consider a simple version of a Bresnahan and Reiss (1990) 2x2 entry game:
			\begin{align*}
				y_{1m} &= 1[\alpha_1' X_{1m} + \delta_2 y_{2m} + \epsilon_{1m} \geq 0] \\
				y_{2m} &= 1[\alpha_2' X_{2m} + \delta_1 y_{1m} + \epsilon_{2m} \geq 0] 
			\end{align*}
			\item Here:
			\item $(X_{1m}, X_{2m})$ is a vector of observed exogenous regressors that contain market $m$ specific variables
			\item $(y_{1m}, y_{2m})$: whether firm 1 and firm 2 enter
			\item Error terms $(\epsilon_{1m}$, $\epsilon_{2m})$ are observed by firms but not by econometrician
			\item Choices are interdependent
			\item \textbf{Note:} consider only pure strategy equilibria, also assume perfect information
		\end{wideitemize}
\end{frame}

\begin{frame}{Ciliberto and Tamer (2009): simple example}
	\begin{wideitemize}
		\item Consider a simple version of a Bresnahan and Reiss (1990) 2x2 entry game:
		\begin{align*}
			y_{1m} &= 1[\alpha_1' X_{1m} + \delta_2 y_{2m} + \epsilon_{1m} \geq 0] \\
			y_{2m} &= 1[\alpha_2' X_{2m} + \delta_1 y_{1m} + \epsilon_{2m} \geq 0] 
		\end{align*}
		\item We will now see that with large enough support for the $\epsilon$'s, we will now see that if $\delta_1 , \delta_2 <0$ (i.e. duopoly profits are smaller than monopoly profits) there are multiple equilibria. 
		\item (Note: there are multiple equilibria if $\delta_1 , \delta_2 >0$)
	\end{wideitemize}
\end{frame}

\begin{frame}{Ciliberto and Tamer (2009): simple example}
	\begin{figure}
		\includegraphics[scale=0.23]{tamer_1.jpeg}
	\end{figure}
\end{frame}

\begin{frame}{Ciliberto and Tamer (2009): simple example}
	\begin{wideitemize}
		\item Based on the previous figure we can construct the following choice probabilities:
		\begin{align*}
			Pr(1,1 \vert X) &= Pr(\epsilon_1 \geq - \alpha_1' X_1 -\delta_2; \epsilon_2 \geq -\alpha_2' X_2 - \delta_1 ) \\
			Pr(0,0 \vert X) &= Pr(\epsilon_1 \leq - \alpha_1' X_1 ; \epsilon_2 \leq -\alpha_2' X_2 ) \\
			Pr(1,0 \vert X) &= Pr((\epsilon_1, \epsilon_2) \in R_1(X, \theta)) + \int Pr((1,0) \vert \epsilon_1, \epsilon_2, X) 1[(\epsilon_1, \epsilon_2) \in R_2(\theta, X)] dF_{\epsilon_1, \epsilon_2}
		\end{align*}
		\item Here:
		\begin{wideitemize}
			\item $R_1 (\theta, X)$: (1,0) is the unique equilibrium of the game
			\item $R_2 (\theta, X)$: (1,0) \textit{potentially observable outcome of the game} + was ``selected''
		\end{wideitemize}
	\end{wideitemize}
\end{frame}

\begin{frame}{Ciliberto and Tamer (2009): simple example}
	\begin{wideitemize}
		\item The ``selection mechanism'' is the function $Pr((1,0) \vert \epsilon_1, \epsilon_2, X)$. 
		\item This is unknown to the econometrician and can differ across markets
		\item Noticing that the selection mechanism is a probability and so lies in $[0,1]$, an implication of the above model is:
		\begin{align*}
			Pr((\epsilon_1, \epsilon_2) \in R_1) \leq Pr((1,0)) \leq Pr((\epsilon_1, \epsilon_2) \in R_1) + Pr((\epsilon_1, \epsilon_2) \in R_2)
		\end{align*}
		\item (Notice the bounds and recall similarities to the `missing data' partial identification example before)
	\end{wideitemize}
\end{frame}

\begin{frame}{Ciliberto and Tamer (2009)}
	\begin{wideitemize}
		\item Profit for firm $i$ in market $m$:
		\begin{align*}
			\pi_{im} = S_m' \alpha_i + Z'_{im} \beta_i + W'_{im} \gamma_i + \sum_{j \neq i} \delta_j^i y_{jm} + \sum_{j \neq i} Z'_{jm} \phi_j^i y_{jm} + \epsilon_{im}
		\end{align*}
		\item Where:
		\begin{wideitemize}
			\item $S_m$: vector of market characteristics which are common among the firms in market $m$
			\item $Z_m$: matrix of firm characteristics which enter into all firms (e.g. some product attributes that consumers value)
			\item $K$: total number of potential entrants in market $m$
			\item $W_m=(W_{1m}, ..., W_{Km})$: vector of firm characteristics where $W_{im}$ enters only into firm $i$'s profit in market $m$ (e.g. cost variables)
			\item $y_{jm}$: indicator if firm $j$ enters into market $m$
			\item $\epsilon_{im}$: part of profits that is unobserved to the econometrician (but observable to the participants $\rightarrow$ this is a game of complete information).
		\end{wideitemize}
	\end{wideitemize}
\end{frame}

\begin{frame}{Ciliberto and Tamer (2009)}
	\begin{wideitemize}
		\item Nash equilibrium in each market:
		\begin{align*}
			y_{im} \pi_{im} = y_{im} (S_m' \alpha_i + Z'_{im} \beta_i + W'_{im} \gamma_i + \sum_{j \neq i} \delta_j^i y_{jm} + \sum_{j \neq i} Z'_{jm} \phi_j^i y_{jm} + \epsilon_{im}) \geq 0
		\end{align*}
		\item By similar arguments to the 2x2 example before, the predicted choice probabilities are:
		\begin{align*}
			Pr(y' \vert X) &= \int Pr(y' \vert \epsilon, X) dF \\ 
				&= \underbrace{ \int_{R_1(\theta,X)} dF }_{ \text{unique outcome region} } + \underbrace{ \int_{R_2 (\theta, X)} Pr(y' \vert \epsilon, X) dF }_{\text{multiple outcome region}}
		\end{align*}
		\item Note: $y'=(y_1', ..., y_K')$ is some outcome which is a sequence of 0's or 1's corresponding to different airlines serving the market.
	\end{wideitemize}
\end{frame}

\begin{frame}{Ciliberto and Tamer (2009)}
	\begin{wideitemize}
		\item Using that the selection function is bounded between 0 and 1:
		\begin{align*}
			\int_{R_1(\theta, X)} dF \leq Pr(y' \vert X) \leq \int_{R_1 (\theta, X)} dF + \int_{R_2 (\theta, X)} dF 
		\end{align*}
		\item The above conditions yields many \textbf{moment inequalities} (denote - for later - $H_1(\theta,X)$ as the LHS and $H_2(\theta,X)$ as the RHS)
		\item These define an identified set:
		\begin{align*}
			\Phi_I = \{ \theta: \text{ inequalities above are satisfied for all X} \}
		\end{align*}
	\end{wideitemize}
\end{frame}

\begin{frame}{Ciliberto and Tamer (2009)}
	\begin{wideitemize}
		\item To estimate, use the following objective function 
		\begin{align*}
			Q(\theta) = \int [ \vert \vert (P(y' \vert X) -H_1(\theta, X_m) )_- \vert \vert + \vert \vert (P(y' \vert X) -H_2(\theta, X_m) )_+ \vert \vert d F_X ]
		\end{align*}
		\item Work with the sample analogue:
		\begin{align*}
			Q_n(\theta) =\frac{1}{n} \sum_{m=1}^n \left[ \vert \vert (\hat{P}(y' \vert X_m) -H_1(\theta, X_m) )_- \vert \vert + \vert \vert (\hat{P}(y' \vert X_m) -H_2(\theta, X_m) )_+ \vert \vert  \right]
		\end{align*}
		\item Will need to estimate P (estimate denoted $\hat{P}$). Do this flexibly/non-parametrically. 
		%\item Also, will need to simulate $H_1$ and $H_2$ for each value of $\theta$.
		\item (Paper has a lot more detail about how they discretize the covariates etc)
	\end{wideitemize}
\end{frame}

\begin{frame}{Ciliberto and Tamer (2009)}
	\begin{wideitemize}
		\item The procedure on the previous slide allows for the estimation of the parameters. 
		\item However, inference is a lot trickier. There are quite a few alternative procedures to do inference in the literature (and many have been created since this paper was written).
		\item See IO Handbook Vol 4 Chapter ``Moment Inequalities and Partial Identification in Industrial Organization'' for an up-to-date overview of current best-practices 
	\end{wideitemize}
\end{frame}

\begin{frame}{Plan}
	\begin{wideenumerate}
		\item Partial identification: warm-up
		\item Ciliberto and Tamer (2009)
		\item \textbf{Wollmann (2018)}
	\end{wideenumerate}
\end{frame}

\begin{frame}{Wollmann (2018)}
		\begin{columns}
		\begin{column}{0.65\textwidth}
			\begin{wideitemize}
				\item \textbf{Question:} What would have happened if the auto industry had not been bailed out in 2009?
				\begin{wideitemize}
					\item (Mitt Romney: ``Let Detroit Go Bankrupt'')
				\end{wideitemize}
				\item  \textbf{Application:} Market for trucks in the US
				\begin{wideitemize}
					\item (Note: `chicken tax' implies all trucks domestically produced) 
				\end{wideitemize}
				\item \textbf{Method:} Allows for BLP-style demand while also endogenizing entry of different products with different characteristics
				\item \textbf{Main takeaway:} Product entry/exit moderates the markup increases and output decreases from a liquidation of GM and Chrysler by up to three-quarters.
			\end{wideitemize}
		\end{column}
		\begin{column}{0.35\textwidth}
			\includegraphics[scale=0.2]{truck_1.jpg}
		\end{column}
	\end{columns}
\end{frame}

\begin{frame}{Wollmann (2018): Main Sources of Data}
		\begin{wideitemize}
			\item 1. All commercial vehicles sold in USA 1986-2012 from `The Truck Blue Book'
			\begin{wideitemize}
				\item Each observation includes brand, model, year, suggested retail price, other characteristics (like the load capacity, cab, chassis, etc)
			\end{wideitemize}
			\item 2. Unit sales data from R.L. Polk Database
			\item 3. Microdata on commercial vehicle purchases available through the US Census 
			\begin{wideitemize}
				\item Up to 2002 the Census mailed 130000 owners of trucks and vans and asked about how they used their vehicle, and industry/state buyer operates in.
			\end{wideitemize}
		\end{wideitemize}
\end{frame}

\begin{frame}{Wollmann (2018): Model - Stage 2, Demand}
	\begin{wideitemize}
		\item Each buyer $r$ chooses whether to purchase vehicle $j$:
		\begin{align*}
			U_{r,j} = x_j (\beta_x + \beta_x^o z_r^o + \beta_x^u z_r^u) + p_j \beta_p + \xi_j + \epsilon_{r,j}
		\end{align*}
		\item Where:
		\begin{wideitemize}
			\item $x_j$: vehicle characteristics (e.g. Gross Weight Rating (GWR), if it has a `cabover')
			\item $z^o_r$: observed buyer attributes
			\item $z^u_r$: unobserved buyer attributes
			\item Other notation similar to what we have seen before.
		\end{wideitemize}
	\end{wideitemize}
\end{frame}

\begin{frame}{Wollmann (2018): Model - Stage 2, Prices}
	\begin{wideitemize}
		\item Firms $f$ offer a set of products $J_{f,t}$ and choose prices to maximize profits:
		\begin{align*}
			\Pi_{f,t} = \sum_{j \in J_{f,t}} [p_{j,t} - mc_{j,t}] s(x_{j,t}, x_{-j,t}, p_t, z_t ; \beta, \xi_t) M_t
		\end{align*}
		\item Where:
		\begin{wideitemize}
			\item $mc_{j,t}$: denotes the marginal costs of producing $j$ at $t$
			\item $x_{-j,t}$: matrix of characteristics for products other than $j$ at $t$
			\item $M_t$: market size (note: disappears after taking first-order condition)
		\end{wideitemize}
			\item Parametrize marginal cost: 
		\begin{wideitemize}
			\item $\log(mc_{j,t})$ is  linear in observable product characteristics and some other cost components, as well as an unobserved factor specific to the product and time $\omega_{j,t}$.
			\item $\rightarrow$ additional parameter $\gamma$ which is the vector of coefficients needs to be estimated. 
		\end{wideitemize}
	\end{wideitemize}
\end{frame}

\begin{frame}{Wollmann (2018): Model - Stage 1, Product Offerings}
		\begin{wideitemize}
			\item In the second stage, expected profits (expectation taken over the distribution of the disturbances $(F_{\xi}, F_{\omega})$) are:
			\begin{align*}
				\pi(J_{f,t}, J_{-f,t}, z_t, w_t) = \int_{\xi',\omega'} \Pi(J_{f,t}, J_{-f,t}, z_t, w_t ; \beta, \gamma, \xi', \omega') dF_{\xi'} dF_{\omega'}
			\end{align*}
			\item The sunk cost/scrap value of a product is:
			\begin{align*}
				SC_{f,j,t,J_{f,t-1}} = x_j' \tilde{\theta}_{f,x_j,t} \times \left[ \{j \in J_{f,t} , j \notin J_{f,t-1}\} + \lambda \{ j \notin J_{f,t}, j \in J_{f,t-1}\} \right]
			\end{align*}
			\item In above equation, $\{.\}$ is the indicator function, $\lambda$ indexes how much is returned in scrap value relative to the entry cost. The term $\tilde{\theta}_{f,x_j,t}$ is a sunk cost parameter (see paper for details on parametrization).
		\end{wideitemize}
\end{frame}

\begin{frame}{Wollmann (2018): Model - Stage 1, Product Offerings}
	\begin{wideitemize}
		\item Decisions to introduce products (or to not introduce)  produces the following moment inequalities (which I write using the sample analogues):
		\begin{align*}
			&\frac{1}{XTF} \sum_{x_j} \sum_t \sum_f h^i_{f,x_j,t} \{ j \in J_{f,t} \} \\ 
			&\times [\Delta \pi (J_{f,t} ,J_{f,t} \backslash j, J_{-f,t}, z_t, w_t) + x_j' \theta_f (\{j \notin J_{f,t-1}\} - \lambda \{j \in J_{f,t-1}\})] \geq 0 
		\end{align*}
		\item And:
		\begin{align*}
			&\frac{1}{XTF} \sum_{x_j} \sum_t \sum_f h^i_{f,x_j,t} \{ j \notin J_{f,t} \} \\ 
			&\times [\Delta \pi (J_{f,t} , J_{f,t} \cup j, J_{-f,t}, z_t, w_t) - x_j' \theta_f (\{j \notin J_{f,t-1}\} - \lambda \{j \in J_{f,t-1}\})] \geq 0 
		\end{align*}
		\item Here, $h^i_{f,x_j,t}$ are moment inequality weights (see paper for how these are constructed)
		\item X, T, F: the number of unique $x_j$ vectors, time periods, firms
		\item $\Delta \pi$: change in the profits
		\end{wideitemize}
\end{frame}

\begin{frame}{Wollmann (2018): Estimation}
	\begin{wideitemize}
		\item Stage 2: `standard' BLP estimation
		\item Stage 1: Use moment inequalities 
	\end{wideitemize}
\end{frame}

\begin{frame}{Wollmann (2018): Results - sunk costs}
	\begin{figure}
		\centering
		\includegraphics[scale=0.4]{sunk.jpeg}
		\caption{95\% confidence interval in brackets.}
	\end{figure}
	\begin{wideitemize}
		\item Above is just one part of the sunk costs (see paper for all the other parameters)
		\item Note that it is \underline{set-identified}
		\item Interpretation: ``firms recover about 55-60\% of the sunk cost when they retire vehicles''
	\end{wideitemize}
\end{frame}

\begin{frame}{Wollmann (2018): Counterfactuals}
	\begin{wideitemize}
		\item What would have happened in the absence of the \$85 billion rescue of GM and Chrysler?
		\item 1. Acquisition: GM and Chrysler products acquired by Ford
		\item 2. Acquisition: GM and Chrysler products acquired by PACCAR (this is a truck manufacturer)
		\item 3. Liquidation
	\end{wideitemize}
\end{frame}

\begin{frame}{Wollmann (2018): Counterfactuals}
	\begin{figure}
		\includegraphics[scale=0.17]{gwr}
	\end{figure}
\end{frame}

\begin{comment}
\begin{frame}{Wollmann (2018): Counterfactuals}
	\begin{figure}
		\includegraphics[scale=0.3]{trucks3.jpeg}
	\end{figure}
\end{frame}
\end{comment}

\end{document}

