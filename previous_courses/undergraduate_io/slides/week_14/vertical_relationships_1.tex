\documentclass[notes,11pt, aspectratio=169]{beamer}

\usepackage{pgfpages}
% These slides also contain speaker notes. You can print just the slides,
% just the notes, or both, depending on the setting below. Comment out the want
% you want.
\setbeameroption{hide notes} % Only slide
%\setbeameroption{show only notes} % Only notes
%\setbeameroption{show notes on second screen=right} % Both

%\usepackage[scaled=1.0]{helvet}
\usepackage{array}

\usepackage{graphicx}
\usepackage{tikz}
\usetikzlibrary{calc}
\usetikzlibrary{matrix}
\usetikzlibrary{positioning}

\newcommand{\payoff}[4][below]{\node[#1]at(#2){$(#3,#4)$};}
\usepackage{verbatim}
\setbeamertemplate{note page}{\pagecolor{gray!5}\insertnote}
\usetikzlibrary{positioning}
\usetikzlibrary{snakes}
\usetikzlibrary{calc}
\usetikzlibrary{arrows}
\usetikzlibrary{decorations.markings}
\usetikzlibrary{shapes.misc}
\usetikzlibrary{matrix,shapes,arrows,fit,tikzmark}
\usepackage{amsmath}
\usepackage{mathpazo}
\usepackage{hyperref}
\usepackage{lipsum}
\usepackage{multimedia}
\usepackage{graphicx}
\usepackage{multirow}
\usepackage{graphicx}
\usepackage{dcolumn}
\usepackage{bbm}
\newcolumntype{d}[0]{D{.}{.}{5}}

\usepackage{changepage}
\usepackage{appendixnumberbeamer}
\newcommand{\beginbackup}{
   \newcounter{framenumbervorappendix}
   \setcounter{framenumbervorappendix}{\value{framenumber}}
   \setbeamertemplate{footline}
   {
     \leavevmode%
     \hline
     box{%
       \begin{beamercolorbox}[wd=\paperwidth,ht=2.25ex,dp=1ex,right]{footlinecolor}%
%         \insertframenumber  \hspace*{2ex} 
       \end{beamercolorbox}}%
     \vskip0pt%
   }
 }
\newcommand{\backupend}{
   \addtocounter{framenumbervorappendix}{-\value{framenumber}}
   \addtocounter{framenumber}{\value{framenumbervorappendix}} 
}


\usepackage{graphicx}
\usepackage[space]{grffile}
\usepackage{booktabs}

% These are my colors -- there are many like them, but these ones are mine.
\definecolor{blue}{RGB}{0,114,178}
\definecolor{red}{RGB}{213,94,0}
\definecolor{yellow}{RGB}{240,228,66}
\definecolor{green}{RGB}{0,158,115}

\hypersetup{
  colorlinks=false,
  linkbordercolor = {white},
  linkcolor = {blue}
}

\usepackage{graphicx,stackengine,xcolor}
\newcommand\Circle[1]{%
	\def\useanchorwidth{T}%
	\def\stacktype{L}%
	\stackon[0pt]{#1}{\scalebox{2.0}[1.15]{\textcolor{red}{$\bigcirc$}}}%
}

%% I use a beige off white for my background
\definecolor{MyBackground}{RGB}{255,253,218}

%% Uncomment this if you want to change the background color to something else
%\setbeamercolor{background canvas}{bg=MyBackground}

%% Change the bg color to adjust your transition slide background color!
\newenvironment{transitionframe}{
  \setbeamercolor{background canvas}{bg=white}
  \begin{frame}}{
    \end{frame}
}

\setbeamercolor{frametitle}{fg=blue}
\setbeamercolor{title}{fg=black}
\setbeamertemplate{footline}[frame number]
\setbeamertemplate{navigation symbols}{} 
\setbeamertemplate{itemize items}{-}
\setbeamercolor{itemize item}{fg=blue}
\setbeamercolor{itemize subitem}{fg=blue}
\setbeamercolor{enumerate item}{fg=blue}
\setbeamercolor{enumerate subitem}{fg=blue}
\setbeamercolor{button}{bg=MyBackground,fg=blue,}

%%% TIKZ STUFF
\tikzset{   
	every picture/.style={remember picture,baseline},
	every node/.style={anchor=base,align=center,outer sep=1.5pt},
	every path/.style={thick},
}
\newcommand\marktopleft[1]{%
	\tikz[overlay,remember picture] 
	\node (marker-#1-a) at (-.3em,.3em) {};%
}
\newcommand\markbottomright[2]{%
	\tikz[overlay,remember picture] 
	\node (marker-#1-b) at (0em,0em) {};%
}
\tikzstyle{every picture}+=[remember picture] 
\tikzstyle{mybox} =[draw=black, very thick, rectangle, inner sep=10pt, inner ysep=20pt]
\tikzstyle{fancytitle} =[draw=black,fill=red, text=white]
%%%% END TIKZ STUFF


% If you like road maps, rather than having clutter at the top, have a roadmap show up at the end of each section 
% (and after your introduction)
% Uncomment this is if you want the roadmap!
% \AtBeginSection[]
% {
%    \begin{frame}
%        \frametitle{Roadmap of Talk}
%        \tableofcontents[currentsection]
%    \end{frame}
% }
\setbeamercolor{section in toc}{fg=blue}
\setbeamercolor{subsection in toc}{fg=red}
\setbeamersize{text margin left=1em,text margin right=1em} 

\newenvironment{wideitemize}{\itemize\addtolength{\itemsep}{10pt}}{\enditemize}
\newenvironment{wideenumerate}{\enumerate\addtolength{\itemsep}{10pt}}{\endenumerate}

\usepackage{environ}
\NewEnviron{videoframe}[1]{
  \begin{frame}
    \vspace{-8pt}
    \begin{columns}[onlytextwidth, T] % align columns
      \begin{column}{.58\textwidth}
        \begin{minipage}[t][\textheight][t]
          {\dimexpr\textwidth}
          \vspace{8pt}
          \hspace{4pt} {\Large \sc \textcolor{blue}{#1}}
          \vspace{8pt}
          
          \BODY
        \end{minipage}
      \end{column}%
      \hfill%
      \begin{column}{.42\textwidth}
        \colorbox{green!20}{\begin{minipage}[t][1.2\textheight][t]
            {\dimexpr\textwidth}
            Face goes here
          \end{minipage}}
      \end{column}%
    \end{columns}
  \end{frame}
}

\title[]{\textcolor{blue}{ECN 453: Vertical Relationships 1}}
\author[PGP]{}
\institute[FRBNY]{\small{\begin{tabular}{c c c}
Nicholas Vreugdenhil \\
\end{tabular}}}
\date{} 

\begin{document}

% Title Slide
\begin{frame}
\maketitle
  \centering
\end{frame}

% INTRO

\begin{frame}{Vertical relationships between firms}
\begin{wideitemize}
	\item \textbf{Vertical relationships}:  relationships between two firms in a sequence along the value chain.
	\item  \textbf{Examples}:
	\begin{wideitemize}
		\item Cement producers $\rightarrow$ sell cement to concrete producers $\rightarrow$ sell concrete to construction firms.
		\item Commodities $\rightarrow$ car parts $\rightarrow$ assembled into cars
		\item Manufacturer-retailer relationships
	\end{wideitemize}
	\begin{figure}
		\centering
	\includegraphics[scale=0.15]{retailer.jpeg}
	\end{figure}
\end{wideitemize}
\end{frame}

\begin{frame}{Motivation: how might vertical relationships be different to firm-consumer relationships?}
	\begin{wideitemize}
		\item Demand faced by a manufacturer depends not just on the price it sets
		\begin{wideitemize}
			\item ...but also on a host of other factors most of which is doesn't directly control (e.g. the retail price is set by the retailer)
		\end{wideitemize}
		\item Manufacturers are selling to retailers who are competing with each other.
		\begin{wideitemize}
			\item consumers usually do not compete with each other
		\end{wideitemize}
	\end{wideitemize}
\end{frame}

\begin{frame}{Some jargon}
	\begin{wideitemize}
		\item In this section of the course, we will usually simplify to the situation of a single manufacturer selling to one or several retailers.
		\item \textbf{Upstream firm}: the manufacturer e.g. cement producer, flour producer
		\item \textbf{Downstream firm}: the retailer e.g. concrete producer, bakery
		\item \textbf{Vertical integration}: when an upstream firm and a downstream firm are a single firm (e.g. through a merger, or just because that is the way the industry has evolved)
	\end{wideitemize}
\end{frame}

\begin{frame}{Plan}
	\begin{wideenumerate}
		\item Vertical integration: graphical version
		\item Vertical integration: math version
		\item Alternative solutions to double marginalization
	\end{wideenumerate}
\end{frame}

\begin{frame}{Plan}
	\begin{wideenumerate}
		\item \textbf{Vertical integration: graphical version}
		\item Vertical integration: math version
		\item Alternative solutions to double marginalization
	\end{wideenumerate}
\end{frame}

\begin{frame}{Vertical integration: setup}
	\begin{wideitemize}
		\item Consider an upstream firm (M) selling inputs to a downstream firm (R) which produces output.
		\begin{wideitemize}
			\item e.g. M is an oil refiner and R is a gas station.
		\end{wideitemize} 
		\item Demand for final product: $D(p)$
		\begin{wideitemize}
			\item Assume to produce one unit of output $R$ needs one unit of input.
		\end{wideitemize} 
		\item Costs:
		\begin{wideitemize}
			\item  $R$ pays the wholesale price $w$ to its supplier (assume it has no other costs)
			\item Assume $M$ has constant marginal cost $c$
		\end{wideitemize} 
		\item \textbf{Questions:}
		\item What are the joint profits of the firms if $M$ and $R$ are \textbf{vertically integrated}?
		\item What are the joint profits of the firms if $M$ and $R$ are \textbf{vertically separated}?
	\end{wideitemize}
\end{frame}

\begin{frame}{Vertical integration: graphical version}
	\begin{wideitemize}
		\item What are the joint profits of the firms if $M$ and $R$ are \textbf{vertically integrated}?
		\item Here the firm maximizes total profit:
		\begin{align*} 
			\pi = (\underbrace{w-c}_{\text{what M gets}} + \underbrace{p-w}_{\text{what R gets}}) D(p) = (p-c)D(p)
		\end{align*}
		\item Maximize profit $\rightarrow$ our typical monopoly solution
		\item Denote $p^M$ the price that maximizes total profit here (and $q^M$ the resulting quantity)
		\item Then, $\pi_M = (p^M-c)q^M$
	\end{wideitemize}
\end{frame}

\begin{frame}{Vertical integration: graphical version}
	\begin{wideitemize}
		\item What are the joint profits of the firms if $M$ and $R$ are \textbf{vertically separated}?
		\item R chooses $p$ to maximize:
		\begin{align*} 
			\pi =(p - w) D(p) 
		\end{align*}
		\item Here, $w$ is effectively R's marginal cost.
		\item To replicate the solution under vertical integration, $M$ would need to set $w=c$.
		\begin{wideitemize}
			\item But, $w=c$ would result in $M$'s profit = 0. 
		\end{wideitemize}
			\item Instead, $M$ sets $w$ above marginal cost. 
		\begin{wideitemize}
			\item But then, $R$ sets a price \textit{greater} than the optimal price $p_M$ (call this price $p^R$).
		\end{wideitemize}
		\item On the graph on the following page, see graphically that the total profit for both firms under vertical separation $(\pi^M + \pi^R)$ is smaller than under vertical integration.
	\end{wideitemize}
\end{frame}

\begin{frame}{Vertical integration: graphical version}
	\begin{figure}
		\includegraphics[scale=0.3]{double_marginalization.jpeg}
	\end{figure}
\end{frame}

\begin{frame}{Plan}
	\begin{wideenumerate}
		\item Vertical integration: graphical version
		\item \textbf{Vertical integration: math version}
		\item Alternative solutions to double marginalization
	\end{wideenumerate}
\end{frame}

\begin{frame}{Vertical integration: math version}
	\begin{wideitemize}
		\item \textbf{Setup:} 		
		\item 1. Manufacturer has a constant marginal cost $c=1$ and sets input price $w$ to maximize profit.
		\item 2. Retailer buys input from manufacturer for price $w$. Retailer sets price $p$ to maximize profit with demand $D(p)= 9-p$.
		\item \textbf{Question:}
		\item 1. What are the total profits (i.e. the joint profits of both firms) under vertical integration?
		\item 2. What are the total profits (i.e. the joint profits of both firms) under vertical separation?
		\item 3. Are profits higher under vertical integration or vertical separation?
	\end{wideitemize}
\end{frame}

\begin{frame}{Vertical integration: math version - solution}
	\begin{wideitemize}
		\item  1. What are the total profits (i.e. the joint profits of both firms) under vertical integration?
		\item Here, $MR=9-2q$. Set $MR=MC$:
		\item Then, $q^M=4$, $p^M=5$
		\item Profit = $16$.
	\end{wideitemize}
\end{frame}

\begin{frame}{Vertical integration: math version - solution}
	\begin{wideitemize}
		\item  2. What are the total profits (i.e. the joint profits of both firms) under vertical separation?
		\item Start with retailer's problem.
			\begin{wideitemize}
				\item Optimal price: $MR=9-2q$. $MC=w$.
				\item So, $q = \frac{9-w}{2}$, $p=\frac{9+w}{2}$.
			\end{wideitemize}
		\item Now consider the manufacturer:
			\begin{wideitemize}
				\item Manufacturer's profit: $\pi_m = (w-c)q = (w-c)\frac{9-w}{2}$.
				\item Take derivative of profit with respect to $w$ and set to 0 to find the optimal wholesale price $w$.
				\item Solving for $w$ and then substituting $w$ into the retailer's problem: $w=5, p^R=7,q^M=2$.
			\end{wideitemize}
		\item Total profits: $W$ gets 8, $R$ gets 4. Total profits = 12.
	\end{wideitemize}
\end{frame}

\begin{frame}{Vertical integration: math version - solution}
	\begin{wideitemize}
		\item   3. Are profits higher under vertical integration or vertical separation?
		\item Profits are higher under vertical integration. Under vertical separation the retailer's price is too high due to \textit{double marginalization}.
	\end{wideitemize}
\end{frame}

\begin{frame}{Solving double marginalization problems: math steps}
	\begin{wideitemize}
		\item \underline{Vertical separation}:
		\item 1. Begin with retailer's problem. Find the price and quantity that maximizes profit given the wholesale price $w$ is the retailer's marginal cost.
		\item 2. Next, solve the (upstream) manufacturer's problem. Find the wholesale price $w$ which maximizes profit given the retailer's optimal choice.
		\item \underline{Vertical integration}: 
		\item Solve using the standard monopoly solution.
	\end{wideitemize}
\end{frame}

\begin{frame}{Double marginalization: example - regional sport networks}
	\begin{wideitemize}
		\item Regional sport networks: $90\%$ of the 116.4 million television households subscribe to multichannel TV.
		\item Content providers like Disney and ESPN sell to Distributors like Comcast and TimeWarner. 
		\end{wideitemize}
\end{frame}

\begin{frame}{Double marginalization: example - regional sport networks}
	\begin{wideitemize}
		\item Regional sports networks: variation in ownership.
		\begin{wideitemize}
			\item  E.g. in 2007 Comcast Sport Northwest (owned by Comcast) was carried by Comcast but not DirectTV and Disk.
			\item Independently owned YES (Yankees Entertainment and Sports) was carried by TimeWarner and DirectTV but not by Disk.
		\end{wideitemize}
		\begin{figure}
			\centering
			\includegraphics[scale=0.25]{comcast.png}
		\end{figure}
	\end{wideitemize}
\end{frame}

\begin{frame}{Double marginalization: example - regional sport networks}
	\begin{wideitemize}
		\item In this market, when there is vertical integration:
		\item 1. Price paid by consumers is lower
		\item 2. Likelihood that a RSN is carried by a distributor is higher
		\item 3. Likelihood that RSN is carried by rival distributor is lower
	\end{wideitemize}
\end{frame}

\begin{frame}{Plan}
	\begin{wideenumerate}
		\item Vertical integration: graphical version
		\item Vertical integration: math version
		\item  \textbf{Alternative solutions to double marginalization}
	\end{wideenumerate}
\end{frame}

\begin{frame}{Alternative solutions to the double marginalization problem: nonlinear contracts}
	\begin{wideitemize}
		\item  Strong assumption in the previous analysis: payment between the firms is given by a single wholesale price $w$.
		\item What if the firms can write a more complicated contract? Recall, a two-part tariff looks like:
		\begin{align*}
			\text{tariff} = f + w q
		\end{align*}
		\begin{wideitemize}
			\item Fixed fee: f
			\item Wholesale price: w
		\end{wideitemize}
	\end{wideitemize}
\end{frame}

\begin{frame}{Alternative solutions to the double marginalization problem: nonlinear contracts}
	\begin{wideitemize}
		\item Consider the following nonlinear contract (which the retailer pays to the manufacturer):
		\begin{wideitemize}
			\item $w=c$
			\item $f=\pi^M$ (the monopoly profit of an integrated firm)
		\end{wideitemize}
		\item Three things:
		\begin{enumerate}
			\item This contract maximizes joint profits of the firms (i.e. achieves the monopoly outcome). Why? \pause This is because the wholesale price equals marginal cost which induces the retailer $R$ to set the monopoly price.
			\item The retailer $R$ will enter the contract. Why? \pause The retailer receives gross profits $\pi^M$ so is willing to pay a fixed fee up to $\pi^M$.
			\item The manufacturer receives 0 in terms of variable profit (since $w=c$) but is able to recover monopoly profit through the fixed fee.
		\end{enumerate}
	\end{wideitemize}
\end{frame}

\begin{frame}{Alternative solutions to the double marginalization problem: nonlinear contracts}
	\begin{wideitemize}
		\item Bottom line of previous discussion:
		\item \textbf{If nonlinear contracts are possible, then the optimal solution under vertical separation is identical to that under vertical integration.}
		\item \underline{Qualification}: assumed no competition at either stage.
	\end{wideitemize}
\end{frame}

\begin{comment}
\begin{frame}{Alternative solutions to the double marginalization problem: maximum retail price}
	\begin{wideitemize}
		\item \textbf{Maximium retail price}: 
		\item Manufacturer $M$ imposes $p^M$ as the maximum retail price and sells to the retailer at $w=p^M$.
		\item Here, no need to collect a fixed fee, but will still get the optimal `vertical integration' solution.
	\end{wideitemize}
\end{frame}
\end{comment}

\begin{frame}{Summary of key points*}
	%\vspace{-20pt}
	\begin{wideitemize}
		\item Know how to compute the profit under vertical integration and also under vertical separation
		\item Understand that vertical separation may result in a \textit{double marginalization problem}
		\item Know that an alternative solution to vertical separation that eliminates double marginalization is a correctly specific two-part tariff.
	\end{wideitemize}
	\vspace{20pt}
	*To clarify, all the material in the slides, problem sets, etc is assessable unless stated otherwise, but I hope this summary might be a useful place to start when studying the material.
\end{frame}

\begin{comment}
\begin{frame}{Practice question}
	\begin{wideitemize}
		\item Suppose that there is a manufacturer who sells to a retailer at a wholesale price $w$. The retailer then sells to consumers at price $p$ with demand curve $q=10-2p$.
		\begin{wideitemize}
			\item 1. Compute the increase in profit moving from vertical separation to vertical integration (i.e. quantify the costs of double marginalizaton).
			\item 2. Provide a two-part tariff that would achieve the profit of a vertically integrated firm.
		\end{wideitemize}
	\end{wideitemize}
\end{frame}
\end{comment}

\end{document}
