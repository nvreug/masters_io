\documentclass[notes,11pt, aspectratio=169]{beamer}

\usepackage{pgfpages}
% These slides also contain speaker notes. You can print just the slides,
% just the notes, or both, depending on the setting below. Comment out the want
% you want.
\setbeameroption{hide notes} % Only slide
%\setbeameroption{show only notes} % Only notes
%\setbeameroption{show notes on second screen=right} % Both

%\usepackage[scaled=1.0]{helvet}
\usepackage{array}

\usepackage{graphicx}
\usepackage{tikz}
\usetikzlibrary{calc}
\usetikzlibrary{matrix}
\usetikzlibrary{positioning}

\newcommand{\payoff}[4][below]{\node[#1]at(#2){$(#3,#4)$};}
\usepackage{verbatim}
\setbeamertemplate{note page}{\pagecolor{gray!5}\insertnote}
\usetikzlibrary{positioning}
\usetikzlibrary{snakes}
\usetikzlibrary{calc}
\usetikzlibrary{arrows}
\usetikzlibrary{decorations.markings}
\usetikzlibrary{shapes.misc}
\usetikzlibrary{matrix,shapes,arrows,fit,tikzmark}
\usepackage{amsmath}
\usepackage{mathpazo}
\usepackage{hyperref}
\usepackage{lipsum}
\usepackage{multimedia}
\usepackage{graphicx}
\usepackage{multirow}
\usepackage{graphicx}
\usepackage{dcolumn}
\usepackage{bbm}
\newcolumntype{d}[0]{D{.}{.}{5}}

\usepackage{changepage}
\usepackage{appendixnumberbeamer}
\newcommand{\beginbackup}{
   \newcounter{framenumbervorappendix}
   \setcounter{framenumbervorappendix}{\value{framenumber}}
   \setbeamertemplate{footline}
   {
     \leavevmode%
     \hline
     box{%
       \begin{beamercolorbox}[wd=\paperwidth,ht=2.25ex,dp=1ex,right]{footlinecolor}%
%         \insertframenumber  \hspace*{2ex} 
       \end{beamercolorbox}}%
     \vskip0pt%
   }
 }
\newcommand{\backupend}{
   \addtocounter{framenumbervorappendix}{-\value{framenumber}}
   \addtocounter{framenumber}{\value{framenumbervorappendix}} 
}


\usepackage{graphicx}
\usepackage[space]{grffile}
\usepackage{booktabs}

% These are my colors -- there are many like them, but these ones are mine.
\definecolor{blue}{RGB}{0,114,178}
\definecolor{red}{RGB}{213,94,0}
\definecolor{yellow}{RGB}{240,228,66}
\definecolor{green}{RGB}{0,158,115}

\hypersetup{
  colorlinks=false,
  linkbordercolor = {white},
  linkcolor = {blue}
}

\usepackage{graphicx,stackengine,xcolor}
\newcommand\Circle[1]{%
	\def\useanchorwidth{T}%
	\def\stacktype{L}%
	\stackon[0pt]{#1}{\scalebox{2.0}[1.15]{\textcolor{red}{$\bigcirc$}}}%
}

%% I use a beige off white for my background
\definecolor{MyBackground}{RGB}{255,253,218}

%% Uncomment this if you want to change the background color to something else
%\setbeamercolor{background canvas}{bg=MyBackground}

%% Change the bg color to adjust your transition slide background color!
\newenvironment{transitionframe}{
  \setbeamercolor{background canvas}{bg=white}
  \begin{frame}}{
    \end{frame}
}

\setbeamercolor{frametitle}{fg=blue}
\setbeamercolor{title}{fg=black}
\setbeamertemplate{footline}[frame number]
\setbeamertemplate{navigation symbols}{} 
\setbeamertemplate{itemize items}{-}
\setbeamercolor{itemize item}{fg=blue}
\setbeamercolor{itemize subitem}{fg=blue}
\setbeamercolor{enumerate item}{fg=blue}
\setbeamercolor{enumerate subitem}{fg=blue}
\setbeamercolor{button}{bg=MyBackground,fg=blue,}

%%% TIKZ STUFF
\tikzset{   
	every picture/.style={remember picture,baseline},
	every node/.style={anchor=base,align=center,outer sep=1.5pt},
	every path/.style={thick},
}
\newcommand\marktopleft[1]{%
	\tikz[overlay,remember picture] 
	\node (marker-#1-a) at (-.3em,.3em) {};%
}
\newcommand\markbottomright[2]{%
	\tikz[overlay,remember picture] 
	\node (marker-#1-b) at (0em,0em) {};%
}
\tikzstyle{every picture}+=[remember picture] 
\tikzstyle{mybox} =[draw=black, very thick, rectangle, inner sep=10pt, inner ysep=20pt]
\tikzstyle{fancytitle} =[draw=black,fill=red, text=white]
%%%% END TIKZ STUFF


% If you like road maps, rather than having clutter at the top, have a roadmap show up at the end of each section 
% (and after your introduction)
% Uncomment this is if you want the roadmap!
% \AtBeginSection[]
% {
%    \begin{frame}
%        \frametitle{Roadmap of Talk}
%        \tableofcontents[currentsection]
%    \end{frame}
% }
\setbeamercolor{section in toc}{fg=blue}
\setbeamercolor{subsection in toc}{fg=red}
\setbeamersize{text margin left=1em,text margin right=1em} 

\newenvironment{wideitemize}{\itemize\addtolength{\itemsep}{10pt}}{\enditemize}
\newenvironment{wideenumerate}{\enumerate\addtolength{\itemsep}{10pt}}{\endenumerate}

\usepackage{environ}
\NewEnviron{videoframe}[1]{
  \begin{frame}
    \vspace{-8pt}
    \begin{columns}[onlytextwidth, T] % align columns
      \begin{column}{.58\textwidth}
        \begin{minipage}[t][\textheight][t]
          {\dimexpr\textwidth}
          \vspace{8pt}
          \hspace{4pt} {\Large \sc \textcolor{blue}{#1}}
          \vspace{8pt}
          
          \BODY
        \end{minipage}
      \end{column}%
      \hfill%
      \begin{column}{.42\textwidth}
        \colorbox{green!20}{\begin{minipage}[t][1.2\textheight][t]
            {\dimexpr\textwidth}
            Face goes here
          \end{minipage}}
      \end{column}%
    \end{columns}
  \end{frame}
}

\title[]{\textcolor{blue}{ECN 453: Vertical Relationships 2}}
\author[PGP]{}
\institute[FRBNY]{\small{\begin{tabular}{c c c}
Nicholas Vreugdenhil \\
\end{tabular}}}
\date{} 

\begin{document}

% Title Slide
\begin{frame}
\maketitle
  \centering
\end{frame}

% INTRO

\begin{frame}{Vertical relationships between firms}
\begin{wideitemize}
	\item \textbf{Vertical relationships}:  relationships between two firms in a sequence along the value chain.
	\item Main idea of last time: vertical integration eliminates double marginalization.
	\item Today: other important economic behavior in supply chains.
\end{wideitemize}
\end{frame}

\begin{frame}{Plan}
	\begin{wideenumerate}
		\item Downstream competition
		\item Investment incentives
	\end{wideenumerate}
\end{frame}

\begin{frame}{Plan}
	\begin{wideenumerate}
		\item \textbf{Downstream competition}
		\item Investment incentives
	\end{wideenumerate}
\end{frame}

\begin{frame}{Downstream competition}
	\begin{wideitemize}
		\item Last time we looked at the situation where there was only one firm downstream 
		\item What if there is $>1$ firm downstream, and these firms are competing against each other?
		\item Example: Samsung makes phone parts and phones. It also supplies Apple with phone parts, and Apple then competes with Samsung downstream by selling phones to consumers.
	\end{wideitemize}
	\begin{figure}
		\includegraphics[scale=0.2]{samsung.jpeg}
	\end{figure}
\end{frame}

\begin{frame}{Downstream competition}
	\begin{wideitemize}
		\item \textbf{Setup:} 
		\item A single upstream firm (manufacturer M)
		\item Two downstream firms (retailers $R_1$ and $R_2$)
		\begin{wideitemize}
			\item Denote $w_i$ the wholesale price paid by $R_i$
			\item Denote $p_i$ the retail price paid by $R_i$
		\end{wideitemize}
		\item \textbf{Question:} Suppose that firm $M$ merges with retailer $R_i$. What impact would we expect this to have on prices?
	\end{wideitemize}
\end{frame}

\begin{frame}{Downstream competition (diagram from book)}
	\begin{figure}
		\includegraphics[scale=0.2]{downstream.jpeg}
	\end{figure}
\end{frame}

\begin{frame}{Downstream competition}
	\begin{wideitemize}
		\item \textbf{Question:} Suppose that firm $M$ merges with retailer $R_1$. What impact would we expect this to have on prices?
		\item \underline{Effect on $w_1$}:\pause Effect is to decrease $w_1$. Why? \pause Eliminate double marginalization.
		\item \underline{Effect on $w_2$ and $p_2$}: \pause Effect is to increase $w_2$ (and also $p_2$). Why? \pause
		\begin{wideitemize}
			\item Firm $R_2$, which was a customer of firm M, is now a \textit{rival} of the newly merged firm. 
			\item The merged firm can now increase its profits by increasing $w_2$.
			\item This is because an increase in $w_2$ induces $R_2$ to \textbf{increase} $p_2$, which in turn helps $R_1$, which in turns helps the newly merged firm.
			\item This is called the incentive to \textbf{raise rivals' costs}.
		\end{wideitemize}
	\end{wideitemize}
\end{frame}

\begin{frame}{Downstream competition}
	\begin{wideitemize}
		\item \textbf{Question:} Suppose that firm $M$ merges with retailer $R_1$. What impact would we expect this to have on prices?
		\item \underline{Effect on $p_1$}: \pause Ambiguous effect on $p_1$ (i.e. could increase or decrease $p_1$). Why? \pause 		\begin{wideitemize}
			\item Vertical integration eliminates double marginalization: tends to decrease $p_1$
			\item But, vertical integration causes \textbf{competition softening} that tends to push the price $p_1$ up.
			\item Why does competition softening happen? \pause
			\item Answer: When $R_1$ increase price it loses market share to $R_2$, causing $R_2$ to buy more from the manufacturer, which benefits the merged firm. Therefore, the benefits to decreasing price are lower which causes $p_1$ to increase.
		\end{wideitemize}
	\end{wideitemize}
\end{frame}

\begin{frame}{Downstream competition}
	\begin{wideitemize}
		\item Overall effects from M and $R_1$ merging:
		\begin{wideitemize}
			\item Profits of the merged firm increase.
			\item Profits of $R_2$ decrease 
			\item Conflicting effects on consumers: eliminate double marginalization but also soften downstream competition. Net effect can go either way.
		\end{wideitemize}
	\end{wideitemize}
\end{frame}


\begin{frame}{Plan}
	\begin{wideenumerate}
		\item Downstream competition
		\item \textbf{Investment incentives}
	\end{wideenumerate}
\end{frame}

\begin{frame}{Investment incentives}
	\begin{wideitemize}
		\item Suppose the $R$ comes up with a new car model worth $v$ to consumers.
		\item But: production is only possible if $M$ makes an investment (e.g. building a mould to make a car part)
		\begin{wideitemize}
			\item (Also assume that this is a \textbf{specific asset} i.e. mould can only be used to make R's car.)
		\end{wideitemize}
	\end{wideitemize}
\end{frame}

\begin{frame}{Investment incentives}
	\begin{wideitemize}
		\item \textbf{Issue that comes from the timing}: There are unforseen contingencies that make it hard to write a contract (e.g. how much it will cost to make the part, or how many cars will be demanded), but M's investment needs to be made from the get-go. 
		\item This generates a commitment problem: once M's investment has been made, $R$ could renege on its promise to pay for the investment and only agree to pay a price lower than the cost of the investment itself.
		\begin{wideitemize}
			\item $M$ would then agree to pay the lower price (since the investment is relationship-specific, M will not be able to find an alternative buyer, so will accept any price $> 0$)
		\end{wideitemize}
	\end{wideitemize}
\end{frame}

\begin{frame}{Investment incentives}
	\begin{wideitemize}
		\item \textbf{Setup:} (game tree on the next slide)
		\item The investment costs $c=5$, whereas the total value created by the investment is $v=8$ (which R gets)
		\begin{wideitemize}
			\item Since $v>c$, this is a worthwhile investment.
		\end{wideitemize}
		\item \underline{After} the investment is made, wholesale price (which retailer pays to manufacturer) is then negotiated. Assume there are two possible prices $p^H=6$ and $p^L=3$.
		\item \textbf{Payoffs:}
		\item So, final payoff for the Manufacturer is $6-5=1$ if investment is made and high price is chosen, and $3-5=-2$ if the investment is made and low price is chosen.
		\item Final payoff for the Retailer is $8-6=2$ if high price is chosen, and $8-3=5$ if low price is chosen.
	\end{wideitemize}
\end{frame}

\begin{frame}{Investment incentives}
	
		\tikzset{
		% Two node styles for game trees: solid and hollow
		solid node/.style={circle,draw,inner sep=1.5,fill=black},
		hollow node/.style={circle,draw,inner sep=1.5}
	}

	\begin{figure}
		\centering
		\begin{tikzpicture}[scale=1.5,font=\footnotesize]
			% Specify spacing for each level of the tree
			\tikzstyle{level 1}=[level distance=15mm,sibling distance=35mm]
			\tikzstyle{level 2}=[level distance=15mm,sibling distance=15mm]
			% The Tree
			\node(0)[solid node,label=above:{Manufacturer}]{}
			child{node(1)[solid node]{}
				child{node[hollow node,label=below:{$(1,2)$}]{} edge from parent node[left]{high price $p^H$}}
				child{node[hollow node,label=below:{$(-2,5)$}]{} edge from parent node[right]{low price $p^L$}}
				edge from parent node[left,xshift=-3]{invest}
			}
			child{node(2)[solid node]{}
				%child{node[hollow node,label=below:{$(10,20)$}]{} edge from parent node[left]{enter}}
				child{node[hollow node,label=below:{$(0,0)$}]{} edge from parent node[right]{}}
				edge from parent node[right,xshift=-3]{don't invest}
			};
			\node at ($(1)!.5!(2)$) {Retailer};
		\end{tikzpicture}
	\end{figure}
\end{frame}

\begin{frame}{Investment incentives}
	\begin{wideitemize}
		\item Solving the game, the equilibrium is `don't invest' even though the investment overall is worthwhile (i.e. the value of 8 is greater than the cost of 5).
		\begin{wideitemize}
		\item This is because the investment is a sunk cost by the time prices are determined.
			\end{wideitemize}
		\item This is called a \textbf{hold-up problem}: once the Manufacturer pays for the relationship-specific asset, the seller can charge a lower price.
		\item In this context, vertical integration solves the hold-up problem. Why? \pause
			\begin{wideitemize}
		\item The decision to invest is made by a single firm who simply chooses whether the investment is worthwhile (i.e. invest if $v>c$)
			\end{wideitemize}
		\item \textbf{When investments in specific assets are at stake, vertical integration alleviates the hold-up problem.}
	\end{wideitemize}
\end{frame}

\begin{frame}{Summary of key points*}
	%\vspace{-20pt}
	\begin{wideitemize}
		\item Understand two other economic issues in vertical relationships:
		\begin{wideenumerate}
			\item The hold-up problem
			\item How downstream competition affects wholesale prices and the final prices when two firms merge in a supply chain (a trade-off between double marginalization vs raising rivals' costs)
		\end{wideenumerate}
	\end{wideitemize}
	\vspace{40pt}
	*To clarify, all the material in the slides, problem sets, etc is assessable unless stated otherwise, but I hope this summary might be a useful place to start when studying the material.
\end{frame}

\begin{comment}
\begin{frame}{Practice question}
	\begin{wideitemize}
		\item Suppose that there is a manufacturer who sells to a retailer at a wholesale price $w$. The retailer then sells to consumers at price $p$ with demand curve $q=10-2p$.
		\begin{wideitemize}
			\item 1. Compute the increase in profit moving from vertical separation to vertical integration (i.e. quantify the costs of double marginalizaton).
			\item 2. Provide a two-part tariff that would achieve the profit of a vertically integrated firm.
		\end{wideitemize}
	\end{wideitemize}
\end{frame}
\end{comment}

\end{document}
