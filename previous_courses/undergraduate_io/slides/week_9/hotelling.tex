\documentclass[notes,11pt, aspectratio=169]{beamer}

\usepackage{pgfpages}
% These slides also contain speaker notes. You can print just the slides,
% just the notes, or both, depending on the setting below. Comment out the want
% you want.
\setbeameroption{hide notes} % Only slide
%\setbeameroption{show only notes} % Only notes
%\setbeameroption{show notes on second screen=right} % Both

%\usepackage[scaled=1.0]{helvet}
\usepackage{array}

\usepackage{graphicx}
\usepackage{tikz}
\usetikzlibrary{calc}
\usetikzlibrary{matrix}
\usetikzlibrary{positioning}

\newcommand{\payoff}[4][below]{\node[#1]at(#2){$(#3,#4)$};}
\usepackage{verbatim}
\setbeamertemplate{note page}{\pagecolor{gray!5}\insertnote}
\usetikzlibrary{positioning}
\usetikzlibrary{snakes}
\usetikzlibrary{calc}
\usetikzlibrary{arrows}
\usetikzlibrary{decorations.markings}
\usetikzlibrary{shapes.misc}
\usetikzlibrary{matrix,shapes,arrows,fit,tikzmark}
\usepackage{amsmath}
\usepackage{mathpazo}
\usepackage{hyperref}
\usepackage{lipsum}
\usepackage{multimedia}
\usepackage{graphicx}
\usepackage{multirow}
\usepackage{graphicx}
\usepackage{dcolumn}
\usepackage{bbm}
\newcolumntype{d}[0]{D{.}{.}{5}}

\usepackage{changepage}
\usepackage{appendixnumberbeamer}
\newcommand{\beginbackup}{
   \newcounter{framenumbervorappendix}
   \setcounter{framenumbervorappendix}{\value{framenumber}}
   \setbeamertemplate{footline}
   {
     \leavevmode%
     \hline
     box{%
       \begin{beamercolorbox}[wd=\paperwidth,ht=2.25ex,dp=1ex,right]{footlinecolor}%
%         \insertframenumber  \hspace*{2ex} 
       \end{beamercolorbox}}%
     \vskip0pt%
   }
 }
\newcommand{\backupend}{
   \addtocounter{framenumbervorappendix}{-\value{framenumber}}
   \addtocounter{framenumber}{\value{framenumbervorappendix}} 
}


\usepackage{graphicx}
\usepackage[space]{grffile}
\usepackage{booktabs}

% These are my colors -- there are many like them, but these ones are mine.
\definecolor{blue}{RGB}{0,114,178}
\definecolor{red}{RGB}{213,94,0}
\definecolor{yellow}{RGB}{240,228,66}
\definecolor{green}{RGB}{0,158,115}

\hypersetup{
  colorlinks=false,
  linkbordercolor = {white},
  linkcolor = {blue}
}

\usepackage{graphicx,stackengine,xcolor}
\newcommand\Circle[1]{%
	\def\useanchorwidth{T}%
	\def\stacktype{L}%
	\stackon[0pt]{#1}{\scalebox{2.0}[1.15]{\textcolor{red}{$\bigcirc$}}}%
}

%% I use a beige off white for my background
\definecolor{MyBackground}{RGB}{255,253,218}

%% Uncomment this if you want to change the background color to something else
%\setbeamercolor{background canvas}{bg=MyBackground}

%% Change the bg color to adjust your transition slide background color!
\newenvironment{transitionframe}{
  \setbeamercolor{background canvas}{bg=white}
  \begin{frame}}{
    \end{frame}
}

\setbeamercolor{frametitle}{fg=blue}
\setbeamercolor{title}{fg=black}
\setbeamertemplate{footline}[frame number]
\setbeamertemplate{navigation symbols}{} 
\setbeamertemplate{itemize items}{-}
\setbeamercolor{itemize item}{fg=blue}
\setbeamercolor{itemize subitem}{fg=blue}
\setbeamercolor{enumerate item}{fg=blue}
\setbeamercolor{enumerate subitem}{fg=blue}
\setbeamercolor{button}{bg=MyBackground,fg=blue,}

%%% TIKZ STUFF
\tikzset{   
	every picture/.style={remember picture,baseline},
	every node/.style={anchor=base,align=center,outer sep=1.5pt},
	every path/.style={thick},
}
\newcommand\marktopleft[1]{%
	\tikz[overlay,remember picture] 
	\node (marker-#1-a) at (-.3em,.3em) {};%
}
\newcommand\markbottomright[2]{%
	\tikz[overlay,remember picture] 
	\node (marker-#1-b) at (0em,0em) {};%
}
\tikzstyle{every picture}+=[remember picture] 
\tikzstyle{mybox} =[draw=black, very thick, rectangle, inner sep=10pt, inner ysep=20pt]
\tikzstyle{fancytitle} =[draw=black,fill=red, text=white]
%%%% END TIKZ STUFF


% If you like road maps, rather than having clutter at the top, have a roadmap show up at the end of each section 
% (and after your introduction)
% Uncomment this is if you want the roadmap!
% \AtBeginSection[]
% {
%    \begin{frame}
%        \frametitle{Roadmap of Talk}
%        \tableofcontents[currentsection]
%    \end{frame}
% }
\setbeamercolor{section in toc}{fg=blue}
\setbeamercolor{subsection in toc}{fg=red}
\setbeamersize{text margin left=1em,text margin right=1em} 

\newenvironment{wideitemize}{\itemize\addtolength{\itemsep}{10pt}}{\enditemize}
\newenvironment{wideenumerate}{\enumerate\addtolength{\itemsep}{10pt}}{\endenumerate}

\usepackage{environ}
\NewEnviron{videoframe}[1]{
  \begin{frame}
    \vspace{-8pt}
    \begin{columns}[onlytextwidth, T] % align columns
      \begin{column}{.58\textwidth}
        \begin{minipage}[t][\textheight][t]
          {\dimexpr\textwidth}
          \vspace{8pt}
          \hspace{4pt} {\Large \sc \textcolor{blue}{#1}}
          \vspace{8pt}
          
          \BODY
        \end{minipage}
      \end{column}%
      \hfill%
      \begin{column}{.42\textwidth}
        \colorbox{green!20}{\begin{minipage}[t][1.2\textheight][t]
            {\dimexpr\textwidth}
            Face goes here
          \end{minipage}}
      \end{column}%
    \end{columns}
  \end{frame}
}

\title[]{\textcolor{blue}{ECN 453: Hotelling Competition}}
\author[PGP]{}
\institute[FRBNY]{\small{\begin{tabular}{c c c}
Nicholas Vreugdenhil \\
\end{tabular}}}
\date{} 

\begin{document}

% Title Slide
\begin{frame}
\maketitle
  \centering
\end{frame}

% INTRO

\begin{frame}{Static Models of Oligopoly: Hotelling Competition}
\begin{wideitemize}
	\item So far we have studied three models of competition with \textbf{homogeneous} products.
	\item In many markets firms sell \textbf{differentiated} (heterogeneous) products. 
	\item How should we model competition in these markets?
	\item Today we will see one model: \textbf{Hotelling competition}.
\end{wideitemize}
\end{frame}

\begin{frame}{Static Models of Oligopoly: Hotelling Competition}
	\begin{wideitemize}
		\item Hotelling competition relates to selling products with \textbf{horizontal differentiation}.
		\item `Horizontal differentiation' relates to characteristics where there is no clear quality ordering
		\item Examples:
		\begin{wideitemize}
			\item Geographical differentiation (one supermarket might be closer to where you live than another)
			\item Product characteristics that appeal to different `tastes' (Pepsi vs Coke)
			\item Ice-cream vendors on a 1 mile beach (this was the original example given when this model was written)
		\end{wideitemize}
		\item Contrast: `Vertical differentiation' - consumers agree on which is the better product (i.e. quality differentiation)
	\end{wideitemize}
\end{frame}

\begin{frame}{Static Models of Oligopoly: Horizontal Differentiation On Campus}
	\begin{figure}
		\item \includegraphics[scale=0.18]{tempe_map.jpeg}
	\end{figure}
\end{frame}

\begin{frame}{Plan}
	\begin{wideenumerate}
		\item Hotelling Competition: Setup
		\item Hotelling Competition: Solution
		\item Hotelling Competition: Application
	\end{wideenumerate}
\end{frame}

\begin{frame}{Plan}
	\begin{wideenumerate}
		\item \textbf{Hotelling Competition: Setup}
		\item Hotelling Competition: Solution
		\item Hotelling Competition: Application
	\end{wideenumerate}
\end{frame}

\begin{frame}{Hotelling Competition: Setup - Preliminaries}
	\begin{wideitemize}
		\item Let's review one math preliminary before getting into the model.
		\item Remember the uniform distribution? 
		\item Uniform distribution: $x \sim U[0,1]$
		\vspace{11pt}
		\begin{wideitemize}
			\item If we pick a point $s \in [0,1]$, $Pr(x \leq s) = s$
		\end{wideitemize}
	\end{wideitemize}
\end{frame}

\begin{frame}{Hotelling Competition: Setup}
	\begin{wideitemize}
		\item \textbf{Players}
		\item Consumers are distributed $x \in U[0,1]$
		\begin{wideitemize}
			\item E.g. x refers to a consumer's `address'
		\end{wideitemize}
		\item Firm 1 is located at address $x=0$ and Firm 2 is located at address $x=1$.
		\item Consumer at location x pays a \textit{transport cost} equal to $t$ multiplied by the distance between its address and the Firm's address when they buy from a particular Firm (as well as price)
		\begin{wideitemize}
			\item Consumer at address $x$ pays transport cost $tx$ to get to Firm 1 and $t(1-x)$ to get to Firm 2.
		\end{wideitemize}
		\item \textbf{Strategies}
		\begin{wideitemize}
			\item Firm 1 and Firm 2 choose prices $p_1$ and $p_2$
		\end{wideitemize}
		\item \textbf{Payoffs}
		\begin{wideitemize}
			\item Firm 1 has marginal cost $c_1$ and Firm 2 has marginal cost $c_2$
			\item Firms maximize their profits
		\end{wideitemize}
	\end{wideitemize}
\end{frame}

\begin{frame}{Hotelling Competition: Setup - graph of consumer's payoffs}
	\begin{figure}
		\includegraphics[scale=0.6]{hotelling_consumers.pdf}
	\end{figure}
\end{frame}

\begin{frame}{Hotelling Competition: Setup}
	\begin{wideitemize}
		\item Before solving this model, let's first think about how the math relates to some real-world applications:
		\item \textbf{Supermarket competition}: here there are two supermarkets and the 'distance' cost might be e.g. cost of driving
		\item \textbf{Coke vs Pepsi (or other products)}: consumers have different tastes - if prices are equal will just choose the brand that is closest to their tastes.
	\end{wideitemize}
\end{frame}

\begin{frame}{Plan}
	\begin{wideenumerate}
		\item Hotelling Competition: Setup
		\item \textbf{Hotelling Competition: Solution}
		\item Hotelling Competition: Application
	\end{wideenumerate}
\end{frame}

\begin{frame}{Hotelling Competition: Solution}
	\begin{wideitemize}
		\item Broadly, we will follow our 'usual steps' to solve these models of competition:
		\item 1. Get the payoffs (profits)
		\item 2. Find the best responses
		\item 3. Solve for the Nash equilibrium
	\end{wideitemize}
\end{frame}

\begin{frame}{Hotelling Competition: Solution - 1. Get the payoffs}
	\begin{wideitemize}
		\item \textbf{Note}: for simplicity, we will assume that consumers always choose to buy \textit{something} (i.e. ignore the possibility that the firms might set prices so high some consumers may choose to buy from either firm)
		\item A consumer with address $x'$ is indifferent between the two firms if:
		\begin{align*}
			tx'+p_1 = t(1-x')+p_2
		\end{align*}
		\item So, consumers to the left of $x'$ buy from Firm 1 and consumers to the right of $x'$ buy from Firm 2.
		\begin{wideitemize}
			\item So, since consumers are uniform distributed Firm 1's demand is $x'$ and Firm 2's demand is $1-x'$.
		\end{wideitemize}
		\item Solving the above equation for $x'$ implies:
		\begin{align*}
			q_1 &= 0.5 + \frac{p_2 - p_1}{2t} \\
			q_2 &= 0.5 + \frac{p_1 - p_2}{2t} 
		\end{align*}
	\end{wideitemize}
\end{frame}

\begin{frame}{Hotelling Competition: Solution - 1. Get the payoffs}
	\begin{wideitemize}
		\item On the previous slide we found demand for each firm. Use this demand to get the payoffs:
		\begin{align*}
			\pi_1 &= q_1 (p_1 - c_1) = (0.5 + \frac{p_2 - p_1}{2t})(p_1-c_1) \\
			\pi_2 &= q_2 (p_2 - c_2) = (0.5 + \frac{p_1 - p_2}{2t})(p_2-c_2) \\
		\end{align*}
	\end{wideitemize}
\end{frame}

\begin{frame}{Hotelling Competition: Solution - 2. Get the best responses}
	\begin{wideitemize}
		\item Take the derivative of the payoffs with respect to price and set to 0 to maximize profit:
		\item \underline{Firm 1}:
		\begin{align*}
			0.5 + \frac{p_2-p_1}{2t} - \frac{1}{2t}(p_1-c_1) = 0
		\end{align*}
		\item So:
		\begin{align*}
			p_1 = 0.5(c_1+t+p_2)
		\end{align*}
		\item \underline{Firm 2}:
		\begin{align*}
			0.5 + \frac{p_1-p_2}{2t} - \frac{1}{2t}(p_2-c_2) = 0
		\end{align*}
		\item So:
		\begin{align*}
			p_2 = 0.5(c_2+t+p_1)
		\end{align*}
	\end{wideitemize}
\end{frame}

\begin{frame}{Hotelling Competition: Solution - 3. Solve for the Nash equilibrium}
	\begin{wideitemize}
		\item Setting $c_1=c_2=c$ and noting that since Firm 1 and Firm 2 are now identical, in equilibrium $p_1=p_2$.
		\item So:
		\begin{align*}
			p_1=p_2=c+t
		\end{align*}
	\end{wideitemize}
\end{frame}

\begin{frame}{Hotelling Competition: Solution Discussion}
	\begin{wideitemize}
		\item Consider the solution:
		\begin{align*}
			p_1=p_2=c+t
		\end{align*}
		\item The transportation costs $t$ (which index how differentiated products are to consumers) govern how intense competition is. 
		\item If $t=0$ consumers flock to the product with the lowest cost and we get the Bertrand solution. So, in the model, product differentiation ($t>0$) solves the 'Bertrand Trap'.
		\item Using the model solution, we can relate this idea to `market power' (the ability of firms to price their products above marginal cost):
		\item \textbf{The greater the degree of product differentiation, the greater the degree of market power.}
		\end{wideitemize}
\end{frame}


\begin{frame}{Hotelling Competition: Solution Discussion}
	\begin{wideitemize}
		\item What if firms could choose their location (product positioning) and then set prices?
		\begin{wideitemize}
		\item Full solution is a little too complicated to go into in detail, but I'll discuss the main intuition 
		\end{wideitemize}
		\item Depends on the interplay between two effects (consider the case of moving from opposite ends of the uniform distribution the center at 0.5):
		\begin{wideitemize}
			\item \underline{Direct effect}: Holding prices fixed, moving closer to the center increases demand and profits
			\item \underline{Strategic effect}: Moving closer to the rival leads the rival to decrease its prices, which in turn decreases a firm's profits.
		\end{wideitemize}
	\end{wideitemize}
\end{frame}

\begin{frame}{Hotelling Competition: Solution Discussion}
	\begin{wideitemize}
		\item Examples of these effects:
		\item \textit{Example:} retail banking in Europe. Prices (interest rates) determined at the country level, so strategic effect is low. So, expect banks to locate branches close to the center.
		\item \textit{Example:} ice-cream vendors on a beach. Consumers choose solely based on price - expect them to locate far from each other.
	\end{wideitemize}
\end{frame}

\begin{frame}{Plan}
	\begin{wideenumerate}
		\item Hotelling Competition: Setup
		\item Hotelling Competition: Solution
		\item \textbf{Hotelling Competition: Application}
	\end{wideenumerate}
\end{frame}

\begin{frame}{Hotelling Competition: Application - Strategic Trade Policy}
	\begin{wideitemize}
		\item \textbf{Background}: Airplane producers: US (Boeing); EU (Airbus)
		\item In 2005 US sued EU before the World Trade Organization, accusing the EU of subsidizing Airbus.
		\item We'd like to know: what was the impact of these subsidies on Boeing's profitability?
	\end{wideitemize}
\end{frame}

\begin{frame}{Hotelling Competition: Application - Strategic Trade Policy}
	\begin{wideitemize}
		\item \textbf{Setup}: Suppose that we have the Hotelling setup from before, where we found the best responses were:
		\begin{align*}
			\text{Firm 1:} \hspace{11pt} p_1 = 0.5(c_1+t+p_2) \\
			\text{Firm 2:} \hspace{11pt} p_2 = 0.5(c_2+t+p_1)
		\end{align*}
		\item \textbf{Questions:}
		\item 1. What is the Nash equilibrium?
		\item 2. Starting from the case where costs are the same ($c_1=c_2$) what is the change in Boeing's profits (Firm 1) for a $\$1$ increase in subsidies for Airbus (Firm 2)?
	\end{wideitemize}
\end{frame}

\begin{frame}{Hotelling Competition: Application - Strategic Trade Policy: Solution}
	\begin{wideitemize}
		\item 1. Nash equilibrium (where $i$ is a particular firm and $j$ is its rival) is:
		\begin{align*}
			p_i = \frac{2}{3} c_i + \frac{1}{3}c_j + t
		\end{align*}
		\item 2. Note that profit is: $\pi_i = \frac{1}{18t} (3t + c_j - c_i)^2$
		\item Compute $\frac{d \pi_1}{d c_2}$
		\item Substitute in $c_1=c_2$ and we get that the derivative here is $\frac{1}{3}$.
		\item A \$1 increase in subsidies results lowers Airbus' marginal costs by \$1. This results in a 33c decrease in Boeing's profit.
	\end{wideitemize}
\end{frame}

\begin{frame}{Summary of key points*}
	%\vspace{-20pt}
	\begin{wideitemize}
		\item Know what 'horizontal differentiation' is
		\item Know how to setup, interpret, and solve the Hotelling model with two firms
		\item Understand the 'strategic' vs 'direct' effects as firms change location
	\end{wideitemize}
	\vspace{20pt}
	*To clarify, all the material in the slides, problem sets, etc is assessable unless stated otherwise, but I hope this summary might be a useful place to start when studying the material.
\end{frame}

\begin{comment}
\begin{frame}{Question}
	%\vspace{-20pt}
	\begin{wideitemize}
		\item \textbf{Setup:} 
		\item Suppose that consumers are distributed uniformly between $0$ and $1$. 
		\item Firm 1 is located at $0$ and Firm 2 is located at $0.75$. 
		\item Marginal costs are equal for both firms and equal to $c$. 
		\item Constant transport costs equal to $td$ where $d$ is distance and $t$ is a number. 
		\item Firms choose prices and compete under Hotelling competition.
		\item \textbf{Question:} 
		\item 1. Find demand for each firm
		\item 2. Find the best responses
		\item 3. Find the Nash equilibrium prices
	\end{wideitemize}
\end{frame}
\end{comment}

\end{document}
