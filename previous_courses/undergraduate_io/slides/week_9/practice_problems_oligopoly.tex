% Don't touch this %%%%%%%%%%%%%%%%%%%%%%%%%%%%%%%%%%%%%%%%%%%
\documentclass[addpoints]{exam}
\usepackage{fullpage}
\usepackage[left=1.0in,top=1.0in,right=1.0in,bottom=1.0in,headheight=3ex,headsep=3ex]{geometry}
\usepackage{graphicx}
\usepackage{float}
\usepackage{adjustbox}
\usepackage{comment}
\usepackage{tikz}
\usetikzlibrary{calc}
\usetikzlibrary{matrix}
\usetikzlibrary{positioning}
\usepackage{amsmath}

\tikzset{   
	every picture/.style={remember picture,baseline},
	every node/.style={anchor=base,align=center,outer sep=1.5pt},
	every path/.style={thick},
}
\newcommand\marktopleft[1]{%
	\tikz[overlay,remember picture] 
	\node (marker-#1-a) at (-.3em,.3em) {};%
}
\newcommand\markbottomright[2]{%
	\tikz[overlay,remember picture] 
	\node (marker-#1-b) at (0em,0em) {};%
}
\tikzstyle{every picture}+=[remember picture] 
\tikzstyle{mybox} =[draw=black, very thick, rectangle, inner sep=10pt, inner ysep=20pt]
\tikzstyle{fancytitle} =[draw=black,fill=red, text=white]


\usepackage{graphicx,stackengine,xcolor}
\newcommand\Circle[1]{%
	\def\useanchorwidth{T}%
	\def\stacktype{L}%
	\stackon[0pt]{#1}{\scalebox{2.0}[1.15]{\textcolor{red}{$\bigcirc$}}}%
}
\newcommand{\blankline}{\quad\pagebreak[2]}
%%%%%%%%%%%%%%%%%%%%%%%%%%%%%%%%%%%%%%%%%%%%%%%%%%%%%%%%%%%%%%

% Modify Course title, instructor name, semester here %%%%%%%%

\title{ECN 453: Static Oligopoly Practice Problems}
% Modify header here %%%%%%%%%%%%%%%%%%%%%%%%%%%%%%%%%%%%%%%%%
\rhead{\footnotesize ECN 453: Mid-term Exam 1: Practice}

\date{} 

\begin{document}
	\maketitle

\section*{Question 1\footnote{Question based on Pepall-Richards-Norman (2014)}}
	Suppose that total demand is given by $P=100-Q$ where $Q=q_1+q_2$ and there are two firms  Firm 1 has a constant marginal cost equal to 20 and Firm 2 has a constant marginal cost equal to $10$.
\begin{enumerate}
	\item What is the Bertrand equilibrium?
	\item What is the Cournot equilbrium?
	\item What is the Stackelberg equilibrium if Firm 1 moves first and Firm 2 moves second?
	\begin{comment}
	\item In the Stackelberg equilibrium, what value would $c$ have to be so that in equilibrium the two firms have the same market share?
	\end{comment}
\end{enumerate}

\section*{Question 2\footnote{Question based on Pepall-Richards-Norman (2014)}}
In Tuftsville, everyone lives along Main Street, which is 10 miles long. There are 1000 people uniformly spread up and down Main Street, and every day the each buy a fruit smoothie from one of the two stores located at either end of Main Street. 

Store 1 is located at the west end and Store 2 is located at the east end. Customers ride their scooters to and from the store using \$0.50 worth of gas per mile. Consumers buy smoothies from the store with the lowest price plus travel expenses.
\begin{enumerate}
	\item What is the demand for smoothies for each store if Store 1 charges the price $p_1$ and Store 2 charges the price $p_2$?
	\item Assume that marginal cost is constant and equal to $\$1$. Find the equilibrium prices.
	\begin{comment}
	\item Suppose that Store 3 enters halfway along Main street and that Store 1 and Store 2 do not change their prices. What is the optimal price for Store 3?
	\end{comment}
\end{enumerate}

\begin{comment}
\section*{Question 3\footnote{Based on 8.12 from Cabral textbook}}
Suppose there are two producers of wolframium in the world. Wolframium is a homogenous product. The two producers compete on quantities (Cournot competition) and have constant marginal costs of \$900 per tonne. One producer is in the US, another in Mexico. Demand for wolframium is found exclusively in the US. It is estimated that, at $p=\$1000$, world demand for Wolframium is 130000 metric tons per year, and demand elasticity is $\epsilon=-0.5$.
\begin{enumerate}
	\item Suppose the US government imposes an import tax of 20\% on wolframium imports. What are equilibrium price and profits?
	\item Suppose a third producer enters the industry. It is located in China and has a marginal cost of \$600 per metric ton. What impact does this have on equilibrium price and profits?
\end{enumerate}
\end{comment}

\section*{Question 3}
 Suppose that consumers are distributed uniformly between $0$ and $1$. Firm 1 is located at $0$ and Firm 2 is located at $0.75$. Marginal costs are equal for both firms and equal to $c$. Constant transport costs equal to $0.5d$ where $d$ is distance. Firms choose prices and compete under Hotelling competition.
	\begin{enumerate}
		\item Find the demand for each firm
		\item Find the best responses
		\item Find the Nash equilibrium prices
	\end{enumerate}

\section*{Question 4}
Suppose that there are two firms that compute on quantities. Firm 1 is the incumbent and moves first and Firm 2 is a potential entrant and moves second (so this is a Stackelberg oligopoly). Market demand is given by $P=40-Q$. The marginal cost is $c=10$.
\begin{enumerate}
	\item What is the optimal price that Firm 1 should charge if it is a monopolist? 
	\item What is Firm 1's profit if Firm 2 enters?
	\item Suppose that Firm 2 must pay an entry cost of \$100 to enter. What value of $q_1$ deters Firm 2's entry?
	\item Should the incumbent firm deter entry?
\end{enumerate}

\begin{comment}
\section*{Question 4}
Suppose that there are two supermarkets located exactly one mile away from each other. Between the two supermarkets live 500 people uniformly spread apart. Each person has a transport cost of going to the supermarket of $2 d$ where $d$ is the distance between the person's address and the supermarket.
\begin{enumerate}
	\item What are the payoffs to choosing each store for each consumer?
	\item What is the demand for each supermarket given they set prices denoted $p_1$ and $p_2$?
	\item What are the best response functions?
	\item What are the equilibrium profits both firms charge?
	\item Suppose these firms were forced to charge the prices above, but could now choose their locations. Would the firms want to relocate? Explain your answer.
\end{enumerate}

\section*{Question 5\footnote{Based on 8.10 from the Cabral textbook.}}
In the ethanol industry, each firm chooses what output to produce and price is determined by aggregate output. Market demand is given by $Q=1500-2p$, where Q is millions/tons and $p$ is in \$/ton. There are two producers and their marginal costs are constant and given by $c_1=340, c_2=420$ (in \$/ton). 
\begin{enumerate}
	\item Determine equilibrium price, output, and market shares.
	\item Firm 2 is considering a public opinion campaign that would cost \$1.14 billion and shift demand curve to $Q=1520-2p$. Is this investment worthwhile?
	\item Firm 2 is considering a capital investment of \$4.9 billion that would reduce marginal cost $c_2$ to 400
	\$/ton. Is this investment worthwhile?
\end{enumerate}
\end{comment}

\end{document}