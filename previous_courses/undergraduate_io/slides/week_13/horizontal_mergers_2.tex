\documentclass[notes,11pt, aspectratio=169]{beamer}

\usepackage{pgfpages}
% These slides also contain speaker notes. You can print just the slides,
% just the notes, or both, depending on the setting below. Comment out the want
% you want.
\setbeameroption{hide notes} % Only slide
%\setbeameroption{show only notes} % Only notes
%\setbeameroption{show notes on second screen=right} % Both

%\usepackage[scaled=1.0]{helvet}
\usepackage{array}

\usepackage{graphicx}
\usepackage{tikz}
\usetikzlibrary{calc}
\usetikzlibrary{matrix}
\usetikzlibrary{positioning}

\newcommand{\payoff}[4][below]{\node[#1]at(#2){$(#3,#4)$};}
\usepackage{verbatim}
\setbeamertemplate{note page}{\pagecolor{gray!5}\insertnote}
\usetikzlibrary{positioning}
\usetikzlibrary{snakes}
\usetikzlibrary{calc}
\usetikzlibrary{arrows}
\usetikzlibrary{decorations.markings}
\usetikzlibrary{shapes.misc}
\usetikzlibrary{matrix,shapes,arrows,fit,tikzmark}
\usepackage{amsmath}
\usepackage{mathpazo}
\usepackage{hyperref}
\usepackage{lipsum}
\usepackage{multimedia}
\usepackage{graphicx}
\usepackage{multirow}
\usepackage{graphicx}
\usepackage{dcolumn}
\usepackage{bbm}
\newcolumntype{d}[0]{D{.}{.}{5}}

\usepackage{changepage}
\usepackage{appendixnumberbeamer}
\newcommand{\beginbackup}{
   \newcounter{framenumbervorappendix}
   \setcounter{framenumbervorappendix}{\value{framenumber}}
   \setbeamertemplate{footline}
   {
     \leavevmode%
     \hline
     box{%
       \begin{beamercolorbox}[wd=\paperwidth,ht=2.25ex,dp=1ex,right]{footlinecolor}%
%         \insertframenumber  \hspace*{2ex} 
       \end{beamercolorbox}}%
     \vskip0pt%
   }
 }
\newcommand{\backupend}{
   \addtocounter{framenumbervorappendix}{-\value{framenumber}}
   \addtocounter{framenumber}{\value{framenumbervorappendix}} 
}


\usepackage{graphicx}
\usepackage[space]{grffile}
\usepackage{booktabs}

% These are my colors -- there are many like them, but these ones are mine.
\definecolor{blue}{RGB}{0,114,178}
\definecolor{red}{RGB}{213,94,0}
\definecolor{yellow}{RGB}{240,228,66}
\definecolor{green}{RGB}{0,158,115}

\hypersetup{
  colorlinks=false,
  linkbordercolor = {white},
  linkcolor = {blue}
}

\usepackage{graphicx,stackengine,xcolor}
\newcommand\Circle[1]{%
	\def\useanchorwidth{T}%
	\def\stacktype{L}%
	\stackon[0pt]{#1}{\scalebox{2.0}[1.15]{\textcolor{red}{$\bigcirc$}}}%
}

%% I use a beige off white for my background
\definecolor{MyBackground}{RGB}{255,253,218}

%% Uncomment this if you want to change the background color to something else
%\setbeamercolor{background canvas}{bg=MyBackground}

%% Change the bg color to adjust your transition slide background color!
\newenvironment{transitionframe}{
  \setbeamercolor{background canvas}{bg=white}
  \begin{frame}}{
    \end{frame}
}

\setbeamercolor{frametitle}{fg=blue}
\setbeamercolor{title}{fg=black}
\setbeamertemplate{footline}[frame number]
\setbeamertemplate{navigation symbols}{} 
\setbeamertemplate{itemize items}{-}
\setbeamercolor{itemize item}{fg=blue}
\setbeamercolor{itemize subitem}{fg=blue}
\setbeamercolor{enumerate item}{fg=blue}
\setbeamercolor{enumerate subitem}{fg=blue}
\setbeamercolor{button}{bg=MyBackground,fg=blue,}

%%% TIKZ STUFF
\tikzset{   
	every picture/.style={remember picture,baseline},
	every node/.style={anchor=base,align=center,outer sep=1.5pt},
	every path/.style={thick},
}
\newcommand\marktopleft[1]{%
	\tikz[overlay,remember picture] 
	\node (marker-#1-a) at (-.3em,.3em) {};%
}
\newcommand\markbottomright[2]{%
	\tikz[overlay,remember picture] 
	\node (marker-#1-b) at (0em,0em) {};%
}
\tikzstyle{every picture}+=[remember picture] 
\tikzstyle{mybox} =[draw=black, very thick, rectangle, inner sep=10pt, inner ysep=20pt]
\tikzstyle{fancytitle} =[draw=black,fill=red, text=white]
%%%% END TIKZ STUFF


% If you like road maps, rather than having clutter at the top, have a roadmap show up at the end of each section 
% (and after your introduction)
% Uncomment this is if you want the roadmap!
% \AtBeginSection[]
% {
%    \begin{frame}
%        \frametitle{Roadmap of Talk}
%        \tableofcontents[currentsection]
%    \end{frame}
% }
\setbeamercolor{section in toc}{fg=blue}
\setbeamercolor{subsection in toc}{fg=red}
\setbeamersize{text margin left=1em,text margin right=1em} 

\newenvironment{wideitemize}{\itemize\addtolength{\itemsep}{10pt}}{\enditemize}
\newenvironment{wideenumerate}{\enumerate\addtolength{\itemsep}{10pt}}{\endenumerate}

\usepackage{environ}
\NewEnviron{videoframe}[1]{
  \begin{frame}
    \vspace{-8pt}
    \begin{columns}[onlytextwidth, T] % align columns
      \begin{column}{.58\textwidth}
        \begin{minipage}[t][\textheight][t]
          {\dimexpr\textwidth}
          \vspace{8pt}
          \hspace{4pt} {\Large \sc \textcolor{blue}{#1}}
          \vspace{8pt}
          
          \BODY
        \end{minipage}
      \end{column}%
      \hfill%
      \begin{column}{.42\textwidth}
        \colorbox{green!20}{\begin{minipage}[t][1.2\textheight][t]
            {\dimexpr\textwidth}
            Face goes here
          \end{minipage}}
      \end{column}%
    \end{columns}
  \end{frame}
}

\title[]{\textcolor{blue}{ECN 453: Horizontal Mergers 2}}
\author[PGP]{}
\institute[FRBNY]{\small{\begin{tabular}{c c c}
Nicholas Vreugdenhil \\
\end{tabular}}}
\date{} 

\begin{document}

% Title Slide
\begin{frame}
\maketitle
  \centering
\end{frame}

% INTRO

\begin{frame}{Horizontal Mergers}
\begin{wideitemize}
	\item \textbf{Merger}: When two firms become one.
	\item \textbf{Horizontal Mergers:} Mergers between two firms in the same industry.
	\item \textbf{Review from last time:} when deciding whether to approve a merger, a regulator needs to tradeoff the positive effects from cost efficiencies with the negative effects of market power.
\end{wideitemize}
\end{frame}

\begin{frame}{Plan}
	\begin{wideenumerate}
		\item Horizontal merger dynamics
		\item Horizontal merger policy
	\end{wideenumerate}
\end{frame}

\begin{frame}{Plan}
	\begin{wideenumerate}
		\item \textbf{Horizontal merger dynamics}
		\item Horizontal merger policy
	\end{wideenumerate}
\end{frame}

\begin{frame}{Horizontal Merger Dynamics}
\begin{wideitemize}
	\item Before, considered one merger happening in isolation.
	\item Now, we will discuss some dynamic issues in horizontal mergers.
	\item First, we will talk about \textbf{merger waves}. Multiple mergers in an industry often happen in a short span of time. Why?
\end{wideitemize}
\end{frame}

\begin{frame}{Horizontal Merger Dynamics: Merger Waves in US Radio Stations}
\begin{figure}
	\centering
	\includegraphics[scale=0.2]{us_radio_waves.jpeg}
\end{figure}
\end{frame}

\begin{frame}{Horizontal Merger Dynamics: Merger Waves}
\begin{wideitemize}
	\item What caused this wave in US Radio Station mergers?
	\item \textbf{Exogenous factors}: US 1996 Telecommunications Act raised ownership caps in local markets and relaxed some other regulations.
	\item Alternatively, merger waves may be caused by endogenous factors.
	\item \textbf{Endogenous factors}: US supermarket business had a merger waves in the 1990s. 
	\begin{wideitemize}
		\item Some argue this merger wave was driven by need to cut costs and remain competitive with Wal-Mart
		\item As several firms merged together, pressure to cut costs became even greater, leading to more mergers.
	\end{wideitemize}
	\textbf{Merger waves may results from exogenous events (e.g. industry deregulation) or from endogenous events (e.g. a merger between two large firms).}
\end{wideitemize}
\end{frame}

\begin{frame}{Horizontal Merger Dynamics: Merger Waves (textbook Q11.6)}
\begin{wideitemize}
	\item \textbf{Setup:} Consider an industry where firms choose quantities (Cournot). Market demand is given by $Q=150-p$. Firms are identical. Marginal cost is constant and $c=50$. Fixed cost is $F=150$.
	\item Useful formula (from the previous lecture):
	\begin{align*}
		\pi_i = \left( \frac{a-c}{n+1} \right)^2 - F_i
	\end{align*}
	\item \textbf{Questions:} 
	\begin{wideitemize}
		\item (a) What are the profits per firm as the number of firms is equal to 2,3, and, 4?
		\item Suppose a merger leads to a new firm with the same fixed cost and the same marginal cost.
		\item (b) Suppose that initially there are 4 firms. Show that a merger between Firms 1 and 2 is unprofitable.
		\item (c) Suppose that Firms 3 and 4 decide to merge, forming Firm 3\&4. Show that now a merger between FIrms 1 and 2 is profitable.
	\end{wideitemize}
\end{wideitemize}
\end{frame}

\begin{frame}{Horizontal Merger Dynamics: Preemptive Mergers}
	\begin{wideitemize}
		\item Primary goal of a merger/acquisition may be to preempt a rival from doing so.
		\item Example (quote from an analyst):
		\item \textit{Google bought Waze not just because the company offers a potentially good product that Google can link to its own dominant map service, but possible because its purchase keeps Waze out of the hands of its rival Facebook, which was also a rumored bidder.}
		\item Under these conditions, the merger might even \textit{reduce} value!
	\end{wideitemize}
\end{frame}

\begin{frame}{Horizontal Merger Dynamics: Mergers and Entry}
	\begin{wideitemize}
		\item Consider again the US radio broadcasting industry.
		\item After the merger wave, the next slide shows that in most markets (72\%) mergers are followed by higher entry rates.
	\end{wideitemize}
\end{frame}

\begin{frame}{Horizontal Merger Dynamics: Mergers and Entry - monthly net entry rates}
\begin{figure}
	\centering
	\includegraphics[scale=0.2]{us_radio_wave_entry.jpeg}
\end{figure}
\end{frame}

\begin{frame}{Horizontal Merger Dynamics: Mergers and Entry}
	\begin{wideitemize}
		\item Why do you think entry might increase after a merger?
		\item \pause One reason: a merger can be thought of as a firm exit. 
		\item \textbf{If barriers to entry are not very high, then mergers tend to be followed by new firm entry.}
		\item Another way of thinking about this: mergers and entry jointly create `self-correcting' dynamics. This will play an important role in public policy towards mergers which we will now see.
	\end{wideitemize}
\end{frame}

\begin{frame}{Plan}
	\begin{wideenumerate}
		\item Horizontal merger dynamics
		\item \textbf{Horizontal merger policy}
	\end{wideenumerate}
\end{frame}

\begin{frame}{Horizontal merger policy}
	\begin{wideitemize}
		\item Three interested parties in a horizontal merger: merging firms, non-merging firms, and consumers.
		\item Task for public policy: evaluate the relative importance of each gain/loss, and to assess the overall effect.
		\item This is very challenging! E.g. information about cost savings comes from the firms themselves, usually, and they have a clear incentive to overstate the benefits.
	\end{wideitemize}
\end{frame}

\begin{frame}{Horizontal merger policy}
	\begin{wideitemize}
		\item Also important to merger analysis: what is the increase in price following a merger?
		\item From our previous discussion about market structure, equilibrium price is increasing in market concentration.
		\begin{wideitemize}
			\item Two large firms merging implies a greater increase in price than two small firms.
		\end{wideitemize}
		\item Price increase channels:
		\begin{wideitemize}
			\item \textbf{Unilateral effects} (less competition) 
			\item \textbf{Collusion effects} (easier to collude with fewer firms)
		\end{wideitemize}
	\end{wideitemize}
\end{frame}

\begin{frame}{Horizontal merger policy: practical aspects}
	\begin{wideitemize}
		\item \textbf{How much concentration is `too much'?}
		\item Department of Justice (DOJ) and FTC follow 2023 horizontal merger guidelines which depend on the Herfindahl-Hirschman Index (HHI):
		\item Post-merger $HHI > 1800$ AND change in $HHI>100$ due to merger:
		\begin{wideitemize}
			\item ``merger is presumed to substantially lessen competition''
			\item big change from pre-2023 ($HHI > 2500$ and change in $HHI > 200$ due to merger)
		\end{wideitemize}
		\item In addition, merger can be presumed anticompetitive if:
		\begin{wideitemize}
			\item 1. combined firm has market share $>30\%$ AND
			\item 2. the merger results in change in $HHI > 100$
		\end{wideitemize}
		\item These are only guidelines, but mergers in these category will attract a lot of scrutiny from regulators. Note that merger guidelines also set out other cases where a merger is presumed to lessen competition.
	\end{wideitemize}
\end{frame}

\begin{frame}{Horizontal merger policy: practical aspects - summary}{Table from 2023 Horizontal Merger Guidelines}
	\begin{figure}
		\centering
		\vspace{-100pt}
		\includegraphics[angle=270, scale=0.6]{guidelines_2023.pdf}
		\caption{From: 2023 Horizontal Merger Guidelines}
	\end{figure}
\end{frame}

\begin{frame}{Practice question: HP-Compaq Merger Q11.4}
	\begin{wideitemize}
		\item \textbf{Setup:} In 2001, HP acquired Compaq. The merger had an impact on two different markets: desktop PCs and servers. 
		\item Pre-merger market shares in the desktop PC market were: Dell 13; Compaq 12; HP 8; IBM 6; Gateway 4. 
		\item Pre-merger market shares in the servers market were: IBM 26; Compaq 16; HP 14; Dell 7. 
		\item Assume that the `leftover' market share is made up of tiny firms that can be ignored when calculating the HHI.
		\item \textbf{Question:} 
		\begin{wideitemize}
			\item 1. Determine the value of HHI in each market before the merger.
			\item 2. Assuming market shares of each firm remain constant, determine the value of HHI after the merger.
			\item 3. Considering the values determined above and the merger guidelines, was the Department of Justice right in allowing the merger to take place under the new 2023 guidelines?
		\end{wideitemize}
	\end{wideitemize}
\end{frame}

\begin{frame}{Horizontal merger policy: practical aspects}
	\begin{wideitemize}
		\item \textbf{What is the relevant `market'?}
		\item To compute the HHI, need to define market shares. What is the denominator in the `share'?
		\item Definition of market is an obvious way for firms to skirt merger enforcement: try to define the market as large as possible.
		\item Example: 1996 Staples and Office Depot (two largest US chains of office supplies superstores) proposed a merger.
		\begin{wideitemize}
			\item If market is `office superstores': combined market share of merging parties is $>70\%$.
			\item If market is `stores that sell office supplies': combined market share is much lower.
		\end{wideitemize}
		\item To get around these debates about market definition, recently FTC	has favored a more direct approach of estimating the impact of a merger on consumer prices.
	\end{wideitemize}
\end{frame}

\begin{frame}{Horizontal merger policy: merger remedies}
	\begin{wideitemize}
		\item In the US, mergers are challenged in court. 
		\begin{wideitemize}
			\item That is, regulators do not block them directly. In the EU, the European Commission blocks mergers directly and this can then be appealed in court.
		\end{wideitemize}
		\item Possible outcomes:
		\item \textbf{Behavioral remedies}: e.g. prices cannot be increased by x\% during the next $n$ years
		\item \textbf{Structural remedies}: e.g. sell assets to competitor
		\item Merger might be blocked
		\item Merger might be allowed to go ahead
	\end{wideitemize}
\end{frame}

\begin{frame}{Summary of key points*}
	%\vspace{-20pt}
	\begin{wideitemize}
		\item Mergers usually involve a public policy tradeoff: lower costs vs increased market power
		\item Understand the dynamic effects of mergers
		\item Know about two practical aspects of merger policy: 1. how regulators target enforcement based on concentration and 2. market definition
	\end{wideitemize}
	\vspace{20pt}
	*To clarify, all the material in the slides, problem sets, etc is assessable unless stated otherwise, but I hope this summary might be a useful place to start when studying the material.
\end{frame}

\begin{comment}
\begin{frame}{Practice question}
	\begin{wideitemize}
		\item \textbf{Setup:} Consider three identical firms computing under Cournot competition.
		\begin{wideitemize}
			\item Costs $C=2+3q$
			\item Demand $Q=20-p$
		\end{wideitemize}
		\item \textbf{Question:} Suppose that Firm 1 and Firm 2 decide to merge. If they merge then the costs of the merged firm are $C=3+2q$. 
		\begin{wideitemize}
			\item 1. What are the effects on the profits of the merging firms?
			\item 2. What are the effects on the profits of the non-merging firms?
			\item 3. What are the effects on consumers?
			\item 4. Overall, should the Department of Justice approve this merger?
		\end{wideitemize}
	\end{wideitemize}
\end{frame}
\end{comment}

\end{document}
