\documentclass[notes,11pt, aspectratio=169]{beamer}

\usepackage{pgfpages}
% These slides also contain speaker notes. You can print just the slides,
% just the notes, or both, depending on the setting below. Comment out the want
% you want.
\setbeameroption{hide notes} % Only slide
%\setbeameroption{show only notes} % Only notes
%\setbeameroption{show notes on second screen=right} % Both

%\usepackage[scaled=1.0]{helvet}
\usepackage{array}

\usepackage{graphicx}
\usepackage{tikz}
\usetikzlibrary{calc}
\usetikzlibrary{matrix}
\usetikzlibrary{positioning}

\newcommand{\payoff}[4][below]{\node[#1]at(#2){$(#3,#4)$};}
\usepackage{verbatim}
\setbeamertemplate{note page}{\pagecolor{gray!5}\insertnote}
\usetikzlibrary{positioning}
\usetikzlibrary{snakes}
\usetikzlibrary{calc}
\usetikzlibrary{arrows}
\usetikzlibrary{decorations.markings}
\usetikzlibrary{shapes.misc}
\usetikzlibrary{matrix,shapes,arrows,fit,tikzmark}
\usepackage{amsmath}
\usepackage{mathpazo}
\usepackage{hyperref}
\usepackage{lipsum}
\usepackage{multimedia}
\usepackage{graphicx}
\usepackage{multirow}
\usepackage{graphicx}
\usepackage{dcolumn}
\usepackage{bbm}
\newcolumntype{d}[0]{D{.}{.}{5}}

\usepackage{changepage}
\usepackage{appendixnumberbeamer}
\newcommand{\beginbackup}{
   \newcounter{framenumbervorappendix}
   \setcounter{framenumbervorappendix}{\value{framenumber}}
   \setbeamertemplate{footline}
   {
     \leavevmode%
     \hline
     box{%
       \begin{beamercolorbox}[wd=\paperwidth,ht=2.25ex,dp=1ex,right]{footlinecolor}%
%         \insertframenumber  \hspace*{2ex} 
       \end{beamercolorbox}}%
     \vskip0pt%
   }
 }
\newcommand{\backupend}{
   \addtocounter{framenumbervorappendix}{-\value{framenumber}}
   \addtocounter{framenumber}{\value{framenumbervorappendix}} 
}


\usepackage{graphicx}
\usepackage[space]{grffile}
\usepackage{booktabs}

% These are my colors -- there are many like them, but these ones are mine.
\definecolor{blue}{RGB}{0,114,178}
\definecolor{red}{RGB}{213,94,0}
\definecolor{yellow}{RGB}{240,228,66}
\definecolor{green}{RGB}{0,158,115}

\hypersetup{
  colorlinks=false,
  linkbordercolor = {white},
  linkcolor = {blue}
}

\usepackage{graphicx,stackengine,xcolor}
\newcommand\Circle[1]{%
	\def\useanchorwidth{T}%
	\def\stacktype{L}%
	\stackon[0pt]{#1}{\scalebox{2.0}[1.15]{\textcolor{red}{$\bigcirc$}}}%
}

%% I use a beige off white for my background
\definecolor{MyBackground}{RGB}{255,253,218}

%% Uncomment this if you want to change the background color to something else
%\setbeamercolor{background canvas}{bg=MyBackground}

%% Change the bg color to adjust your transition slide background color!
\newenvironment{transitionframe}{
  \setbeamercolor{background canvas}{bg=white}
  \begin{frame}}{
    \end{frame}
}

\setbeamercolor{frametitle}{fg=blue}
\setbeamercolor{title}{fg=black}
\setbeamertemplate{footline}[frame number]
\setbeamertemplate{navigation symbols}{} 
\setbeamertemplate{itemize items}{-}
\setbeamercolor{itemize item}{fg=blue}
\setbeamercolor{itemize subitem}{fg=blue}
\setbeamercolor{enumerate item}{fg=blue}
\setbeamercolor{enumerate subitem}{fg=blue}
\setbeamercolor{button}{bg=MyBackground,fg=blue,}

%%% TIKZ STUFF
\tikzset{   
	every picture/.style={remember picture,baseline},
	every node/.style={anchor=base,align=center,outer sep=1.5pt},
	every path/.style={thick},
}
\newcommand\marktopleft[1]{%
	\tikz[overlay,remember picture] 
	\node (marker-#1-a) at (-.3em,.3em) {};%
}
\newcommand\markbottomright[2]{%
	\tikz[overlay,remember picture] 
	\node (marker-#1-b) at (0em,0em) {};%
}
\tikzstyle{every picture}+=[remember picture] 
\tikzstyle{mybox} =[draw=black, very thick, rectangle, inner sep=10pt, inner ysep=20pt]
\tikzstyle{fancytitle} =[draw=black,fill=red, text=white]
%%%% END TIKZ STUFF


% If you like road maps, rather than having clutter at the top, have a roadmap show up at the end of each section 
% (and after your introduction)
% Uncomment this is if you want the roadmap!
% \AtBeginSection[]
% {
%    \begin{frame}
%        \frametitle{Roadmap of Talk}
%        \tableofcontents[currentsection]
%    \end{frame}
% }
\setbeamercolor{section in toc}{fg=blue}
\setbeamercolor{subsection in toc}{fg=red}
\setbeamersize{text margin left=1em,text margin right=1em} 

\newenvironment{wideitemize}{\itemize\addtolength{\itemsep}{10pt}}{\enditemize}
\newenvironment{wideenumerate}{\enumerate\addtolength{\itemsep}{10pt}}{\endenumerate}

\usepackage{environ}
\NewEnviron{videoframe}[1]{
  \begin{frame}
    \vspace{-8pt}
    \begin{columns}[onlytextwidth, T] % align columns
      \begin{column}{.58\textwidth}
        \begin{minipage}[t][\textheight][t]
          {\dimexpr\textwidth}
          \vspace{8pt}
          \hspace{4pt} {\Large \sc \textcolor{blue}{#1}}
          \vspace{8pt}
          
          \BODY
        \end{minipage}
      \end{column}%
      \hfill%
      \begin{column}{.42\textwidth}
        \colorbox{green!20}{\begin{minipage}[t][1.2\textheight][t]
            {\dimexpr\textwidth}
            Face goes here
          \end{minipage}}
      \end{column}%
    \end{columns}
  \end{frame}
}

\title[]{\textcolor{blue}{ECN 453: Horizontal Mergers 1}}
\author[PGP]{}
\institute[FRBNY]{\small{\begin{tabular}{c c c}
Nicholas Vreugdenhil \\
\end{tabular}}}
\date{} 

\begin{document}

% Title Slide
\begin{frame}
\maketitle
  \centering
\end{frame}

% INTRO

\begin{frame}{Horizontal Mergers}
\begin{wideitemize}
	\item \textbf{Merger}: When two firms become one.
	\item \textbf{Horizontal Mergers:} Mergers between two firms in the same industry
	\begin{wideitemize}
		\item Phillip Morris and Kraft (food products)
		\item Nestle and General Mills (breakfast cereals)
		\item InBev and Anheuser-Busch (beer)
	\end{wideitemize}
	\item \textbf{Vertical Mergers:} Mergers between two firms at different stages of the value chain
	\begin{wideitemize}
		\item e.g. Gasoline Refinery and a Gas Station
	\end{wideitemize}
	\item In this course, we will not focus on some other forms of mergers e.g. conglomerate mergers
\end{wideitemize}
\end{frame}

\begin{frame}{Largest merger and acquisitions as of 2014}
\begin{figure}
	\centering
	\includegraphics[scale=0.18]{largest_mergers.jpeg}
\end{figure}
\end{frame}

\begin{frame}{Plan}
	\begin{wideenumerate}
		\item Why do firms merge?
		\item Determining the effects of mergers.
		\item Horizontal merger policy
	\end{wideenumerate}
\end{frame}

\begin{frame}{Plan}
	\begin{wideenumerate}
		\item \textbf{Why do firms merge?}
		\item Determining the effects of mergers.
				\item Horizontal merger policy
	\end{wideenumerate}
\end{frame}

\begin{frame}{Why do firms merge?}
	\begin{wideitemize}
		\item \underline{Sony and Columbia}: ``synergies''; Columbia's collection of movies was seen as a guarantee of a minimum supply of `software' to complement the `hardware' offered by Sony
		\item \underline{Philip Morris and Kraft}: sell food products to supermarkets; merger allowed firms to increase bargaining power with respect to retailers
		\item \underline{Nestle and Rowntree}: Allowed Nestle to enter a new market for chocolate (UK), avoiding high cost of launching new brands (Rowntree owned KitKat, Smarties, etc)
		\item \underline{Nestle and General Mills}: Distributional efficiencies in breakfast cereals;  Nestle great at distribution, General Mills leader in production.
	\end{wideitemize}
\end{frame}

\begin{frame}{Economic effects of horizontal mergers}
	\begin{wideitemize}
		\item \textbf{Cost efficiencies/savings}
		\begin{wideitemize}
			\item Fixed costs e.g. reduce duplication
			\item Variable costs e.g the Nestle and General Mills example from before
		\end{wideitemize}
		\item \textbf{Market power}
		\begin{wideitemize}
			\item Unilateral effect (market has less competition)
			\item Coordination effects (easier to sustain collusion)
		\end{wideitemize}
		\item \textbf{Mergers normally imply an increase in prices and a reduction in costs.}
		\item Regulators need to decide whether to approve mergers: 
		\begin{wideitemize}
			\item Test: do the positive effects of efficiencies outweigh the negative effects of market power?
		\end{wideitemize}
	\end{wideitemize}
\end{frame}

\begin{frame}{Plan}
	\begin{wideenumerate}
		\item Why do firms merge?
		\item \textbf{Determining the effects of mergers.}
				\item Horizontal merger policy
	\end{wideenumerate}
\end{frame}

\begin{frame}{Determining the effects of mergers: useful formulas}
	\begin{wideitemize}
		\item \textbf{Setup}
		\begin{wideitemize}
			\item Firm i's cost: $C_i = F_i + c_i q$
			\item Market demand: $D=a-p$
			\item Cournot competition with $n$ firms.
		\end{wideitemize}
	\item Then, profits and consumer surplus are:
	\end{wideitemize}
	\begin{align*}
		\hat{\pi}_i &= \left( \frac{a-n c_i + \sum_{j \neq i} c_j}{n+1} \right)^2 - F_i \\
		CS &= \frac{1}{2} \left( \frac{n}{n+1} \right)^2 \left( a - \frac{1}{n}\sum_{i=1}^n c_i \right)^2
	\end{align*}
	\begin{wideitemize}
		\item Note: $ \sum_{j \neq i} c_j$ is `sum of all other firms' marginal costs'
		\item Note: $ \sum_{i=1}^n c_i$ is `sum of all firms' marginal costs'
	\end{wideitemize}
\end{frame}


\begin{frame}{Determining the effects of mergers: useful formulas}
	\begin{wideitemize}
		\item We can derive the formulas on the previous slide using our `usual method' for solving Cournot competition models.
		\item We've derived other formulas that are very similar, so I will just take these useful formulas as given, and use them directly for analysis.
	\end{wideitemize}
\end{frame}


\begin{frame}{Determining the effects of mergers}
	\begin{wideitemize}
		\item \textbf{Setup:}
		\begin{wideitemize}
			\item Initially, n=3, all firms: $C=F+cq$
			\item Assume Firm 2 and Firm 3 merge $\rightarrow$ Firm 2\&3.
			\item Firm 2\&3 has $C=F'+c'q$. 
			\item Merger efficiencies: $F < F' < 2F$, $c'<c$
		\end{wideitemize}
		\item \textbf{Question:} What is the effect of the merger on:
		\begin{wideitemize}
			\item 1. The merging firms?
			\item 2. The non-merging firm?
			\item 3. Consumers?
		\end{wideitemize}
	\end{wideitemize}
\end{frame}

\begin{frame}{Determining the effects of mergers: solution}
	\begin{wideitemize}
		\item  \textbf{Question:} 1. What is the effect of the merger on the merging firms?
		\item Each firm's pre-merger profit:
		\begin{align*}
			\pi_1 = \pi_2 = \pi_3 = \left( \frac{a-c}{4} \right)^2 - F
		\end{align*}
		\item Firm 2\&3's profit post-merger:
		\begin{align*}
			\pi_{2 \& 3} = \left( \frac{a+c-2c'}{3} \right)^2 - F'
		\end{align*}
	\end{wideitemize}
\end{frame}

\begin{frame}{Determining the effects of mergers: solution}
	\begin{wideitemize}
		\item  \textbf{Question:} 1. What is the effect of the merger on the merging firms?
		\item Change in profit (taking differences):
		\begin{align*}
			\pi_{2 \& 3} - (\pi_2 + \pi_3)= (2F - F') + \left( \frac{a+c-2c'}{3} \right)^2 - 2 \left( \frac{a-c}{4} \right)^2
		\end{align*}
		\item Four effects on profits that this equation shows:
		\begin{wideenumerate}
			\item Fixed cost savings $F' < 2F$ (positive effect)
			\item Marginal cost savings: if $c' <c$ then $a+c-2c'>a-c$ (positive effect)
			\item Market power: number of firms $4 \rightarrow 3$ (positive effect) 
			\item Exit: two profits are turned into one (negative)
		\end{wideenumerate}
	\end{wideitemize}
\end{frame}

\begin{frame}{Determining the effects of mergers: solution}
	\begin{wideitemize}
		\item  \textbf{Question:} 2. What is the effect of the merger on the non-merging firms?
		\item Firm 1's profit after the merger:
		\begin{align*}
			\pi_{1}' =  \left( \frac{a+c'-2c}{3} \right)^2 -F
		\end{align*}
		\item Change in profit (taking differences):
		\begin{align*}
			\pi_{1}' - \pi_1=  \left( \frac{a+c'-2c}{3} \right)^2 - \left( \frac{a-c}{4} \right)^2
		\end{align*}
		\item Effects:
		\begin{wideenumerate}
		\item Rival is more efficient: if $c'<c$ then $a+c'-2c<a-c$ (negative effect)
		\item Market power: $2<3$ (positive effect)
		\end{wideenumerate}
	\end{wideitemize}
\end{frame}

\begin{frame}{Determining the effects of mergers: solution - effect on outsiders, examples}
	\begin{wideitemize}
		\item \underline{Oil industry}: BP acquired Amoco. As a result, Mobil's stock price increased by \$2.625 after the announcement
		\item \underline{Hard drive industry}: in March 2011, Western Digital announced it would buy Hitachi (number 1 and 2 suppliers of disk drives). Stock prices of Seagate (third largest supplier) grew by 9 \%.
		\item \underline{Airline industry}: British Airways and American Airlines announced a proposed merger. As a results, Virgin Atlantic painted its aircraft with  
	\begin{figure}
		\includegraphics[scale=0.2]{ba.jpeg}
	\end{figure}
	\item \textbf{The value of non-merging firms may decrease or increase as the result of a merger, depending on the cost efficiencies generated by the merger.}
	\end{wideitemize}
\end{frame}

\begin{frame}{Determining the effects of mergers: solution}
	\begin{wideitemize}
		\item  \textbf{Question:} 3. What is the effect of the merger on consumers?
		\item Change in consumer surplus (taking differences):
		\begin{align*}
			CS'-CS=\frac{1}{2} \left( \frac{2}{3} \right)^2 (a- \frac{1}{2} (c+c'))^2 - \frac{1}{2} \left( \frac{3}{4} \right)^2 (a- c)^2 
		\end{align*}
		\item Effects:
		\begin{wideenumerate}
			\item Part of the merger's cost reductions are passed on to consumers: if $c'<c$ then  $a- \frac{1}{2} (c+c')>a-c$ (positive effect)
			\item Market power: the factor $\frac{2}{3} < \frac{3}{4}$ decreases due to number of firms $3 \rightarrow 2$ (negative effect)
		\end{wideenumerate}
	\end{wideitemize}
\end{frame}

\begin{frame}{Plan}
	\begin{wideenumerate}
		\item Why do firms merge?
		\item Determining the effects of mergers.
				\item \textbf{Horizontal merger policy}
	\end{wideenumerate}
\end{frame}



\begin{frame}{Horizontal merger policy}
	\begin{wideitemize}
		\item Three interested parties in a horizontal merger: merging firms, non-merging firms, and consumers.
		\item Task for public policy: evaluate the relative importance of each gain/loss, and to assess the overall effect.
		\item This is very challenging! E.g. information about cost savings comes from the firms themselves, usually, and they have a clear incentive to overstate the benefits.
	\end{wideitemize}
\end{frame}

\begin{frame}{Horizontal merger policy}
	\begin{wideitemize}
		\item Also important to merger analysis: what is the increase in price following a merger?
		\item From our previous discussion about market structure, equilibrium price is increasing in market concentration.
		\begin{wideitemize}
			\item Two large firms merging implies a greater increase in price than two small firms.
		\end{wideitemize}
		\item Price increase channels:
		\begin{wideitemize}
			\item \textbf{Unilateral effects} (less competition) 
			\item \textbf{Collusion effects} (easier to collude with fewer firms)
		\end{wideitemize}
	\end{wideitemize}
\end{frame}

\begin{frame}{Horizontal merger policy: practical aspects - summary}{Table from 2023 Horizontal Merger Guidelines}
	\begin{figure}
		\centering
		\vspace{-100pt}
		\includegraphics[angle=270, scale=0.6]{guidelines_2023.pdf}
		\caption{From: 2023 Horizontal Merger Guidelines}
	\end{figure}
\end{frame}

\begin{frame}{Horizontal merger policy: practical aspects}
	\begin{wideitemize}
		\item \textbf{What is the relevant `market'?}
		\item To compute the HHI, need to define market shares. What is the denominator in the `share'?
		\item Definition of market is an obvious way for firms to skirt merger enforcement: try to define the market as large as possible.
		\item Example: 1996 Staples and Office Depot (two largest US chains of office supplies superstores) proposed a merger.
		\begin{wideitemize}
			\item If market is `office superstores': combined market share of merging parties is $>70\%$.
			\item If market is `stores that sell office supplies': combined market share is much lower.
		\end{wideitemize}
		\item To get around these debates about market definition, recently FTC	has favored a more direct approach of estimating the impact of a merger on consumer prices.
	\end{wideitemize}
\end{frame}

\begin{frame}{Horizontal merger policy: merger remedies}
	\begin{wideitemize}
		\item In the US, mergers are challenged in court. 
		\begin{wideitemize}
			\item That is, regulators do not block them directly. In the EU, the European Commission blocks mergers directly and this can then be appealed in court.
		\end{wideitemize}
		\item Possible outcomes:
		\item \textbf{Behavioral remedies}: e.g. prices cannot be increased by x\% during the next $n$ years
		\item \textbf{Structural remedies}: e.g. sell assets to competitor
		\item Merger might be blocked
		\item Merger might be allowed to go ahead
	\end{wideitemize}
\end{frame}

\begin{frame}{Summary of key points*}
	%\vspace{-20pt}
	\begin{wideitemize}
		\item Mergers usually involve a public policy tradeoff: lower costs vs increased market power
		\item Know the formulas for profit and consumer surplus and how to use them to compute the effect of the merger on:
		\begin{wideitemize}
			\item The merging firms
			\item The non-merging firms
			\item Consumers
		\end{wideitemize}
		\item Know how the above effects increase or decrease as a result of a merger, and depend on the parameters of the problem.
				\item Know about two practical aspects of merger policy: 1. how regulators target enforcement based on concentration and 2. market definition
	\end{wideitemize}
	\vspace{20pt}
	*To clarify, all the material in the slides, problem sets, etc is assessable unless stated otherwise, but I hope this summary might be a useful place to start when studying the material.
\end{frame}

\begin{comment}
\begin{frame}{Practice question}
	\begin{wideitemize}
		\item \textbf{Setup:} Consider three identical firms computing under Cournot competition.
		\begin{wideitemize}
			\item Costs $C=2+3q$
			\item Demand $Q=20-p$
		\end{wideitemize}
		\item \textbf{Question:} Suppose that Firm 1 and Firm 2 decide to merge. If they merge then the costs of the merged firm are $C=3+2q$. 
		\begin{wideitemize}
			\item 1. What are the effects on the profits of the merging firms?
			\item 2. What are the effects on the profits of the non-merging firms?
			\item 3. What are the effects on consumers?
			\item 4. Overall, should the Department of Justice approve this merger?
		\end{wideitemize}
	\end{wideitemize}
\end{frame}
\end{comment}

\end{document}
