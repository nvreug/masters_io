\documentclass[notes,11pt, aspectratio=169]{beamer}

\usepackage{pgfpages}
% These slides also contain speaker notes. You can print just the slides,
% just the notes, or both, depending on the setting below. Comment out the want
% you want.
\setbeameroption{hide notes} % Only slide
%\setbeameroption{show only notes} % Only notes
%\setbeameroption{show notes on second screen=right} % Both

%\usepackage[scaled=1.0]{helvet}
\usepackage{array}

\usepackage{graphicx}
\usepackage{tikz}
\usetikzlibrary{calc}
\usetikzlibrary{matrix}
\usetikzlibrary{positioning}

\newcommand{\payoff}[4][below]{\node[#1]at(#2){$(#3,#4)$};}
\usepackage{verbatim}
\setbeamertemplate{note page}{\pagecolor{gray!5}\insertnote}
\usetikzlibrary{positioning}
\usetikzlibrary{snakes}
\usetikzlibrary{calc}
\usetikzlibrary{arrows}
\usetikzlibrary{decorations.markings}
\usetikzlibrary{shapes.misc}
\usetikzlibrary{matrix,shapes,arrows,fit,tikzmark}
\usepackage{amsmath}
\usepackage{mathpazo}
\usepackage{hyperref}
\usepackage{lipsum}
\usepackage{multimedia}
\usepackage{graphicx}
\usepackage{multirow}
\usepackage{graphicx}
\usepackage{dcolumn}
\usepackage{bbm}
\newcolumntype{d}[0]{D{.}{.}{5}}

\usepackage{changepage}
\usepackage{appendixnumberbeamer}
\newcommand{\beginbackup}{
   \newcounter{framenumbervorappendix}
   \setcounter{framenumbervorappendix}{\value{framenumber}}
   \setbeamertemplate{footline}
   {
     \leavevmode%
     \hline
     box{%
       \begin{beamercolorbox}[wd=\paperwidth,ht=2.25ex,dp=1ex,right]{footlinecolor}%
%         \insertframenumber  \hspace*{2ex} 
       \end{beamercolorbox}}%
     \vskip0pt%
   }
 }
\newcommand{\backupend}{
   \addtocounter{framenumbervorappendix}{-\value{framenumber}}
   \addtocounter{framenumber}{\value{framenumbervorappendix}} 
}


\usepackage{graphicx}
\usepackage[space]{grffile}
\usepackage{booktabs}

% These are my colors -- there are many like them, but these ones are mine.
\definecolor{blue}{RGB}{0,114,178}
\definecolor{red}{RGB}{213,94,0}
\definecolor{yellow}{RGB}{240,228,66}
\definecolor{green}{RGB}{0,158,115}

\hypersetup{
  colorlinks=false,
  linkbordercolor = {white},
  linkcolor = {blue}
}

\usepackage{graphicx,stackengine,xcolor}
\newcommand\Circle[1]{%
	\def\useanchorwidth{T}%
	\def\stacktype{L}%
	\stackon[0pt]{#1}{\scalebox{2.0}[1.15]{\textcolor{red}{$\bigcirc$}}}%
}

%% I use a beige off white for my background
\definecolor{MyBackground}{RGB}{255,253,218}

%% Uncomment this if you want to change the background color to something else
%\setbeamercolor{background canvas}{bg=MyBackground}

%% Change the bg color to adjust your transition slide background color!
\newenvironment{transitionframe}{
  \setbeamercolor{background canvas}{bg=white}
  \begin{frame}}{
    \end{frame}
}

\setbeamercolor{frametitle}{fg=blue}
\setbeamercolor{title}{fg=black}
\setbeamertemplate{footline}[frame number]
\setbeamertemplate{navigation symbols}{} 
\setbeamertemplate{itemize items}{-}
\setbeamercolor{itemize item}{fg=blue}
\setbeamercolor{itemize subitem}{fg=blue}
\setbeamercolor{enumerate item}{fg=blue}
\setbeamercolor{enumerate subitem}{fg=blue}
\setbeamercolor{button}{bg=MyBackground,fg=blue,}

%%% TIKZ STUFF
\tikzset{   
	every picture/.style={remember picture,baseline},
	every node/.style={anchor=base,align=center,outer sep=1.5pt},
	every path/.style={thick},
}
\newcommand\marktopleft[1]{%
	\tikz[overlay,remember picture] 
	\node (marker-#1-a) at (-.3em,.3em) {};%
}
\newcommand\markbottomright[2]{%
	\tikz[overlay,remember picture] 
	\node (marker-#1-b) at (0em,0em) {};%
}
\tikzstyle{every picture}+=[remember picture] 
\tikzstyle{mybox} =[draw=black, very thick, rectangle, inner sep=10pt, inner ysep=20pt]
\tikzstyle{fancytitle} =[draw=black,fill=red, text=white]
%%%% END TIKZ STUFF


% If you like road maps, rather than having clutter at the top, have a roadmap show up at the end of each section 
% (and after your introduction)
% Uncomment this is if you want the roadmap!
% \AtBeginSection[]
% {
%    \begin{frame}
%        \frametitle{Roadmap of Talk}
%        \tableofcontents[currentsection]
%    \end{frame}
% }
\setbeamercolor{section in toc}{fg=blue}
\setbeamercolor{subsection in toc}{fg=red}
\setbeamersize{text margin left=1em,text margin right=1em} 

\newenvironment{wideitemize}{\itemize\addtolength{\itemsep}{10pt}}{\enditemize}
\newenvironment{wideenumerate}{\enumerate\addtolength{\itemsep}{10pt}}{\endenumerate}

\usepackage{environ}
\NewEnviron{videoframe}[1]{
  \begin{frame}
    \vspace{-8pt}
    \begin{columns}[onlytextwidth, T] % align columns
      \begin{column}{.58\textwidth}
        \begin{minipage}[t][\textheight][t]
          {\dimexpr\textwidth}
          \vspace{8pt}
          \hspace{4pt} {\Large \sc \textcolor{blue}{#1}}
          \vspace{8pt}
          
          \BODY
        \end{minipage}
      \end{column}%
      \hfill%
      \begin{column}{.42\textwidth}
        \colorbox{green!20}{\begin{minipage}[t][1.2\textheight][t]
            {\dimexpr\textwidth}
            Face goes here
          \end{minipage}}
      \end{column}%
    \end{columns}
  \end{frame}
}

\title[]{\textcolor{blue}{ECN 453: Market Structure}}
\author[PGP]{}
\institute[FRBNY]{\small{\begin{tabular}{c c c}
Nicholas Vreugdenhil \\
\end{tabular}}}
\date{} 

\begin{document}

% Title Slide
\begin{frame}
\maketitle
  \centering
\end{frame}

% INTRO

\begin{frame}{Market Structure}
\begin{wideitemize}
	\item So far we have studied markets and competition \textit{given} the number of firms.
	\item Taking a step back:
	\item How many firms would you expect to find in a given market?
	\item How large would you expect these firms to be?
\end{wideitemize}
\end{frame}

\begin{frame}{Market share of the four largest firms}
	\includegraphics[scale=0.16]{concentration.jpg}
	\pause
\begin{wideitemize}
	\item Suggests industry-specific factors determine each firm's size
\end{wideitemize}
\end{frame}

\begin{frame}{Plan}
	\begin{wideenumerate}
		\item Measuring market concentration and market power
		\item Entry costs and market structure
		\item Model assumptions vs reality
	\end{wideenumerate}
\end{frame}

\begin{frame}{Plan}
	\begin{wideenumerate}
		\item \textbf{Measuring market concentration and market power}
		\item Entry costs and market structure
		\item Model assumptions vs reality
	\end{wideenumerate}
\end{frame}

\begin{frame}{Measuring market concentration and market power}
	\begin{wideitemize}
		\item \textbf{Market concentration}: the extent to which a small number of firms account for a large percentage of the market
		\item \textbf{Market power}: ability of firms to raise price above marginal cost
	\end{wideitemize}
\end{frame}

\begin{frame}{Measuring market concentration and market power: different methods}
	\begin{wideitemize}
		\item Count the number of firms (why is this not a good measure?)
		\item Largest four firms:
		\begin{align*}
			C_4 = \sum_{i=1}^4 s_i = s_1 + s_2 + s_3 + s_4
		\end{align*}
		\item Herfindahl index:
		\begin{align*}
			HHI = 10000 \sum_{i=1}^n s^2_i = 10000 * (s_1^2 + ... + s^2_n)
		\end{align*}
		\item Lerner index:
		\begin{align*}
			L = \sum_{i=1}^n s_i \frac{p-MC_i}{p}
		\end{align*}
		\item Note: if all firms have same MC then this is just the (common) margin set by all firms
	\end{wideitemize}
\end{frame}

\begin{frame}{Measuring market concentration and market power: different methods}
	\begin{wideitemize}
		\item Suppose that there are three firms. Firm 1 and Firm 2 each produce $q=100$. Firm 3 produces $q=200$. The market price is $p=5$. Firm 1 and Firm 2 have a marginal cost of 4, and Firm 3 has a marginal cost of 3.
		\item \textbf{Question:}
		\item What is the largest four firm concentration ratio?
		\item What is the HHI?
		\item What is the Lerner index?
	\end{wideitemize}
\end{frame}

\begin{frame}{Measuring market concentration and market power: different methods}
	\begin{wideitemize}
		\item Suppose that there are three firms. Firm 1 and Firm 2 each produce $q=100$. Firm 3 produces $q=200$. The market price is $p=5$. Firm 1 and Firm 2 have a marginal cost of 4, and Firm 3 has a marginal cost of 3.
		\item \textbf{Question:}
		\item What is the largest four firm concentration ratio? 1.0
		\item What is the HHI? 3750
		\item What is the Lerner index? 0.3
	\end{wideitemize}
\end{frame}

\begin{frame}{Measuring market concentration and market power: different methods}
	\begin{wideitemize}
		\item Suppose that there is one firm. The market price is $p=5$ and the firm has a marginal cost of 3.
		\item \textbf{Question:}
		\item What is the largest four firm concentration ratio? 1.0
		\item What is the HHI? 10000
		\item What is the Lerner index? 0.4
	\end{wideitemize}
\end{frame}

\begin{frame}{Plan}
	\begin{wideenumerate}
		\item Measuring market concentration and market power
		\item \textbf{Entry costs and market structure}
		\item Model assumptions vs reality
	\end{wideenumerate}
\end{frame}

\begin{frame}{Entry costs and market structure}
	\begin{wideitemize}
		\item What is the relationship between technology, market size, and industry concentration?
		\item Start by considering a model with $n$ identical firms. Assume cournot competition.
		\item Cost function: $C=F+c q_i$
		\item Demand: $Q=(a-P)S$ where $S$ is a number that refers to `market size'.
		\item \textbf{Question:}
		\item 1. What is the profit of each firm? $\Pi(n)$
		\item 2. What is the free entry equilibrium? (I.e. what is the number of active firms so that (i) no active firm wishes to leave the market (ii) no inactive firm wants to enter the market)
	\end{wideitemize}
\end{frame}

\begin{frame}{Entry costs and market structure}
	\begin{wideitemize}
		\item 1. What is the profit of each firm? $\Pi(n)$
		\item From n-firm cournot example in a previous lecture, we found that for $n$ firms and demand $P=a-bQ$: 
		\begin{align*}
			q &= \frac{a-c}{(n+1)b} \\
			p &= \frac{a+nc}{n+1}
		\end{align*}
		\item Note that if $Q=(a-P)S$ then $b =1/S$. Therefore:
		\begin{align*}
			\Pi(n) = S \bigg(\frac{a-c}{n+1} \bigg)^2 - F
		\end{align*}
	\end{wideitemize}
\end{frame}

\begin{frame}{Entry costs and market structure}
	\begin{wideitemize}
		\item 2. What is the free entry equilibrium? (I.e. what is the number of active firms so that (i) no active firm wishes to leave the market (ii) no inactive firm wants to enter the market)
		\item Find $\hat{n}$ where $\Pi(\hat{n}) \geq 0 \geq \Pi (\hat{n} + 1)$
		\item Set $\Pi(n) = 0$ and rearrange:
		\begin{align*}
			n = (a-c) \sqrt{\frac{S}{F}} - 1
		\end{align*}
		\item So:
		\begin{align*}
			\hat{n} = \bigg[ (a-c) \sqrt{\frac{S}{F}} - 1 \bigg]
		\end{align*}
		\item Where $[x]$ denotes highest integer lower than $x$. E.g. if $n=32.4$ than $\hat{n}=[32.4]=32$.
	\end{wideitemize}
\end{frame}

\begin{frame}{Entry costs and market structure: observations}
	\begin{align*}
		\hat{n} = \bigg[ (a-c) \sqrt{\frac{S}{F}} - 1 \bigg]
	\end{align*}
	\begin{wideitemize}
		\item Equilibrium number of firms:
		\begin{wideitemize}
			\item An increasing function of market size $S$ (and $a$)
			\item Decreasing in $F$ and $c$
			\item Relation between $S$ and $n$ is not proportional: actually in order to 2x the number of firms, market size $S$ must increase around 4x
			\item Why? \pause Price changes as firms enter: High S $\rightarrow$ higher $n$ $\rightarrow$ lower $p-c$, which limits the increase in $n$
		\end{wideitemize}
	\item \textbf{Due to increased price competition, the equilibrium number of active firms varies less than proportionally with respect to market size.}
	\end{wideitemize}
\end{frame}

\begin{frame}{Entry costs and market structure: observations}
	\begin{figure}
		\centering
		\includegraphics[scale=0.27]{scale.jpeg}
	\end{figure}
\end{frame}

\begin{frame}{Minimum efficient scale and concentration}
	\begin{wideitemize}
		\item The cost structure of firms is a key determinant of market structure.
		\item The particular cost structure we have used in this class $C(q) = F + c q$ has \textbf{increasing returns to scale}. (increasing returns to scale means that the average cost is decreasing.)
		\item A common way of measuring the degree of returns to scale in an industry is through \textbf{minimum efficient scale}. 
		\begin{wideitemize}
			\item This is defined as when a firms average cost is `close' to the minimum average cost (c).
		\end{wideitemize}
		\item $AC=F/q+c$. Let MES be the scale where average cost is equal to $c'$. Solving for $q$:
		\begin{align*}
			MES = \frac{F}{c-c'}
		\end{align*}
	\end{wideitemize}
\end{frame}

\begin{frame}{Minimum efficient scale and concentration}
	\begin{align*}
		MES = \frac{F}{c-c'}
	\end{align*}
	\begin{wideitemize}
		\item Interpret changes in MES as coming from changes in F.
		\item If F doubles then MES doubles.
		\item If MES doubles (due to F doubling) then the number of firms decreases by around $\sqrt{2} < 2$
		\begin{wideitemize}
			\item Intuition? \pause for why it decreases less than 2 is due to the price effects previously discussed when talking about changes in $S$.
		\end{wideitemize}
		\item \textbf{Concentration is generally greater the greater the minimum efficient scale.}
	\end{wideitemize}
\end{frame}


\begin{frame}{Plan}
	\begin{wideenumerate}
		\item Measuring market concentration and market power
		\item Entry costs and market structure
		\item \textbf{Model assumptions vs reality}
	\end{wideenumerate}
\end{frame}

\begin{frame}{Model assumptions vs reality}
	\begin{wideitemize}
		\item Model before made a few implicit assumptions:
		\item 1. All firms have access to the same technology (corresponding to cost $C=F+cq$)
		\item 2. Firms have perfect information about the demand (i.e. they know demand)
		\item 3. Entry process is well-coordinated 
		\begin{wideitemize}
			\item Firms are choosing their entry decision sequentially knowing the previous decisions entrants have made.
		\end{wideitemize}
	\end{wideitemize}
\end{frame}

\begin{frame}{Model assumptions vs reality: history matters}
	\begin{wideitemize}
		\item If these assumptions hold then for a given set of parameter values (a, c, F, etc) then we can predict exactly the number of firms in the market, and all these firms should be the same size.
		\item Do these predictions hold in reality? Usually not...
		\item Example: prepared soups industry in the US and the UK.
		\begin{wideitemize}
			\item Markets differ in size but similar in most other dimensions (e.g. canned vs dried)
			\item Campbell was the first entrant in the US; Heinz established an early lead in the UK
			\item Despite attempts to expand market share, Heinz still dominates UK (41\% share) and Campbell dominates US market (63\%).
		\end{wideitemize}
		\item Example: industries include firms of different sizes
		\begin{wideitemize}
			\item e.g. US car market	
		\end{wideitemize}
	\end{wideitemize}
\end{frame}

\begin{frame}{Model assumptions vs reality: history matters}
	\begin{wideitemize}
		\item How can we relax these assumptions?
		\item Perhaps firms don't have access to the same technology. \pause
		\begin{wideitemize}
			\item Example: Dupont
			\item For a period of time, Dupont had exclusive rights over a new production process for titanium dioxide
			\item Even after the patents expired, Dupont maintained a cost advantage due to the fact that it had moved down the \textit{learning curve}.
		\end{wideitemize}
	\end{wideitemize}
\end{frame}

\begin{frame}{Model assumptions vs reality: history matters}
	\begin{wideitemize}
		\item How can we relax these assumptions?
		\item Imperfect information about market conditions \pause
		\item \textbf{Forecasting mistakes}: 
		\begin{wideitemize}
			\item several oil companies built large refineries in early 1970s
			\item 1973: oil shock, demand cuts, excess capacity, n too large
		\end{wideitemize}
	\end{wideitemize}
\end{frame}

\begin{frame}{Model assumptions vs reality: history matters}
	\begin{wideitemize}
		\item How can we relax these assumptions?
		\item Imperfect information about market conditions
		\item \textbf{Coordination mistakes}:
		\begin{wideitemize}
			\item Commercial aircraft in 1960s. Lockheed and McDonnell Douglas were considering whether to enter the market for large commercial aircraft.
			\item Boeing had entered with B747 and there was room for only one more firm $\hat{n}=2$. Both ended up entering the market and made huge losses.
			\item A different example in the same industry: Boeing and Airbus in the `super-jumbo' segment.
			\item In 1990, firms agreed there wasn't room for two firms. Initially thought about a joint design, but then Airbus decided to go it alone.  \item Both firms delayed their entry decision until late 2000, A380 developed a decade later due to the coordination `mistake'
		\end{wideitemize}
	\end{wideitemize}
\end{frame}

\begin{comment}
\begin{frame}{Model assumptions vs reality: history matters}
	\begin{wideitemize}
		\item How can we relax these assumptions?
		\item \textbf{Agglomeration externalities}: why are so many high-tech firms in the bay area?
		\begin{wideitemize}
			\item Snowball effect: HP and some other firms initially, who attracted new ones and so forth		
		\end{wideitemize}
		\item \textbf{Overall: The particular historical details of the evolution of an industry may in some cases determine the long-run market structure in ways that go beyond simple determinants}
	\end{wideitemize}
\end{frame}

\begin{frame}{Evolution of US Beer Industry}
	\begin{figure}
		\includegraphics[scale=0.23]{beer.jpg}
	\end{figure}
		\begin{wideitemize}
	\item Exogenous factors changing over time can change industry structure: 
	\item E.g. TV created a national market, interstate highway system decreased transport costs, created ``shipping brewers'', better bottling tech, consumer demand for variety increased
\end{wideitemize}
\end{frame}

\begin{frame}{Evolution of new industries}
	\begin{figure}
		\includegraphics[scale=0.15]{hdd.jpg}
	\end{figure}
	\begin{wideitemize}
		\item Observed regularities: fast growth followed by a period of consolidation (sometimes, a `shakeout' where many firms exit by merger or bankruptcy). 
	\end{wideitemize}
\end{frame}

\begin{frame}{Summary of key points*}
	%\vspace{-20pt}
	\begin{wideitemize}
		\item Know different measures of concentration and how to compute them, particularly HHI
		\item Know the model of entry costs and market structure
		\item Understand the comparative statics of the model of entry costs and market structure, and minimum efficient scale
		\item Understand the assumptions behind the model and that sometimes they may need to be relaxed to fit the particular historical details of the industry.
	\end{wideitemize}
	\vspace{20pt}
	*To clarify, all the material in the slides, problem sets, etc is assessable unless stated otherwise, but I hope this summary might be a useful place to start when studying the material.
\end{frame}

\begin{frame}{Question: 10.12}
	%\vspace{-20pt}
	\begin{wideitemize}
		\item Consider an industry with market demand $Q=a-p$ and an infinite number of potential entrants with access to the same technology. Initially, the technology is given by $C=F+cq$. A new technology allows for a lower marginal cost $c'<c$ at the expense of a higher fixed cost $F'>F$.
		\item (a) What can you say about the effect of the new technology on equilibrium price?
		\item (b) Suppose that $a=10, F=2, F'=3,c=2,c'=1, S=1$. Determine the equilibrium price under each of the new technologies.
	\end{wideitemize}
\end{frame}
\end{comment}

\end{document}
