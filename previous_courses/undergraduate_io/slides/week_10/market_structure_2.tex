\documentclass[notes,11pt, aspectratio=169]{beamer}

\usepackage{pgfpages}
% These slides also contain speaker notes. You can print just the slides,
% just the notes, or both, depending on the setting below. Comment out the want
% you want.
\setbeameroption{hide notes} % Only slide
%\setbeameroption{show only notes} % Only notes
%\setbeameroption{show notes on second screen=right} % Both

%\usepackage[scaled=1.0]{helvet}
\usepackage{array}

\usepackage{graphicx}
\usepackage{tikz}
\usetikzlibrary{calc}
\usetikzlibrary{matrix}
\usetikzlibrary{positioning}

\newcommand{\payoff}[4][below]{\node[#1]at(#2){$(#3,#4)$};}
\usepackage{verbatim}
\setbeamertemplate{note page}{\pagecolor{gray!5}\insertnote}
\usetikzlibrary{positioning}
\usetikzlibrary{snakes}
\usetikzlibrary{calc}
\usetikzlibrary{arrows}
\usetikzlibrary{decorations.markings}
\usetikzlibrary{shapes.misc}
\usetikzlibrary{matrix,shapes,arrows,fit,tikzmark}
\usepackage{amsmath}
\usepackage{mathpazo}
\usepackage{hyperref}
\usepackage{lipsum}
\usepackage{multimedia}
\usepackage{graphicx}
\usepackage{multirow}
\usepackage{graphicx}
\usepackage{dcolumn}
\usepackage{bbm}
\newcolumntype{d}[0]{D{.}{.}{5}}

\usepackage{changepage}
\usepackage{appendixnumberbeamer}
\newcommand{\beginbackup}{
   \newcounter{framenumbervorappendix}
   \setcounter{framenumbervorappendix}{\value{framenumber}}
   \setbeamertemplate{footline}
   {
     \leavevmode%
     \hline
     box{%
       \begin{beamercolorbox}[wd=\paperwidth,ht=2.25ex,dp=1ex,right]{footlinecolor}%
%         \insertframenumber  \hspace*{2ex} 
       \end{beamercolorbox}}%
     \vskip0pt%
   }
 }
\newcommand{\backupend}{
   \addtocounter{framenumbervorappendix}{-\value{framenumber}}
   \addtocounter{framenumber}{\value{framenumbervorappendix}} 
}


\usepackage{graphicx}
\usepackage[space]{grffile}
\usepackage{booktabs}

% These are my colors -- there are many like them, but these ones are mine.
\definecolor{blue}{RGB}{0,114,178}
\definecolor{red}{RGB}{213,94,0}
\definecolor{yellow}{RGB}{240,228,66}
\definecolor{green}{RGB}{0,158,115}

\hypersetup{
  colorlinks=false,
  linkbordercolor = {white},
  linkcolor = {blue}
}

\usepackage{graphicx,stackengine,xcolor}
\newcommand\Circle[1]{%
	\def\useanchorwidth{T}%
	\def\stacktype{L}%
	\stackon[0pt]{#1}{\scalebox{2.0}[1.15]{\textcolor{red}{$\bigcirc$}}}%
}

%% I use a beige off white for my background
\definecolor{MyBackground}{RGB}{255,253,218}

%% Uncomment this if you want to change the background color to something else
%\setbeamercolor{background canvas}{bg=MyBackground}

%% Change the bg color to adjust your transition slide background color!
\newenvironment{transitionframe}{
  \setbeamercolor{background canvas}{bg=white}
  \begin{frame}}{
    \end{frame}
}

\setbeamercolor{frametitle}{fg=blue}
\setbeamercolor{title}{fg=black}
\setbeamertemplate{footline}[frame number]
\setbeamertemplate{navigation symbols}{} 
\setbeamertemplate{itemize items}{-}
\setbeamercolor{itemize item}{fg=blue}
\setbeamercolor{itemize subitem}{fg=blue}
\setbeamercolor{enumerate item}{fg=blue}
\setbeamercolor{enumerate subitem}{fg=blue}
\setbeamercolor{button}{bg=MyBackground,fg=blue,}

%%% TIKZ STUFF
\tikzset{   
	every picture/.style={remember picture,baseline},
	every node/.style={anchor=base,align=center,outer sep=1.5pt},
	every path/.style={thick},
}
\newcommand\marktopleft[1]{%
	\tikz[overlay,remember picture] 
	\node (marker-#1-a) at (-.3em,.3em) {};%
}
\newcommand\markbottomright[2]{%
	\tikz[overlay,remember picture] 
	\node (marker-#1-b) at (0em,0em) {};%
}
\tikzstyle{every picture}+=[remember picture] 
\tikzstyle{mybox} =[draw=black, very thick, rectangle, inner sep=10pt, inner ysep=20pt]
\tikzstyle{fancytitle} =[draw=black,fill=red, text=white]
%%%% END TIKZ STUFF


% If you like road maps, rather than having clutter at the top, have a roadmap show up at the end of each section 
% (and after your introduction)
% Uncomment this is if you want the roadmap!
% \AtBeginSection[]
% {
%    \begin{frame}
%        \frametitle{Roadmap of Talk}
%        \tableofcontents[currentsection]
%    \end{frame}
% }
\setbeamercolor{section in toc}{fg=blue}
\setbeamercolor{subsection in toc}{fg=red}
\setbeamersize{text margin left=1em,text margin right=1em} 

\newenvironment{wideitemize}{\itemize\addtolength{\itemsep}{10pt}}{\enditemize}
\newenvironment{wideenumerate}{\enumerate\addtolength{\itemsep}{10pt}}{\endenumerate}

\usepackage{environ}
\NewEnviron{videoframe}[1]{
  \begin{frame}
    \vspace{-8pt}
    \begin{columns}[onlytextwidth, T] % align columns
      \begin{column}{.58\textwidth}
        \begin{minipage}[t][\textheight][t]
          {\dimexpr\textwidth}
          \vspace{8pt}
          \hspace{4pt} {\Large \sc \textcolor{blue}{#1}}
          \vspace{8pt}
          
          \BODY
        \end{minipage}
      \end{column}%
      \hfill%
      \begin{column}{.42\textwidth}
        \colorbox{green!20}{\begin{minipage}[t][1.2\textheight][t]
            {\dimexpr\textwidth}
            Face goes here
          \end{minipage}}
      \end{column}%
    \end{columns}
  \end{frame}
}

\title[]{\textcolor{blue}{ECN 453: Market Structure 2}}
\author[PGP]{}
\institute[FRBNY]{\small{\begin{tabular}{c c c}
Nicholas Vreugdenhil \\
\end{tabular}}}
\date{} 

\begin{document}

% Title Slide
\begin{frame}
\maketitle
  \centering
\end{frame}

% INTRO

\begin{frame}{Market Structure}
\begin{wideitemize}
	\item Today we will continue our discussion of the following questions:
	\item How many firms would you expect to find in a given market?
	\item How large would you expect these firms to be?
\end{wideitemize}
\end{frame}

\begin{frame}{Plan}
	\begin{wideenumerate}
		\item Model assumptions vs reality (continued from last time)
		\item Endogenous entry costs
		\item Market concentration determines market power
	\end{wideenumerate}
\end{frame}

\begin{frame}{Plan}
	\begin{wideenumerate}
		\item \textbf{Model assumptions vs reality (continued from last time)}
		\item Endogenous entry costs
		\item Market concentration determines market power
	\end{wideenumerate}
\end{frame}

\begin{frame}{Model assumptions vs reality: history matters}
	\begin{wideitemize}
		\item Last time: we wrote down a model of entry and competition that made predictions about how many firms would be in a given market given the parameters
		\item Today, continue discussion of the model assumptions...
	\end{wideitemize}
\end{frame}

\begin{frame}{Model assumptions vs reality: history matters}
	\begin{wideitemize}
		\item How can we relax these assumptions?
		\item \textbf{Agglomeration externalities}: why are so many high-tech firms in the bay area? \pause
		\begin{wideitemize}
			\item Snowball effect: HP and some other firms initially, who attracted new ones and so forth		
		\end{wideitemize}
		\item \textbf{Overall: The particular historical details of the evolution of an industry may in some cases determine the long-run market structure in ways that go beyond simple determinants}
	\end{wideitemize}
\end{frame}

\begin{frame}{Evolution of US Beer Industry}
	\begin{figure}
		\includegraphics[scale=0.23]{beer.jpg}
	\end{figure}
		\begin{wideitemize}
	\item Exogenous factors changing over time can change industry structure: 
	\item E.g. TV created a national market, interstate highway system decreased transport costs, created ``shipping brewers'', better bottling tech, consumer demand for variety increased
\end{wideitemize}
\end{frame}

\begin{frame}{Evolution of new industries}
	\begin{figure}
		\includegraphics[scale=0.15]{hdd.jpg}
	\end{figure}
	\begin{wideitemize}
		\item Observed regularities: fast growth followed by a period of consolidation (sometimes, a `shakeout' where many firms exit by merger or bankruptcy). 
	\end{wideitemize}
\end{frame}

\begin{frame}{Plan}
	\begin{wideenumerate}
		\item Model assumptions vs reality (continued from last time)
		\item \textbf{Endogenous entry costs}
		\item Market concentration determines market power
	\end{wideenumerate}
\end{frame}

\begin{frame}{Another example of a real-world market at odds with the previous model...}
	\begin{columns}
		\begin{column}{0.3\textwidth}
			\begin{figure}
				\includegraphics[scale=0.3]{beer.jpeg}
			\end{figure}
		\end{column}
		\begin{column}{0.7\textwidth}
		\begin{wideitemize}
			\item Beer industry in Portugal similar to the US
			\item 3 firms dominate US, 2 firms dominate Portugal
			\item Why is this observation at odds with the model from the last lecture?
			\item \pause A: US is 30-50x bigger than Portugal, model predicts number of firms in US should be $\sqrt{30}$ to $\sqrt{50}$ more than in Portugal!
		\end{wideitemize}
		\end{column}
	\end{columns}
\end{frame}

\begin{frame}{Another example of a real-world market at odds with the previous model...}
	\begin{columns}
		\begin{column}{0.3\textwidth}
			\begin{figure}
				\includegraphics[scale=0.3]{beer.jpeg}
			\end{figure}
		\end{column}
		\begin{column}{0.7\textwidth}
			\begin{wideitemize}
				\item Why is the number of firms in reality different to what the model would predict?
				\item \pause One aspect: \textbf{advertising} \pause
				\item Value of advertising as a \textit{percentage of sales} is similar between countries
				\item In order to enter the US industry and compete with Budweiser and Miller Lite, new entrant would need to pay a greater entry cost than in Portugal
				\item \textbf{Entry costs are endogenous} (endogenous - i.e. entry costs change - with respect to market size)
			\end{wideitemize}
		\end{column}
	\end{columns}
\end{frame}

\begin{frame}{Endogenous entry costs}
\begin{wideitemize}
	\item Idea with the previous model: due to price competition, if market scales by 2x then there will be room for \textit{less} than 2x as many firms.
	\item If entry costs increase with market size, this is an \textit{additional} reason whereby the number of firms does not increase as much as market size.
	\begin{wideitemize}
		\item Idea: Bigger market induces firms to make bigger investments
	\end{wideitemize}
\end{wideitemize}
\end{frame}

\begin{frame}{Endogenous entry costs: example}
	\begin{columns}
		\begin{column}{0.6\textwidth}
			\begin{wideitemize}
				\item Suppose a country wants to deregulate its telecommunications sector, government sells a single license to operate
				\item Revenues for this license are $S$
				\item To get the license the firm must already be established as a telecommunications company; doing so costs $F$
				\item We will consider two ways to allocate the license. 
			\end{wideitemize}
		\end{column}
		\begin{column}{0.4\textwidth}
			\includegraphics[scale=0.22]{telecom.jpeg}
		\end{column}
	\end{columns}
\end{frame}

\begin{frame}{Endogenous entry costs: example - allocation with lottery}
	\begin{wideitemize}
		\item $n$ potential firms, each gets license with probability $\frac{1}{n}$
		\item How many firms enter? Set $\pi = \frac{S}{n} - F = 0$
		\item Then: $\hat{n}= \left[ \frac{S}{F} \right]$
		\item Here, number of firms is proportional to the market size (no price competition!)
	\end{wideitemize}
\end{frame}


\begin{frame}{Endogenous entry costs: example - allocation with auction}
	\begin{wideitemize}
		\item Assume the `auction' is like Bertrand competition (but the highest price wins so each firm competes equilibrium bid
		 \textit{up} to S if $n>1$, bids 0 if $n=1$)
		\begin{wideitemize}
			\item (Auction: used in NZ, Australia, US, some countries in Europe and South America)
		\end{wideitemize}
		\item How many firms will enter (assuming $F>0$)? \pause Only enter if you are the only firm \textit{no matter what the market size is}!
		\begin{wideitemize}
			\item So: while the value of winning increases as S increases, the bids submitted by the other firms also increase, and profit is still = 0.
		\item Auction essentially creates an endogenous entry cost: if $B$ is the bid for the license then total `entry cost' is $F+B$, and B scales with market size $S$
		\end{wideitemize}
		\item \textbf{If entry costs are endogenous, then the number of firms is less sensitive to changes in market size}
	\end{wideitemize}
\end{frame}

\begin{frame}{Common sources of endogenous entry costs}
	\begin{wideitemize}
		\item Advertising
		\item Bidding for a government license/`escalation wars'
		\item R\&D expenditures (like patents for a medical drug which might spur a patent race)
	\end{wideitemize}
\end{frame}

\begin{frame}{Endogenous entry costs: empirical evidence}
	\begin{wideitemize}
		\item If the theory of endogenous entry costs is correct, what would we expect to see in the data?
		\item More precise question: What relationship do we expect to see between industry size and concentration in \textit{more} vs \textit{less} advertising-intensive industries.
		\item One test: relationship between industry size and concentration is \textit{greater} in \textit{less} advertising-intensive industries.
	\end{wideitemize}
\end{frame}

\begin{frame}{Endogenous entry costs: empirical evidence}
	\begin{figure}
		\includegraphics[scale=0.12]{advertising.jpg}
		\caption{Advertising/retail sale ratios for various industries across 5 countries}
	\end{figure}
\end{frame}

\begin{frame}{Endogenous entry costs: empirical evidence}
	\begin{figure}
		\includegraphics[scale=0.15]{size_concentration.jpeg}
		\caption{Industry size and concentration; left: `low advertising industries', right: `high advertising industries'. Use $ln(size/MES)$ on horizontal axis as a proxy for $S/F$, motivated by our previous model that related the number of firms (i.e. concentration) to $S/F$.}
	\end{figure}
\end{frame}

\begin{frame}{Plan}
	\begin{wideenumerate}
		\item Model assumptions vs reality (continued from last time)
		\item Endogenous entry costs
		\item  \textbf{Market concentration determines market power}
	\end{wideenumerate}
\end{frame}


\begin{frame}{Market concentration determines market power}
	\begin{wideitemize}
		\item It can be shown that, under Cournot competition with identical firms:
		\begin{align*}
			L = \frac{H}{-\epsilon}
		\end{align*}
		\item $L = \sum_{i=1}^n s_i \frac{p-MC_i}{p}$
		\begin{wideitemize}
		\item `Market power' measure
	\end{wideitemize}
		\item $H = \sum_{i=1}^n s^2_i $
		\begin{wideitemize}
			\item `Market concentration' measure
			\item Note: don't multiply by 10000 for this formula
		\end{wideitemize}
		\item $\epsilon$: demand elasticity
		\item Note: If $n=1$ this is our previous optimal pricing formula for a monopolist!
	\end{wideitemize}
\end{frame}

\begin{frame}{Market concentration determines market power: example}
	\begin{wideitemize}
		\item \textbf{Setup:} Two markets, same demand elasticity
		\item Market 1: 2 firms with same market share
		\item Market 2: 1 firm with 70\% share, two small firms with 15\% share each
		\item \textbf{Question:} Under Cournot competition, where is market power the greatest?
		\item \pause A: find market with greatest H:
		\item Market 1: $H=5000$
		\item Market 2: $H=5350$
		\item Market 2 has the greatest market power
	\end{wideitemize}
\end{frame}

\begin{frame}{Summary of key points*}
	%\vspace{-20pt}
	\begin{wideitemize}
		\item Know different measures of concentration and how to compute them, particularly HHI
		\item Know the model of entry costs and market structure
		\item Understand the comparative statics of the model of entry costs and market structure, and minimum efficient scale
		\item Understand the assumptions behind the model and that sometimes they may need to be relaxed to fit the particular historical details of the industry.
		\item Know why endogenous entry costs are useful for explaining some common empirical patterns in the data.
		\item Know the formula: $L = \frac{H}{-\epsilon}$ and how to use it.
	\end{wideitemize}
	\vspace{20pt}
	*To clarify, all the material in the slides, problem sets, etc is assessable unless stated otherwise, but I hope this summary might be a useful place to start when studying the material.
\end{frame}

\begin{comment}
\begin{frame}{Question: 10.12}
	%\vspace{-20pt}
	\begin{wideitemize}
		\item Consider an industry with market demand $Q=a-p$ and an infinite number of potential entrants with access to the same technology. Initially, the technology is given by $C=F+cq$. A new technology allows for a lower marginal cost $c'<c$ at the expense of a higher fixed cost $F'>F$.
		\item (a) What can you say about the effect of the new technology on equilibrium price?
		\item (b) Suppose that $a=10, F=2, F'=3,c=2,c'=1, S=1$. Determine the equilibrium price under each of the new technologies.
	\end{wideitemize}
\end{frame}
\end{comment}

\end{document}
