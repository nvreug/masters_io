\documentclass[notes,11pt, aspectratio=169]{beamer}

\usepackage{pgfpages}
% These slides also contain speaker notes. You can print just the slides,
% just the notes, or both, depending on the setting below. Comment out the want
% you want.
\setbeameroption{hide notes} % Only slide
%\setbeameroption{show only notes} % Only notes
%\setbeameroption{show notes on second screen=right} % Both

%\usepackage[scaled=1.0]{helvet}
\usepackage{array}


\usepackage{tikz}
\usetikzlibrary{calc}
\usetikzlibrary{matrix}
\usetikzlibrary{positioning}

\newcommand{\payoff}[4][below]{\node[#1]at(#2){$(#3,#4)$};}
\usepackage{verbatim}
\setbeamertemplate{note page}{\pagecolor{gray!5}\insertnote}
\usetikzlibrary{positioning}
\usetikzlibrary{snakes}
\usetikzlibrary{calc}
\usetikzlibrary{arrows}
\usetikzlibrary{decorations.markings}
\usetikzlibrary{shapes.misc}
\usetikzlibrary{matrix,shapes,arrows,fit,tikzmark}
\usepackage{amsmath}
\usepackage{mathpazo}
\usepackage{hyperref}
\usepackage{lipsum}
\usepackage{multimedia}
\usepackage{graphicx}
\usepackage{multirow}
\usepackage{graphicx}
\usepackage{dcolumn}
\usepackage{bbm}
\newcolumntype{d}[0]{D{.}{.}{5}}

\usepackage{changepage}
\usepackage{appendixnumberbeamer}
\newcommand{\beginbackup}{
	\newcounter{framenumbervorappendix}
	\setcounter{framenumbervorappendix}{\value{framenumber}}
	\setbeamertemplate{footline}
	{
		\leavevmode%
		\hline
		box{%
			\begin{beamercolorbox}[wd=\paperwidth,ht=2.25ex,dp=1ex,right]{footlinecolor}%
				%         \insertframenumber  \hspace*{2ex} 
		\end{beamercolorbox}}%
		\vskip0pt%
	}
}
\newcommand{\backupend}{
	\addtocounter{framenumbervorappendix}{-\value{framenumber}}
	\addtocounter{framenumber}{\value{framenumbervorappendix}} 
}


\usepackage{graphicx}
\usepackage[space]{grffile}
\usepackage{booktabs}

% These are my colors -- there are many like them, but these ones are mine.
\definecolor{blue}{RGB}{0,114,178}
\definecolor{red}{RGB}{213,94,0}
\definecolor{yellow}{RGB}{240,228,66}
\definecolor{green}{RGB}{0,158,115}

\hypersetup{
	colorlinks=false,
	linkbordercolor = {white},
	linkcolor = {blue}
}

\usepackage{graphicx,stackengine,xcolor}
\newcommand\Circle[1]{%
	\def\useanchorwidth{T}%
	\def\stacktype{L}%
	\stackon[0pt]{#1}{\scalebox{2.0}[1.15]{\textcolor{red}{$\bigcirc$}}}%
}

%% I use a beige off white for my background
\definecolor{MyBackground}{RGB}{255,253,218}

%% Uncomment this if you want to change the background color to something else
%\setbeamercolor{background canvas}{bg=MyBackground}

%% Change the bg color to adjust your transition slide background color!
\newenvironment{transitionframe}{
	\setbeamercolor{background canvas}{bg=white}
	\begin{frame}}{
	\end{frame}
}

\setbeamercolor{frametitle}{fg=blue}
\setbeamercolor{title}{fg=black}
\setbeamertemplate{footline}[frame number]
\setbeamertemplate{navigation symbols}{} 
\setbeamertemplate{itemize items}{-}
\setbeamercolor{itemize item}{fg=blue}
\setbeamercolor{itemize subitem}{fg=blue}
\setbeamercolor{enumerate item}{fg=blue}
\setbeamercolor{enumerate subitem}{fg=blue}
\setbeamercolor{button}{bg=MyBackground,fg=blue,}

%%% TIKZ STUFF
\tikzset{   
	every picture/.style={remember picture,baseline},
	every node/.style={anchor=base,align=center,outer sep=1.5pt},
	every path/.style={thick},
}
\newcommand\marktopleft[1]{%
	\tikz[overlay,remember picture] 
	\node (marker-#1-a) at (-.3em,.3em) {};%
}
\newcommand\markbottomright[2]{%
	\tikz[overlay,remember picture] 
	\node (marker-#1-b) at (0em,0em) {};%
}
\tikzstyle{every picture}+=[remember picture] 
\tikzstyle{mybox} =[draw=black, very thick, rectangle, inner sep=10pt, inner ysep=20pt]
\tikzstyle{fancytitle} =[draw=black,fill=red, text=white]
%%%% END TIKZ STUFF


% If you like road maps, rather than having clutter at the top, have a roadmap show up at the end of each section 
% (and after your introduction)
% Uncomment this is if you want the roadmap!
% \AtBeginSection[]
% {
	%    \begin{frame}
		%        \frametitle{Roadmap of Talk}
		%        \tableofcontents[currentsection]
		%    \end{frame}
	% }
\setbeamercolor{section in toc}{fg=blue}
\setbeamercolor{subsection in toc}{fg=red}
\setbeamersize{text margin left=1em,text margin right=1em} 

\newenvironment{wideitemize}{\itemize\addtolength{\itemsep}{10pt}}{\enditemize}
\newenvironment{wideenumerate}{\enumerate\addtolength{\itemsep}{10pt}}{\endenumerate}

\usepackage{environ}
\NewEnviron{videoframe}[1]{
	\begin{frame}
		\vspace{-8pt}
		\begin{columns}[onlytextwidth, T] % align columns
			\begin{column}{.58\textwidth}
				\begin{minipage}[t][\textheight][t]
					{\dimexpr\textwidth}
					\vspace{8pt}
					\hspace{4pt} {\Large \sc \textcolor{blue}{#1}}
					\vspace{8pt}
					
					\BODY
				\end{minipage}
			\end{column}%
			\hfill%
			\begin{column}{.42\textwidth}
				\colorbox{green!20}{\begin{minipage}[t][1.2\textheight][t]
						{\dimexpr\textwidth}
						Face goes here
				\end{minipage}}
			\end{column}%
		\end{columns}
	\end{frame}
}

\title[]{\textcolor{blue}{ECN 453: Calculus for Cournot Competition}}
\author[PGP]{}
\institute[FRBNY]{\small{\begin{tabular}{c c c}
			Nicholas Vreugdenhil \\
\end{tabular}}}
\date{} 

\begin{document}
	
	% Title Slide
	\begin{frame}
		\maketitle
		\centering
	\end{frame}
		
	\begin{frame}{Calculus for Cournot Competition}
		\begin{wideitemize}
			\item Next we will study Cournot Competition (competition on quantities)
			\item Before we get into the details, let's review some (relatively simple) calculus we will use when analyzing this model.
		\end{wideitemize}
	\end{frame}

	\begin{frame}{Maximizing Profit}
	\begin{wideitemize}
		\item Before, we used that profit is maximized at $MR=MC$.
		\item We can also do profit maximization with calculus 
	\end{wideitemize}
\end{frame}
	
	\begin{frame}{Finding the quantity that maximizes profit: steps}
	\begin{wideitemize}
		\item 1. Write down profit (profit=TR-TC)
		\item 2. Take the derivative of profit with respect to $q$
		\item 3. Set the derivative equal to 0
		\item 4. Solve for $q$
	\end{wideitemize}
	\end{frame}

	\begin{frame}{Finding the quantity that maximizes profit: example from calculus.pdf}
	\begin{wideitemize}
		\item To remember how to use these steps, let's do the example from the notes.
		\item \textbf{Question:} Suppose that demand is $q=10-2p$ and marginal cost is $2$.
		\item 1. What is the firms profit?
		\item 2. What level of output maximizes profit?
	\end{wideitemize}
	\end{frame}

	\begin{frame}{Finding the quantity that maximizes profit: typical Cournot example}
	\begin{wideitemize}
		\item \textbf{Question:} Total demand is given by $Q=q_1+q_2$ where $q_1$ is the output of firm 1 and $q_2$ is the output of firm 2. Suppose that the total demand curve is $Q=10-2p$ and marginal cost is $2$.
		\item 1. What is firm 1's profit?
		\item 2. If firm 2's quantity $q_2$ is fixed, what level of firm 1's output $q_1$ maximizes profit?
	\end{wideitemize}
\end{frame}

	\begin{frame}{Finding the quantity that maximizes profit: typical Cournot example}
	\begin{wideitemize}
		\item \textbf{Question:} Total demand is given by $Q=q_1+q_2$ where $q_1$ is the output of firm 1 and $q_2$ is the output of firm 2. Suppose that the total demand curve is $Q=10-2p$ and marginal cost is $2$.
		\item 1. What is firm 1's profit? 
		\item Answer: Profit = $5q_1-0.5q_1^2-0.5q_2q_1-2q_1$.
		\item 2. If firm 2's quantity $q_2$ is fixed, what level of firm 1's output $q_1$ maximizes profit? 
		\item Answer: $q_1=3-0.5q_2$.
	\end{wideitemize}
\end{frame}
	
\end{document}
