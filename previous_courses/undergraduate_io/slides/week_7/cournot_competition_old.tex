\documentclass[notes,11pt, aspectratio=169]{beamer}

\usepackage{pgfpages}
% These slides also contain speaker notes. You can print just the slides,
% just the notes, or both, depending on the setting below. Comment out the want
% you want.
\setbeameroption{hide notes} % Only slide
%\setbeameroption{show only notes} % Only notes
%\setbeameroption{show notes on second screen=right} % Both

%\usepackage[scaled=1.0]{helvet}
\usepackage{array}


\usepackage{tikz}
\usetikzlibrary{calc}
\usetikzlibrary{matrix}
\usetikzlibrary{positioning}

\newcommand{\payoff}[4][below]{\node[#1]at(#2){$(#3,#4)$};}
\usepackage{verbatim}
\setbeamertemplate{note page}{\pagecolor{gray!5}\insertnote}
\usetikzlibrary{positioning}
\usetikzlibrary{snakes}
\usetikzlibrary{calc}
\usetikzlibrary{arrows}
\usetikzlibrary{decorations.markings}
\usetikzlibrary{shapes.misc}
\usetikzlibrary{matrix,shapes,arrows,fit,tikzmark}
\usepackage{amsmath}
\usepackage{mathpazo}
\usepackage{hyperref}
\usepackage{lipsum}
\usepackage{multimedia}
\usepackage{graphicx}
\usepackage{multirow}
\usepackage{graphicx}
\usepackage{dcolumn}
\usepackage{bbm}
\newcolumntype{d}[0]{D{.}{.}{5}}

\usepackage{changepage}
\usepackage{appendixnumberbeamer}
\newcommand{\beginbackup}{
   \newcounter{framenumbervorappendix}
   \setcounter{framenumbervorappendix}{\value{framenumber}}
   \setbeamertemplate{footline}
   {
     \leavevmode%
     \hline
     box{%
       \begin{beamercolorbox}[wd=\paperwidth,ht=2.25ex,dp=1ex,right]{footlinecolor}%
%         \insertframenumber  \hspace*{2ex} 
       \end{beamercolorbox}}%
     \vskip0pt%
   }
 }
\newcommand{\backupend}{
   \addtocounter{framenumbervorappendix}{-\value{framenumber}}
   \addtocounter{framenumber}{\value{framenumbervorappendix}} 
}


\usepackage{graphicx}
\usepackage[space]{grffile}
\usepackage{booktabs}

% These are my colors -- there are many like them, but these ones are mine.
\definecolor{blue}{RGB}{0,114,178}
\definecolor{red}{RGB}{213,94,0}
\definecolor{yellow}{RGB}{240,228,66}
\definecolor{green}{RGB}{0,158,115}

\hypersetup{
  colorlinks=false,
  linkbordercolor = {white},
  linkcolor = {blue}
}

\usepackage{graphicx,stackengine,xcolor}
\newcommand\Circle[1]{%
	\def\useanchorwidth{T}%
	\def\stacktype{L}%
	\stackon[0pt]{#1}{\scalebox{2.0}[1.15]{\textcolor{red}{$\bigcirc$}}}%
}

%% I use a beige off white for my background
\definecolor{MyBackground}{RGB}{255,253,218}

%% Uncomment this if you want to change the background color to something else
%\setbeamercolor{background canvas}{bg=MyBackground}

%% Change the bg color to adjust your transition slide background color!
\newenvironment{transitionframe}{
  \setbeamercolor{background canvas}{bg=white}
  \begin{frame}}{
    \end{frame}
}

\setbeamercolor{frametitle}{fg=blue}
\setbeamercolor{title}{fg=black}
\setbeamertemplate{footline}[frame number]
\setbeamertemplate{navigation symbols}{} 
\setbeamertemplate{itemize items}{-}
\setbeamercolor{itemize item}{fg=blue}
\setbeamercolor{itemize subitem}{fg=blue}
\setbeamercolor{enumerate item}{fg=blue}
\setbeamercolor{enumerate subitem}{fg=blue}
\setbeamercolor{button}{bg=MyBackground,fg=blue,}

%%% TIKZ STUFF
\tikzset{   
	every picture/.style={remember picture,baseline},
	every node/.style={anchor=base,align=center,outer sep=1.5pt},
	every path/.style={thick},
}
\newcommand\marktopleft[1]{%
	\tikz[overlay,remember picture] 
	\node (marker-#1-a) at (-.3em,.3em) {};%
}
\newcommand\markbottomright[2]{%
	\tikz[overlay,remember picture] 
	\node (marker-#1-b) at (0em,0em) {};%
}
\tikzstyle{every picture}+=[remember picture] 
\tikzstyle{mybox} =[draw=black, very thick, rectangle, inner sep=10pt, inner ysep=20pt]
\tikzstyle{fancytitle} =[draw=black,fill=red, text=white]
%%%% END TIKZ STUFF


% If you like road maps, rather than having clutter at the top, have a roadmap show up at the end of each section 
% (and after your introduction)
% Uncomment this is if you want the roadmap!
% \AtBeginSection[]
% {
%    \begin{frame}
%        \frametitle{Roadmap of Talk}
%        \tableofcontents[currentsection]
%    \end{frame}
% }
\setbeamercolor{section in toc}{fg=blue}
\setbeamercolor{subsection in toc}{fg=red}
\setbeamersize{text margin left=1em,text margin right=1em} 

\newenvironment{wideitemize}{\itemize\addtolength{\itemsep}{10pt}}{\enditemize}
\newenvironment{wideenumerate}{\enumerate\addtolength{\itemsep}{10pt}}{\endenumerate}

\usepackage{environ}
\NewEnviron{videoframe}[1]{
  \begin{frame}
    \vspace{-8pt}
    \begin{columns}[onlytextwidth, T] % align columns
      \begin{column}{.58\textwidth}
        \begin{minipage}[t][\textheight][t]
          {\dimexpr\textwidth}
          \vspace{8pt}
          \hspace{4pt} {\Large \sc \textcolor{blue}{#1}}
          \vspace{8pt}
          
          \BODY
        \end{minipage}
      \end{column}%
      \hfill%
      \begin{column}{.42\textwidth}
        \colorbox{green!20}{\begin{minipage}[t][1.2\textheight][t]
            {\dimexpr\textwidth}
            Face goes here
          \end{minipage}}
      \end{column}%
    \end{columns}
  \end{frame}
}

\title[]{\textcolor{blue}{ECN 453: Cournot Competition}}
\author[PGP]{}
\institute[FRBNY]{\small{\begin{tabular}{c c c}
Nicholas Vreugdenhil \\
\end{tabular}}}
\date{} 

\begin{document}

% Title Slide
\begin{frame}
\maketitle
  \centering
\end{frame}

% INTRO

\begin{frame}{Static Models of Oligopoly: Cournot Competition}
\begin{wideitemize}
	\item Last time we studied Bertrand competition (competition where firms choose \textbf{prices})
	\item Today we will study Cournot competition (competition when firms choose \textbf{quantities})
\end{wideitemize}
\end{frame}

\begin{frame}{Plan}
	\begin{wideenumerate}
		\item Cournot competition: setup
		\item Cournot competition: solution using a graph
		\item Cournot competition: solution using math
		\item Connection between Bertrand competition and Cournot competition
	\end{wideenumerate}
\end{frame}

\begin{frame}{Plan}
	\begin{wideenumerate}
		\item \textbf{Cournot competition: setup}
		\item Cournot competition: solution using a graph
		\item Cournot competition: solution using math
		\item Connection between Bertrand competition and Cournot competition
	\end{wideenumerate}
\end{frame}

\begin{frame}{Cournot competition: setup}
	\begin{wideitemize}
		\item \textbf{Players}: Two firms (denote each by i where i = 1 or 2)
		\item \textbf{Strategies}: Each firm chooses output level $q_1$, $q_2$
		\begin{wideitemize}
			\item Sell homogeneous (identical) products
		\end{wideitemize}
		\item \textbf{Payoffs:}  Each firm $i$'s payoff is profit: $\pi_i = q_i(P(q_1+q_2) - c)$
		\begin{wideitemize}
			\item Prices are determined by a demand curve $P(Q)$ where $Q=q_1+q_2$.
			\item Marginal cost is $c$
			\item Note: observe that the price that firm 1 gets $P(q_1+q_2)$ is not just dependent on how much firm 1 produces $q_1$, but also how much firm 2 produces $q_2$
		\end{wideitemize}
		\item Observe that the setup is the same as Bertrand \underline{except} that firms now choose quantities.
	\end{wideitemize}
\end{frame}

\begin{frame}{Cournot competition: solution overview}
	\begin{wideitemize}
		\item We will follow our `usual steps' to find the equilibrium in Cournot competition:
	\end{wideitemize}
	\begin{wideenumerate}
		\vspace{11pt}
		\item Find the best response of firm 1 to firm 2's choice (denote this $q_1^*(q_2)$)
		\item Find the best response of firm 2 to firm 1's choice (denote this $q_2^*(q_1)$)
		\item Find where the two best reponse curves cross: this is the Nash equilibrium
	\end{wideenumerate}
	\begin{wideitemize}
		\vspace{11pt}
		\item We will now see two equivalent ways of doing these three steps: a solution using a graph and a solution using math.
	\end{wideitemize}
\end{frame}

\begin{frame}{Plan}
	\begin{wideenumerate}
		\item Cournot competition: setup
		\item \textbf{Cournot competition: solution using a graph}
		\item Cournot competition: solution using math
		\item Connection between Bertrand competition and Cournot competition
	\end{wideenumerate}
\end{frame}

\begin{frame}{Cournot competition: solution using a graph - preliminaries}
	\begin{wideitemize}
		\item Suppose that firm 2 is producing $q_2$. How can we find firm 1's best response $q_1^*(q_2)$?
		\item Start by defining the \textbf{residual demand curve}.
		\begin{wideitemize}
			\item This is denoted $d_1(q_2)$: defined as all possible combinations of Firm 1's quantity and price for \textit{a given value of $q_2$}.
			\item Idea: suppose that firm 2 produces $q_2$. The demand `left over' for firm 1 is the residual demand curve.
			\item Example: residual demand curve at $q_2=0$ is $P(q_1+0)=P(q_1)$.
		\end{wideitemize}
		\item The residual demand curve has a \textbf{residual marginal revenue curve}: $r_1(q_2)$.
		\item The optimal quantity is then found by applying the monopoly solution \textit{on the marginal revenue curve}: i.e. set $r_1(q_2)=MC$.
	\end{wideitemize}
\end{frame}

\begin{frame}{Cournot competition: solution using a graph - best responses}
\begin{wideitemize}
	\item Let's find the best responses of Firm 1 at two extremes. These two extremes will allow us to `trace out' the best response curve for Firm 1. (Graphical version of this slide is on the next slide.)
	\item \underline{Case 1:} Firm 2 produces $q_2=0$. We can show $q_1^*(0)=q^M$
	\begin{wideitemize}

		\item If Firm 2 produces $q_2=0$ then the residual demand curve for Firm 1 is $P(0+q_2)=P(q_2)$.
		\item So, Firm 1's demand curve is the entire market demand. Thus, this is equivalent to when Firm 1 is a monopolist and the optimal quantity is the monopoly quantity.
	\end{wideitemize}
	\item \underline{Case 2:} Firm 2 produces $q_2=q^c$ (the perfect competition quantity). We can show $q_1^*(q^c)=0$
	\begin{wideitemize}
		\item On the following slide I draw the residual demand curve $d_1(q^c)$ (along with some other points).
		\item This residual demand curve intersects marginal cost at $q_1=0$; since the marginal revenue curve has the same vertical intercept it also intersects marginal cost at $q_1=0$. Hence $q_1=0$ is the optimal quantity.
	\end{wideitemize}
\end{wideitemize}
\end{frame}

\begin{frame}{Cournot competition: solution using a graph - best responses}
	\begin{figure}
		\centering
	\includegraphics[scale=0.52]{firm_1_optimum.pdf}
	\end{figure}
\end{frame}

\begin{frame}{Cournot competition: solution using a graph - best responses and Nash equilibrium}
	\begin{wideitemize}
		\item On the next slide I plot all of Firm 1's best responses $q_1^*(q_2)$ by drawing a line between these two extreme cases.
		\item By the same arguments for Firm 1, on the next slide I also trace out Firm 2's best responses  $q_2^*(q_1)$.
		\item Finally, on the next slide, I plot the point where the curves cross: this is the Nash equilibrium (denoted ($\hat{q}_2$,$\hat{q}_1)$)
\end{wideitemize}
\end{frame}

\begin{frame}{Cournot competition: solution using a graph - Nash equilibrium}
	\begin{figure}
		\centering
		\includegraphics[scale=0.52]{best_responses.pdf}
	\end{figure}
\end{frame}

\begin{frame}{Cournot competition: any initial solution converges to Nash equilibrium}
	\begin{figure}
		\centering
		\includegraphics[scale=0.38]{convergence.pdf}
	\end{figure}
\end{frame}


\begin{frame}{Plan}
	\begin{wideenumerate}
		\item Cournot competition: setup
		\item Cournot competition: solution using a graph
		\item \textbf{Cournot competition: solution using math}
		\item Connection between Bertrand competition and Cournot competition
	\end{wideenumerate}
\end{frame}

\begin{frame}{Warm up: maximizing profit with calculus for a monopolist}
		\begin{wideitemize}
			\item 1. Write down profit (profit=TR-TC)
			\item 2. Take the derivative of profit with respect to $q$
			\item 3. Set the derivative equal to 0
			\item 4. Solve for $q$
		\end{wideitemize}
	\end{frame}
	
	\begin{frame}{Warm up: example from calculus.pdf}
		\begin{wideitemize}
			\item To remember how to use these steps, let's do the example from the calculus practice material.
			\item \textbf{Question:} Suppose that demand is $q=10-2p$ and marginal cost is $2$.
			\item 1. What is the firms profit?
			\item 2. What level of output maximizes profit?
		\end{wideitemize}
\end{frame}

\begin{frame}{Cournot competition: solution using math}
	\begin{wideitemize}
		\item \textbf{Question:} Suppose that we have the Cournot competition setup. Assume that demand is $Q=10-2p$ where $Q=q_1+q_2$ and marginal cost is $2$.
		\item 1. What is the profit of Firm 1 and Firm 2? 
		\item 2. What are the best responses of Firm 1 and  Firm 2?
		\item 3. What is the Nash equilibrium?
	\end{wideitemize}
\end{frame}

\begin{frame}{Cournot competition: solution using math}
	\begin{wideitemize}
		\item \textbf{Solution:} 1. What is the profit of Firm 1 and Firm 2? 
		\item \underline{Firm 1:}
		\item $\pi_1 = q_1(P(q_1+q_2)-c))=q_1 (5-0.5(q_1+q_2) - 2)=q_1(3-0.5(q_1+q_2))$
		\item \underline{Firm 1:}
		\item $\pi_2 = q_2(P(q_1+q_2)-c))=q_2 (5-0.5(q_1+q_2) - 2)=q_2(3-0.5(q_1+q_2))$
	\end{wideitemize}
\end{frame}

\begin{frame}{Cournot competition: solution using math}
	\begin{wideitemize}
		\item \textbf{Solution:} 2. What are the best responses of Firm 1 and  Firm 2?
		\item Start with Firm 1 and follow the `maximizing profit' steps from before. 
		\item Note that when maximizing profit for Firm 1, treat Firm 2's quantity choice $q_2$ \underline{as a constant}.
		\item First, take the derivative:
	\begin{align*}
		\frac{d \pi_1}{d q_1} = 3 - q_1- 0.5q_2
	\end{align*}
		\item Set the derivative = 0 to maximize profit:
	\begin{align*}
		0 = 3 - q_1- 0.5q_2
	\end{align*}
		\item Rearrange for $q_1$: this is Firm 1's best response to $q_2$.
		\begin{align*}
		q_1^*(q_2)=3- 0.5q_2
	\end{align*}
	\end{wideitemize}
\end{frame}

\begin{frame}{Cournot competition: solution using math}
	\begin{wideitemize}
		\item \textbf{Solution:} 2. What are the best responses of Firm 1 and  Firm 2?
		\item Use exactly the same arguments to get Firm 2's best response:
		\begin{align*}
			q_2^*(q_1)=3- 0.5q_1
		\end{align*}
	\end{wideitemize}
\end{frame}

\begin{frame}{Cournot competition: solution using math}
	\begin{wideitemize}
		\item \textbf{Solution:} 3. What is the Nash equilibrium?
		\item We now have two equations (the best responses) and two unknowns (the Nash equilibrium quantities $q_1$, $q_2$)
		\item Substitute Firm 2's best response equation into Firm 1's best response equation:
		\begin{align*}
			q_1=3- 0.5(3-0.5q_1)
		\end{align*}
		\item Rearranging for $q_1$ (and substituting this $q_1$ into Firm 2's best response to get $q_2$)
		\begin{align*}
			q_1&=2 \\
			q_2&=2
		\end{align*} 
		\item So, $(q_1,q_2) = (2,2)$ is the Nash equilibrium.
	\end{wideitemize}
\end{frame}

\begin{frame}{Cournot competition: solution reprinted with more math sub-steps}
	\begin{wideenumerate}
		\item Find the best response of firm 1 to firm 2's choice (denote this $q_1^*(q_2)$)
		\begin{wideenumerate}
			\item Write down Firm 1's profit.
			\item Take the derivative of Firm 1's profit with respect to $q_1$ and set this derivative = 0.
			\item Rearrange for $q_1$
		\end{wideenumerate}
		\item Find the best response of firm 2 to firm 1's choice (denote this $q_2^*(q_1)$)
		\begin{wideenumerate}
			\item Write down Firm 2's profit.
			\item Take the derivative of Firm 2's profit with respect to $q_2$ and set this derivative = 0.
			\item Rearrange for $q_2$
		\end{wideenumerate}
		\item Find where the two best reponse curves cross: this is the Nash equilibrium
		\begin{wideenumerate}
			\item We have two equations (the two best response curves) and two unknowns (the Nash equilibrium $q_1$ and $q_2$). 
			\item Solve these two equations for $q_1$ and $q_2$: this is the Nash equilibrium.
		\end{wideenumerate}
	\end{wideenumerate}
\end{frame}

\begin{frame}{Cournot competition: general solution with two \underline{identical} firms; p196}
	\begin{wideitemize}
		\item \textbf{Setup:}
		\item Two identical firms with (inverse) demand $P(Q)=a-bQ$ where $Q=q_1+q_2$ and $a,b$ are constants. Marginal cost = $c$. 
		\item \textbf{Solution:}
		\item Best responses:
		\begin{align*}
			q_1^*(q_2)= \frac{a-c}{2b} - \frac{q_2}{2} \\
			q_2^*(q_1)= \frac{a-c}{2b} - \frac{q_1}{2} \\
		\end{align*}
		\item Nash equilibrium:
		\begin{align*}
			q_1=q_2=\frac{a-c}{3b} 
		\end{align*}
	\end{wideitemize}
\end{frame}

\begin{frame}{Plan}
	\begin{wideenumerate}
		\item Cournot competition: setup
		\item Cournot competition: solution using a graph
		\item Cournot competition: solution using math
		\item \textbf{Connection between Bertrand competition and Cournot competition}
	\end{wideenumerate}
\end{frame}

\begin{frame}{Cournot competition: Cournot vs Bertrand}
	\begin{wideitemize}
		\item How is Cournot competition related to Bertrand competition?
		\item Remember the capacity constrained Bertrand competition from last week?
		\begin{wideitemize}
			\item Then, we argued that if capacity constraints were $q_1$ and $q_2$ then the price under Bertrand competition was $p_1=p_2=P(q_1+q_2)$.
			\item (Note: before, we denoted the capacity constraints by $k_i$ but now I'm denoting them by $q_i$)
		\end{wideitemize} 
		\item So, the setup to Cournot competition is equivalent to the choice of capacity in the following two stage game:
		\begin{wideenumerate}
			\item Choose capacity constraints $q_1$ and $q_2$
			\item Given these capacity constraints, compete under Bertrand competition
		\end{wideenumerate}
		\item The above relationship between Bertrand and Cournot competition is known as the \textbf{Kreps and Scheinkman (1983)} result.
	\end{wideitemize}
\end{frame}

\begin{frame}{Cournot competition: Cournot vs Bertrand}
	\begin{wideitemize}
		\item As economists, we choose which model best fits a particular industry.
		\item \underline{General way to choose between Cournot vs Bertrand}:
		\begin{wideitemize}
			\item If capacity/output can be adjusted easily $\rightarrow$ Bertrand
			\item If capacity/output are hard to adjust $\rightarrow$ Cournot
		\end{wideitemize}
		\item Markets more suited to modeling with Bertrand competition:
		\begin{wideitemize}
			\item Software, insurance, banking
		\end{wideitemize}
		\item Markets more suited to modeling with Cournot competition:
		\begin{wideitemize}
			\item Airlines
			\item Many other industries that manufacture physical goods e.g. wheat, cement, steel, cars etc
			\item Idea: capacity investments are long-run choices in these industries
		\end{wideitemize}
	\end{wideitemize}
\end{frame}


\begin{frame}{Summary of key points*}
	%\vspace{-20pt}
	\begin{wideitemize}
		\item Know the assumptions behind Cournot competition (quantity competition)
		\item Know how to solve a Cournot model using a graph
		\item \underline{Very important!} Know how to solve a Cournot model using math (including: take derivative of profit, get the best responses, get the Nash equilibrium etc)
		\item Know the connection between Cournot and Bertrand competition and know how to choose between the two models to fit a particular industry.
	\end{wideitemize}
	\vspace{20pt}
	*To clarify, all the material in the slides, problem sets, etc is assessable unless stated otherwise, but I hope this summary might be a useful place to start when studying the material.
\end{frame}


\end{document}
