\documentclass[notes,11pt, aspectratio=169]{beamer}

\usepackage{pgfpages}
% These slides also contain speaker notes. You can print just the slides,
% just the notes, or both, depending on the setting below. Comment out the want
% you want.
\setbeameroption{hide notes} % Only slide
%\setbeameroption{show only notes} % Only notes
%\setbeameroption{show notes on second screen=right} % Both

%\usepackage[scaled=1.0]{helvet}
\usepackage{array}

\usepackage{graphicx}
\usepackage{tikz}
\usetikzlibrary{calc}
\usetikzlibrary{matrix}
\usetikzlibrary{positioning}

\newcommand{\payoff}[4][below]{\node[#1]at(#2){$(#3,#4)$};}
\usepackage{verbatim}
\setbeamertemplate{note page}{\pagecolor{gray!5}\insertnote}
\usetikzlibrary{positioning}
\usetikzlibrary{snakes}
\usetikzlibrary{calc}
\usetikzlibrary{arrows}
\usetikzlibrary{decorations.markings}
\usetikzlibrary{shapes.misc}
\usetikzlibrary{matrix,shapes,arrows,fit,tikzmark}
\usepackage{amsmath}
\usepackage{mathpazo}
\usepackage{hyperref}
\usepackage{lipsum}
\usepackage{multimedia}
\usepackage{graphicx}
\usepackage{multirow}
\usepackage{graphicx}
\usepackage{dcolumn}
\usepackage{bbm}
\newcolumntype{d}[0]{D{.}{.}{5}}

\usepackage{changepage}
\usepackage{appendixnumberbeamer}
\newcommand{\beginbackup}{
   \newcounter{framenumbervorappendix}
   \setcounter{framenumbervorappendix}{\value{framenumber}}
   \setbeamertemplate{footline}
   {
     \leavevmode%
     \hline
     box{%
       \begin{beamercolorbox}[wd=\paperwidth,ht=2.25ex,dp=1ex,right]{footlinecolor}%
%         \insertframenumber  \hspace*{2ex} 
       \end{beamercolorbox}}%
     \vskip0pt%
   }
 }
\newcommand{\backupend}{
   \addtocounter{framenumbervorappendix}{-\value{framenumber}}
   \addtocounter{framenumber}{\value{framenumbervorappendix}} 
}


\usepackage{graphicx}
\usepackage[space]{grffile}
\usepackage{booktabs}

% These are my colors -- there are many like them, but these ones are mine.
\definecolor{blue}{RGB}{0,114,178}
\definecolor{red}{RGB}{213,94,0}
\definecolor{yellow}{RGB}{240,228,66}
\definecolor{green}{RGB}{0,158,115}

\hypersetup{
  colorlinks=false,
  linkbordercolor = {white},
  linkcolor = {blue}
}

\usepackage{graphicx,stackengine,xcolor}
\newcommand\Circle[1]{%
	\def\useanchorwidth{T}%
	\def\stacktype{L}%
	\stackon[0pt]{#1}{\scalebox{2.0}[1.15]{\textcolor{red}{$\bigcirc$}}}%
}

%% I use a beige off white for my background
\definecolor{MyBackground}{RGB}{255,253,218}

%% Uncomment this if you want to change the background color to something else
%\setbeamercolor{background canvas}{bg=MyBackground}

%% Change the bg color to adjust your transition slide background color!
\newenvironment{transitionframe}{
  \setbeamercolor{background canvas}{bg=white}
  \begin{frame}}{
    \end{frame}
}

\setbeamercolor{frametitle}{fg=blue}
\setbeamercolor{title}{fg=black}
\setbeamertemplate{footline}[frame number]
\setbeamertemplate{navigation symbols}{} 
\setbeamertemplate{itemize items}{-}
\setbeamercolor{itemize item}{fg=blue}
\setbeamercolor{itemize subitem}{fg=blue}
\setbeamercolor{enumerate item}{fg=blue}
\setbeamercolor{enumerate subitem}{fg=blue}
\setbeamercolor{button}{bg=MyBackground,fg=blue,}

%%% TIKZ STUFF
\tikzset{   
	every picture/.style={remember picture,baseline},
	every node/.style={anchor=base,align=center,outer sep=1.5pt},
	every path/.style={thick},
}
\newcommand\marktopleft[1]{%
	\tikz[overlay,remember picture] 
	\node (marker-#1-a) at (-.3em,.3em) {};%
}
\newcommand\markbottomright[2]{%
	\tikz[overlay,remember picture] 
	\node (marker-#1-b) at (0em,0em) {};%
}
\tikzstyle{every picture}+=[remember picture] 
\tikzstyle{mybox} =[draw=black, very thick, rectangle, inner sep=10pt, inner ysep=20pt]
\tikzstyle{fancytitle} =[draw=black,fill=red, text=white]
%%%% END TIKZ STUFF


% If you like road maps, rather than having clutter at the top, have a roadmap show up at the end of each section 
% (and after your introduction)
% Uncomment this is if you want the roadmap!
% \AtBeginSection[]
% {
%    \begin{frame}
%        \frametitle{Roadmap of Talk}
%        \tableofcontents[currentsection]
%    \end{frame}
% }
\setbeamercolor{section in toc}{fg=blue}
\setbeamercolor{subsection in toc}{fg=red}
\setbeamersize{text margin left=1em,text margin right=1em} 

\newenvironment{wideitemize}{\itemize\addtolength{\itemsep}{10pt}}{\enditemize}
\newenvironment{wideenumerate}{\enumerate\addtolength{\itemsep}{10pt}}{\endenumerate}

\usepackage{environ}
\NewEnviron{videoframe}[1]{
  \begin{frame}
    \vspace{-8pt}
    \begin{columns}[onlytextwidth, T] % align columns
      \begin{column}{.58\textwidth}
        \begin{minipage}[t][\textheight][t]
          {\dimexpr\textwidth}
          \vspace{8pt}
          \hspace{4pt} {\Large \sc \textcolor{blue}{#1}}
          \vspace{8pt}
          
          \BODY
        \end{minipage}
      \end{column}%
      \hfill%
      \begin{column}{.42\textwidth}
        \colorbox{green!20}{\begin{minipage}[t][1.2\textheight][t]
            {\dimexpr\textwidth}
            Face goes here
          \end{minipage}}
      \end{column}%
    \end{columns}
  \end{frame}
}

\title[]{\textcolor{blue}{ECN 453: Cournot Competition 2}}
\author[PGP]{}
\institute[FRBNY]{\small{\begin{tabular}{c c c}
Nicholas Vreugdenhil \\
\end{tabular}}}
\date{} 

\begin{document}

% Title Slide
\begin{frame}
\maketitle
  \centering
\end{frame}

% INTRO

\begin{frame}{Static Models of Oligopoly: Cournot Competition}
\begin{wideitemize}
	\item Today we will continue to study Cournot competition (competition on quantities)
	\item We will focus on extensions and applications of the basic Cournot model we saw last time.
\end{wideitemize}
\end{frame}

\begin{frame}{Plan}
	\begin{wideenumerate}
		\item Connection between Bertrand competition and Cournot competition
		\item Cournot competition with many firms
		\item Comparative statics: changes to input prices
		\item Comparative statics: new technology
		\item Comparative statics: exchange rates 
	\end{wideenumerate}
\end{frame}

\begin{frame}{Plan}
	\begin{wideenumerate}
		\item \textbf{Connection between Bertrand competition and Cournot competition}
		\item Cournot competition with many firms
		\item Comparative statics: changes to input prices
		\item Comparative statics: new technology
		\item Comparative statics: exchange rates 
	\end{wideenumerate}
\end{frame}

\begin{frame}{Cournot competition: Cournot vs Bertrand}
	\begin{wideitemize}
		\item How is Cournot competition related to Bertrand competition?
		\item Remember the capacity constrained Bertrand competition from last week?
		\begin{wideitemize}
			\item Then, we argued that if capacity constraints were $q_1$ and $q_2$ then the price under Bertrand competition was $p_1=p_2=P(q_1+q_2)$.
			\item (Note: before, we denoted the capacity constraints by $k_i$ but now I'm denoting them by $q_i$)
		\end{wideitemize} 
		\item So, the setup to Cournot competition is equivalent to the choice of capacity in the following two stage game:
		\begin{wideenumerate}
			\item Choose capacity constraints $q_1$ and $q_2$
			\item Given these capacity constraints, compete under Bertrand competition
		\end{wideenumerate}
		\item The above relationship between Bertrand and Cournot competition is known as the \textbf{Kreps and Scheinkman (1983)} result.
	\end{wideitemize}
\end{frame}

\begin{frame}{Cournot competition: Cournot vs Bertrand}
	\begin{wideitemize}
		\item As economists, we choose which model best fits a particular industry.
		\item \underline{General way to choose between Cournot vs Bertrand}:
		\begin{wideitemize}
			\item If capacity/output can be adjusted easily $\rightarrow$ Bertrand
			\item If capacity/output are hard to adjust $\rightarrow$ Cournot
		\end{wideitemize}
	\end{wideitemize}
\end{frame}

\begin{frame}{Cournot competition: Cournot vs Bertrand}
	\begin{columns}
		\begin{column}{0.55\textwidth}
			\begin{wideitemize}
				\item Markets more suited to modeling with Bertrand competition:
				\begin{wideitemize}
					\item Software, insurance, banking
				\end{wideitemize}
				\item Markets more suited to modeling with Cournot competition:
				\begin{wideitemize}
					\item Airlines
					\item Many other industries that manufacture physical goods e.g. wheat, cement, steel, cars etc
					\item Idea: capacity investments are long-run choices in these industries
				\end{wideitemize}
			\end{wideitemize}
		\end{column}
		\begin{column}{0.45\textwidth}
			\begin{figure}
				\includegraphics[scale=0.15]{airline.jpeg}
			\end{figure}
		\end{column}
	\end{columns}
\end{frame}

\begin{frame}{Plan}
	\begin{wideenumerate}
		\item Connection between Bertrand competition and Cournot competition
		\item \textbf{Cournot competition with many firms}
		\item Comparative statics: changes to input prices
		\item Comparative statics: new technology
		\item Comparative statics: exchange rates 
	\end{wideenumerate}
\end{frame}

\begin{frame}{Cournot competition with many firms}
	\begin{wideitemize}
		\item \textbf{Setup:}
		\item n firms
		\item Market demand: P(Q) = a-bQ where: 
		\begin{wideitemize}
			\item a,b are constants 
			\item total output: $Q=q_1+q_2+...+q_n$
		\end{wideitemize}
		\item Constant marginal cost: c
		\item \textbf{Question:} What are the Cournot equilibrium quantities?
	\end{wideitemize}
\end{frame}

\begin{frame}{Cournot competition with many firms}
	\begin{wideitemize}
		\item \textbf{Question:} What are the Cournot equilibrium quantities?
		\item To solve, we will use our typical steps, starting with writing down the payoffs for each firm.
		\item Payoffs:
		\begin{align*}
			\pi_i &= q_i P(Q) - c q_i 
		\end{align*}
	\end{wideitemize}
\end{frame}

\begin{frame}{Cournot competition with many firms}
	\begin{wideitemize}
		\item Next, we need to find the best response of firm $i$.
		\item First, expand profit:
		\begin{align*}
			\pi_i &= q_i P(Q) - c q_i \\
			&= q_i (a - bQ) - c q_i \\
			&= q_i (a - b(q_1 + q_2 + ... + q_n)) - c q_i \\
			&= q_i a - q_i b q_1 - q_i b q_2 - ... - b q_i^2 - ... - b q_i q_n - c q_i
		\end{align*}
		\item Take the derivative and set it equal to zero (to maximize profit):
		\begin{align*}
			\frac{d \pi_i}{d q_i} = -bq_i + a - b Q - c = 0 
		\end{align*}
	\end{wideitemize}
\end{frame}

\begin{frame}{Cournot competition with many firms}
	\begin{wideitemize}
		\item Finally, we need to compute the Nash equilibrium from the best responses.
		\item \underline{The trick is to observe that the firms are identical.} Therefore, in equilibrium, the quantities will be identical with $q=q_1=...=q_n$ and so $Q=nq$.
		\item Therefore the best response of firm $i$ is:
	\end{wideitemize}
	\begin{align*}
		-bq+a-bnq-c=0
	\end{align*}
	\begin{wideitemize}
		\item So, optimal quantity and price (getting price from substituting optimal quantity into the demand curve) is:
	\end{wideitemize}
	\begin{align*}
			q &= \frac{a-c}{(n+1)b} \\
			p &= \frac{a+nc}{n+1}
	\end{align*}
\end{frame}

\begin{frame}{Cournot competition with many firms}
	\begin{wideitemize}
		\item Let's see how prices change with the number of firms:
	\begin{align*}
		p &= \frac{a+nc}{n+1}
	\end{align*}
		\item As $n \rightarrow \infty$, $p \rightarrow c$.
		\item I.e. as the number of firms gets large, the market converges to the perfect competition price level.
	\end{wideitemize}
\end{frame}

\begin{frame}{Plan}
	\begin{wideenumerate}
		\item Connection between Bertrand competition and Cournot competition
		\item Cournot competition with many firms
		\item \textbf{Comparative statics: changes to input prices}
		\item Comparative statics: new technology
		\item Comparative statics: exchange rates 
	\end{wideenumerate}
\end{frame}

\begin{frame}{Comparative statics: changes to input prices}
	\begin{wideitemize}
		\item We will now look at some \textbf{comparative statics}.
		\item \textbf{Comparative statics} means evaluating how the equilibrium changes when we change the parameters to the problem.
		\begin{wideitemize}
			\item For example, in the entry deterrence midterm question, we evaluated how the equilibrium changed when we changed $x$.
		\end{wideitemize}
		\item We will start by looking at an increase in input costs.
		\begin{wideitemize}
			\item We are particularly in interested in computing \textbf{pass-through}: the change in the final price for a change in the input price (marginal cost).
		\end{wideitemize}
	\end{wideitemize}
\end{frame}

\begin{frame}{Comparative statics: changes to input prices - carbon price example}
\begin{figure}
	\centering
	\includegraphics[scale=0.32]{ets.pdf}
\end{figure}
\begin{wideitemize}
	\item  \textit{`Pass Through of Emissions Costs in Electricity Markets'}: Fabra and Reguant (2014)
\end{wideitemize}
\end{frame}

\begin{frame}{Comparative statics: changes to input prices - graph with example of 40\% increase in costs}
	\begin{figure}
		\centering
		\includegraphics[scale=0.4]{tech1.pdf}
	\end{figure}
\end{frame}

\begin{frame}{Comparative statics: changes to input prices - graph with example of 40\% increase in costs}
	\begin{figure}
		\centering
		\includegraphics[scale=0.35]{tech2.pdf}
	\end{figure}
\end{frame}

\begin{frame}{Comparative statics: changes to input prices - graph with example of 40\% increase in costs}
	\begin{figure}
		\centering
		\includegraphics[scale=0.29]{tech3.pdf}
	\end{figure}
\end{frame}

\begin{frame}{Comparative statics: changes to input prices (math version)}
	\begin{wideitemize}
		\item \textbf{Setup:} Cournot duopoly with total demand $p=a-bQ$ and two identical firms with marginal cost $c$.
		\item \textbf{Question:} Suppose that marginal cost increases by \$20. What is the change in output price $p$?
	\end{wideitemize}
\end{frame}

\begin{frame}{Comparative statics: changes to input prices}
	\begin{wideitemize}
		\item \textbf{Solution:} 
		\item With 2 firms, we saw before that equilibrium prices are given by:
		\begin{align*}
			p=\frac{a+2c}{3}
		\end{align*}
		\item Finding the change in price for a change in $c$:
		\begin{align*}
			\frac{dp}{dc}= \frac{2}{3}
		\end{align*}
		\item So, model predicts price increases by 66\% (\$13.33) for a \$20 increase in costs.
	\end{wideitemize}
\end{frame}

\begin{frame}{Plan}
	\begin{wideenumerate}
		\item Connection between Bertrand competition and Cournot competition
		\item Cournot competition with many firms
		\item Comparative statics: changes to input prices
		\item \textbf{Comparative statics: new technology}
		\item Comparative statics: exchange rates 
	\end{wideenumerate}
\end{frame}

\begin{frame}{Comparative statics: new technology}
	\begin{wideitemize}
		\item \textbf{Setup}:
		\item Firm 1: old technology $c_1=\$15$
		\item Firm 2: new technology $c_2=\$12$
		\item Demand: p=30-Q
		\item Firms compete on quantities (Cournot competition)
		\item \textbf{Question}: How much would Firm 1 be willing to pay for the new technology?
	\end{wideitemize}
\end{frame}

\begin{frame}{Comparative statics: new technology}
	\begin{wideitemize}
		\item \textbf{Solution}:
		\item Apply our `usual steps' to solve for the Cournot equilibrium with and without the new technology.
		\item Without the new technology can show that the initial profit level for Firm 1 is: \$16. After the new technology is introduced, solving for the equilibrium, the profit for Firm 1 is: \$36.
		\item So, Firm 1 would pay \$20 for this new technology. 
	\end{wideitemize}
\end{frame}

\begin{frame}{Comparative statics: new technology}
	\begin{wideitemize}
		\item For problems like these with: 
		\begin{wideitemize}
			\item Asymmetric (i.e. different) constant marginal costs $c_1 \neq c_2$
			\item Two firms
			\item Cournot competition
			\item Linear demand in the form $p=a-bQ$ where a,b are constants
		\end{wideitemize}
		\item By \underline{exactly the same method from the previous slide} (but more complicated algebra on p205) where a hat on the variables denotes the Cournot equilibrium values (see next slide):
	\end{wideitemize}
\end{frame}

\begin{frame}{Comparative statics: new technology}
		\begin{align*}
			\hat{q}_1 &= \frac{a-2c_1+c_2}{3b} \\
			\hat{q}_2 &= \frac{a-2c_2+c_1}{3b} \\
			\hat{Q} &=\hat{q}_1+\hat{q}_2 = \frac{2a - c_1 -c_2}{3b} \\
			\hat{p} &=a-b \hat{Q}= \frac{a+c_1+c_2}{3}
		\end{align*}
	\begin{wideitemize}
		\item \textbf{Caution}: although these formulas are useful, in many of the problems I will set some assumptions will not be satisfied and so you will need to go through the `usual steps' to solve the Cournot equilibrium.
	\end{wideitemize}
\end{frame}

\begin{frame}{Plan}
	\begin{wideenumerate}
		\item Connection between Bertrand competition and Cournot competition
		\item Cournot competition with many firms
		\item Comparative statics: changes to input prices
		\item Comparative statics: new technology
		\item \textbf{Comparative statics: exchange rates}
	\end{wideenumerate}
\end{frame}

\begin{frame}{Comparative statics: exchange rates}
	\begin{wideitemize}
		\item \textbf{Setup}:
		\item Microprocessor duopoly; Cournot competition
		\item Two firms, Firm 1: in Japan Firm 2: in the US
		\item Initially, firms have the same cost, $c_1=c_2=\$12$ 
		\item Suppose we have a Yen devaluation so that $c_1=\$8$
		\item Assume at the initial equilibrium $p=24$.
		\item \textbf{Question:} How do firms' market shares change due to the devaluation?
	\end{wideitemize}
\end{frame}

\begin{frame}{Comparative statics: exchange rates}
	\begin{wideitemize}
		\item \textbf{Solution:} Use `calibration' (i.e. use data to determine values of parameters)
		\item Firm 1's market share (using the asymmetric equilibrium formulas):
		\begin{align*}
			s_1 = \frac{q_1}{q_1+q_2} = \frac{a-2c_1+c_2}{2a-c_1-c_2}
		\end{align*}
		\item At initial equilibrium (where costs = \$12 for both firms):
		\begin{align*}
			p=\frac{a+2c}{3}
		\end{align*}
		\item Solving for $a$:
		\begin{align*}
			a=3p-2c=3*24-2*12=48
		\end{align*}
		\item Substituting into the formula:
		\begin{align*}
			s_1 \approx 58\%
		\end{align*}
		\item So, Japan's market share goes from 50\% $\rightarrow$ 58\%
	\end{wideitemize}
\end{frame}

\begin{frame}{Summary of key points*}
	%\vspace{-20pt}
	\begin{wideitemize}
		\item Know the connection between Cournot and Bertrand competition and know how to choose between the two models to fit a particular industry.
		\item Know how to solve Cournot models with $>2$ firms (and the `trick' which is to set the equilibrium production level of identical firms to be the same value)
		\item Understand the three comparative statics examples:
		\begin{wideitemize}
			\item Changes to input prices
			\item New technology
			\item Exchange rates
		\end{wideitemize}
		\item Know the formulas for linear demand (n firms, asymmetric costs, etc...)
	\end{wideitemize}
	\vspace{20pt}
	*To clarify, all the material in the slides, problem sets, etc is assessable unless stated otherwise, but I hope this summary might be a useful place to start when studying the material.
\end{frame}


\end{document}
