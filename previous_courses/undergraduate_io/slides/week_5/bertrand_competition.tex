\documentclass[notes,11pt, aspectratio=169]{beamer}

\usepackage{pgfpages}
% These slides also contain speaker notes. You can print just the slides,
% just the notes, or both, depending on the setting below. Comment out the want
% you want.
\setbeameroption{hide notes} % Only slide
%\setbeameroption{show only notes} % Only notes
%\setbeameroption{show notes on second screen=right} % Both

%\usepackage[scaled=1.0]{helvet}
\usepackage{array}


\usepackage{tikz}
\usetikzlibrary{calc}
\usetikzlibrary{matrix}
\usetikzlibrary{positioning}

\newcommand{\payoff}[4][below]{\node[#1]at(#2){$(#3,#4)$};}
\usepackage{verbatim}
\setbeamertemplate{note page}{\pagecolor{gray!5}\insertnote}
\usetikzlibrary{positioning}
\usetikzlibrary{snakes}
\usetikzlibrary{calc}
\usetikzlibrary{arrows}
\usetikzlibrary{decorations.markings}
\usetikzlibrary{shapes.misc}
\usetikzlibrary{matrix,shapes,arrows,fit,tikzmark}
\usepackage{amsmath}
\usepackage{mathpazo}
\usepackage{hyperref}
\usepackage{lipsum}
\usepackage{multimedia}
\usepackage{graphicx}
\usepackage{multirow}
\usepackage{graphicx}
\usepackage{dcolumn}
\usepackage{bbm}
\newcolumntype{d}[0]{D{.}{.}{5}}

\usepackage{changepage}
\usepackage{appendixnumberbeamer}
\newcommand{\beginbackup}{
   \newcounter{framenumbervorappendix}
   \setcounter{framenumbervorappendix}{\value{framenumber}}
   \setbeamertemplate{footline}
   {
     \leavevmode%
     \hline
     box{%
       \begin{beamercolorbox}[wd=\paperwidth,ht=2.25ex,dp=1ex,right]{footlinecolor}%
%         \insertframenumber  \hspace*{2ex} 
       \end{beamercolorbox}}%
     \vskip0pt%
   }
 }
\newcommand{\backupend}{
   \addtocounter{framenumbervorappendix}{-\value{framenumber}}
   \addtocounter{framenumber}{\value{framenumbervorappendix}} 
}


\usepackage{graphicx}
\usepackage[space]{grffile}
\usepackage{booktabs}

% These are my colors -- there are many like them, but these ones are mine.
\definecolor{blue}{RGB}{0,114,178}
\definecolor{red}{RGB}{213,94,0}
\definecolor{yellow}{RGB}{240,228,66}
\definecolor{green}{RGB}{0,158,115}

\hypersetup{
  colorlinks=false,
  linkbordercolor = {white},
  linkcolor = {blue}
}

\usepackage{graphicx,stackengine,xcolor}
\newcommand\Circle[1]{%
	\def\useanchorwidth{T}%
	\def\stacktype{L}%
	\stackon[0pt]{#1}{\scalebox{2.0}[1.15]{\textcolor{red}{$\bigcirc$}}}%
}

%% I use a beige off white for my background
\definecolor{MyBackground}{RGB}{255,253,218}

%% Uncomment this if you want to change the background color to something else
%\setbeamercolor{background canvas}{bg=MyBackground}

%% Change the bg color to adjust your transition slide background color!
\newenvironment{transitionframe}{
  \setbeamercolor{background canvas}{bg=white}
  \begin{frame}}{
    \end{frame}
}

\setbeamercolor{frametitle}{fg=blue}
\setbeamercolor{title}{fg=black}
\setbeamertemplate{footline}[frame number]
\setbeamertemplate{navigation symbols}{} 
\setbeamertemplate{itemize items}{-}
\setbeamercolor{itemize item}{fg=blue}
\setbeamercolor{itemize subitem}{fg=blue}
\setbeamercolor{enumerate item}{fg=blue}
\setbeamercolor{enumerate subitem}{fg=blue}
\setbeamercolor{button}{bg=MyBackground,fg=blue,}

%%% TIKZ STUFF
\tikzset{   
	every picture/.style={remember picture,baseline},
	every node/.style={anchor=base,align=center,outer sep=1.5pt},
	every path/.style={thick},
}
\newcommand\marktopleft[1]{%
	\tikz[overlay,remember picture] 
	\node (marker-#1-a) at (-.3em,.3em) {};%
}
\newcommand\markbottomright[2]{%
	\tikz[overlay,remember picture] 
	\node (marker-#1-b) at (0em,0em) {};%
}
\tikzstyle{every picture}+=[remember picture] 
\tikzstyle{mybox} =[draw=black, very thick, rectangle, inner sep=10pt, inner ysep=20pt]
\tikzstyle{fancytitle} =[draw=black,fill=red, text=white]
%%%% END TIKZ STUFF


% If you like road maps, rather than having clutter at the top, have a roadmap show up at the end of each section 
% (and after your introduction)
% Uncomment this is if you want the roadmap!
% \AtBeginSection[]
% {
%    \begin{frame}
%        \frametitle{Roadmap of Talk}
%        \tableofcontents[currentsection]
%    \end{frame}
% }
\setbeamercolor{section in toc}{fg=blue}
\setbeamercolor{subsection in toc}{fg=red}
\setbeamersize{text margin left=1em,text margin right=1em} 

\newenvironment{wideitemize}{\itemize\addtolength{\itemsep}{10pt}}{\enditemize}
\newenvironment{wideenumerate}{\enumerate\addtolength{\itemsep}{10pt}}{\endenumerate}

\usepackage{environ}
\NewEnviron{videoframe}[1]{
  \begin{frame}
    \vspace{-8pt}
    \begin{columns}[onlytextwidth, T] % align columns
      \begin{column}{.58\textwidth}
        \begin{minipage}[t][\textheight][t]
          {\dimexpr\textwidth}
          \vspace{8pt}
          \hspace{4pt} {\Large \sc \textcolor{blue}{#1}}
          \vspace{8pt}
          
          \BODY
        \end{minipage}
      \end{column}%
      \hfill%
      \begin{column}{.42\textwidth}
        \colorbox{green!20}{\begin{minipage}[t][1.2\textheight][t]
            {\dimexpr\textwidth}
            Face goes here
          \end{minipage}}
      \end{column}%
    \end{columns}
  \end{frame}
}

\title[]{\textcolor{blue}{ECN 453: Bertrand Competition}}
\author[PGP]{}
\institute[FRBNY]{\small{\begin{tabular}{c c c}
Nicholas Vreugdenhil \\
\end{tabular}}}
\date{} 

\begin{document}

% Title Slide
\begin{frame}
\maketitle
  \centering
\end{frame}

% INTRO

\begin{frame}{Static Models of Oligopoly}
\begin{wideitemize}
	\item We will start the second part of the course today.
	\begin{wideitemize}
		\item (So, the material we will talk about today will not be on the first mid-term exam)
	\end{wideitemize}
	\item In this part of the course, we will study different \textbf{static models of oligopoly}
\end{wideitemize}
\end{frame}

\begin{frame}{Static Models of Oligopoly}
	\begin{wideitemize}
		\item What are \textbf{static models of oligopoly}?
		\begin{wideitemize}
			\item \underline{`Static'}: this means that the game is only played once 
			\begin{wideitemize}
				\item Note that there might be a sequential element to it e.g. players take turns to choose like in the entry deterrence example we saw in previous lectures, but ultimately the game is played only once
				\item (In the third part of the course, we will contrast `static' games with `dynamic' or `repeated' games which are played again and again)
			\end{wideitemize}
			\item \underline{`Oligopoly'}: this means when we have (potentially) more than one firm competing in the market
		\end{wideitemize}
		\item These models (arguably) the central building blocks of industrial organization, and the models will correspond to different types of competition we observe in real-world markets.
	\end{wideitemize}
\end{frame}

\begin{frame}{Plan}
	  \begin{wideenumerate}
		\item Bertrand competition
		\item Bertrand competition with capacity constraints
	\end{wideenumerate}
\end{frame}

\begin{frame}{Plan}
	\begin{wideenumerate}
		\item \textbf{Bertrand competition}
		\item Bertrand competition with capacity constraints
	\end{wideenumerate}
\end{frame}

\begin{frame}{Bertrand Competition}
	\begin{columns}
		\begin{column}{0.5\textwidth}
			\begin{wideitemize}
				\item Named after Joseph Bertrand 
				\item 11 March 1822 – 5 April 1900
				\item He was a mathematician
			\end{wideitemize}
		\end{column}
		\begin{column}{0.5\textwidth}
			\includegraphics[scale=0.5]{bertrand.jpg}
		\end{column}
	\end{columns}
\end{frame}

\begin{frame}{Bertrand Competition}
	\begin{wideitemize}
		\item \textbf{Players}: two firms (firm 1 and firm 2)
		\item \textbf{Strategies}
		\begin{wideitemize}
			\item Firms choose prices
			\item Prices are set simultaneously
		\end{wideitemize}
		\item \textbf{Payoffs}
		\begin{wideitemize}
			\item Each firm has a constant marginal cost (for now, assume they have the \textit{same} marginal cost)
			\item Consumers buy from the firm which sets the lower price
			\item If the prices are the same, consumers split their demand equally between the firms	
			\item So: total demand is $Q=D(p)$ where $p = min \{ p_1, p_2 \}$
		\end{wideitemize}
	\end{wideitemize}
\end{frame}

\begin{frame}{Bertrand Competition}
	\begin{wideitemize}
		\item Before getting into the implications of Bertrand competition, first consider: what kind of real-world competitive situations does Bertrand competition correspond to?
		\item It is a model of \textbf{price competition} (the decision is `what price do I choose'?)
		\item Best suited to markets where firms offer a \textbf{homogeneous} (identical) product
		\begin{wideitemize}
			\item In other words, the products are \textbf{perfect substitutes}
			\item E.g. gas stations next to each other offer the same gas
			\item If the price is just a few cents lower, all demand will go to the station with the lowest price
			\item If the products were \textit{not} homogeneous, then the strong assumption that `consumers buy from the firm which sets the lower price' would probably be violated and we would need to use a different model.
		\end{wideitemize}
	\end{wideitemize}
\end{frame}


\begin{frame}{Bertrand Competition: Continuous Strategies}
	\begin{wideitemize}
		\item Unlike in the previous lectures, where strategies were limited to just a few discrete alternatives (e.g. choose a high price/low price), here firms can choose \textit{any} price (`continuous strategies').
		\item All of our definitions about best responses, Nash equilibrium, etc still work!
		\item I will first show you the mathematical definitions (which are pretty abstract), but they will hopefully make more sense when applied to the Bertrand competition example.
	\end{wideitemize}
\end{frame}

\begin{frame}{Bertrand Competition: Continuous Strategies - Best Responses}
	\begin{wideitemize}
		\item Here, the \textbf{best responses} are prices (where $p_1^*(p_2)$ is just another way of writing $BR_1(p_2)$ and $p_2^*(p_1)$ is just another way of writing $BR_2(p_1)$):
		\begin{align*}
			p_1^*(p_2) = BR_1(p_2) \in argmax_{p_1} \Pi_1(p_1,p_2) \\
			p_2^*(p_1) = BR_2(p_1) \in argmax_{p_2} \Pi_2(p_1,p_2) 
		\end{align*}
		\item Notation:
		\begin{wideitemize}
			\item $BR_1(p_2)$: the best response of firm 1 given that firm 2 chooses the price $p_2$
			\item $BR_2(p_1)$: the best response of firm 2 given that firm 1 chooses the price $p_1$
			\item $argmax$: the argument(s) (i.e. prices) that maximize profit
			\item $\in$: `in' the set of best responses
			\item $p_1,p_2$: prices that firm 1 and firm 2 set, respectively
			\item $\Pi_1(p_1,p_2)$: profit of firm 1 given that firm 1 and firm 2 set the prices $p_1,p_2$.
		\end{wideitemize}
		%\item In words: `the best response of firm 1 to $p_2$ is the profit maximizing price $p_1$ for firm 1, given firm 2 plays $p_2$'
\end{wideitemize}
\end{frame}

\begin{frame}{Bertrand Competition: Continuous Strategies - Nash Equilibrium}
	\begin{wideitemize}
		\item The idea behind the Nash equilibrium is the same as before: `a Nash equilibrium is two prices ($p_1, p_2$) where each firm has no incentive to unilaterally deviate by choosing a different price'
		\item Equivalently: a Nash equilibrium is a price for each firm $(p_1, p_2)$ that is the best response to the price of the other firm.
		\begin{align*}
			p_1 \in p_1^*(p_2) \\
			p_2 \in p_2^*(p_1) \\
		\end{align*}
	\end{wideitemize}
\end{frame}

\begin{frame}{Bertrand Competition: Best Responses - Firm 1}
	\begin{wideitemize}
		\item Let's apply the definitions on the previous two slides to Bertrand competition, starting with firm 1.
		\item We want to find the best response $p_1^*(p_2)$ 
		\begin{wideitemize}
			\item i.e. the optimal price for firm 1 given firm 2 plays $p_2$.
		\end{wideitemize}
		\item There are three main cases to consider, corresponding to different potential $p_2$
		\begin{wideitemize}
			\item \textbf{Case 1}: Firm 2 plays $p_2 > p^M$ (where $p^M$ is the monopoly price)
			\item \textbf{Case 2}: Firm 2 plays $MC < p_2 \leq p^M$
			\item \textbf{Case 3}: Firm 2 plays $p_2 \leq MC$
		\end{wideitemize}
	\end{wideitemize}
\end{frame}

\begin{frame}{Bertrand Competition: Best Responses - Firm 1}
	\begin{wideitemize}
		\item \textbf{Case 1}: Firm 2 plays $p_2 > p^M$ (where $p^M$ is the monopoly price)
		\item The best response: $p_1^*(p_2) = p^M$.
		\item Why? 
		\begin{wideitemize}
			\item Since $p_2 > p^M$, for any $p_1 \leq p^M$ firm 1 gets all of the total demand.
			\item So, firm 1 can just set prices like it is a monopoly for the entire market and can set the monopoly price.
		\end{wideitemize}
	\end{wideitemize}
\end{frame}

\begin{frame}{Bertrand Competition: Best Responses - Firm 1}
	\begin{wideitemize}
		\item \textbf{Case 2}: Firm 2 plays $MC < p_2 \leq p^M$
		\item The best response: $p_1^*(p_2) = p_2 - \epsilon$
		\begin{wideitemize}
			\item Define $\epsilon$: a tiny change in the price
			\item \underline{Idea}: undercut firm 2 by a tiny amount.
		\end{wideitemize}
		\item Why? (Intuition)
		\begin{wideitemize}
			\item Firm 1 should undercut firm 2 by a tiny amount, and receive the entirety of total demand, rather than setting a price equal to firm 2 (and splitting the market), or setting a price higher than firm 2 (and receiving 0 demand).
		\end{wideitemize}
	\end{wideitemize}
\end{frame}

\begin{frame}{Bertrand Competition: Best Responses - Firm 1}
	\begin{wideitemize}
		\item \textbf{Case 2}: Firm 2 plays $MC < p_2 \leq p^M$
		\item The best response: $p_1^*(p_2) = p_2 - \epsilon$
		\begin{wideitemize}
			\item Define $\epsilon$: a tiny change in the price
			\item \underline{Idea}: undercut firm 2 by a tiny amount.
		\end{wideitemize}
		\item Why? (Math)
		\begin{wideitemize}
			\item Note that the profit at $p_1 = p_2 - \epsilon$ is: 
			\begin{align*}
				\Pi_1(p_1=p_2-\epsilon,p_2) = D(p_2 - \epsilon)(p_2 - \epsilon - MC)
			\end{align*}
			\item Since firm 1 already gets all of total demand at $p_1 = p_2 - \epsilon$, any price lower than this will just reduce profit.
			\item If firm 1 were to set a price that exactly matched $p_2$, the profit would be $0.5 D(p_2)(p_2 - MC) < D(p_2 - \epsilon)(p_2 - \epsilon - MC)$ if $\epsilon$ is sufficiently small
			\item If firm 1 were to set a price $p_1 > p_2$ then firm 1's profits would be = 0 since noone would buy from them, but firm 1's profits would be positive if $\epsilon$ is small enough because $p_2 > MC$.
		\end{wideitemize}
	\end{wideitemize}
\end{frame}

\begin{frame}{Bertrand Competition: Best Responses - Firm 1}
	\begin{wideitemize}
		\item \textbf{Case 3}: Firm 2 plays $MC \geq p_2$
		\item The best response: $p_1^*(p_2) = MC$
		\item Why? 
		\begin{wideitemize}
			\item Any price $p_1 < p_2$ would give firm 1 negative profit, since $p_2 < MC$.
			\item So, set $p_1 = MC$: here you get 0 profit.
			\item (Note: technically any $p_1 > p_2$ is a best response if $MC > p_2$ and any $p_1 \geq MC$ is a best response if $p_2 = MC$, since for all these $p_1$ prices profits are zero for firm 1. But, dealing with these situations just complicates the proof and does not change the equilibrium; the textbook just ignores them and so will I).
		\end{wideitemize}
	\end{wideitemize}
\end{frame}

\begin{frame}{Bertrand Competition: Best Responses - Firm 1, summary}
	\begin{wideitemize}
		\item $p_1^*(p_2) = MC$ if $p_2 \leq MC$
		\item $p_1^*(p_2) = p_2 - \epsilon$ if $MC < p_2 \leq p^M$
		\item $p_1^*(p_2) = p^M$ if $p_2 > p^M$
	\end{wideitemize}
\end{frame}

\begin{frame}{Bertrand Competition: Best Responses - Firm 2, summary}
	\begin{wideitemize}
		\item By exactly the same arguments, we can find the best responses for firm 2:
		\item $p_2^*(p_1) = MC$ if $p_1 \leq MC$
		\item $p_2^*(p_1) = p_1 - \epsilon$ if $MC < p_1 \leq p^M$
		\item $p_2^*(p_1) = p^M$ if $p_1 > p^M$
	\end{wideitemize}
\end{frame}

\begin{frame}{Bertrand Competition: Best Responses - Firm 1, graph}
\begin{figure}
	\includegraphics[scale=0.3]{firm_1_response.jpeg}
\end{figure}
\end{frame}

\begin{frame}{Bertrand Competition: Best Responses - Firm 2, graph}
	\begin{figure}
		\includegraphics[scale=0.36]{firm_2_response.jpeg}
	\end{figure}
\end{frame}

\begin{frame}{Bertrand Competition: Nash Equilibrium, graph}
	\begin{figure}
		\includegraphics[scale=0.34]{nash_equilibrium.jpeg}
	\end{figure}
	\begin{wideitemize}
		\item Nash equilibrium is where the two best response curves cross.
		\item This is at $p_1=p_2=MC$.
	\end{wideitemize}
\end{frame}

\begin{frame}{Bertrand Competition: recipe for how to solve it}
	\begin{wideenumerate}
		\item Find the best responses for firm 1 (i.e. find the optimal prices $p_1$ for all prices $p_2$ that firm 2 could set)
		\item Find the best responses for firm 2 (i.e. find the optimal prices $p_2$ for all prices $p_1$ that firm 1 could set)
		\item Find where the two best responses cross: this is a Nash equilibrium!
	\end{wideenumerate}
	\begin{wideitemize}
		\item Note: This recipe is exactly the same as in the simultaneous games we saw before, the `trick' is splitting the best responses into different cases.
	\end{wideitemize}
\end{frame}

\begin{frame}{Bertrand Competition: alternative way to think about it}
	\begin{wideitemize}
		\item Essentially, \textbf{Bertrand competition is a model of a price war}.
		\item Suppose that firm 1 chooses a price $p^M > p_1 > MC$.
		\begin{wideitemize}
		\item Firm 2 will then slightly undercut it by a tiny amount...
		\item Firm 1 then responds by undercutting by a tiny amount...
		\item Firm 2 then responds by undercutting by a tiny amount...
		\item ...this continues until each firm is setting price = MC.
		\end{wideitemize}
	\end{wideitemize}
\end{frame}

\begin{frame}{Bertrand Competition: example 1}
	\begin{wideitemize}
		\item \textbf{Question:} Assume we have two firms with the same marginal cost (=2) and these firms produce homogenous products. These two firms compete under Bertrand competition and the total demand curve: $Q=100-p$.
		\item 1. What are the best response functions?
		\item 2. Are the prices $p_1=p_2=4$ a Nash equilibrium? 
		\item 3. What is the Nash equilibrium? 
		\item 4. What are the profits of the firms? 
		\item 5. What is consumer surplus? 
	\end{wideitemize}
\end{frame}

\begin{frame}{Bertrand Competition: example 1}
	\begin{wideitemize}
		\item \textbf{Question:} Assume we have two firms with the same marginal cost (=2) and these firms produce homogenous products. These two firms compete under Bertrand competition and the total demand curve: $Q=100-p$.
		\item 1. What are the best response functions? (Write down the three cases as before with the monopoly price $q^M=49,p^M=51$ and $MC=2$)
		\item 2. Are the prices $p_1=p_2=4$ a Nash equilibrium? No, firms will undercut each other.
		\item 3. What is the Nash equilibrium? (set $p=MC=2$)
		\item 4. What are the profits of the firms? ($0$)
		\item 5. What is consumer surplus? ($CS = 0.5 * 98 * 98 = 4802$)
	\end{wideitemize}
\end{frame}

\begin{frame}{Bertrand Competition: example 2}
	\begin{wideitemize}
		\item \textbf{Question:} Assume we have two firms with the different marginal costs $p_1^M>p_2^M>c_1>c_2$ and these firms produce homogenous products. These two firms compete under Bertrand competition with total demand curve denoted $D(p)$.
		\item 1. What are the best response functions?
		\item 2. What is the Nash equilibrium? 
		\item 3. What are the firm profits?
		\item (Note: the textbook does this example but with $c_2>c_1$ on p 191)
	\end{wideitemize}
\end{frame}

\begin{frame}{Bertrand Competition: example 2}
	\begin{wideitemize}
		\item \textbf{Question:} Assume we have two firms with the different marginal costs $p_1^M>p_2^M>c_1>c_2$ and these firms produce homogeneous products. These two firms compete under Bertrand competition with total demand curve denoted $D(p)$.
		\item 1. What are the best response functions? Similar to before, except the same `MC' from before is now replaced with each firm's specific marginal cost (either $c_1$ or $c_2$ for firm 1 and firm 2 respectively)
		\item 2. What is the Nash equilibrium? $p_1=c_1, p_2=c_1-\epsilon$ (note that if $\epsilon$ is really small then $p_2 \approx p_1=c_1$)
		\item 3. What are the firm profits? Firm 1: 0. Firm 2: $D(p_2)(p_2-c_2)=D(c_1-\epsilon)(c_1-\epsilon-c_2) \approx D(c_1)(c_1-c_2)>0$ since $c_1>c_2$. I.e. firm 2 now competes firm 1 down to its marginal cost and makes a profit on the rest of demand.
	\end{wideitemize}
\end{frame}

\begin{frame}{Bertrand Competition}
	\begin{wideitemize}
		\item We just showed that the equilibrium of the Bertrand model is for both firms to price at marginal cost (for the case where they have the same marginal cost) and make zero profit.
		\item This is really surprising! Particularly since the assumptions behind the Bertrand model seemed (at least at first glance) quite reasonable.
		\item The result is so surprising that economists have names for the predictions of the model:
		\begin{wideitemize}
				\item The \textbf{Bertrand trap}: when firms get caught in a fierce price war where they compete prices down to marginal cost.
				\item The \textbf{Bertrand paradox}: the predictions of the Bertrand model imply that as we move from monopoly (1 firm) to duopoly (2 firms), price will change from the monopoly price to the perfect com\part{title}petition price.
				\begin{wideitemize}
					\item If this is the case, no role for competition policy in markets with $>1$ firm!
				\end{wideitemize}
		\end{wideitemize}
	\end{wideitemize}
\end{frame}

\begin{frame}{Bertrand Competition: the Bertrand trap}
	\begin{wideitemize}
		\item Example of the Bertrand trap: the case of encyclopedias
		\begin{wideitemize}
			\item Encyclopedia Britannica: Until the 1990s, 32 volume hardback sold for \$1600
			\item Entry by Microsoft Encarta in the 1990s, sold on CD for less than \$100
			\item In 2000: both Encarta and Britannica sold for \$89.99
		\end{wideitemize}
		\item Example of the Bertrand trap: airline industry
		\begin{wideitemize}
			\item American Airlines: in 1992 introduce a `value pricing' plan that cut fares
			\item Competitors announced even bigger cuts, American Airlines undercut these further, rest of the industry also cut prices
			\item Total cost (to the airlines) of the price war: 4 billion dollars.
		\end{wideitemize}
	\end{wideitemize}
\end{frame}

\begin{frame}{Bertrand Competition}
	\begin{wideitemize}
		\item How do the predictions of the Bertrand model hold up in real-world settings? \pause
		\begin{wideitemize}
			\item The answer is - for the vast majority of markets - not very well.
			\item We typically see firms making positive profits.
			\item Therefore, *something* about the assumptions of the Bertrand model must be wrong.
		\end{wideitemize}
		\end{wideitemize}
\end{frame}

\begin{frame}{Potential solutions to the Bertrand paradox/ways out of the Bertrand trap}
	\begin{wideitemize}
		\item How could we change the assumptions of the Bertrand model to get a more realistic model with positive profits (i.e. firms pricing about MC)? \pause
		\item \textbf{Asymmetric (i.e. different) costs}
		\begin{wideitemize}
			\item Like `cost leadership'
		\end{wideitemize}
		\item \textbf{Product differentiation/branding}
		\begin{wideitemize}
			\item Undercutting the price by a small amount may no longer deliver all of total demand
		\end{wideitemize}
		\item \textbf{Dynamic competition}
		\begin{wideitemize}
			\item What if rivals can retaliate? E.g. you set a high price but threaten to retaliate next time if your rival undercuts you?
			\item We will see this case in Part 3 of the course when we study `dynamic models of competition'
		\end{wideitemize}
		\item \textbf{Capacity constraints}
		\begin{wideitemize}
			\item  What good is undercutting your rival to get total demand if you cannot actually supply all this demand due to capacity constraints?
			%\item We will now look at this case.
		\end{wideitemize}
	\end{wideitemize}
\end{frame}

\begin{frame}{Bertrand Competition: example 3}
	\begin{wideitemize}
		\item \textbf{Question:} Suppose that total demand for golf balls is $Q=90-3P$ and $Q$ is measured in kilos of balls. There are two firms that supply the market. Firm 1 can produce a kilo of balls at a constant unit cost of \$15 whereas firm 2 has a constant unit cost equal to \$10.
		\item 1. Suppose the firms compete in price. How much does each firm sell in a Bertrand equilibrium? What is the market price and what are firms profits?
		\item 2. How would your answer to 1. change if there were three firms, one with unit cost = \$20 and two with unit cost = \$10?
		\item 3. How would your answer to 2b change if firm 1's golf balls were green and endorsed by a famous golfer, but firm 2's were plain and white?
	\end{wideitemize}
\end{frame}

\begin{frame}{Plan}
	\begin{wideenumerate}
		\item Bertrand competition
		\item \textbf{Bertrand competition with capacity constraints}
	\end{wideenumerate}
\end{frame}

\begin{frame}{Bertrand competition with capacity constraints}
	\begin{wideitemize}
		\item \textbf{Setup:}
		\item Same assumptions as before
		\begin{wideitemize}
			\item Firms set prices simultaneously; constant MC (set = 0 for simplicity), homogeneous product
			\item Denote the inverse demand curve by $P(Q)$.
		\end{wideitemize}
		\item But: each firm is constrained to not be able to sell more than $k_i$ (where $i$ is either 1 or 2 depending on the firm)
		\item \textbf{Question:} What is the Nash equilibrium?
		\item \textbf{Answer}(solution on the next slide): $p_1=p_2=P(k_1+k_2)$.
	\end{wideitemize}
\end{frame}

\begin{frame}{Bertrand competition with capacity constraints}
	\begin{wideitemize}
		\item \textbf{Question:} What is the Nash equilibrium?
		\item \textbf{Answer}: $p_1=p_2=P(k_1+k_2)$.
		\item \textbf{Why is this a Nash equilibrium?}
		\begin{wideitemize}
			\item Suppose that Firm 1 is setting $p_1 = P(k_1+k_2)$. Consider Firm 2's decision:
			\item Can Firm 2 do better than $p_1=p_2$ by deviating and setting $p_2 < P(k_1+k_2)$?
			\begin{wideitemize}
				\item No. Although Firm 2 now gets \textit{all} demand, this price actually lowers its profits: it can still only sell $k_2$ units but it now does this at a lower price.
			\end{wideitemize}
			\item Can Firm 2 do better  than $p_1=p_2$ by deviating and setting $p_2 > P(k_1+k_2)$?
			\begin{wideitemize}
				\item Firm 2 now receives positive demand even though it prices above Firm 1, since Firm 1 is capacity constrained.
				\item Specifically, Firm 2 gets the `residual demand' $d_1=D(p_2)-k_1$, with the corresponding marginal revenue curve $r_1$.
				\item But, looking at the diagram on the next slide, $MR>MC=0$ for all the quantities below its capacity $k_2$. Hence, Firm 2 is not setting the optimal price but should in fact lower its price.
			\end{wideitemize}
		\end{wideitemize}
	\end{wideitemize}
\end{frame}

\begin{frame}{Bertrand competition with capacity constraints}
	\begin{figure}
		\includegraphics[scale=0.3]{capacity_constraints.jpeg}
	\end{figure}
\end{frame}

\begin{frame}{Bertrand competition with capacity constraints}
	\begin{wideitemize}
		\item \textbf{Question:} What is the Nash equilibrium?
		\item \textbf{Answer}: $p_1=p_2=P(k_1+k_2)$.
		\item \textbf{Why is this a Nash equilibrium?} (continued)
		\begin{wideitemize}
			\item A similar argument holds for Firm 1's decision.
			\item So, neither firm has an incentive to unilaterally deviate from $p_1=p_2=P(k_1+k_2)$ and so it is a Nash equilibrium.
		\end{wideitemize}
			\item \textbf{Note:} this only works if capacity levels are low. If capacity levels are high, it may be optimal to undercut the rival's price.
	\end{wideitemize}
\end{frame}

\begin{frame}{Summary of key points*}
	%\vspace{-20pt}
	\begin{wideitemize}
		\item Know the assumptions behind Bertrand competition
		\item Know that Bertrand competition is a model of a price war that ends in firms charging marginal cost
		\item Know what is the Bertrand paradox/Bertrand trap
		\item Understand how to construct and read best responses/Nash equilibria with continuous strategies (including graphing them)
		\item Understand how to `solve' the Bertrand paradox with capacity constraints
	\end{wideitemize}
	\vspace{20pt}
	*To clarify, all the material in the slides, problem sets, etc is assessable unless stated otherwise, but I hope this summary might be a useful place to start when studying the material.
\end{frame}

\end{document}
