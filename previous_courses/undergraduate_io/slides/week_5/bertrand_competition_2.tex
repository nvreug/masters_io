\documentclass[notes,11pt, aspectratio=169]{beamer}

\usepackage{pgfpages}
% These slides also contain speaker notes. You can print just the slides,
% just the notes, or both, depending on the setting below. Comment out the want
% you want.
\setbeameroption{hide notes} % Only slide
%\setbeameroption{show only notes} % Only notes
%\setbeameroption{show notes on second screen=right} % Both

%\usepackage[scaled=1.0]{helvet}
\usepackage{array}


\usepackage{tikz}
\usetikzlibrary{calc}
\usetikzlibrary{matrix}
\usetikzlibrary{positioning}

\newcommand{\payoff}[4][below]{\node[#1]at(#2){$(#3,#4)$};}
\usepackage{verbatim}
\setbeamertemplate{note page}{\pagecolor{gray!5}\insertnote}
\usetikzlibrary{positioning}
\usetikzlibrary{snakes}
\usetikzlibrary{calc}
\usetikzlibrary{arrows}
\usetikzlibrary{decorations.markings}
\usetikzlibrary{shapes.misc}
\usetikzlibrary{matrix,shapes,arrows,fit,tikzmark}
\usepackage{amsmath}
\usepackage{mathpazo}
\usepackage{hyperref}
\usepackage{lipsum}
\usepackage{multimedia}
\usepackage{graphicx}
\usepackage{multirow}
\usepackage{graphicx}
\usepackage{dcolumn}
\usepackage{bbm}
\newcolumntype{d}[0]{D{.}{.}{5}}

\usepackage{changepage}
\usepackage{appendixnumberbeamer}
\newcommand{\beginbackup}{
   \newcounter{framenumbervorappendix}
   \setcounter{framenumbervorappendix}{\value{framenumber}}
   \setbeamertemplate{footline}
   {
     \leavevmode%
     \hline
     box{%
       \begin{beamercolorbox}[wd=\paperwidth,ht=2.25ex,dp=1ex,right]{footlinecolor}%
%         \insertframenumber  \hspace*{2ex} 
       \end{beamercolorbox}}%
     \vskip0pt%
   }
 }
\newcommand{\backupend}{
   \addtocounter{framenumbervorappendix}{-\value{framenumber}}
   \addtocounter{framenumber}{\value{framenumbervorappendix}} 
}


\usepackage{graphicx}
\usepackage[space]{grffile}
\usepackage{booktabs}

% These are my colors -- there are many like them, but these ones are mine.
\definecolor{blue}{RGB}{0,114,178}
\definecolor{red}{RGB}{213,94,0}
\definecolor{yellow}{RGB}{240,228,66}
\definecolor{green}{RGB}{0,158,115}

\hypersetup{
  colorlinks=false,
  linkbordercolor = {white},
  linkcolor = {blue}
}

\usepackage{graphicx,stackengine,xcolor}
\newcommand\Circle[1]{%
	\def\useanchorwidth{T}%
	\def\stacktype{L}%
	\stackon[0pt]{#1}{\scalebox{2.0}[1.15]{\textcolor{red}{$\bigcirc$}}}%
}

%% I use a beige off white for my background
\definecolor{MyBackground}{RGB}{255,253,218}

%% Uncomment this if you want to change the background color to something else
%\setbeamercolor{background canvas}{bg=MyBackground}

%% Change the bg color to adjust your transition slide background color!
\newenvironment{transitionframe}{
  \setbeamercolor{background canvas}{bg=white}
  \begin{frame}}{
    \end{frame}
}

\setbeamercolor{frametitle}{fg=blue}
\setbeamercolor{title}{fg=black}
\setbeamertemplate{footline}[frame number]
\setbeamertemplate{navigation symbols}{} 
\setbeamertemplate{itemize items}{-}
\setbeamercolor{itemize item}{fg=blue}
\setbeamercolor{itemize subitem}{fg=blue}
\setbeamercolor{enumerate item}{fg=blue}
\setbeamercolor{enumerate subitem}{fg=blue}
\setbeamercolor{button}{bg=MyBackground,fg=blue,}

%%% TIKZ STUFF
\tikzset{   
	every picture/.style={remember picture,baseline},
	every node/.style={anchor=base,align=center,outer sep=1.5pt},
	every path/.style={thick},
}
\newcommand\marktopleft[1]{%
	\tikz[overlay,remember picture] 
	\node (marker-#1-a) at (-.3em,.3em) {};%
}
\newcommand\markbottomright[2]{%
	\tikz[overlay,remember picture] 
	\node (marker-#1-b) at (0em,0em) {};%
}
\tikzstyle{every picture}+=[remember picture] 
\tikzstyle{mybox} =[draw=black, very thick, rectangle, inner sep=10pt, inner ysep=20pt]
\tikzstyle{fancytitle} =[draw=black,fill=red, text=white]
%%%% END TIKZ STUFF


% If you like road maps, rather than having clutter at the top, have a roadmap show up at the end of each section 
% (and after your introduction)
% Uncomment this is if you want the roadmap!
% \AtBeginSection[]
% {
%    \begin{frame}
%        \frametitle{Roadmap of Talk}
%        \tableofcontents[currentsection]
%    \end{frame}
% }
\setbeamercolor{section in toc}{fg=blue}
\setbeamercolor{subsection in toc}{fg=red}
\setbeamersize{text margin left=1em,text margin right=1em} 

\newenvironment{wideitemize}{\itemize\addtolength{\itemsep}{10pt}}{\enditemize}
\newenvironment{wideenumerate}{\enumerate\addtolength{\itemsep}{10pt}}{\endenumerate}

\usepackage{environ}
\NewEnviron{videoframe}[1]{
  \begin{frame}
    \vspace{-8pt}
    \begin{columns}[onlytextwidth, T] % align columns
      \begin{column}{.58\textwidth}
        \begin{minipage}[t][\textheight][t]
          {\dimexpr\textwidth}
          \vspace{8pt}
          \hspace{4pt} {\Large \sc \textcolor{blue}{#1}}
          \vspace{8pt}
          
          \BODY
        \end{minipage}
      \end{column}%
      \hfill%
      \begin{column}{.42\textwidth}
        \colorbox{green!20}{\begin{minipage}[t][1.2\textheight][t]
            {\dimexpr\textwidth}
            Face goes here
          \end{minipage}}
      \end{column}%
    \end{columns}
  \end{frame}
}

\title[]{\textcolor{blue}{ECN 453: Bertrand Competition}}
\author[PGP]{}
\institute[FRBNY]{\small{\begin{tabular}{c c c}
Nicholas Vreugdenhil \\
\end{tabular}}}
\date{} 

\begin{document}

% Title Slide
\begin{frame}
\maketitle
  \centering
\end{frame}

% INTRO

\begin{frame}{Plan}
	  \begin{wideenumerate}
		\item Bertrand competition continued
		\item Bertrand competition with capacity constraints
	\end{wideenumerate}
\end{frame}

\begin{frame}{Plan}
	\begin{wideenumerate}
		\item \textbf{Bertrand competition continued}
		\item Bertrand competition with capacity constraints
	\end{wideenumerate}
\end{frame}

\begin{frame}{Bertrand Competition: Nash Equilibrium, graph}
	\begin{figure}
		\includegraphics[scale=0.34]{nash_equilibrium.jpeg}
	\end{figure}
	\begin{wideitemize}
		\item Nash equilibrium is where the two best response curves cross.
		\item This is at $p_1=p_2=MC$.
	\end{wideitemize}
\end{frame}

\begin{frame}{Bertrand Competition: recipe for how to solve it}
	\begin{wideenumerate}
		\item Find the best responses for firm 1 (i.e. find the optimal prices $p_1$ for all prices $p_2$ that firm 2 could set)
		\item Find the best responses for firm 2 (i.e. find the optimal prices $p_2$ for all prices $p_1$ that firm 1 could set)
		\item Find where the two best responses cross: this is a Nash equilibrium!
	\end{wideenumerate}
	\begin{wideitemize}
		\item Note: This recipe is exactly the same as in the simultaneous games we saw before, the `trick' is splitting the best responses into different cases.
	\end{wideitemize}
\end{frame}

\begin{frame}{Bertrand Competition: alternative way to think about it}
	\begin{wideitemize}
		\item Essentially, \textbf{Bertrand competition is a model of a price war}.
		\item Suppose that firm 1 chooses a price $p^M > p_1 > MC$.
		\begin{wideitemize}
		\item Firm 2 will then slightly undercut it by a tiny amount...
		\item Firm 1 then responds by undercutting by a tiny amount...
		\item Firm 2 then responds by undercutting by a tiny amount...
		\item ...this continues until each firm is setting price = MC.
		\end{wideitemize}
	\end{wideitemize}
\end{frame}

\begin{frame}{Bertrand Competition: example 1}
	\begin{wideitemize}
		\item \textbf{Question:} Assume we have two firms with the same marginal cost (=2) and these firms produce homogenous products. These two firms compete under Bertrand competition and the total demand curve: $Q=100-p$.
		\item 1. What are the best response functions?
		\item 2. Are the prices $p_1=p_2=4$ a Nash equilibrium? 
		\item 3. What is the Nash equilibrium? 
		\item 4. What are the profits of the firms? 
		\item 5. What is consumer surplus? 
	\end{wideitemize}
\end{frame}

\begin{frame}{Bertrand Competition: example 1}
	\begin{wideitemize}
		\item \textbf{Question:} Assume we have two firms with the same marginal cost (=2) and these firms produce homogenous products. These two firms compete under Bertrand competition and the total demand curve: $Q=100-p$.
		\item 1. What are the best response functions? (Write down the three cases as before with the monopoly price $q^M=49,p^M=51$ and $MC=2$)
		\item 2. Are the prices $p_1=p_2=4$ a Nash equilibrium? No, firms will undercut each other.
		\item 3. What is the Nash equilibrium? (set $p=MC=2$)
		\item 4. What are the profits of the firms? ($0$)
		\item 5. What is consumer surplus? ($CS = 0.5 * 98 * 98 = 4802$)
	\end{wideitemize}
\end{frame}

\begin{frame}{Bertrand Competition: example 2}
	\begin{wideitemize}
		\item \textbf{Question:} Assume we have two firms with the different marginal costs $p_1^M>p_2^M>c_1>c_2$ and these firms produce homogenous products. These two firms compete under Bertrand competition with total demand curve denoted $D(p)$.
		\item 1. What are the best response functions?
		\item 2. What is the Nash equilibrium? 
		\item 3. What are the firm profits?
		\item (Note: the textbook does this example but with $c_2>c_1$ on p 191)
	\end{wideitemize}
\end{frame}

\begin{frame}{Bertrand Competition: example 2}
	\begin{wideitemize}
		\item \textbf{Question:} Assume we have two firms with the different marginal costs $p_1^M>p_2^M>c_1>c_2$ and these firms produce homogeneous products. These two firms compete under Bertrand competition with total demand curve denoted $D(p)$.
		\item 1. What are the best response functions? Similar to before, except the same `MC' from before is now replaced with each firm's specific marginal cost (either $c_1$ or $c_2$ for firm 1 and firm 2 respectively)
		\item 2. What is the Nash equilibrium? $p_1=c_1, p_2=c_1-\epsilon$ (note that if $\epsilon$ is really small then $p_2 \approx p_1=c_1$)
		\item 3. What are the firm profits? Firm 1: 0. Firm 2: $D(p_2)(p_2-c_2)=D(c_1-\epsilon)(c_1-\epsilon-c_2) \approx D(c_1)(c_1-c_2)>0$ since $c_1>c_2$. I.e. firm 2 now competes firm 1 down to its marginal cost and makes a profit on the rest of demand.
	\end{wideitemize}
\end{frame}

\begin{frame}{Bertrand Competition}
	\begin{wideitemize}
		\item We just showed that the equilibrium of the Bertrand model is for both firms to price at marginal cost (for the case where they have the same marginal cost) and make zero profit.
		\item This is really surprising! Particularly since the assumptions behind the Bertrand model seemed (at least at first glance) quite reasonable.
		\item The result is so surprising that economists have names for the predictions of the model:
		\begin{wideitemize}
				\item The \textbf{Bertrand trap}: when firms get caught in a fierce price war where they compete prices down to marginal cost.
				\item The \textbf{Bertrand paradox}: the predictions of the Bertrand model imply that as we move from monopoly (1 firm) to duopoly (2 firms), price will change from the monopoly price to the perfect com\part{title}petition price.
				\begin{wideitemize}
					\item If this is the case, no role for competition policy in markets with $>1$ firm!
				\end{wideitemize}
		\end{wideitemize}
	\end{wideitemize}
\end{frame}

\begin{frame}{Bertrand Competition: the Bertrand trap}
	\begin{wideitemize}
		\item Example of the Bertrand trap: the case of encyclopedias
		\begin{wideitemize}
			\item Encyclopedia Britannica: Until the 1990s, 32 volume hardback sold for \$1600
			\item Entry by Microsoft Encarta in the 1990s, sold on CD for less than \$100
			\item In 2000: both Encarta and Britannica sold for \$89.99
		\end{wideitemize}
		\item Example of the Bertrand trap: airline industry
		\begin{wideitemize}
			\item American Airlines: in 1992 introduce a `value pricing' plan that cut fares
			\item Competitors announced even bigger cuts, American Airlines undercut these further, rest of the industry also cut prices
			\item Total cost (to the airlines) of the price war: 4 billion dollars.
		\end{wideitemize}
	\end{wideitemize}
\end{frame}

\begin{frame}{Bertrand Competition}
	\begin{wideitemize}
		\item How do the predictions of the Bertrand model hold up in real-world settings? \pause
		\begin{wideitemize}
			\item The answer is - for the vast majority of markets - not very well.
			\item We typically see firms making positive profits.
			\item Therefore, *something* about the assumptions of the Bertrand model must be wrong.
		\end{wideitemize}
		\end{wideitemize}
\end{frame}

\begin{frame}{Potential solutions to the Bertrand paradox/ways out of the Bertrand trap}
	\begin{wideitemize}
		\item How could we change the assumptions of the Bertrand model to get a more realistic model with positive profits (i.e. firms pricing about MC)? \pause
		\item \textbf{Asymmetric (i.e. different) costs}
		\begin{wideitemize}
			\item Like `cost leadership'
		\end{wideitemize}
		\item \textbf{Product differentiation/branding}
		\begin{wideitemize}
			\item Undercutting the price by a small amount may no longer deliver all of total demand
		\end{wideitemize}
		\item \textbf{Dynamic competition}
		\begin{wideitemize}
			\item What if rivals can retaliate? E.g. you set a high price but threaten to retaliate next time if your rival undercuts you?
			\item We will see this case in Part 3 of the course when we study `dynamic models of competition'
		\end{wideitemize}
		\item \textbf{Capacity constraints}
		\begin{wideitemize}
			\item  What good is undercutting your rival to get total demand if you cannot actually supply all this demand due to capacity constraints?
			%\item We will now look at this case.
		\end{wideitemize}
	\end{wideitemize}
\end{frame}

\begin{frame}{Bertrand Competition: example 3}
	\begin{wideitemize}
		\item \textbf{Question:} Suppose that total demand for golf balls is $Q=90-3P$ and $Q$ is measured in kilos of balls. There are two firms that supply the market. Firm 1 can produce a kilo of balls at a constant unit cost of \$15 whereas firm 2 has a constant unit cost equal to \$10.
		\item 1. Suppose the firms compete in price. How much does each firm sell in a Bertrand equilibrium? What is the market price and what are firms profits?
		\item 2. How would your answer to 1. change if there were three firms, one with unit cost = \$20 and two with unit cost = \$10?
		\item 3. How would your answer to 2b change if firm 1's golf balls were green and endorsed by a famous golfer, but firm 2's were plain and white?
	\end{wideitemize}
\end{frame}

\begin{frame}{Plan}
	\begin{wideenumerate}
		\item Bertrand competition
		\item \textbf{Bertrand competition with capacity constraints}
	\end{wideenumerate}
\end{frame}

\begin{frame}{Bertrand competition with capacity constraints}
	\begin{wideitemize}
		\item \textbf{Setup:}
		\item Same assumptions as before
		\begin{wideitemize}
			\item Firms set prices simultaneously; constant MC (set = 0 for simplicity), homogeneous product
			\item Denote the inverse demand curve by $P(Q)$.
		\end{wideitemize}
		\item But: each firm is constrained to not be able to sell more than $k_i$ (where $i$ is either 1 or 2 depending on the firm)
		\item \textbf{Question:} What is the Nash equilibrium?
		\item \textbf{Answer}(solution on the next slide): $p_1=p_2=P(k_1+k_2)$.
	\end{wideitemize}
\end{frame}

\begin{frame}{Bertrand competition with capacity constraints}
	\begin{wideitemize}
		\item \textbf{Question:} What is the Nash equilibrium?
		\item \textbf{Answer}: $p_1=p_2=P(k_1+k_2)$.
		\item \textbf{Why is this a Nash equilibrium?}
		\begin{wideitemize}
			\item Suppose that Firm 2 is setting $p_2 = P(k_1+k_2)$. Consider Firm 1's decision:
			\item Can Firm 1 do better than $p_1=p_2$ by deviating and setting $p_1 < P(k_1+k_2)$?
			\begin{wideitemize}
				\item No. Although Firm 1 now gets \textit{all} demand, this price actually lowers its profits: it can still only sell $k_1$ units but it now does this at a lower price.
			\end{wideitemize}m
		
			\item Can Firm 1 do better  than $p_1=p_2$ by deviating and setting $p_1 > P(k_1+k_2)$?
			\begin{wideitemize}
				\item Firm 1 now receives positive demand even though it prices above Firm 2, since Firm 2 is capacity constrained.
				\item Specifically, Firm 2 gets the `residual demand' $d_1=D(p_1)-k_2$, with the corresponding marginal revenue curve $r_1$.
				\item But, looking at the diagram on the next slide, $MR>MC=0$ for all the quantities below its capacity $k_1$. Hence, Firm 2 is not setting the optimal price but should in fact lower its price.
			\end{wideitemize}
		\end{wideitemize}
	\end{wideitemize}
\end{frame}

\begin{frame}{Bertrand competition with capacity constraints}
	\begin{figure}
		\includegraphics[scale=0.3]{capacity_constraints.jpeg}
	\end{figure}
\end{frame}

\begin{frame}{Bertrand competition with capacity constraints}
	\begin{wideitemize}
		\item \textbf{Question:} What is the Nash equilibrium?
		\item \textbf{Answer}: $p_1=p_2=P(k_1+k_2)$.
		\item \textbf{Why is this a Nash equilibrium?} (continued)
		\begin{wideitemize}
			\item A similar argument holds for Firm 2's decision.
			\item So, neither firm has an incentive to unilaterally deviate from $p_1=p_2=P(k_1+k_2)$ and so it is a Nash equilibrium.
		\end{wideitemize}
			\item \textbf{Note:} this only works if capacity levels are low. If capacity levels are high, it may be optimal to undercut the rival's price.
	\end{wideitemize}
\end{frame}

\begin{frame}{Summary of key points*}
	%\vspace{-20pt}
	\begin{wideitemize}
		\item Know the assumptions behind Bertrand competition
		\item Know that Bertrand competition is a model of a price war that ends in firms charging marginal cost
		\item Know what is the Bertrand paradox/Bertrand trap
		\item Understand how to construct and read best responses/Nash equilibria with continuous strategies (including graphing them)
		\item Understand how to `solve' the Bertrand paradox with capacity constraints
	\end{wideitemize}
	\vspace{20pt}
	*To clarify, all the material in the slides, problem sets, etc is assessable unless stated otherwise, but I hope this summary might be a useful place to start when studying the material.
\end{frame}

\end{document}
