% Don't touch this %%%%%%%%%%%%%%%%%%%%%%%%%%%%%%%%%%%%%%%%%%%
\documentclass[11pt]{article}
\usepackage{fullpage}
\usepackage[left=1.0in,top=1.0in,right=1.0in,bottom=1.0in,headheight=3ex,headsep=3ex]{geometry}
\usepackage{graphicx}
\usepackage{float}
\usepackage{adjustbox}
\usepackage{tikz}
\usetikzlibrary{calc}
\usetikzlibrary{matrix}
\usetikzlibrary{positioning}
\usepackage{comment}

\tikzset{   
	every picture/.style={remember picture,baseline},
	every node/.style={anchor=base,align=center,outer sep=1.5pt},
	every path/.style={thick},
}
\newcommand\marktopleft[1]{%
	\tikz[overlay,remember picture] 
	\node (marker-#1-a) at (-.3em,.3em) {};%
}
\newcommand\markbottomright[2]{%
	\tikz[overlay,remember picture] 
	\node (marker-#1-b) at (0em,0em) {};%
}
\tikzstyle{every picture}+=[remember picture] 
\tikzstyle{mybox} =[draw=black, very thick, rectangle, inner sep=10pt, inner ysep=20pt]
\tikzstyle{fancytitle} =[draw=black,fill=red, text=white]


\usepackage{graphicx,stackengine,xcolor}
\newcommand\Circle[1]{%
	\def\useanchorwidth{T}%
	\def\stacktype{L}%
	\stackon[0pt]{#1}{\scalebox{2.0}[1.15]{\textcolor{red}{$\bigcirc$}}}%
}
\newcommand{\blankline}{\quad\pagebreak[2]}
%%%%%%%%%%%%%%%%%%%%%%%%%%%%%%%%%%%%%%%%%%%%%%%%%%%%%%%%%%%%%%

% Modify Course title, instructor name, semester here %%%%%%%%

\title{ECN 453: Mid-term Exam 2 (Practice)}
%\date{Fall, 2021}

%%%%%%%%%%%%%%%%%%%%%%%%%%%%%%%%%%%%%%%%%%%%%%%%%%%%%%%%%%%%%%

% Don't touch this %%%%%%%%%%%%%%%%%%%%%%%%%%%%%%%%%%%%%%%%%%%
%\usepackage[sc]{mathpazo}
\linespread{1.3} % Palatino needs more leading (space between lines)
\usepackage[T1]{fontenc}
\usepackage[mmddyyyy]{datetime}% http://ctan.org/pkg/datetime
\usepackage{advdate}% http://ctan.org/pkg/advdate
%\newdateformat{syldate}{\twodigit{\THEMONTH}/\twodigit{\THEDAY}}
\newsavebox{\MONDAY}\savebox{\MONDAY}{Mon}% Mon
\newcommand{\week}[1]{%
%  \cleardate{mydate}% Clear date
% \newdate{mydate}{\the\day}{\the\month}{\the\year}% Store date
  \paragraph*{\kern-2ex\quad #1, \syldate{\today} - \AdvanceDate[4]\syldate{\today}:}% Set heading  \quad #1
%  \setbox1=\hbox{\shortdayofweekname{\getdateday{mydate}}{\getdatemonth{mydate}}{\getdateyear{mydate}}}%
  \ifdim\wd1=\wd\MONDAY
    \AdvanceDate[7]
  \else
    \AdvanceDate[7]
  \fi%
}
\usepackage{setspace}
\usepackage{multicol}
%\usepackage{indentfirst}
\usepackage{fancyhdr,lastpage}
\usepackage{url}
\pagestyle{fancy}
\usepackage{hyperref}
\usepackage{lastpage}
\usepackage{amsmath}
\usepackage{layout}
\usepackage[shortlabels]{enumitem}
%\renewcommand{\theenumi}{\alph{enumi}}


\lhead{}
\chead{}
%%%%%%%%%%%%%%%%%%%%%%%%%%%%%%%%%%%%%%%%%%%%%%%%%%%%%%%%%%%%%%

% Modify header here %%%%%%%%%%%%%%%%%%%%%%%%%%%%%%%%%%%%%%%%%
\rhead{\footnotesize ECN 453: Mid-term Exam 2 (Practice)}

%%%%%%%%%%%%%%%%%%%%%%%%%%%%%%%%%%%%%%%%%%%%%%%%%%%%%%%%%%%%%%
% Don't touch this %%%%%%%%%%%%%%%%%%%%%%%%%%%%%%%%%%%%%%%%%%%
\lfoot{}
\cfoot{\small \thepage/\pageref*{LastPage}}
\rfoot{}

\usepackage{array, xcolor}
\usepackage{color,hyperref}
\definecolor{clemsonorange}{HTML}{EA6A20}
\hypersetup{colorlinks,breaklinks,linkcolor=clemsonorange,urlcolor=clemsonorange,anchorcolor=clemsonorange,citecolor=black}

\date{} 

\begin{document}
\maketitle

\subsection*{1. Short answer questions (30 points)}

\begin{enumerate}[(a)]
	\item Hotelling
	\item True.
	\item True.
	\item 20.
	\item 80.
	\item False.
	\item Monopoly output.
	\item 0.
	\item 0.
	\item Marginal cost.
	
\end{enumerate}


\subsection*{2. Cournot Competition With Asymmetric Marginal Costs (30 points)}

\begin{enumerate}[(a)]
		\item Given:
		\begin{align*}
			Q = 100 - p \implies \\ 
			p = 100 - (q_1+q_2)  
		\end{align*}
		For firms $i \in {1,2}$,profit  maximization problem is (denoting by $-i$ the firm that is not $i$): 
		\begin{equation*}
		 \pi_{i}  = (100 - q_i -q_{-i})q_i - c_iq_i 
		\end{equation*}
		\begin{align*}
        MR = MC \implies \\
       q_i(q_{-i}) = \frac{100 - c_i - q_{-i}}{2} 
		\end{align*}
		Therefore, the respective best responses for both firms can be derived as follows:
		\begin{align*}
		q_1(q_2) = 30 - q_2/2 \\
		q_2(q_1) = 35 - q_1/2 
		\end{align*}
		Both graphs are shown in part c
		
		
		
		\item Substitute one best response into another and solve:
		\begin{align*}
		q_1 = 16.66, q_2 = 26.67
		\end{align*}

		\item If marginal costs of both firms reduce by 10, then the new best responses can be written as:
		\begin{align*}
		q_1(q_2) = 35 - q_2/2 \\
		q_2(q_1) = 40 - q_1/2 
		\end{align*}	
	
	
		
	\begin{figure}[!htb] 
		\centering
				\includegraphics[width=\textwidth]{m2f2.pdf}		
	\end{figure}
		
\pagebreak 


\end{enumerate}

\subsection*{3. Stackelberg (30 points)}
\begin{enumerate}[(a)]
	\item 
Solve by backward induction. Firm 2's profit maximization problem is:
\begin{equation*}
\pi_2(q_1) = (100-2q_1-2q_2)q_2-20q_2
\end{equation*}
\begin{align*}
MR=MC \implies \\
100-2q_1-4q_2-20 = 0 \implies \\
q_2(q_1) = \frac{80-2q_1}{4}
\end{align*}

Plugging this into firm 1's problem:
\begin{align*}
\pi_1 = (100-2q_1-2q_2(q_1))q_2-4q_1^2 \\ 
\pi_1 = (60-q_1)q_1-4q_1^2 
\end{align*}
Setting FOC = 0
\begin{align*}
60 - 2q_1 -8q_1 = 0 \implies \\ 
q_1 = 6 \implies q_2 = 17
\end{align*}
\end{enumerate}
\begin{comment}
\item We know from the part (a) that 
\begin{equation*}
q_2(q_1) = \frac{99-2q_1}{4}
\end{equation*}

Use that to find the profit function for firm 2 in terms of output of firm 1:
\begin{align*}
\pi_2 = (100-2q_1-2q_2(q_1) - 1)q_2(q_1) \\ 
\pi_2 = (99-2q_1-2(\frac{99-2q_1}{4}))(\frac{99-2q_1}{4}) \implies \\
\pi_2 = \frac{(99-2q_1)}{2}\frac{(99-2q_1)}{4} \implies \\
\pi_2 = \frac{(99-2q_1)^2}{8}
\end{align*}

In order to deter entry, we need the following condition:
\begin{align*}
E \geq> \pi_2(q_1) \\
16 \geq \frac{(99-2q_1)^2}{8} \implies \\
q_1 \geq \frac{99 - 8\sqrt{2}}{2}
\end{align*}
\end{comment}

\subsection*{3. Hotelling Model (30 points)}

\begin{enumerate}[(a)]
\item 
Let's assume prices are set in such a way that the indifferent consumer will be in between the two firms, i.e. a the indifferent consumer has address $x'$ where $x' \in [0,0.6]$. Therefore, all consumers to the right of firm 2 will choose firm 2 due to transport costs. The indifference consumer for consumer at $x'$ is:

\begin{equation*}
	tx'+p_1 = t(0.6-x')+p_2
\end{equation*}

So consumers to the left of $x'$ buy from Firm 1, and consumers to the right of $x'$ including those in between 0.6 and 1 buy from Firm 2. So Firm 1's demand is $x'$ and Firm 2's demand is $(0.6-x)+0.4$

Solving the above equation for $x'$ and substituting $t = 0.5$, we can get the demands for each firm:
\begin{align*}
	q_1 = 100 \times (0.3 + (p_2-p_1)) \\
	q_2 = 100 \times (0.7 + (p_1-p_2))
\end{align*} 

\item 
Using demands, we can get the payoffs:
\begin{align*}
\pi_1 = q_1(p_1-c_1) =100( 0.3 + (p_2-p_1))p_1  
\end{align*} 
Taking the derivative with respect to price and setting to zero:

\begin{align*}
	\frac{0.3+p_2}{2} = p_1(p_2) 
\end{align*}
At $p_2=0.5$, $p_1=0.4$.


\end{enumerate}
\end{document}