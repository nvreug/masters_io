\documentclass[notes,11pt, aspectratio=169]{beamer}

\usepackage{pgfpages}
\setbeameroption{hide notes} % Only slide

\usepackage{array}
\usepackage{tikz}
\usepackage{verbatim}
\setbeamertemplate{note page}{\pagecolor{gray!5}\insertnote}
\usetikzlibrary{positioning}
\usetikzlibrary{snakes}
\usetikzlibrary{calc}
\usetikzlibrary{arrows}
\usetikzlibrary{decorations.markings}
\usetikzlibrary{shapes.misc}
\usetikzlibrary{matrix,shapes,arrows,fit,tikzmark}
\usepackage{amsmath}
\usepackage{mathpazo}
\usepackage{hyperref}
\usepackage{lipsum}
\usepackage{multimedia}
\usepackage{graphicx}
\usepackage{multirow}
\usepackage{dcolumn}
\usepackage{bbm}
\newcolumntype{d}[0]{D{.}{.}{5}}

\usepackage{changepage}
\usepackage{appendixnumberbeamer}

\usepackage[space]{grffile}
\usepackage{booktabs}

% Colors
\definecolor{blue}{RGB}{0,114,178}
\definecolor{red}{RGB}{213,94,0}
\definecolor{yellow}{RGB}{240,228,66}
\definecolor{green}{RGB}{0,158,115}

\hypersetup{
  colorlinks=false,
  linkbordercolor = {white},
  linkcolor = {blue}
}

\definecolor{MyBackground}{RGB}{255,253,218}

\newenvironment{transitionframe}{
  \setbeamercolor{background canvas}{bg=white}
  \begin{frame}}{
    \end{frame}
}

\setbeamercolor{frametitle}{fg=blue}
\setbeamercolor{title}{fg=black}
\setbeamertemplate{footline}[frame number]
\setbeamertemplate{navigation symbols}{}
\setbeamertemplate{itemize items}{-}
\setbeamercolor{itemize item}{fg=blue}
\setbeamercolor{itemize subitem}{fg=blue}
\setbeamercolor{enumerate item}{fg=blue}
\setbeamercolor{enumerate subitem}{fg=blue}
\setbeamercolor{button}{bg=MyBackground,fg=blue,}

\setbeamercolor{section in toc}{fg=blue}
\setbeamercolor{subsection in toc}{fg=red}
\setbeamersize{text margin left=1em,text margin right=1em}

\newenvironment{wideitemize}{\itemize\addtolength{\itemsep}{10pt}}{\enditemize}
\newenvironment{wideenumerate}{\enumerate\addtolength{\itemsep}{10pt}}{\endenumerate}

\title[]{\textcolor{blue}{ECN 594: Collusion}}
\author[PGP]{}
\institute[FRBNY]{\small{\begin{tabular}{c c c}
Nicholas Vreugdenhil \\
\end{tabular}}}
\date{\today}

\begin{document}

% Title Slide
\begin{frame}
\maketitle
  \centering
\end{frame}

\begin{frame}{Plan for today}
  \begin{wideenumerate}
    \item Collusion refresher (from ECN 532)
    \item Critical discount factor with $N$ firms
    \item Cournot vs Bertrand collusion
    \item[] \rule{0.5\textwidth}{0.4pt}
    \item Detection and fines
    \item Leniency programs
    \item Antitrust enforcement
  \end{wideenumerate}
\end{frame}

%%%%%%%%%%%%%%%%%%%%%%%%%%%%%%%%%%%%%%%%%%%%%%%%%%%%%%%%%%%%%
% PART 1: COLLUSION THEORY
%%%%%%%%%%%%%%%%%%%%%%%%%%%%%%%%%%%%%%%%%%%%%%%%%%%%%%%%%%%%%

\begin{transitionframe}
	\begin{center}
		{\Huge Part 1: Collusion Theory}
	\end{center}
\end{transitionframe}

\begin{frame}{From ECN 532: Collusion basics}
	\begin{wideitemize}
		\item \textbf{Collusion:} Firms coordinate to raise prices/restrict output
		\item Problem: each firm has incentive to deviate (undercut)
		\item \textbf{Solution:} Repeated game with punishment
		\item \textbf{Grim trigger strategy:}
		\begin{wideitemize}
			\vspace{5pt}
			\item Collude as long as everyone colludes
			\item If anyone deviates $\rightarrow$ Nash forever
		\end{wideitemize}
		\item You derived this in Hector's class
	\end{wideitemize}
\end{frame}

\begin{frame}{The collusion condition}
	\begin{wideitemize}
		\item \textbf{Three profit levels:}
		\begin{wideitemize}
			\vspace{5pt}
			\item $\pi^C$: Collusive profit (per period)
			\item $\pi^D$: Deviation profit (one-shot gain)
			\item $\pi^{NE}$: Nash equilibrium profit (punishment)
		\end{wideitemize}
		\item \textbf{Collusion sustained if:}
		\begin{align*}
			\frac{\pi^C}{1 - \delta} \geq \pi^D + \frac{\delta \pi^{NE}}{1 - \delta}
		\end{align*}
		\item Rearranging:
		\begin{align*}
			\delta \geq \delta^* = \frac{\pi^D - \pi^C}{\pi^D - \pi^{NE}}
		\end{align*}
	\end{wideitemize}
\end{frame}

\begin{frame}{Critical discount factor: intuition}
	\begin{wideitemize}
		\item $\delta^* = \frac{\pi^D - \pi^C}{\pi^D - \pi^{NE}}$
		\item \textbf{Numerator:} Gain from deviating ($\pi^D - \pi^C$)
		\item \textbf{Denominator:} Total loss from punishment ($\pi^D - \pi^{NE}$)
		\item \textbf{Higher $\delta^*$ means collusion is harder}
		\begin{wideitemize}
			\vspace{5pt}
			\item Need more patient firms
			\item More frequent interaction helps (reduces effective $\delta$)
		\end{wideitemize}
	\end{wideitemize}
\end{frame}

\begin{frame}{Cournot collusion with $N$ firms}
	\begin{wideitemize}
		\item Linear demand: $P = a - bQ$, symmetric firms with $MC = c$
		\item \textbf{Collusive profit per firm:}
		\begin{align*}
			\pi^C = \frac{\pi^M}{N} = \frac{(a-c)^2}{4bN}
		\end{align*}
		\item \textbf{Nash profit per firm:}
		\begin{align*}
			\pi^{NE} = \frac{(a-c)^2}{b(N+1)^2}
		\end{align*}
		\item \textbf{Deviation profit:} Best response to $N-1$ firms playing $q^C$
	\end{wideitemize}
\end{frame}

\begin{frame}{Critical discount factor: Cournot formula}
	\begin{wideitemize}
		\item For symmetric linear Cournot with $N$ firms:
		\begin{align*}
			\delta^* = \frac{(N+1)^2}{N^2 + (N+1)^2}
		\end{align*}
		\item \textbf{Examples:}
		\begin{center}
			\begin{tabular}{|c|c|}
				\hline
				$N$ & $\delta^*$ \\
				\hline
				2 & $9/17 \approx 0.53$ \\
				3 & $16/25 = 0.64$ \\
				4 & $25/41 \approx 0.61$ \\
				10 & $121/221 \approx 0.55$ \\
				\hline
			\end{tabular}
		\end{center}
		\item Key insight: Collusion harder with more firms
	\end{wideitemize}
\end{frame}

\begin{frame}{Worked example: Cournot collusion}
	\begin{wideitemize}
		\item \textbf{Question:} 3 symmetric Cournot firms. $P = 100 - Q$, $MC = 10$.
		\item (a) Calculate $\pi^C$, $\pi^{NE}$, and $\pi^D$ for each firm.
		\item (b) Find the minimum $\delta$ for collusion.
	\end{wideitemize}
	\vspace{10pt}
	\centering
	\textit{Take 7 minutes.}
\end{frame}

\begin{frame}{Worked example: Cournot collusion (solution)}
	\begin{wideitemize}
		\item \textbf{(a) Profit calculations:}
		\item $\pi^M = (90)^2/4 = 2025$, so $\pi^C = 2025/3 = 675$
		\item $q^C = 45/3 = 15$ per firm (monopoly quantity split)
		\item Nash: $q^{NE} = 90/4 = 22.5$, $\pi^{NE} = 90^2/16 = 506.25$
		\item Deviation: BR to $2 \times 15 = 30$ is $q^D = (90-30)/2 = 30$
		\item $P = 100 - 60 = 40$, $\pi^D = (40-10) \times 30 = 900$
	\end{wideitemize}
\end{frame}

\begin{frame}{Worked example: Cournot collusion (solution cont.)}
	\begin{wideitemize}
		\item \textbf{(b) Critical discount factor:}
		\begin{align*}
			\delta^* = \frac{\pi^D - \pi^C}{\pi^D - \pi^{NE}} = \frac{900 - 675}{900 - 506.25} = \frac{225}{393.75} = 0.571
		\end{align*}
		\item Or use formula: $\delta^* = \frac{(3+1)^2}{3^2 + (3+1)^2} = \frac{16}{9+16} = \frac{16}{25} = 0.64$
		\item (Small difference due to rounding in worked example)
		\item \textbf{Interpretation:} Firms must value future at 64\% of present
	\end{wideitemize}
\end{frame}

\begin{frame}{Bertrand collusion with $N$ firms}
	\begin{wideitemize}
		\item Homogeneous Bertrand: $\pi^{NE} = 0$ (price = cost)
		\item Collusion: split monopoly profits
		\item \textbf{Key difference:} Punishment is more severe ($\pi^{NE} = 0$)
		\item \textbf{Critical discount factor for Bertrand:}
		\begin{align*}
			\delta^* = \frac{\pi^D - \pi^C}{\pi^D - 0} = \frac{\pi^M - \pi^M/N}{\pi^M} = \frac{N-1}{N}
		\end{align*}
		\item \textbf{Examples:}
		\begin{wideitemize}
			\vspace{5pt}
			\item $N = 2$: $\delta^* = 0.5$
			\item $N = 4$: $\delta^* = 0.75$
		\end{wideitemize}
	\end{wideitemize}
\end{frame}

\begin{frame}{Cournot vs Bertrand collusion}
	\begin{center}
		\begin{tabular}{|c|c|c|}
			\hline
			$N$ & $\delta^*$ (Cournot) & $\delta^*$ (Bertrand) \\
			\hline
			2 & 0.53 & 0.50 \\
			3 & 0.64 & 0.67 \\
			4 & 0.61 & 0.75 \\
			\hline
		\end{tabular}
	\end{center}
	\vspace{10pt}
	\begin{wideitemize}
		\item At $N = 2$: Bertrand collusion \textbf{easier}
		\item \textbf{Why?} Bertrand punishment is harsher ($\pi^{NE} = 0$)
		\item At higher $N$: Bertrand collusion harder
		\item \textbf{Why?} Deviation captures entire market (bigger temptation)
	\end{wideitemize}
\end{frame}

%%%%%%%%%%%%%%%%%%%%%%%%%%%%%%%%%%%%%%%%%%%%%%%%%%%%%%%%%%%%%
% PART 2: DETECTION AND POLICY
%%%%%%%%%%%%%%%%%%%%%%%%%%%%%%%%%%%%%%%%%%%%%%%%%%%%%%%%%%%%%

\begin{transitionframe}
	\begin{center}
		{\Huge Part 2: Detection and Policy}
	\end{center}
\end{transitionframe}

\begin{frame}{Detection probability and fines}
	\begin{wideitemize}
		\item In reality: cartels may be detected and punished
		\item \textbf{Each period:}
		\begin{wideitemize}
			\vspace{5pt}
			\item Detection probability: $\rho$
			\item Fine if detected: $F$
		\end{wideitemize}
		\item \textbf{Modified collusion condition:}
		\begin{align*}
			\delta^* = \frac{\pi^D - \pi^C + \rho F}{\pi^D - \pi^{NE} + \rho F}
		\end{align*}
		\item Higher $\rho$ or higher $F$ $\rightarrow$ higher $\delta^*$ $\rightarrow$ harder to collude
	\end{wideitemize}
\end{frame}

\begin{frame}{Worked example: Detection and fines}
	\begin{wideitemize}
		\item \textbf{Question:}
		\item Cartel earns $\pi^C = 100$ per period
		\item $\pi^{NE} = 25$, $\pi^D = 150$
		\item Detection probability $\rho = 0.1$, fine $F = 500$
		\item Find the minimum $\delta$ for collusion.
	\end{wideitemize}
	\vspace{10pt}
	\centering
	\textit{Take 3 minutes.}
\end{frame}

\begin{frame}{Worked example: Detection (solution)}
	\begin{wideitemize}
		\item Expected fine per period: $\rho F = 0.1 \times 500 = 50$
		\item Apply formula:
		\begin{align*}
			\delta^* = \frac{\pi^D - \pi^C + \rho F}{\pi^D - \pi^{NE} + \rho F} = \frac{150 - 100 + 50}{150 - 25 + 50} = \frac{100}{175} = 0.571
		\end{align*}
		\item \textbf{Compare to no detection:}
		\begin{align*}
			\delta^*_{\text{no detection}} = \frac{150 - 100}{150 - 25} = \frac{50}{125} = 0.4
		\end{align*}
		\item Detection and fines make collusion harder (0.4 $\rightarrow$ 0.57)
	\end{wideitemize}
\end{frame}

\begin{frame}{Leniency programs}
	\begin{wideitemize}
		\item \textbf{Leniency:} First firm to report cartel gets reduced/zero fine
		\item \textbf{US Corporate Leniency Program (1993):}
		\begin{wideitemize}
			\vspace{5pt}
			\item First to report: automatic immunity
			\item Second: significant reduction possible
		\end{wideitemize}
		\item \textbf{Effect on incentives:}
		\begin{wideitemize}
			\vspace{5pt}
			\item Creates ``race to report''
			\item Each firm fears others will report first
			\item Destabilizes existing cartels
		\end{wideitemize}
	\end{wideitemize}
\end{frame}

\begin{frame}{Why leniency works}
	\begin{wideitemize}
		\item \textbf{Without leniency:}
		\begin{wideitemize}
			\vspace{5pt}
			\item If detected, everyone pays fine
			\item No incentive to report
		\end{wideitemize}
		\item \textbf{With leniency:}
		\begin{wideitemize}
			\vspace{5pt}
			\item First to report gets immunity
			\item Creates Prisoner's Dilemma within cartel
			\item Each firm thinks: ``Better report before they do''
		\end{wideitemize}
		\item \textbf{Result:} Cartel detection increased dramatically after 1993
		\item \textbf{Exam question:} ``Explain why leniency programs help detect cartels.''
	\end{wideitemize}
\end{frame}

\begin{frame}{Factors facilitating collusion}
	\begin{wideenumerate}
		\item \textbf{Few firms:} Easier to coordinate and monitor
		\item \textbf{Frequent interaction:} Higher effective $\delta$
		\item \textbf{Similar costs:} Easier to agree on price
		\item \textbf{Stable demand:} Easier to detect deviations
		\item \textbf{Homogeneous products:} Easier to monitor prices
		\item \textbf{Industry associations:} Facilitate communication
	\end{wideenumerate}
\end{frame}

\begin{frame}{Famous cartel cases}
	\begin{wideitemize}
		\item \textbf{Lysine cartel (1990s):}
		\begin{wideitemize}
			\vspace{5pt}
			\item Price-fixing among feed additive producers
			\item FBI surveillance, recorded meetings
		\end{wideitemize}
		\item \textbf{LCD screen cartel (2000s):}
		\begin{wideitemize}
			\vspace{5pt}
			\item Samsung, LG, Sharp, others
			\item \$1.4 billion in fines
		\end{wideitemize}
		\item \textbf{LIBOR scandal (2012):}
		\begin{wideitemize}
			\vspace{5pt}
			\item Banks manipulated interest rate benchmark
			\item \$9 billion in fines
		\end{wideitemize}
	\end{wideitemize}
\end{frame}

\begin{frame}{Detecting collusion: what regulators look for}
	\begin{wideitemize}
		\item \textbf{Pricing patterns:}
		\begin{wideitemize}
			\vspace{5pt}
			\item Parallel price changes
			\item Price rigidity despite cost changes
			\item Similar prices despite different costs
		\end{wideitemize}
		\item \textbf{Market characteristics:}
		\begin{wideitemize}
			\vspace{5pt}
			\item High concentration
			\item Frequent meetings/communication
			\item History of antitrust violations
		\end{wideitemize}
		\item \textbf{Whistleblowers:} Leniency program tips
	\end{wideitemize}
\end{frame}

%%%%%%%%%%%%%%%%%%%%%%%%%%%%%%%%%%%%%%%%%%%%%%%%%%%%%%%%%%%%%
% KEY POINTS
%%%%%%%%%%%%%%%%%%%%%%%%%%%%%%%%%%%%%%%%%%%%%%%%%%%%%%%%%%%%%

\begin{frame}{Key Points}
	\vspace{11pt}
	\begin{wideenumerate}
		\item \textbf{Critical discount factor:} $\delta^* = \frac{\pi^D - \pi^C}{\pi^D - \pi^{NE}}$
		\item \textbf{Cournot with N firms:} $\delta^* = \frac{(N+1)^2}{N^2 + (N+1)^2}$
		\item \textbf{Bertrand with N firms:} $\delta^* = \frac{N-1}{N}$
		\item More firms $\rightarrow$ generally harder to collude
		\item \textbf{Detection and fines} raise $\delta^*$: $\delta^* = \frac{\pi^D - \pi^C + \rho F}{\pi^D - \pi^{NE} + \rho F}$
		\item \textbf{Leniency programs:} Create ``race to report,'' destabilize cartels
		\item Collusion easier with: few firms, frequent interaction, similar costs
	\end{wideenumerate}
\end{frame}

\begin{frame}{Next time}
	\begin{wideitemize}
		\item \textbf{Lecture 13:} Final Review
		\begin{wideitemize}
			\vspace{5pt}
			\item Comprehensive review of Part 1 and Part 2
			\item Practice problems for final exam
		\end{wideitemize}
		\item \textbf{HW2 due before Lecture 13}
	\end{wideitemize}
\end{frame}

\end{document}
