\documentclass[notes,11pt, aspectratio=169]{beamer}

\usepackage{pgfpages}
\setbeameroption{hide notes} % Only slide

\usepackage{array}
\usepackage{tikz}
\usepackage{verbatim}
\setbeamertemplate{note page}{\pagecolor{gray!5}\insertnote}
\usetikzlibrary{positioning}
\usetikzlibrary{snakes}
\usetikzlibrary{calc}
\usetikzlibrary{arrows}
\usetikzlibrary{decorations.markings}
\usetikzlibrary{shapes.misc}
\usetikzlibrary{matrix,shapes,arrows,fit,tikzmark}
\usepackage{amsmath}
\usepackage{mathpazo}
\usepackage{hyperref}
\usepackage{lipsum}
\usepackage{multimedia}
\usepackage{graphicx}
\usepackage{multirow}
\usepackage{dcolumn}
\usepackage{bbm}
\newcolumntype{d}[0]{D{.}{.}{5}}

\usepackage{changepage}
\usepackage{appendixnumberbeamer}

\usepackage[space]{grffile}
\usepackage{booktabs}

% Colors
\definecolor{blue}{RGB}{0,114,178}
\definecolor{red}{RGB}{213,94,0}
\definecolor{yellow}{RGB}{240,228,66}
\definecolor{green}{RGB}{0,158,115}

\hypersetup{
  colorlinks=false,
  linkbordercolor = {white},
  linkcolor = {blue}
}

\definecolor{MyBackground}{RGB}{255,253,218}

\newenvironment{transitionframe}{
  \setbeamercolor{background canvas}{bg=white}
  \begin{frame}}{
    \end{frame}
}

\setbeamercolor{frametitle}{fg=blue}
\setbeamercolor{title}{fg=black}
\setbeamertemplate{footline}[frame number]
\setbeamertemplate{navigation symbols}{}
\setbeamertemplate{itemize items}{-}
\setbeamercolor{itemize item}{fg=blue}
\setbeamercolor{itemize subitem}{fg=blue}
\setbeamercolor{enumerate item}{fg=blue}
\setbeamercolor{enumerate subitem}{fg=blue}
\setbeamercolor{button}{bg=MyBackground,fg=blue,}

\setbeamercolor{section in toc}{fg=blue}
\setbeamercolor{subsection in toc}{fg=red}
\setbeamersize{text margin left=1em,text margin right=1em}

\newenvironment{wideitemize}{\itemize\addtolength{\itemsep}{10pt}}{\enditemize}
\newenvironment{wideenumerate}{\enumerate\addtolength{\itemsep}{10pt}}{\endenumerate}

\title[]{\textcolor{blue}{ECN 594: Entry and Market Structure}}
\author[PGP]{}
\institute[FRBNY]{\small{\begin{tabular}{c c c}
Nicholas Vreugdenhil \\
\end{tabular}}}
\date{\today}

\begin{document}

% Title Slide
\begin{frame}
\maketitle
  \centering
\end{frame}

\begin{frame}{Plan for today}
  \begin{wideenumerate}
    \item What determines market structure?
    \item Free entry condition
    \item Entry with Cournot: worked example
    \item[] \rule{0.5\textwidth}{0.4pt}
    \item Entry barriers
    \item Entry deterrence strategies
    \item Sequential game analysis (from ECN 532)
  \end{wideenumerate}
\end{frame}

%%%%%%%%%%%%%%%%%%%%%%%%%%%%%%%%%%%%%%%%%%%%%%%%%%%%%%%%%%%%%
% PART 1: ENTRY AND MARKET STRUCTURE
%%%%%%%%%%%%%%%%%%%%%%%%%%%%%%%%%%%%%%%%%%%%%%%%%%%%%%%%%%%%%

\begin{transitionframe}
	\begin{center}
		{\Huge Part 1: Entry and Market Structure}
	\end{center}
\end{transitionframe}

\begin{frame}{What determines market structure?}
	\begin{wideitemize}
		\item Why do some industries have many firms (restaurants) and others few (aircraft)?
		\item \textbf{Key factors:}
		\begin{wideenumerate}
			\vspace{5pt}
			\item Fixed costs of entry
			\item Market size (demand)
			\item Nature of competition
			\item Entry barriers
		\end{wideenumerate}
		\item Today: focus on entry decisions and barriers
	\end{wideitemize}
\end{frame}

\begin{frame}{Free entry condition}
	\begin{wideitemize}
		\item Suppose entry requires fixed cost $F$
		\item \textbf{Entry decision:} Enter if $\pi(N) > F$
		\begin{wideitemize}
			\vspace{5pt}
			\item $\pi(N)$ = profit when $N$ firms in market
		\end{wideitemize}
		\item \textbf{Free entry equilibrium:} Number of firms $N^*$ such that:
		\begin{align*}
			\pi(N^*) \geq F > \pi(N^* + 1)
		\end{align*}
		\item \textbf{Intuition:}
		\begin{wideitemize}
			\vspace{5pt}
			\item Current firms earn enough to cover $F$
			\item One more firm would push profits below $F$
		\end{wideitemize}
	\end{wideitemize}
\end{frame}

\begin{frame}{Entry reduces profits}
	\begin{wideitemize}
		\item As $N$ increases:
		\begin{wideitemize}
			\vspace{5pt}
			\item Competition intensifies
			\item Prices fall
			\item Each firm's profit falls
		\end{wideitemize}
		\item $\pi(N)$ is decreasing in $N$
		\item Eventually: $\pi(N) < F$ and entry stops
	\end{wideitemize}
\end{frame}

\begin{frame}{Entry with Cournot: setup}
	\begin{wideitemize}
		\item Inverse demand: $P = a - bQ$
		\item $N$ symmetric firms, each with $MC = c$
		\item Fixed cost of entry: $F$
		\item Cournot equilibrium with $N$ firms:
		\begin{align*}
			q_i^* &= \frac{a - c}{b(N + 1)} \\
			P^* &= \frac{a + Nc}{N + 1} \\
			\pi_i^* &= \frac{(a - c)^2}{b(N + 1)^2}
		\end{align*}
	\end{wideitemize}
\end{frame}

\begin{frame}{Worked example: Entry with Cournot}
	\begin{wideitemize}
		\item \textbf{Question:} $P = 100 - Q$, $c = 20$, $F = 100$.
		\item How many firms will enter in equilibrium?
	\end{wideitemize}
	\vspace{15pt}
	\centering
	\textit{Take 5 minutes.}
\end{frame}

\begin{frame}{Worked example: Entry (solution)}
	\begin{wideitemize}
		\item With $a = 100$, $b = 1$, $c = 20$:
		\begin{align*}
			\pi(N) = \frac{(100 - 20)^2}{(N + 1)^2} = \frac{6400}{(N + 1)^2}
		\end{align*}
		\item Check different values of $N$:
		\begin{center}
			\begin{tabular}{|c|c|c|}
				\hline
				$N$ & $\pi(N)$ & Enter? \\
				\hline
				1 & $6400/4 = 1600$ & Yes ($> 100$) \\
				2 & $6400/9 = 711$ & Yes \\
				3 & $6400/16 = 400$ & Yes \\
				5 & $6400/36 = 178$ & Yes \\
				7 & $6400/64 = 100$ & Indifferent \\
				8 & $6400/81 = 79$ & No ($< 100$) \\
				\hline
			\end{tabular}
		\end{center}
		\item \textbf{Answer:} $N^* = 7$ firms enter
	\end{wideitemize}
\end{frame}

\begin{frame}{Fixed costs and natural monopoly}
	\begin{wideitemize}
		\item When $F$ is very high relative to demand:
		\begin{wideitemize}
			\vspace{5pt}
			\item Only one firm can profitably operate
			\item \textbf{Natural monopoly}
		\end{wideitemize}
		\item \textbf{Examples:}
		\begin{wideitemize}
			\vspace{5pt}
			\item Utilities (water, electricity distribution)
			\item Railroad tracks
			\item Cable infrastructure
		\end{wideitemize}
		\item Average cost is declining over relevant range
		\item One firm can serve entire market more cheaply than multiple firms
	\end{wideitemize}
\end{frame}

\begin{frame}{Excess entry theorem (brief)}
	\begin{wideitemize}
		\item Free entry may produce ``too many'' firms
		\item \textbf{Why?} Each entrant ignores:
		\begin{wideenumerate}
			\vspace{5pt}
			\item Business-stealing effect: takes customers from incumbents
			\item Consumer surplus: captured by entrant, not new value created
		\end{wideenumerate}
		\item Private incentive to enter $>$ social incentive
		\item \textbf{Result:} Free entry equilibrium can have more firms than socially optimal
		\item Caveat: depends on model specifics
	\end{wideitemize}
\end{frame}

%%%%%%%%%%%%%%%%%%%%%%%%%%%%%%%%%%%%%%%%%%%%%%%%%%%%%%%%%%%%%
% PART 2: ENTRY DETERRENCE
%%%%%%%%%%%%%%%%%%%%%%%%%%%%%%%%%%%%%%%%%%%%%%%%%%%%%%%%%%%%%

\begin{transitionframe}
	\begin{center}
		{\Huge Part 2: Entry Deterrence}
	\end{center}
\end{transitionframe}

\begin{frame}{Entry barriers}
	\begin{wideitemize}
		\item \textbf{Structural barriers:} inherent to industry
		\begin{wideitemize}
			\vspace{5pt}
			\item Economies of scale (high $F$)
			\item Capital requirements
			\item Patents and intellectual property
			\item Network effects
		\end{wideitemize}
		\item \textbf{Strategic barriers:} created by incumbents
		\begin{wideitemize}
			\vspace{5pt}
			\item Limit pricing
			\item Capacity commitment
			\item Product proliferation
			\item Long-term contracts with customers
		\end{wideitemize}
	\end{wideitemize}
\end{frame}

\begin{frame}{Entry deterrence: the key question}
	\begin{wideitemize}
		\item Can an incumbent prevent entry?
		\item \textbf{The credibility problem:}
		\begin{wideitemize}
			\vspace{5pt}
			\item Incumbent threatens to ``fight'' if entry occurs
			\item But is this threat credible?
			\item Once entry happens, fighting may hurt the incumbent too
		\end{wideitemize}
		\item \textbf{From ECN 532:} Need subgame perfect equilibrium (SPE)
		\begin{wideitemize}
			\vspace{5pt}
			\item Check what incumbent would actually do if entry occurs
			\item Non-credible threats are ignored
		\end{wideitemize}
	\end{wideitemize}
\end{frame}

\begin{frame}{Limit pricing}
	\begin{wideitemize}
		\item Incumbent sets price low enough that entry is unprofitable
		\item \textbf{Idea:} If $P$ is low, post-entry profits are low
		\item \textbf{Problem:} Why would incumbent maintain low $P$ after entry?
		\begin{wideitemize}
			\vspace{5pt}
			\item If entrant enters anyway, incumbent may prefer to accommodate
			\item Threat to keep $P$ low may not be credible
		\end{wideitemize}
		\item Works better if low $P$ is a commitment device
		\item Or if low $P$ signals low costs
	\end{wideitemize}
\end{frame}

\begin{frame}{Capacity commitment}
	\begin{wideitemize}
		\item Incumbent builds excess capacity before entry decision
		\item \textbf{Why it works:}
		\begin{wideitemize}
			\vspace{5pt}
			\item Capacity is a sunk cost
			\item With excess capacity, incumbent's marginal cost is low
			\item Post-entry, incumbent will produce more (use the capacity)
			\item Entrant anticipates this $\rightarrow$ lower post-entry profits
		\end{wideitemize}
		\item \textbf{Key insight:} Capacity makes ``fight'' credible
		\item Building capacity is a commitment device
	\end{wideitemize}
\end{frame}

\begin{frame}{Entry deterrence game: structure}
	\begin{wideitemize}
		\item \textbf{Stage 1:} Incumbent chooses capacity $K$
		\item \textbf{Stage 2:} Entrant observes $K$ and decides: Enter or Stay Out
		\item \textbf{Stage 3:} If entry, firms compete (Cournot or Bertrand)
		\item Solve by \textbf{backward induction} (from ECN 532):
		\begin{wideenumerate}
			\vspace{5pt}
			\item Find post-entry equilibrium profits given $K$
			\item Determine when entrant enters
			\item Find incumbent's optimal $K$
		\end{wideenumerate}
	\end{wideitemize}
\end{frame}

\begin{frame}{Worked example: Entry deterrence}
	\begin{wideitemize}
		\item \textbf{Setup:}
		\item Inverse demand: $P = 100 - Q$
		\item Incumbent has capacity $K$ (sunk), $MC = 0$ up to $K$
		\item Entrant has $MC = 20$, fixed cost $F = 200$
		\item If entry: Cournot competition
		\item \textbf{Question:} What $K$ deters entry?
	\end{wideitemize}
	\vspace{10pt}
	\centering
	\textit{Take 5 minutes to set up the backward induction.}
\end{frame}

\begin{frame}{Worked example: Entry deterrence (solution 1)}
	\begin{wideitemize}
		\item \textbf{Step 1: Post-entry equilibrium}
		\item Incumbent produces $q_I \leq K$ at $MC = 0$
		\item Entrant FOC: $100 - 2q_E - q_I - 20 = 0 \Rightarrow q_E = 40 - q_I/2$
		\item If $K$ is large, incumbent produces $q_I = K$
		\item Entrant produces: $q_E = 40 - K/2$
		\item Price: $P = 100 - K - (40 - K/2) = 60 - K/2$
		\item Entrant profit: $\pi_E = (60 - K/2 - 20)(40 - K/2) = (40 - K/2)^2$
	\end{wideitemize}
\end{frame}

\begin{frame}{Worked example: Entry deterrence (solution 2)}
	\begin{wideitemize}
		\item \textbf{Step 2: Entry decision}
		\item Entrant enters if: $\pi_E - F > 0$
		\begin{align*}
			(40 - K/2)^2 > 200
		\end{align*}
		\item Entry occurs if: $40 - K/2 > \sqrt{200} \approx 14.1$
		\item Entry is deterred if: $K \geq 2(40 - 14.1) = 51.8$
		\item \textbf{Step 3: Incumbent's choice}
		\item If deterrence is profitable: Set $K \geq 52$
		\item Compare: monopoly profits with $K = 52$ vs accommodation
	\end{wideitemize}
\end{frame}

\begin{frame}{When is deterrence profitable?}
	\begin{wideitemize}
		\item Incumbent compares:
		\begin{wideenumerate}
			\vspace{5pt}
			\item \textbf{Deter:} Monopoly profits minus cost of excess capacity
			\item \textbf{Accommodate:} Duopoly profits
		\end{wideenumerate}
		\item Deterrence is profitable when:
		\begin{wideitemize}
			\vspace{5pt}
			\item Cost of deterrence (excess capacity) is low
			\item Post-entry competition is intense
			\item Monopoly profits are high
		\end{wideitemize}
		\item Sometimes: accommodation is better (``puppy dog'' strategy)
	\end{wideitemize}
\end{frame}

\begin{frame}{Predatory pricing (brief)}
	\begin{wideitemize}
		\item \textbf{Predatory pricing:} Price below cost to drive out competitor
		\item \textbf{Requirements:}
		\begin{wideitemize}
			\vspace{5pt}
			\item ``Deep pockets'' - survive losses longer than rival
			\item Recoupment - raise prices after rival exits
		\end{wideitemize}
		\item \textbf{Antitrust concern:}
		\begin{wideitemize}
			\vspace{5pt}
			\item Hard to distinguish from aggressive competition
			\item Low prices benefit consumers (in short run)
		\end{wideitemize}
		\item \textbf{Legal standard:} Price below appropriate cost measure + recoupment likely
	\end{wideitemize}
\end{frame}

\begin{frame}{Summary: barriers and deterrence}
	\begin{center}
		\begin{tabular}{|l|p{4cm}|p{4cm}|}
			\hline
			& \textbf{Structural} & \textbf{Strategic} \\
			\hline
			\textbf{Examples} & Economies of scale, patents, network effects & Capacity, limit pricing, product proliferation \\
			\hline
			\textbf{Created by} & Industry characteristics & Incumbent behavior \\
			\hline
			\textbf{Key issue} & -- & Credibility of threat \\
			\hline
		\end{tabular}
	\end{center}
\end{frame}

%%%%%%%%%%%%%%%%%%%%%%%%%%%%%%%%%%%%%%%%%%%%%%%%%%%%%%%%%%%%%
% KEY POINTS
%%%%%%%%%%%%%%%%%%%%%%%%%%%%%%%%%%%%%%%%%%%%%%%%%%%%%%%%%%%%%

\begin{frame}{Key Points}
	\vspace{11pt}
	\begin{wideenumerate}
		\item \textbf{Free entry:} Enter until $\pi(N^*) \geq F > \pi(N^* + 1)$
		\item As $N$ increases, profits decrease
		\item High fixed costs $\rightarrow$ few firms (natural monopoly at extreme)
		\item Free entry may lead to \textbf{excess entry} (business stealing)
		\item \textbf{Barriers:} Structural (inherent) vs Strategic (created)
		\item Entry deterrence requires \textbf{credible} commitment
		\item \textbf{Capacity commitment:} Makes ``fight'' credible (sunk cost)
		\item Solve deterrence games by \textbf{backward induction} (SPE)
	\end{wideenumerate}
\end{frame}

\begin{frame}{Next time}
	\begin{wideitemize}
		\item \textbf{Lecture 10:} Mergers and Merger Policy
		\begin{wideitemize}
			\vspace{5pt}
			\item Merger effects on prices
			\item Merger simulation (connects demand estimation to policy)
			\item Antitrust review and HHI
		\end{wideitemize}
		\item \textbf{HW2 released:} Merger simulation exercise
	\end{wideitemize}
\end{frame}

\end{document}
