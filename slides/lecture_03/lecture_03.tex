\documentclass[notes,11pt, aspectratio=169]{beamer}

\usepackage{pgfpages}
\setbeameroption{hide notes} % Only slide

\usepackage{array}
\usepackage{tikz}
\usepackage{verbatim}
\setbeamertemplate{note page}{\pagecolor{gray!5}\insertnote}
\usetikzlibrary{positioning}
\usetikzlibrary{snakes}
\usetikzlibrary{calc}
\usetikzlibrary{arrows}
\usetikzlibrary{decorations.markings}
\usetikzlibrary{shapes.misc}
\usetikzlibrary{matrix,shapes,arrows,fit,tikzmark}
\usepackage{amsmath}
\usepackage{mathpazo}
\usepackage{hyperref}
\usepackage{lipsum}
\usepackage{multimedia}
\usepackage{graphicx}
\usepackage{multirow}
\usepackage{dcolumn}
\usepackage{bbm}
\usepackage{listings}
\usepackage{tcolorbox}
\tcbuselibrary{listings,skins}
\newcolumntype{d}[0]{D{.}{.}{5}}

\usepackage{changepage}
\usepackage{appendixnumberbeamer}

\usepackage[space]{grffile}
\usepackage{booktabs}

% Colors
\definecolor{blue}{RGB}{0,114,178}
\definecolor{red}{RGB}{213,94,0}
\definecolor{yellow}{RGB}{240,228,66}
\definecolor{green}{RGB}{0,158,115}
\definecolor{codebg}{RGB}{245,245,245}
\definecolor{codeframe}{RGB}{200,200,200}
\definecolor{codecomment}{RGB}{0,128,0}
\definecolor{codestring}{RGB}{163,21,21}
\definecolor{codekeyword}{RGB}{0,0,180}
\definecolor{solutionbg}{RGB}{240,248,240}
\definecolor{solutionframe}{RGB}{0,158,115}

% Solution box environment for worked answers
\newtcolorbox{solutionbox}[1][]{
  enhanced,
  colback=solutionbg,
  colframe=solutionframe,
  boxrule=0pt,
  leftrule=3pt,
  arc=0pt,
  left=8pt,
  right=8pt,
  top=6pt,
  bottom=6pt,
  fonttitle=\bfseries,
  title={#1},
  attach boxed title to top left={yshift=-2mm, xshift=5mm},
  boxed title style={colback=solutionframe, colframe=solutionframe, size=small, arc=2pt}
}

\hypersetup{
  colorlinks=false,
  linkbordercolor = {white},
  linkcolor = {blue}
}

\definecolor{MyBackground}{RGB}{255,253,218}

\newenvironment{transitionframe}{
  \setbeamercolor{background canvas}{bg=white}
  \begin{frame}}{
    \end{frame}
}

\setbeamercolor{frametitle}{fg=blue}
\setbeamercolor{title}{fg=black}
\setbeamertemplate{footline}[frame number]
\setbeamertemplate{navigation symbols}{}
\setbeamertemplate{itemize items}{-}
\setbeamercolor{itemize item}{fg=blue}
\setbeamercolor{itemize subitem}{fg=blue}
\setbeamercolor{enumerate item}{fg=blue}
\setbeamercolor{enumerate subitem}{fg=blue}
\setbeamercolor{button}{bg=MyBackground,fg=blue,}

\setbeamercolor{section in toc}{fg=blue}
\setbeamercolor{subsection in toc}{fg=red}
\setbeamersize{text margin left=1em,text margin right=1em}

\newenvironment{wideitemize}{\itemize\addtolength{\itemsep}{10pt}}{\enditemize}
\newenvironment{wideenumerate}{\enumerate\addtolength{\itemsep}{10pt}}{\endenumerate}

% Code listing style - light theme
\lstset{
  basicstyle=\ttfamily\footnotesize\color{black},
  backgroundcolor=\color{codebg},
  breaklines=true,
  frame=none,
  language=Python,
  showstringspaces=false,
  keywordstyle=\color{codekeyword}\bfseries,
  commentstyle=\color{codecomment}\itshape,
  stringstyle=\color{codestring},
  numbers=none,
  xleftmargin=8pt,
  xrightmargin=8pt,
  aboveskip=10pt,
  belowskip=10pt,
  framesep=8pt,
  rulecolor=\color{codeframe},
  morekeywords={pyblp, Formulation, Problem, solve, compute_elasticities, compute_markups}
}

% Nice code box environment
\newtcblisting{pycode}{
  listing only,
  colback=codebg,
  colframe=codeframe,
  arc=4pt,
  boxrule=0.5pt,
  left=6pt,
  right=6pt,
  top=6pt,
  bottom=6pt,
  listing options={
    basicstyle=\ttfamily\footnotesize\color{black},
    language=Python,
    showstringspaces=false,
    keywordstyle=\color{codekeyword}\bfseries,
    commentstyle=\color{codecomment}\itshape,
    stringstyle=\color{codestring},
    breaklines=true,
    morekeywords={pyblp, Formulation, Problem, solve, compute_elasticities, compute_markups, read_csv}
  }
}

\title[]{\textcolor{blue}{ECN 594: Demographic Interactions and pyblp}}
\author[PGP]{}
\institute[FRBNY]{\small{\begin{tabular}{c c c}
Nicholas Vreugdenhil \\
\end{tabular}}}
\date{\today}

\begin{document}

% Title Slide
\begin{frame}
\maketitle
  \centering
\end{frame}

\begin{frame}{Plan for today}
  \begin{wideenumerate}
    \item \textbf{Recap: endogeneity and IVs}
    \item The demographic interaction model
    \item Why demographics help with IIA
    \item[] \rule{0.5\textwidth}{0.4pt}
    \item Introduction to pyblp
    \item Worked example: car demand
    \item Interpreting output
  \end{wideenumerate}
\end{frame}

%%%%%%%%%%%%%%%%%%%%%%%%%%%%%%%%%%%%%%%%%%%%%%%%%%%%%%%%%%%%%
% DEMOGRAPHIC INTERACTIONS
%%%%%%%%%%%%%%%%%%%%%%%%%%%%%%%%%%%%%%%%%%%%%%%%%%%%%%%%%%%%%

\begin{frame}{Recap: The basic logit model}
	\begin{wideitemize}
		\item Our utility model so far:
		\begin{align*}
			u_{ij} = x_j \beta - \alpha p_j + \xi_j + \varepsilon_{ij}
		\end{align*}
		\item Everyone has the same $\beta$ and $\alpha$
		\item Problem: this is very restrictive
		\begin{wideitemize}
			\vspace{5pt}
			\item Rich and poor consumers have same price sensitivity?
			\item Families and singles value car size the same?
		\end{wideitemize}
	\end{wideitemize}
\end{frame}

\begin{frame}{Why do we need heterogeneous preferences?}
	\begin{wideitemize}
		\item \textbf{Policy questions require knowing WHO values WHAT:}
		\begin{wideitemize}
			\vspace{5pt}
			\item If a product is removed, who loses the most?
			\item If prices rise, which consumers switch away?
			\item Who benefits from new product variety?
		\end{wideitemize}
		\item \textbf{Competitive questions need substitution patterns:}
		\begin{wideitemize}
			\vspace{5pt}
			\item When BMW's price rises, who switches to Mercedes vs Honda?
			\item Basic logit: everyone substitutes proportionally (IIA)
			\item Reality: depends on consumer demographics!
		\end{wideitemize}
	\end{wideitemize}
\end{frame}

\begin{frame}{Extending the model: heterogeneous preferences}
	\begin{wideitemize}
		\item \textbf{Basic logit:} Everyone has same $\beta$
		\item Problem: doesn't capture that different consumers value characteristics differently
		\item \textbf{Solution:} Let preferences vary with observed demographics
	\end{wideitemize}
\end{frame}

\begin{frame}{Plan for today}
  \begin{wideenumerate}
    \item Recap: endogeneity and IVs
    \item \textbf{The demographic interaction model}
    \item Why demographics help with IIA
    \item[] \rule{0.5\textwidth}{0.4pt}
    \item Introduction to pyblp
    \item Worked example: car demand
    \item Interpreting output
  \end{wideenumerate}
\end{frame}

\begin{frame}{The demographic interaction model}
	\begin{wideitemize}
		\item New utility specification:
		\begin{align*}
			u_{ij} = x_j \beta + (D_i \times x_j) \gamma - \alpha p_j + \xi_j + \varepsilon_{ij}
		\end{align*}
		\item $D_i$: observed consumer demographics (income, age, family size, etc.)
		\item $(D_i \times x_j)$: interactions between demographics and characteristics
		\item This creates \textbf{heterogeneous preferences}
	\end{wideitemize}
\end{frame}

\begin{frame}{Classic example: Nevo (2001) - Cereal demand}
	\begin{wideitemize}
		\item ``Measuring Market Power in the Ready-to-Eat Cereal Industry''
		\item Estimated demand for cereal using demographic interactions
		\item Key findings:
		\begin{wideitemize}
			\vspace{5pt}
			\item High-income households less price-sensitive
			\item Families with children prefer kid-targeted cereals
			\item Without demographics: wrong substitution patterns
		\end{wideitemize}
		\item Result: Basic logit would underestimate markups!
	\end{wideitemize}
\end{frame}

\begin{frame}{Examples of demographic interactions}
	\begin{wideitemize}
		\item \textbf{Income $\times$ price:}
		\begin{wideitemize}
			\vspace{5pt}
			\item High-income consumers less price-sensitive
			\item $\gamma_{\text{inc} \times p} > 0$: price hurts less for rich consumers
		\end{wideitemize}
		\item \textbf{Family size $\times$ car size:}
		\begin{wideitemize}
			\vspace{5pt}
			\item Families prefer larger vehicles
			\item $\gamma_{\text{fam} \times \text{size}} > 0$
		\end{wideitemize}
		\item \textbf{Age $\times$ fuel efficiency:}
		\begin{wideitemize}
			\vspace{5pt}
			\item Older consumers may care more about MPG
		\end{wideitemize}
	\end{wideitemize}
\end{frame}

\begin{frame}{Worked example: Income $\times$ price interaction}
	\begin{wideitemize}
		\item \textbf{Question:}
		\item Model: $u_{ij} = \beta x_j - \alpha p_j + \gamma (\text{income}_i \times p_j) + \varepsilon_{ij}$
		\item Estimated: $\alpha = 2.0$, $\gamma = 0.01$
		\item Consumer A has income \$50,000; Consumer B has income \$100,000
		\item What is the effective price coefficient for each consumer?
	\end{wideitemize}
	\vspace{15pt}
	\centering
	\textit{Take 2 minutes to solve this.}
\end{frame}

\begin{frame}{Worked example: Income $\times$ price interaction (solution)}
	\begin{solutionbox}[Solution]
		Effective price coefficient = $-\alpha + \gamma \times \text{income}$
		\vspace{5pt}

		\textbf{Consumer A} (income = \$50,000):
		\vspace{-5pt}
		\begin{align*}
			\text{Effect} = -2.0 + 0.01 \times 50 = -2.0 + 0.5 = -1.5
		\end{align*}
		\vspace{-15pt}

		\textbf{Consumer B} (income = \$100,000):
		\vspace{-5pt}
		\begin{align*}
			\text{Effect} = -2.0 + 0.01 \times 100 = -2.0 + 1.0 = -1.0
		\end{align*}
		\vspace{-10pt}

		\textbf{Interpretation:} Consumer B (richer) is less price-sensitive.\\
		A \$1 price increase hurts B's utility by 1.0, but hurts A's utility by 1.5.
	\end{solutionbox}
\end{frame}

\begin{frame}{Where do demographics come from?}
	\begin{wideitemize}
		\item Two scenarios:
		\begin{wideenumerate}
			\vspace{5pt}
			\item \textbf{Individual-level data:} Observe $D_i$ for each consumer
			\begin{wideitemize}
				\item Survey data, loyalty card data
			\end{wideitemize}
			\item \textbf{Market-level data:} Know distribution of $D_i$ in each market
			\begin{wideitemize}
				\item Census data, Current Population Survey
				\item This is more common in practice
			\end{wideitemize}
		\end{wideenumerate}
		\item pyblp handles both cases
	\end{wideitemize}
\end{frame}

\begin{frame}{Why not random coefficients?}
	\begin{wideitemize}
		\item Full BLP/mixed logit model adds unobserved heterogeneity:
		\begin{align*}
			\beta_i = \bar{\beta} + \Sigma \nu_i, \quad \nu_i \sim N(0, I)
		\end{align*}
		\item This is computationally harder (requires simulation)
		\item Demographic interactions capture a lot of the variation more simply
		\item \textbf{Mixed logit} is beyond our scope, but know it exists
		\item It fully relaxes IIA
	\end{wideitemize}
\end{frame}

\begin{frame}{Plan for today}
  \begin{wideenumerate}
    \item Recap: endogeneity and IVs
    \item The demographic interaction model
    \item \textbf{Why demographics help with IIA}
    \item[] \rule{0.5\textwidth}{0.4pt}
    \item Introduction to pyblp
    \item Worked example: car demand
    \item Interpreting output
  \end{wideenumerate}
\end{frame}

\begin{frame}{Demographics and IIA (preview)}
	\begin{wideitemize}
		\item Recall: basic logit has IIA problem
		\begin{wideitemize}
			\vspace{5pt}
			\item When BMW price rises, same fraction goes to Mercedes as to Civic
		\end{wideitemize}
		\item With demographics, different consumer types substitute differently
		\begin{wideitemize}
			\vspace{5pt}
			\item High-income BMW buyers $\rightarrow$ Mercedes
			\item Low-income BMW buyers $\rightarrow$ Civic
		\end{wideitemize}
		\item Aggregate substitution is richer
		\item But IIA still holds \textit{within} each consumer type
		\item Full discussion in Lecture 4
	\end{wideitemize}
\end{frame}

\begin{frame}{Limitations: Demographics vs Mixed Logit}
	\begin{wideitemize}
		\item \textbf{Demographics model} (what we use):
		\begin{wideitemize}
			\vspace{5pt}
			\item Heterogeneity only from observed characteristics
			\item IIA still holds within demographic groups
			\item Simple and tractable
		\end{wideitemize}
		\item \textbf{Mixed logit / random coefficients} (advanced):
		\begin{wideitemize}
			\vspace{5pt}
			\item Heterogeneity from unobserved factors too
			\item Fully relaxes IIA
			\item Computationally intensive (requires simulation)
		\end{wideitemize}
		\item For this course: demographics are sufficient
		\item Know that mixed logit exists for research
	\end{wideitemize}
\end{frame}

%%%%%%%%%%%%%%%%%%%%%%%%%%%%%%%%%%%%%%%%%%%%%%%%%%%%%%%%%%%%%
% pyblp ESTIMATION
%%%%%%%%%%%%%%%%%%%%%%%%%%%%%%%%%%%%%%%%%%%%%%%%%%%%%%%%%%%%%

\begin{frame}{Plan for today}
  \begin{wideenumerate}
    \item Recap: endogeneity and IVs
    \item The demographic interaction model
    \item Why demographics help with IIA
    \item[] \rule{0.5\textwidth}{0.4pt}
    \item \textbf{Introduction to pyblp}
    \item Worked example: car demand
    \item Interpreting output
  \end{wideenumerate}
\end{frame}

\begin{frame}{What is pyblp?}
	\begin{wideitemize}
		\item Python package for demand estimation
		\item Conlon \& Gortmaker (2020), ``Best Practices for Differentiated Products Demand Estimation with PyBLP''
		\item Handles:
		\begin{wideitemize}
			\vspace{5pt}
			\item Basic logit and logit with demographics
			\item Instrumental variables
			\item Standard errors
			\item Post-estimation (elasticities, markups, etc.)
		\end{wideitemize}
		\item Why use a package?
		\begin{wideitemize}
			\vspace{5pt}
			\item Correct standard errors
			\item Well-tested code
			\item Saves you from making mistakes
		\end{wideitemize}
	\end{wideitemize}
\end{frame}

\begin{frame}{pyblp workflow}
	\begin{wideenumerate}
		\item \textbf{Set up data:} products, markets, shares, characteristics
		\item \textbf{Define formulation:} which variables, which IVs
		\item \textbf{Create problem:} combine data and formulation
		\item \textbf{Solve:} estimate the model
		\item \textbf{Extract results:} coefficients, standard errors, elasticities
	\end{wideenumerate}
	\vspace{11pt}
	\textit{Let's walk through each step with car data}
\end{frame}

\begin{frame}[fragile]{Step 1: Load and inspect data}
\begin{wideitemize}
	\item Load the BLP automobile data
	\item Key columns: \texttt{market\_ids}, \texttt{shares}, \texttt{prices}, \texttt{hpwt}, \texttt{air}, \texttt{mpd}, \texttt{space}, \texttt{demand\_instruments0}, ...
\end{wideitemize}
\vspace{5pt}
\begin{pycode}
import pyblp
import pandas as pd

product_data = pd.read_csv(pyblp.data.BLP_PRODUCTS_LOCATION)
\end{pycode}
\end{frame}

\begin{frame}{What data do we need?}
	\begin{wideitemize}
		\item \textbf{Product data} (for each product $j$ in market $t$):
		\begin{wideitemize}
			\vspace{5pt}
			\item Market share $s_{jt}$
			\item Price $p_{jt}$
			\item Characteristics $x_{jt}$ (size, HP, fuel efficiency, etc.)
			\item Instruments $z_{jt}$ (BLP IVs, cost shifters)
		\end{wideitemize}
		\item \textbf{Demographic data} (optional, for interactions):
		\begin{wideitemize}
			\vspace{5pt}
			\item Census data on income, family size, age by market
			\item Or micro-level survey data
		\end{wideitemize}
	\end{wideitemize}
\end{frame}

\begin{frame}{Plan for today}
  \begin{wideenumerate}
    \item Recap: endogeneity and IVs
    \item The demographic interaction model
    \item Why demographics help with IIA
    \item[] \rule{0.5\textwidth}{0.4pt}
    \item Introduction to pyblp
    \item \textbf{Worked example: car demand}
    \item Interpreting output
  \end{wideenumerate}
\end{frame}

\begin{frame}{The BLP automobile data}
	\begin{center}
		\scriptsize
		\begin{tabular}{lccccccc}
			\toprule
			\textbf{market\_id} & \textbf{firm\_id} & \textbf{shares} & \textbf{prices} & \textbf{hpwt} & \textbf{air} & \textbf{mpd} & \textbf{space} \\
			\midrule
			1971 & 15 & 0.0012 & 4.935 & 0.524 & 0 & 1.888 & 0.917 \\
			1971 & 15 & 0.0011 & 5.516 & 0.494 & 0 & 1.935 & 0.920 \\
			1971 & 15 & 0.0006 & 7.108 & 0.467 & 0 & 1.716 & 1.074 \\
			1971 & 26 & 0.0038 & 4.296 & 0.426 & 0 & 2.449 & 0.853 \\
			1971 & 26 & 0.0018 & 4.080 & 0.521 & 0 & 2.398 & 0.772 \\
			\vdots & \vdots & \vdots & \vdots & \vdots & \vdots & \vdots & \vdots \\
			1990 & 19 & 0.0008 & 12.75 & 0.569 & 1 & 2.012 & 1.145 \\
			\bottomrule
		\end{tabular}
	\end{center}
	\vspace{5pt}
	\begin{wideitemize}
		\item 2,217 products across 20 years (1971--1990)
		\item Each row: one car model in one year
	\end{wideitemize}
\end{frame}

\begin{frame}[fragile]{Step 2: Define the formulation}
\begin{wideitemize}
	\item Specify which variables enter the utility function
	\item Variables: \texttt{1} (constant), \texttt{hpwt} (HP/weight), \texttt{air} (A/C), \texttt{mpd} (MPG), \texttt{space}, \texttt{prices}
\end{wideitemize}
\vspace{5pt}
\begin{pycode}
formulation = pyblp.Formulation('1 + hpwt + air + mpd + space + prices')
\end{pycode}
\end{frame}

\begin{frame}[fragile]{Step 3: Create the problem}
\begin{wideitemize}
	\item Combine formulation and data into a Problem object
	\item pyblp automatically detects: \texttt{demand\_instruments}, \texttt{market\_ids}, \texttt{firm\_ids}
\end{wideitemize}
\vspace{5pt}
\begin{pycode}
problem = pyblp.Problem(formulation, product_data)
\end{pycode}
\end{frame}

\begin{frame}[fragile]{Step 4: Solve}
\begin{wideitemize}
	\item Estimate the model (uses IV if instruments present)
	\item Returns: coefficient estimates, robust standard errors, GMM objective
\end{wideitemize}
\vspace{5pt}
\begin{pycode}
results = problem.solve()
\end{pycode}
\end{frame}

\begin{frame}[fragile]{Step 5: View results}
\begin{wideitemize}
	\item Print coefficient estimates and standard errors
	\item Compute elasticity matrix (own and cross-price)
	\item Compute markups (assuming Nash-Bertrand pricing)
\end{wideitemize}
\vspace{5pt}
\begin{pycode}
print(results)
elasticities = results.compute_elasticities()
markups = results.compute_markups()
\end{pycode}
\end{frame}

\begin{frame}[fragile]{Adding demographic interactions}
\begin{wideitemize}
	\item Load demographic data (income draws for each market)
	\item Add second formulation: demographics interact with price
\end{wideitemize}
\vspace{5pt}
\begin{pycode}
agent_data = pd.read_csv(pyblp.data.BLP_AGENTS_LOCATION)

formulation = pyblp.Formulation(
    '1 + hpwt + air + mpd + space + prices',
    '0 + prices',
    agent_formulation=pyblp.Formulation('0 + income')
)
\end{pycode}
\end{frame}

\begin{frame}[fragile]{Computing aggregate elasticities}
\begin{wideitemize}
	\item \texttt{elasticities[j,k]} = \% change in share of $j$ per \% change in price of $k$
	\item Diagonal = own-price elasticities
\end{wideitemize}
\vspace{5pt}
\begin{pycode}
elasticities = results.compute_elasticities()
own_elasticities = np.diag(elasticities)
print(f"Mean own-price elasticity: {own_elasticities.mean():.2f}")
\end{pycode}
\end{frame}

\begin{frame}{Estimation output}
	\begin{center}
		\begin{tabular}{lcc}
			\toprule
			\textbf{Variable} & \textbf{Coefficient} & \textbf{Std. Error} \\
			\midrule
			Constant & $-10.215$ & $(0.253)$ \\
			HP/Weight & $2.893$ & $(0.367)$ \\
			Air conditioning & $1.521$ & $(0.104)$ \\
			Miles per dollar & $0.158$ & $(0.043)$ \\
			Space & $2.384$ & $(0.125)$ \\
			Price & $-0.142$ & $(0.012)$ \\
			\midrule
			\multicolumn{3}{l}{\small GMM objective: 142.35} \\
			\multicolumn{3}{l}{\small N = 2,217 products, T = 20 markets} \\
			\bottomrule
		\end{tabular}
	\end{center}
	\vspace{3pt}
	\begin{wideitemize}
		\item Price coefficient is \textbf{negative} (as expected)
		\item All characteristics positive: consumers prefer more HP, air conditioning, fuel efficiency, space
	\end{wideitemize}
\end{frame}

\begin{frame}{Plan for today}
  \begin{wideenumerate}
    \item Recap: endogeneity and IVs
    \item The demographic interaction model
    \item Why demographics help with IIA
    \item[] \rule{0.5\textwidth}{0.4pt}
    \item Introduction to pyblp
    \item Worked example: car demand
    \item \textbf{Interpreting output}
  \end{wideenumerate}
\end{frame}

\begin{frame}{Interpreting pyblp output}
	\begin{wideitemize}
		\item \textbf{Coefficients:}
		\begin{wideitemize}
			\vspace{5pt}
			\item $\hat{\alpha}$ (price): should be negative
			\item $\hat{\beta}$ (characteristics): interpret as marginal utility
		\end{wideitemize}
		\item \textbf{Standard errors:}
		\begin{wideitemize}
			\vspace{5pt}
			\item Check statistical significance
			\item pyblp computes robust SEs by default
		\end{wideitemize}
		\item \textbf{First-stage F-statistic:}
		\begin{wideitemize}
			\vspace{5pt}
			\item Check that IVs are relevant
			\item Rule of thumb: $F > 10$
		\end{wideitemize}
	\end{wideitemize}
\end{frame}

\begin{frame}{Worked example: Interpreting coefficients}
	\begin{wideitemize}
		\item Suppose you estimate:
		\begin{wideitemize}
			\vspace{5pt}
			\item $\hat{\alpha} = -0.8$ (price coefficient)
			\item $\hat{\beta}_{HP} = 0.3$ (horsepower coefficient)
		\end{wideitemize}
		\item \textbf{Questions:}
		\begin{wideenumerate}
			\vspace{5pt}
			\item Interpret $\hat{\alpha}$. What does a more negative $\alpha$ mean?
			\item If you had used OLS instead of IV, would $\hat{\alpha}$ be more or less negative?
		\end{wideenumerate}
	\end{wideitemize}
\end{frame}

\begin{frame}{Worked example: Interpreting coefficients (answers)}
	\begin{solutionbox}[Answers]
		\textbf{1.} $\hat{\alpha} = -0.8$: A \$1 price increase reduces mean utility by 0.8 utils
		\begin{itemize}
			\item More negative $\alpha$ = more price-sensitive consumers
		\end{itemize}
		\vspace{8pt}
		\textbf{2.} OLS would give $\hat{\alpha}$ biased toward zero (less negative)
		\begin{itemize}
			\item Because $\text{Cov}(p, \xi) > 0$
			\item OLS thinks high prices don't hurt demand much
			\item So OLS $\hat{\alpha}$ might be $-0.3$ instead of $-0.8$
		\end{itemize}
	\end{solutionbox}
\end{frame}

\begin{frame}{Post-estimation: Elasticities}
	\begin{wideitemize}
		\item pyblp computes elasticities automatically:
		\begin{wideitemize}
			\vspace{5pt}
			\item Own-price elasticity for each product
			\item Cross-price elasticity matrix
		\end{wideitemize}
		\item Check if elasticities are reasonable:
		\begin{wideitemize}
			\vspace{5pt}
			\item Own-price should be negative
			\item Magnitude: typically $-2$ to $-10$ for durable goods
			\item Cross-price: positive for substitutes
		\end{wideitemize}
	\end{wideitemize}
\end{frame}

\begin{frame}{Elasticity matrix example}
	\begin{center}
		\small
		\begin{tabular}{l|ccccc}
			& \textbf{Prod 1} & \textbf{Prod 2} & \textbf{Prod 3} & \textbf{Prod 4} & \textbf{Prod 5} \\
			\hline
			\textbf{Prod 1} & \textcolor{red}{$-0.89$} & $0.002$ & $0.003$ & $0.001$ & $0.002$ \\
			\textbf{Prod 2} & $0.002$ & \textcolor{red}{$-1.24$} & $0.003$ & $0.002$ & $0.003$ \\
			\textbf{Prod 3} & $0.001$ & $0.002$ & \textcolor{red}{$-0.73$} & $0.001$ & $0.002$ \\
			\textbf{Prod 4} & $0.002$ & $0.003$ & $0.002$ & \textcolor{red}{$-1.08$} & $0.002$ \\
			\textbf{Prod 5} & $0.001$ & $0.002$ & $0.001$ & $0.001$ & \textcolor{red}{$-0.95$} \\
		\end{tabular}
	\end{center}
	\vspace{5pt}
	\begin{wideitemize}
		\item \textbf{Diagonal (\textcolor{red}{red}):} Own-price elasticities $\approx -0.7$ to $-1.3$
		\item \textbf{Off-diagonal:} Cross-price elasticities $\approx 0.001$ to $0.003$ (small but positive)
	\end{wideitemize}
\end{frame}

\begin{frame}{Post-estimation: Markups}
	\begin{wideitemize}
		\item Recall: Lerner index $L = (p - MC)/p = 1/|\varepsilon|$
		\item Can recover markups from elasticities:
		\begin{align*}
			\text{markup}_j = \frac{p_j - mc_j}{p_j} = \frac{1}{|\eta_{jj}|}
		\end{align*}
		\item pyblp assumes Nash-Bertrand pricing
		\item Multi-product firms internalize substitution between own products
	\end{wideitemize}
\end{frame}

\begin{frame}{Worked example: Interpret this output}
	\begin{wideitemize}
		\item pyblp gives you this output:
		\begin{center}
		\begin{tabular}{lcc}
			\toprule
			\textbf{Variable} & \textbf{Estimate} & \textbf{Std. Error} \\
			\midrule
			Constant & 3.5 & 0.8 \\
			HP/weight & 1.2 & 0.4 \\
			Air cond. & 0.9 & 0.3 \\
			Price & \textcolor{red}{0.05} & 0.02 \\
			\bottomrule
		\end{tabular}
		\end{center}
		\item \textbf{Question:} Something is wrong. What is it?
	\end{wideitemize}
	\vspace{15pt}
	\centering
	\textit{Take 1 minute to identify the problem.}
\end{frame}

\begin{frame}{Worked example: Interpret this output (solution)}
	\begin{solutionbox}[Solution]
		\textbf{Problem:} Price coefficient is \textcolor{red}{positive} (+0.05)!

		This means: higher prices $\rightarrow$ higher utility. That's economically nonsensical.
		\vspace{8pt}

		\textbf{Why might this happen?}
		\begin{enumerate}
			\item \textbf{Weak instruments:} IVs don't predict prices well
			\item \textbf{Wrong signs:} Maybe you forgot a negative sign somewhere
			\item \textbf{Endogeneity still present:} IVs are correlated with $\xi$
		\end{enumerate}
		\vspace{5pt}
		\textbf{Fix:} Check first-stage F-stat, try different IVs, verify data
	\end{solutionbox}
\end{frame}

\begin{frame}{Common pyblp errors and how to fix them}
	\begin{wideitemize}
		\item \textbf{``Singular matrix'' error:}
		\begin{wideitemize}
			\vspace{3pt}
			\item Likely: collinear variables in formulation
			\item Fix: Remove redundant variables
		\end{wideitemize}
		\item \textbf{Price coefficient positive or near zero:}
		\begin{wideitemize}
			\vspace{3pt}
			\item Likely: weak instruments
			\item Fix: Check first-stage F-stat; try different IVs
		\end{wideitemize}
		\item \textbf{Huge standard errors:}
		\begin{wideitemize}
			\vspace{3pt}
			\item Likely: weak identification or too few observations
			\item Fix: Aggregate to fewer markets; add more IVs
		\end{wideitemize}
	\end{wideitemize}
\end{frame}

\begin{frame}{How to check your results make sense}
	\begin{wideenumerate}
		\item \textbf{Price coefficient negative?} (Required!)
		\item \textbf{Elasticities reasonable?}
		\begin{wideitemize}
			\vspace{3pt}
			\item Own-price: typically $-2$ to $-10$ for durables
			\item Cross-price: positive but small
		\end{wideitemize}
		\item \textbf{Markups reasonable?}
		\begin{wideitemize}
			\vspace{3pt}
			\item Cars: typically 15-30\%
			\item Cereal: typically 30-50\%
		\end{wideitemize}
		\item \textbf{First-stage F-stat $> 10$?} (IV relevance)
		\item \textbf{Coefficients stable across specifications?}
	\end{wideenumerate}
\end{frame}

\begin{frame}{This prepares you for HW1}
	\begin{wideitemize}
		\item HW1 asks you to:
		\begin{wideenumerate}
			\vspace{5pt}
			\item Load car data
			\item Estimate demand with pyblp
			\item Compute elasticities
			\item Interpret your results
			\item Discuss IV choice
		\end{wideenumerate}
		\item Today's worked example is a template for HW1
		\item Start early!
	\end{wideitemize}
\end{frame}

%%%%%%%%%%%%%%%%%%%%%%%%%%%%%%%%%%%%%%%%%%%%%%%%%%%%%%%%%%%%%
% KEY POINTS
%%%%%%%%%%%%%%%%%%%%%%%%%%%%%%%%%%%%%%%%%%%%%%%%%%%%%%%%%%%%%

\begin{frame}{Key Points}
	\vspace{11pt}
	\begin{wideenumerate}
		\item \textbf{Demographic interaction model}: $u_{ij} = x_j \beta + (D_i \times x_j)\gamma - \alpha p_j + \xi_j + \varepsilon_{ij}$
		\item Demographics allow \textbf{heterogeneous preferences} without random coefficients
		\item Demographics \textbf{partially help} with IIA (different types substitute differently)
		\item \textbf{pyblp workflow}: Data $\rightarrow$ Formulation $\rightarrow$ Problem $\rightarrow$ Solve $\rightarrow$ Interpret
		\item \textbf{Key outputs}: Coefficients (check signs), standard errors, elasticities, markups
		\item Price coefficient should be \textbf{negative}; OLS biases it toward zero
		\item \textbf{Check IV relevance}: First-stage F-statistic $> 10$
		\item This is the foundation for HW1
	\end{wideenumerate}
\end{frame}

\begin{frame}{Next time}
	\begin{wideitemize}
		\item \textbf{Lecture 4:} Consumer Surplus, IIA, and Price Discrimination
		\begin{wideitemize}
			\vspace{5pt}
			\item Log-sum formula for consumer surplus
			\item Red bus/blue bus: the IIA problem in detail
			\item Selection by indicators (group pricing)
		\end{wideitemize}
	\end{wideitemize}
\end{frame}

\end{document}
