\documentclass[notes,11pt, aspectratio=169]{beamer}

\usepackage{pgfpages}
\setbeameroption{hide notes} % Only slide

\usepackage{array}
\usepackage{tikz}
\usepackage{verbatim}
\setbeamertemplate{note page}{\pagecolor{gray!5}\insertnote}
\usetikzlibrary{positioning}
\usetikzlibrary{snakes}
\usetikzlibrary{calc}
\usetikzlibrary{arrows}
\usetikzlibrary{decorations.markings}
\usetikzlibrary{shapes.misc}
\usetikzlibrary{matrix,shapes,arrows,fit,tikzmark}
\usepackage{amsmath}
\usepackage{mathpazo}
\usepackage{hyperref}
\usepackage{lipsum}
\usepackage{multimedia}
\usepackage{graphicx}
\usepackage{multirow}
\usepackage{dcolumn}
\usepackage{bbm}
\usepackage{listings}
\newcolumntype{d}[0]{D{.}{.}{5}}

\usepackage{changepage}
\usepackage{appendixnumberbeamer}

\usepackage[space]{grffile}
\usepackage{booktabs}

% Colors
\definecolor{blue}{RGB}{0,114,178}
\definecolor{red}{RGB}{213,94,0}
\definecolor{yellow}{RGB}{240,228,66}
\definecolor{green}{RGB}{0,158,115}

\hypersetup{
  colorlinks=false,
  linkbordercolor = {white},
  linkcolor = {blue}
}

\definecolor{MyBackground}{RGB}{255,253,218}

\newenvironment{transitionframe}{
  \setbeamercolor{background canvas}{bg=white}
  \begin{frame}}{
    \end{frame}
}

\setbeamercolor{frametitle}{fg=blue}
\setbeamercolor{title}{fg=black}
\setbeamertemplate{footline}[frame number]
\setbeamertemplate{navigation symbols}{}
\setbeamertemplate{itemize items}{-}
\setbeamercolor{itemize item}{fg=blue}
\setbeamercolor{itemize subitem}{fg=blue}
\setbeamercolor{enumerate item}{fg=blue}
\setbeamercolor{enumerate subitem}{fg=blue}
\setbeamercolor{button}{bg=MyBackground,fg=blue,}

\setbeamercolor{section in toc}{fg=blue}
\setbeamercolor{subsection in toc}{fg=red}
\setbeamersize{text margin left=1em,text margin right=1em}

\newenvironment{wideitemize}{\itemize\addtolength{\itemsep}{10pt}}{\enditemize}
\newenvironment{wideenumerate}{\enumerate\addtolength{\itemsep}{10pt}}{\endenumerate}

% Code listing style
\lstset{
  basicstyle=\ttfamily\small,
  breaklines=true,
  frame=single,
  language=Python,
  showstringspaces=false,
  keywordstyle=\color{blue},
  commentstyle=\color{green!60!black},
  stringstyle=\color{red}
}

\title[]{\textcolor{blue}{ECN 594: Demographic Interactions and pyblp}}
\author[PGP]{}
\institute[FRBNY]{\small{\begin{tabular}{c c c}
Nicholas Vreugdenhil \\
\end{tabular}}}
\date{\today}

\begin{document}

% Title Slide
\begin{frame}
\maketitle
  \centering
\end{frame}

\begin{frame}{Plan for today}
  \begin{wideenumerate}
    \item Recap: endogeneity and IVs
    \item The demographic interaction model
    \item Why demographics help with IIA
    \item[] \rule{0.5\textwidth}{0.4pt}
    \item Introduction to pyblp
    \item Worked example: car demand
    \item Interpreting output
  \end{wideenumerate}
\end{frame}

%%%%%%%%%%%%%%%%%%%%%%%%%%%%%%%%%%%%%%%%%%%%%%%%%%%%%%%%%%%%%
% PART 1: DEMOGRAPHIC INTERACTIONS
%%%%%%%%%%%%%%%%%%%%%%%%%%%%%%%%%%%%%%%%%%%%%%%%%%%%%%%%%%%%%

\begin{transitionframe}
	\begin{center}
		{\Huge Part 1: Demand with Demographic Interactions}
	\end{center}
\end{transitionframe}

\begin{frame}{Recap: The basic logit model}
	\begin{wideitemize}
		\item Our utility model so far:
		\begin{align*}
			u_{ij} = x_j \beta - \alpha p_j + \xi_j + \varepsilon_{ij}
		\end{align*}
		\item Everyone has the same $\beta$ and $\alpha$
		\item Problem: this is very restrictive
		\begin{wideitemize}
			\vspace{5pt}
			\item Rich and poor consumers have same price sensitivity?
			\item Families and singles value car size the same?
		\end{wideitemize}
	\end{wideitemize}
\end{frame}

\begin{frame}{Extending the model: heterogeneous preferences}
	\begin{wideitemize}
		\item \textbf{Basic logit:} Everyone has same $\beta$
		\item Problem: doesn't capture that different consumers value characteristics differently
		\item \textbf{Solution:} Let preferences vary with observed demographics
	\end{wideitemize}
\end{frame}

\begin{frame}{The demographic interaction model}
	\begin{wideitemize}
		\item New utility specification:
		\begin{align*}
			u_{ij} = x_j \beta + (D_i \times x_j) \gamma - \alpha p_j + \xi_j + \varepsilon_{ij}
		\end{align*}
		\item $D_i$: observed consumer demographics (income, age, family size, etc.)
		\item $(D_i \times x_j)$: interactions between demographics and characteristics
		\item This creates \textbf{heterogeneous preferences}
	\end{wideitemize}
\end{frame}

\begin{frame}{Examples of demographic interactions}
	\begin{wideitemize}
		\item \textbf{Income $\times$ price:}
		\begin{wideitemize}
			\vspace{5pt}
			\item High-income consumers less price-sensitive
			\item $\gamma_{\text{inc} \times p} > 0$: price hurts less for rich consumers
		\end{wideitemize}
		\item \textbf{Family size $\times$ car size:}
		\begin{wideitemize}
			\vspace{5pt}
			\item Families prefer larger vehicles
			\item $\gamma_{\text{fam} \times \text{size}} > 0$
		\end{wideitemize}
		\item \textbf{Age $\times$ fuel efficiency:}
		\begin{wideitemize}
			\vspace{5pt}
			\item Older consumers may care more about MPG
		\end{wideitemize}
	\end{wideitemize}
\end{frame}

\begin{frame}{Where do demographics come from?}
	\begin{wideitemize}
		\item Two scenarios:
		\begin{wideenumerate}
			\vspace{5pt}
			\item \textbf{Individual-level data:} Observe $D_i$ for each consumer
			\begin{wideitemize}
				\item Survey data, loyalty card data
			\end{wideitemize}
			\item \textbf{Market-level data:} Know distribution of $D_i$ in each market
			\begin{wideitemize}
				\item Census data, Current Population Survey
				\item This is more common in practice
			\end{wideitemize}
		\end{wideenumerate}
		\item pyblp handles both cases
	\end{wideitemize}
\end{frame}

\begin{frame}{Why not random coefficients?}
	\begin{wideitemize}
		\item Full BLP/mixed logit model adds unobserved heterogeneity:
		\begin{align*}
			\beta_i = \bar{\beta} + \Sigma \nu_i, \quad \nu_i \sim N(0, I)
		\end{align*}
		\item This is computationally harder (requires simulation)
		\item Demographic interactions capture a lot of the variation more simply
		\item \textbf{Mixed logit} is beyond our scope, but know it exists
		\item It fully relaxes IIA
	\end{wideitemize}
\end{frame}

\begin{frame}{Demographics and IIA (preview)}
	\begin{wideitemize}
		\item Recall: basic logit has IIA problem
		\begin{wideitemize}
			\vspace{5pt}
			\item When BMW price rises, same fraction goes to Mercedes as to Civic
		\end{wideitemize}
		\item With demographics, different consumer types substitute differently
		\begin{wideitemize}
			\vspace{5pt}
			\item High-income BMW buyers $\rightarrow$ Mercedes
			\item Low-income BMW buyers $\rightarrow$ Civic
		\end{wideitemize}
		\item Aggregate substitution is richer
		\item But IIA still holds \textit{within} each consumer type
		\item Full discussion in Lecture 4
	\end{wideitemize}
\end{frame}

%%%%%%%%%%%%%%%%%%%%%%%%%%%%%%%%%%%%%%%%%%%%%%%%%%%%%%%%%%%%%
% PART 2: pyblp ESTIMATION
%%%%%%%%%%%%%%%%%%%%%%%%%%%%%%%%%%%%%%%%%%%%%%%%%%%%%%%%%%%%%

\begin{transitionframe}
	\begin{center}
		{\Huge Part 2: Estimation with pyblp}
	\end{center}
\end{transitionframe}

\begin{frame}{What is pyblp?}
	\begin{wideitemize}
		\item Python package for demand estimation
		\item Conlon \& Gortmaker (2020), ``Best Practices for Differentiated Products Demand Estimation with PyBLP''
		\item Handles:
		\begin{wideitemize}
			\vspace{5pt}
			\item Basic logit and logit with demographics
			\item Instrumental variables
			\item Standard errors
			\item Post-estimation (elasticities, markups, etc.)
		\end{wideitemize}
		\item Why use a package?
		\begin{wideitemize}
			\vspace{5pt}
			\item Correct standard errors
			\item Well-tested code
			\item Saves you from making mistakes
		\end{wideitemize}
	\end{wideitemize}
\end{frame}

\begin{frame}{pyblp workflow}
	\begin{wideenumerate}
		\item \textbf{Set up data:} products, markets, shares, characteristics
		\item \textbf{Define formulation:} which variables, which IVs
		\item \textbf{Create problem:} combine data and formulation
		\item \textbf{Solve:} estimate the model
		\item \textbf{Extract results:} coefficients, standard errors, elasticities
	\end{wideenumerate}
	\vspace{11pt}
	\textit{Let's walk through each step with car data}
\end{frame}

\begin{frame}[fragile]{Step 1: Load and inspect data}
\begin{lstlisting}
import pyblp

# Load the BLP automobile data
product_data = pyblp.data.BLP_PRODUCTS

# Key columns:
# - market_ids: which market (year)
# - shares: market shares
# - prices: prices
# - hpwt, space, mpd: characteristics
# - demand_instruments0, ...: IVs
\end{lstlisting}
\end{frame}

\begin{frame}[fragile]{Step 2: Define the formulation}
\begin{lstlisting}
# Basic logit: no random coefficients
formulation = pyblp.Formulation(
    '1 + hpwt + space + mpd + prices'
)

# With demographic interactions (income x price):
formulation = pyblp.Formulation(
    '1 + hpwt + space + mpd + prices',
    absorb='C(market_ids)'  # market fixed effects
)
\end{lstlisting}
\end{frame}

\begin{frame}[fragile]{Step 3: Create the problem}
\begin{lstlisting}
# Define the problem
problem = pyblp.Problem(
    formulation,
    product_data,
    agent_data=agent_data  # for demographics
)

# Check the problem
print(problem)
\end{lstlisting}
\end{frame}

\begin{frame}[fragile]{Step 4: Solve}
\begin{lstlisting}
# Solve with 2SLS (IV estimation)
results = problem.solve()

# This gives us:
# - Coefficient estimates
# - Standard errors
# - Objective value
\end{lstlisting}
\end{frame}

\begin{frame}[fragile]{Step 5: Extract results}
\begin{lstlisting}
# Print coefficient estimates
print(results)

# Get elasticities
elasticities = results.compute_elasticities()

# Get markups (assuming Nash-Bertrand)
markups = results.compute_markups()
\end{lstlisting}
\end{frame}

\begin{frame}{Interpreting pyblp output}
	\begin{wideitemize}
		\item \textbf{Coefficients:}
		\begin{wideitemize}
			\vspace{5pt}
			\item $\hat{\alpha}$ (price): should be negative
			\item $\hat{\beta}$ (characteristics): interpret as marginal utility
		\end{wideitemize}
		\item \textbf{Standard errors:}
		\begin{wideitemize}
			\vspace{5pt}
			\item Check statistical significance
			\item pyblp computes robust SEs by default
		\end{wideitemize}
		\item \textbf{First-stage F-statistic:}
		\begin{wideitemize}
			\vspace{5pt}
			\item Check that IVs are relevant
			\item Rule of thumb: $F > 10$
		\end{wideitemize}
	\end{wideitemize}
\end{frame}

\begin{frame}{Worked example: Interpreting coefficients}
	\begin{wideitemize}
		\item Suppose you estimate:
		\begin{wideitemize}
			\vspace{5pt}
			\item $\hat{\alpha} = -0.8$ (price coefficient)
			\item $\hat{\beta}_{HP} = 0.3$ (horsepower coefficient)
		\end{wideitemize}
		\item \textbf{Questions:}
		\begin{wideenumerate}
			\vspace{5pt}
			\item Interpret $\hat{\alpha}$. What does a more negative $\alpha$ mean?
			\item If you had used OLS instead of IV, would $\hat{\alpha}$ be more or less negative?
		\end{wideenumerate}
	\end{wideitemize}
\end{frame}

\begin{frame}{Worked example: Interpreting coefficients (answers)}
	\begin{wideenumerate}
		\item $\hat{\alpha} = -0.8$: A \$1 price increase reduces mean utility by 0.8 utils
		\begin{wideitemize}
			\vspace{3pt}
			\item More negative $\alpha$ = more price-sensitive consumers
		\end{wideitemize}
		\vspace{10pt}
		\item OLS would give $\hat{\alpha}$ biased toward zero (less negative)
		\begin{wideitemize}
			\vspace{3pt}
			\item Because $\text{Cov}(p, \xi) > 0$
			\item OLS thinks high prices don't hurt demand much
			\item So OLS $\hat{\alpha}$ might be $-0.3$ instead of $-0.8$
		\end{wideitemize}
	\end{wideenumerate}
\end{frame}

\begin{frame}{Post-estimation: Elasticities}
	\begin{wideitemize}
		\item pyblp computes elasticities automatically:
		\begin{wideitemize}
			\vspace{5pt}
			\item Own-price elasticity for each product
			\item Cross-price elasticity matrix
		\end{wideitemize}
		\item Check if elasticities are reasonable:
		\begin{wideitemize}
			\vspace{5pt}
			\item Own-price should be negative
			\item Magnitude: typically $-2$ to $-10$ for durable goods
			\item Cross-price: positive for substitutes
		\end{wideitemize}
	\end{wideitemize}
\end{frame}

\begin{frame}{Post-estimation: Markups}
	\begin{wideitemize}
		\item Recall: Lerner index $L = (p - MC)/p = 1/|\varepsilon|$
		\item Can recover markups from elasticities:
		\begin{align*}
			\text{markup}_j = \frac{p_j - mc_j}{p_j} = \frac{1}{|\eta_{jj}|}
		\end{align*}
		\item pyblp assumes Nash-Bertrand pricing
		\item Multi-product firms internalize substitution between own products
	\end{wideitemize}
\end{frame}

\begin{frame}{This prepares you for HW1}
	\begin{wideitemize}
		\item HW1 asks you to:
		\begin{wideenumerate}
			\vspace{5pt}
			\item Load car data
			\item Estimate demand with pyblp
			\item Compute elasticities
			\item Interpret your results
			\item Discuss IV choice
		\end{wideenumerate}
		\item Today's worked example is a template for HW1
		\item Start early!
	\end{wideitemize}
\end{frame}

%%%%%%%%%%%%%%%%%%%%%%%%%%%%%%%%%%%%%%%%%%%%%%%%%%%%%%%%%%%%%
% KEY POINTS
%%%%%%%%%%%%%%%%%%%%%%%%%%%%%%%%%%%%%%%%%%%%%%%%%%%%%%%%%%%%%

\begin{frame}{Key Points}
	\vspace{11pt}
	\begin{wideenumerate}
		\item \textbf{Demographic interaction model}: $u_{ij} = x_j \beta + (D_i \times x_j)\gamma - \alpha p_j + \xi_j + \varepsilon_{ij}$
		\item Demographics allow \textbf{heterogeneous preferences} without random coefficients
		\item Demographics \textbf{partially help} with IIA (different types substitute differently)
		\item \textbf{pyblp workflow}: Data $\rightarrow$ Formulation $\rightarrow$ Problem $\rightarrow$ Solve $\rightarrow$ Interpret
		\item \textbf{Key outputs}: Coefficients (check signs), standard errors, elasticities, markups
		\item Price coefficient should be \textbf{negative}; OLS biases it toward zero
		\item \textbf{Check IV relevance}: First-stage F-statistic $> 10$
		\item This is the foundation for HW1
	\end{wideenumerate}
\end{frame}

\begin{frame}{Next time}
	\begin{wideitemize}
		\item \textbf{Lecture 4:} Consumer Surplus, IIA, and Price Discrimination
		\begin{wideitemize}
			\vspace{5pt}
			\item Log-sum formula for consumer surplus
			\item Red bus/blue bus: the IIA problem in detail
			\item Selection by indicators (group pricing)
		\end{wideitemize}
	\end{wideitemize}
\end{frame}

\end{document}
