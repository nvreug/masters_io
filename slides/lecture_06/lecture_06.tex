\documentclass[notes,11pt, aspectratio=169]{beamer}

\usepackage{pgfpages}
\setbeameroption{hide notes} % Only slide

\usepackage{array}
\usepackage{tikz}
\usepackage{verbatim}
\setbeamertemplate{note page}{\pagecolor{gray!5}\insertnote}
\usetikzlibrary{positioning}
\usetikzlibrary{snakes}
\usetikzlibrary{calc}
\usetikzlibrary{arrows}
\usetikzlibrary{decorations.markings}
\usetikzlibrary{shapes.misc}
\usetikzlibrary{matrix,shapes,arrows,fit,tikzmark}
\usepackage{amsmath}
\usepackage{mathpazo}
\usepackage{hyperref}
\usepackage{lipsum}
\usepackage{multimedia}
\usepackage{graphicx}
\usepackage{multirow}
\usepackage{dcolumn}
\usepackage{bbm}
\newcolumntype{d}[0]{D{.}{.}{5}}

\usepackage{changepage}
\usepackage{appendixnumberbeamer}

\usepackage[space]{grffile}
\usepackage{booktabs}

% Colors
\definecolor{blue}{RGB}{0,114,178}
\definecolor{red}{RGB}{213,94,0}
\definecolor{yellow}{RGB}{240,228,66}
\definecolor{green}{RGB}{0,158,115}

\hypersetup{
  colorlinks=false,
  linkbordercolor = {white},
  linkcolor = {blue}
}

\definecolor{MyBackground}{RGB}{255,253,218}

\newenvironment{transitionframe}{
  \setbeamercolor{background canvas}{bg=white}
  \begin{frame}}{
    \end{frame}
}

\setbeamercolor{frametitle}{fg=blue}
\setbeamercolor{title}{fg=black}
\setbeamertemplate{footline}[frame number]
\setbeamertemplate{navigation symbols}{}
\setbeamertemplate{itemize items}{-}
\setbeamercolor{itemize item}{fg=blue}
\setbeamercolor{itemize subitem}{fg=blue}
\setbeamercolor{enumerate item}{fg=blue}
\setbeamercolor{enumerate subitem}{fg=blue}
\setbeamercolor{button}{bg=MyBackground,fg=blue,}

\setbeamercolor{section in toc}{fg=blue}
\setbeamercolor{subsection in toc}{fg=red}
\setbeamersize{text margin left=1em,text margin right=1em}

\newenvironment{wideitemize}{\itemize\addtolength{\itemsep}{10pt}}{\enditemize}
\newenvironment{wideenumerate}{\enumerate\addtolength{\itemsep}{10pt}}{\endenumerate}

\title[]{\textcolor{blue}{ECN 594: Midterm Review}}
\author[PGP]{}
\institute[FRBNY]{\small{\begin{tabular}{c c c}
Nicholas Vreugdenhil \\
\end{tabular}}}
\date{\today}

\begin{document}

% Title Slide
\begin{frame}
\maketitle
  \centering
\end{frame}

\begin{frame}{Plan for today}
  \begin{wideenumerate}
    \item Review: Demand estimation
    \item Practice problems: Demand
    \item[] \rule{0.5\textwidth}{0.4pt}
    \item Review: Pricing and price discrimination
    \item Practice problems: Pricing
    \item Exam logistics
  \end{wideenumerate}
\end{frame}

\begin{frame}{Midterm format}
	\begin{wideitemize}
		\item \textbf{Duration:} 70 minutes
		\item \textbf{Allowed:} Calculator + two-sided cheat sheet (letter-size paper)
		\item \textbf{Coverage:} Lectures 1-5 (all of Part 1)
		\item \textbf{Structure:}
		\begin{wideitemize}
			\vspace{5pt}
			\item Short answer questions (T/F/NEI, definitions, quick calculations)
			\item Longer problems (derivations, pricing, elasticity calculations)
		\end{wideitemize}
	\end{wideitemize}
\end{frame}

%%%%%%%%%%%%%%%%%%%%%%%%%%%%%%%%%%%%%%%%%%%%%%%%%%%%%%%%%%%%%
% PART 1: DEMAND ESTIMATION REVIEW
%%%%%%%%%%%%%%%%%%%%%%%%%%%%%%%%%%%%%%%%%%%%%%%%%%%%%%%%%%%%%

\begin{transitionframe}
	\begin{center}
		{\Huge Part 1: Demand Estimation Review}
	\end{center}
\end{transitionframe}

\begin{frame}{Why demand estimation?}
	\begin{wideitemize}
		\item We need demand models to:
		\begin{wideenumerate}
			\vspace{5pt}
			\item Measure substitution patterns between products
			\item Compute price elasticities
			\item Evaluate policy (mergers, new products, price changes)
			\item Calculate consumer welfare
		\end{wideenumerate}
		\item The dimensionality problem: $J$ products $\rightarrow$ $J^2$ elasticities
		\item Solution: characteristics-based models (Lancaster, BLP)
	\end{wideitemize}
\end{frame}

\begin{frame}{The logit model: key equations}
	\begin{wideitemize}
		\item Utility: $u_{ij} = \delta_j + \varepsilon_{ij}$ where $\delta_j = x_j\beta + \alpha p_j + \xi_j$
		\item Choice probability (market share):
		\begin{align*}
			s_j = \frac{\exp(\delta_j)}{1 + \sum_{k=1}^J \exp(\delta_k)}
		\end{align*}
		\item Outside option: $s_0 = \frac{1}{1 + \sum_{k=1}^J \exp(\delta_k)}$
		\item \textbf{Berry inversion:}
		\begin{align*}
			\ln(s_j) - \ln(s_0) = \delta_j
		\end{align*}
	\end{wideitemize}
\end{frame}

\begin{frame}{Logit elasticities}
	\begin{wideitemize}
		\item \textbf{Own-price elasticity:}
		\begin{align*}
			\eta_{jj} = \frac{\partial s_j}{\partial p_j} \cdot \frac{p_j}{s_j} = \alpha p_j (1 - s_j)
		\end{align*}
		\item \textbf{Cross-price elasticity:}
		\begin{align*}
			\eta_{jk} = \frac{\partial s_j}{\partial p_k} \cdot \frac{p_k}{s_j} = -\alpha p_k s_k
		\end{align*}
		\item Note: $\alpha < 0$ so own-price elasticity is negative
		\item \textbf{IIA problem:} Cross-elasticity depends only on $k$, not on similarity to $j$
	\end{wideitemize}
\end{frame}

\begin{frame}{Practice: Elasticity calculation}
	\begin{wideitemize}
		\item \textbf{Question:} Suppose $\alpha = -0.5$, product A has price $p_A = 20$ and share $s_A = 0.1$.
		\item (a) Calculate the own-price elasticity of product A.
		\item (b) If product B has $p_B = 25$ and $s_B = 0.15$, what is the cross-price elasticity of A with respect to B's price?
	\end{wideitemize}
	\vspace{15pt}
	\centering
	\textit{Take 3 minutes.}
\end{frame}

\begin{frame}{Practice: Elasticity calculation (solution)}
	\begin{wideitemize}
		\item \textbf{(a) Own-price elasticity:}
		\begin{align*}
			\eta_{AA} = \alpha p_A (1 - s_A) = (-0.5)(20)(1 - 0.1) = -9
		\end{align*}
		\item Demand is elastic (a 1\% price increase reduces quantity by 9\%)
		\item \textbf{(b) Cross-price elasticity:}
		\begin{align*}
			\eta_{AB} = -\alpha p_B s_B = -(-0.5)(25)(0.15) = 1.875
		\end{align*}
		\item A 1\% increase in B's price increases A's share by 1.875\%
	\end{wideitemize}
\end{frame}

\begin{frame}{The identification problem}
	\begin{wideitemize}
		\item \textbf{Problem:} We observe equilibrium $(p, q)$ pairs
		\item Can't tell if demand shifted or supply shifted
		\item \textbf{Price endogeneity:} High unobserved quality $\xi_j \rightarrow$ high price
		\begin{wideitemize}
			\vspace{5pt}
			\item OLS sees: high price, still high demand
			\item Concludes: price doesn't matter much
			\item Result: $\hat{\alpha}$ biased toward zero
		\end{wideitemize}
		\item \textbf{Solution:} Instrumental variables
		\begin{wideitemize}
			\vspace{5pt}
			\item Need: correlated with price, uncorrelated with $\xi$
			\item Examples: Hausman IVs, BLP IVs, cost shifters
		\end{wideitemize}
	\end{wideitemize}
\end{frame}

\begin{frame}{Practice: Identification}
	\begin{wideitemize}
		\item \textbf{True, False, or Not Enough Information:}
		\item (a) OLS estimation of demand typically underestimates the price coefficient (in absolute value).
		\item (b) Using prices of the same product in other geographic markets as an IV is valid because prices in other markets don't affect local demand.
		\item (c) The logit model solves the dimensionality problem by assuming all products are equally substitutable.
	\end{wideitemize}
	\vspace{10pt}
	\centering
	\textit{Take 3 minutes.}
\end{frame}

\begin{frame}{Practice: Identification (solutions)}
	\begin{wideitemize}
		\item \textbf{(a) TRUE.} Price endogeneity biases $\alpha$ toward zero. Since $\alpha < 0$, this means $|\hat{\alpha}_{OLS}| < |\alpha_{true}|$.
		\item \textbf{(b) TRUE (with caveat).} Hausman IVs work because common cost shocks affect prices in all markets (relevance), but other markets' prices don't directly affect local demand (exclusion). Caveat: requires no common demand shocks.
		\item \textbf{(c) FALSE.} Logit solves dimensionality by using product characteristics. But the IIA problem means substitution is proportional to share, not similarity.
	\end{wideitemize}
\end{frame}

\begin{frame}{Consumer surplus: log-sum formula}
	\begin{wideitemize}
		\item Expected utility for consumer $i$:
		\begin{align*}
			E[\max_j u_{ij}] = \ln\left[\sum_{j=0}^J \exp(\delta_j)\right] + \text{constant}
		\end{align*}
		\item Consumer surplus (in dollars):
		\begin{align*}
			CS = \frac{1}{|\alpha|} \ln\left[\sum_{j=0}^J \exp(\delta_j)\right]
		\end{align*}
		\item Divide by $|\alpha|$ to convert utils to dollars
		\item \textbf{Application:} Welfare change from adding/removing products
	\end{wideitemize}
\end{frame}

\begin{frame}{The IIA problem}
	\begin{wideitemize}
		\item \textbf{IIA:} Ratio of choice probabilities doesn't depend on other options
		\begin{align*}
			\frac{s_j}{s_k} = \frac{\exp(\delta_j)}{\exp(\delta_k)} = \exp(\delta_j - \delta_k)
		\end{align*}
		\item \textbf{Red Bus / Blue Bus:} Adding identical bus option incorrectly increases welfare
		\item \textbf{Problem:} Logit doesn't know similar products are close substitutes
		\item \textbf{Solutions:}
		\begin{wideitemize}
			\vspace{5pt}
			\item Demographic interactions (partial fix)
			\item Mixed logit / random coefficients (full fix, beyond scope)
		\end{wideitemize}
	\end{wideitemize}
\end{frame}

%%%%%%%%%%%%%%%%%%%%%%%%%%%%%%%%%%%%%%%%%%%%%%%%%%%%%%%%%%%%%
% PART 2: PRICING REVIEW
%%%%%%%%%%%%%%%%%%%%%%%%%%%%%%%%%%%%%%%%%%%%%%%%%%%%%%%%%%%%%

\begin{transitionframe}
	\begin{center}
		{\Huge Part 2: Pricing Review}
	\end{center}
\end{transitionframe}

\begin{frame}{Monopoly pricing: key equations}
	\begin{wideitemize}
		\item Profit maximization: $\max_q \pi = p(q) \cdot q - c(q)$
		\item First-order condition: $MR = MC$
		\item \textbf{Lerner index:}
		\begin{align*}
			L = \frac{p - MC}{p} = \frac{1}{|\varepsilon|}
		\end{align*}
		\item \textbf{Interpretation:} Markup is inversely related to elasticity
		\item Equivalently: $p = \frac{MC}{1 - 1/|\varepsilon|}$
	\end{wideitemize}
\end{frame}

\begin{frame}{Practice: Lerner index}
	\begin{wideitemize}
		\item \textbf{Question:} A monopolist faces demand $p = 100 - 2q$ and has constant marginal cost $MC = 20$.
		\item (a) Find the profit-maximizing price and quantity.
		\item (b) Calculate the Lerner index.
		\item (c) Verify using the elasticity formula.
	\end{wideitemize}
	\vspace{15pt}
	\centering
	\textit{Take 4 minutes.}
\end{frame}

\begin{frame}{Practice: Lerner index (solution)}
	\begin{wideitemize}
		\item \textbf{(a)} $MR = 100 - 4q$. Set $MR = MC$: $100 - 4q = 20 \Rightarrow q = 20$
		\item Price: $p = 100 - 2(20) = 60$
		\item \textbf{(b)} Lerner index: $L = \frac{60 - 20}{60} = \frac{40}{60} = \frac{2}{3}$
		\item \textbf{(c)} Elasticity: $\varepsilon = \frac{dq}{dp} \cdot \frac{p}{q} = -\frac{1}{2} \cdot \frac{60}{20} = -1.5$
		\item Check: $L = \frac{1}{|\varepsilon|} = \frac{1}{1.5} = \frac{2}{3}$ $\checkmark$
	\end{wideitemize}
\end{frame}

\begin{frame}{Types of price discrimination}
	\begin{wideenumerate}
		\item \textbf{Perfect price discrimination}
		\begin{wideitemize}
			\vspace{3pt}
			\item Charge each consumer their exact WTP
			\item Benchmark; rarely feasible
		\end{wideitemize}
		\item \textbf{Selection by indicators}
		\begin{wideitemize}
			\vspace{3pt}
			\item Different prices for observable groups
			\item Example: student discounts, geographic pricing
		\end{wideitemize}
		\item \textbf{Self-selection}
		\begin{wideitemize}
			\vspace{3pt}
			\item Design menu to induce type revelation
			\item Example: versioning, bundling
		\end{wideitemize}
	\end{wideenumerate}
\end{frame}

\begin{frame}{Selection by indicators: inverse elasticity rule}
	\begin{wideitemize}
		\item Two markets with different elasticities
		\item Apply Lerner in each market:
		\begin{align*}
			\frac{p_1 - MC}{p_1} = \frac{1}{|\varepsilon_1|} \quad \text{and} \quad \frac{p_2 - MC}{p_2} = \frac{1}{|\varepsilon_2|}
		\end{align*}
		\item \textbf{Key insight:} Charge higher price in more inelastic market
		\item Rearranging: $p = \frac{MC}{1 - 1/|\varepsilon|}$
	\end{wideitemize}
\end{frame}

\begin{frame}{Practice: Selection by indicators}
	\begin{wideitemize}
		\item \textbf{Question:} A firm sells in two markets. Market A has $\varepsilon_A = -3$. Market B has $\varepsilon_B = -5$. Marginal cost is $MC = 10$.
		\item Find the optimal price in each market.
	\end{wideitemize}
	\vspace{15pt}
	\centering
	\textit{Take 3 minutes.}
\end{frame}

\begin{frame}{Practice: Selection by indicators (solution)}
	\begin{wideitemize}
		\item Using $p = \frac{MC}{1 - 1/|\varepsilon|}$:
		\item \textbf{Market A:}
		\begin{align*}
			p_A = \frac{10}{1 - 1/3} = \frac{10}{2/3} = 15
		\end{align*}
		\item \textbf{Market B:}
		\begin{align*}
			p_B = \frac{10}{1 - 1/5} = \frac{10}{4/5} = 12.50
		\end{align*}
		\item Higher price in more inelastic market (A)
	\end{wideitemize}
\end{frame}

\begin{frame}{Two-part tariffs}
	\begin{wideitemize}
		\item Structure: $\text{Payment} = F + p \cdot q$
		\item \textbf{Optimal (homogeneous consumers):}
		\begin{wideitemize}
			\vspace{5pt}
			\item Set $p = MC$ (maximize total surplus)
			\item Set $F = CS(p = MC)$ (extract all surplus)
		\end{wideitemize}
		\item \textbf{Result:} Firm captures entire surplus; efficient but inequitable
		\item \textbf{Heterogeneous consumers:} Tradeoff between extraction and participation
	\end{wideitemize}
\end{frame}

\begin{frame}{Self-selection: IC and IR constraints}
	\begin{wideitemize}
		\item \textbf{IC (Incentive Compatibility):} Each type prefers their option
		\begin{align*}
			\text{IC}_H: \quad v_H^F - p_F &\geq v_H^S - p_S
		\end{align*}
		\item \textbf{IR (Individual Rationality):} Each type willing to participate
		\begin{align*}
			\text{IR}_L: \quad v_L^S - p_S &\geq 0
		\end{align*}
		\item \textbf{Optimal menu:} IC binds for H, IR binds for L
		\item H gets ``information rent''; L gets zero surplus
	\end{wideitemize}
\end{frame}

\begin{frame}{Practice: Self-selection}
	\begin{wideitemize}
		\item \textbf{Question:} Two products (Premium, Basic), two consumer types.
		\begin{center}
			\begin{tabular}{|c|c|c|}
				\hline
				& Premium & Basic \\
				\hline
				High type & 100 & 60 \\
				Low type & 50 & 40 \\
				\hline
			\end{tabular}
		\end{center}
		\item $MC = 20$ for both. Equal numbers of each type.
		\item A firm considers: $p_P = 100$, $p_B = 40$.
		\item (a) What does each type buy?
		\item (b) Is this menu optimal? If not, what should change?
	\end{wideitemize}
	\vspace{10pt}
	\centering
	\textit{Take 4 minutes.}
\end{frame}

\begin{frame}{Practice: Self-selection (solution)}
	\begin{wideitemize}
		\item \textbf{(a) Consumer choices:}
		\item High type: $CS_P = 100 - 100 = 0$, $CS_B = 60 - 40 = 20$
		\item High type buys Basic! (IC violated)
		\item Low type: $CS_P = 50 - 100 = -50$, $CS_B = 40 - 40 = 0$
		\item Low type buys Basic
		\item \textbf{(b) Not optimal.} Need to lower $p_P$ so IC binds.
		\item IC binds when: $100 - p_P = 60 - 40 = 20 \Rightarrow p_P = 80$
		\item Optimal menu: $p_P = 80$, $p_B = 40$
	\end{wideitemize}
\end{frame}

%%%%%%%%%%%%%%%%%%%%%%%%%%%%%%%%%%%%%%%%%%%%%%%%%%%%%%%%%%%%%
% EXAM LOGISTICS
%%%%%%%%%%%%%%%%%%%%%%%%%%%%%%%%%%%%%%%%%%%%%%%%%%%%%%%%%%%%%

\begin{transitionframe}
	\begin{center}
		{\Huge Exam Logistics}
	\end{center}
\end{transitionframe}

\begin{frame}{Exam format reminder}
	\begin{wideitemize}
		\item \textbf{Date:} Monday, Feb 9 (Lecture 7)
		\item \textbf{Duration:} 70 minutes
		\item \textbf{Allowed:} Calculator + two-sided cheat sheet
		\item \textbf{Coverage:}
		\begin{wideitemize}
			\vspace{5pt}
			\item Demand: logit, Berry inversion, elasticities, IVs, IIA, CS
			\item Pricing: monopoly, Lerner index, price discrimination
			\item Two-part tariffs, self-selection (IC/IR)
		\end{wideitemize}
		\item \textbf{Format:} Mix of T/F/NEI, short answer, and problems
	\end{wideitemize}
\end{frame}

\begin{frame}{What to put on your cheat sheet}
	\begin{wideitemize}
		\item \textbf{Key formulas:}
		\begin{wideitemize}
			\vspace{5pt}
			\item Logit share, Berry inversion
			\item Own and cross elasticities
			\item Log-sum formula for CS
			\item Lerner index
			\item Two-part tariff optimal conditions
		\end{wideitemize}
		\item \textbf{Worked examples:} Similar to practice problems
		\item \textbf{Key intuitions:}
		\begin{wideitemize}
			\vspace{5pt}
			\item Why price endogeneity biases $\alpha$ toward zero
			\item Why IC binds for high type, IR for low type
		\end{wideitemize}
	\end{wideitemize}
\end{frame}

%%%%%%%%%%%%%%%%%%%%%%%%%%%%%%%%%%%%%%%%%%%%%%%%%%%%%%%%%%%%%
% KEY POINTS
%%%%%%%%%%%%%%%%%%%%%%%%%%%%%%%%%%%%%%%%%%%%%%%%%%%%%%%%%%%%%

\begin{frame}{Key Points}
	\vspace{11pt}
	\begin{wideenumerate}
		\item \textbf{Logit:} $s_j = \exp(\delta_j)/[1 + \Sigma\exp(\delta_k)]$
		\item \textbf{Berry inversion:} $\ln(s_j) - \ln(s_0) = \delta_j$
		\item \textbf{Elasticities:} Own: $\alpha p_j(1-s_j)$; Cross: $-\alpha p_k s_k$
		\item \textbf{IVs needed} because $E[p_j \xi_j] \neq 0$; bias toward zero
		\item \textbf{Lerner:} $L = (p-MC)/p = 1/|\varepsilon|$
		\item \textbf{Selection by indicators:} Higher price in inelastic market
		\item \textbf{Two-part tariff:} $p = MC$, $F = CS$
		\item \textbf{Self-selection:} IC binds for high type, IR for low type
	\end{wideenumerate}
\end{frame}

\begin{frame}{Good luck on the midterm!}
	\begin{wideitemize}
		\item \textbf{Office hours:} [TBD]
		\item \textbf{Practice exam:} Posted on course website
		\item Questions?
	\end{wideitemize}
\end{frame}

\end{document}
