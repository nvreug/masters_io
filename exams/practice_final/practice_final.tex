% ECN 594: Practice Final Exam
\documentclass[addpoints]{exam}
\usepackage{fullpage}
\usepackage[left=1.0in,top=1.0in,right=1.0in,bottom=1.0in,headheight=3ex,headsep=3ex]{geometry}
\usepackage{graphicx}
\usepackage{float}
\usepackage{adjustbox}
\usepackage{amsmath}
\usepackage{amssymb}
\usepackage{booktabs}

\title{ECN 594: Practice Final Exam}
\rhead{\footnotesize ECN 594: Practice Final}

\date{}

\begin{document}
\maketitle
\begin{center}
	\fbox{\fbox{\parbox{6in}{\centering
		\textbf{Instructions}:
		\begin{itemize}
		\item You have \textbf{70 minutes}
		\item You may bring a calculator and notes on a two-sided cheat-sheet (letter-size paper)
		\item Please be neat. If your work is too messy it will not be graded.
		\item Be sure to show your working.
		\item This exam is \textbf{cumulative}---it covers all course material
		\item Good luck!
		\end{itemize}
	}}}
\end{center}

\vspace{5mm}
\makebox[0.75\textwidth]{Name: \enspace\hrulefill}
\vspace{30pt}
\begin{center}
	\gradetable[h][questions]
\end{center}

\newpage

\begin{questions}

\subsection*{1. Short Answer Questions (30 points)}
\vspace{11pt}
\question For each question, write either a number/formula, True/False/NEI, or a brief answer.
\vspace{11pt}
\begin{parts}
	\part[3] In a Cournot duopoly with linear demand $P = 100 - Q$ and $MC = 20$, what is the equilibrium price?

	\answerline

	\part[3] Write the formula for the critical discount factor in Cournot collusion with $N$ firms.

	\answerline

	\part[3] True, False, or NEI: In Bertrand competition with homogeneous products and identical costs, the equilibrium price equals marginal cost regardless of the number of firms.

	\answerline

	\part[3] What is ``double marginalization'' and why does it occur in vertical relationships?

	\answerline

	\part[3] True, False, or NEI: A horizontal merger always reduces consumer welfare.

	\answerline

	\part[3] What does HHI stand for, and how is it calculated?

	\answerline

	\part[3] True, False, or NEI: In the Hotelling model of spatial competition, firms locate at the center when transportation costs are linear.

	\answerline

	\part[3] What is an ``efficiency defense'' in merger review?

	\answerline

	\part[3] True, False, or NEI: Entry deterrence through capacity commitment is only credible if capacity is costly to reduce.

	\answerline

	\part[3] Name one factor that makes collusion easier to sustain (higher $\delta^*$).

	\answerline

\end{parts}

\newpage
\subsection*{2. Cournot Competition and Mergers (25 points)}
\question Consider a market with 3 firms competing in quantities (Cournot). Market demand is $P = 120 - Q$ where $Q = q_1 + q_2 + q_3$. All firms have marginal cost $c = 30$.

\begin{parts}
	\part[8] Find the Cournot-Nash equilibrium quantities, price, and per-firm profits.

	\vspace{5cm}

	\part[7] Firms 1 and 2 merge. The merged firm has marginal cost $c = 30$ (no efficiency gains). Find the new equilibrium and compare total output, price, consumer surplus, and total welfare to the pre-merger equilibrium.

	\vspace{6cm}

	\part[5] Now suppose the merger creates efficiency gains, reducing the merged firm's marginal cost to $c = 20$. How does this change your welfare analysis?

	\vspace{4cm}

	\part[5] Calculate the HHI before and after the merger. Would this merger likely face antitrust scrutiny?

\end{parts}

\newpage
\subsection*{3. Collusion (20 points)}
\question Return to the 3-firm Cournot market from Question 2 (before any merger). Firms consider forming a cartel where each produces the monopoly quantity divided by 3.

\begin{parts}
	\part[5] What is the monopoly quantity and price? What would each firm's collusive profit be?

	\vspace{4cm}

	\part[5] If one firm deviates while others stick to the collusive quantity, what is its optimal deviation quantity and profit?

	\vspace{4cm}

	\part[5] Compute the critical discount factor $\delta^*$ for sustaining collusion with grim trigger strategies.

	\vspace{4cm}

	\part[5] The industry is investigated for collusion. Explain how a leniency program works and why it can destabilize cartels.

\end{parts}

\newpage
\subsection*{4. Vertical Relationships (15 points)}
\question A manufacturer (M) sells to a retailer (R), who sells to final consumers. Consumer demand is $Q = 100 - P$. The manufacturer's marginal cost is $c_M = 10$. The retailer's only cost is the wholesale price $w$ paid to the manufacturer.

\begin{parts}
	\part[5] If the manufacturer sets a wholesale price $w$ and the retailer then sets the retail price $P$, find the equilibrium wholesale price, retail price, and total industry profit. This is the ``double marginalization'' outcome.

	\vspace{5cm}

	\part[5] What would be the vertically integrated (single monopolist) outcome? Compare to part (a).

	\vspace{4cm}

	\part[5] Describe two contractual solutions that can eliminate double marginalization without vertical integration.

\end{parts}

\newpage
\subsection*{5. Demand Estimation (10 points)}
\question Consider a logit demand model with 2 products. Product 1 has $\delta_1 = 1$ and $p_1 = 5$. Product 2 has $\delta_2 = 0.5$ and $p_2 = 4$. The price coefficient is $\alpha = -0.3$.

\begin{parts}
	\part[5] Compute the market shares for both products and the outside option.

	\vspace{4cm}

	\part[5] Using the log-sum formula, compute consumer surplus per consumer. If product 2's price increases to $p_2 = 5$, what happens to consumer surplus?

\end{parts}

\end{questions}

\end{document}
