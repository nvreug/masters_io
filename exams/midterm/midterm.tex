% ECN 594: Midterm Exam
\documentclass[addpoints]{exam}
\usepackage{fullpage}
\usepackage[left=1.0in,top=1.0in,right=1.0in,bottom=1.0in,headheight=3ex,headsep=3ex]{geometry}
\usepackage{graphicx}
\usepackage{float}
\usepackage{adjustbox}
\usepackage{amsmath}
\usepackage{amssymb}
\usepackage{booktabs}

\title{ECN 594: Midterm Exam}
\rhead{\footnotesize ECN 594: Midterm Exam}

\date{February 9, 2026}

\begin{document}
\maketitle
\begin{center}
	\fbox{\fbox{\parbox{6in}{\centering
		\textbf{Instructions}:
		\begin{itemize}
		\item You have \textbf{70 minutes}
		\item You may bring a calculator and notes on a two-sided cheat-sheet (letter-size paper)
		\item Please be neat. If your work is too messy it will not be graded.
		\item Be sure to show your working.
		\item This is a long exam, so there are lots of ways to get points. If you get stuck, move on!
		\item Good luck!
		\end{itemize}
	}}}
\end{center}

\vspace{5mm}
\makebox[0.75\textwidth]{Name: \enspace\hrulefill}
\vspace{30pt}
\begin{center}
	\gradetable[h][questions]
\end{center}

\newpage

\begin{questions}

\subsection*{1. Short Answer Questions (30 points)}
\vspace{11pt}
\question For each question, write either:
\begin{itemize}
	\item a number or formula
	\item one of: True, False, or NEI (Not Enough Information)
	\item a brief definition or explanation (one sentence)
\end{itemize}
\vspace{11pt}
\begin{parts}
	\part[3] A monopolist faces constant elasticity demand with $\varepsilon = -4$ and has marginal cost $c = 15$. What is the optimal price?

	\answerline

	\part[3] Write the Berry inversion formula that relates market shares to mean utilities in the logit model.

	\answerline

	\part[3] True, False, or NEI: Adding demographic interactions to a logit model fully solves the IIA problem.

	\answerline

	\part[3] What is the economic interpretation of the price coefficient $\alpha$ in the logit demand model?

	\answerline

	\part[3] True, False, or NEI: If marginal cost increases, a monopolist with linear demand will raise price by exactly the same amount as the cost increase.

	\answerline

	\part[3] True, False, or NEI: Under a two-part tariff, setting the per-unit price equal to marginal cost maximizes total surplus.

	\answerline

	\part[3] Name one reason why the ``BLP instruments'' (characteristics of other products) help identify the price coefficient.

	\answerline

	\part[3] True, False, or NEI: In a self-selection problem, the firm can extract the entire surplus from low-type consumers.

	\answerline

	\part[3] What is ``versioning'' in the context of price discrimination?

	\answerline

	\part[3] True, False, or NEI: Consumer surplus in the logit model can be computed using the log-sum formula, which equals the ``inclusive value'' divided by $|\alpha|$.

	\answerline

\end{parts}

\newpage
\subsection*{2. Demand Estimation (30 points)}
\question Consider a market with 4 differentiated products. The logit demand model is:
$$u_{ij} = \delta_j + \alpha p_j + \varepsilon_{ij}$$
where $\alpha = -0.4$ is the price coefficient.

The following data are observed:
\begin{center}
\begin{tabular}{lccc}
\toprule
Product & Price ($p_j$) & Mean Utility ($\delta_j$) \\
\midrule
1 & \$15 & 3.0 \\
2 & \$12 & 2.5 \\
3 & \$18 & 3.5 \\
4 & \$10 & 2.0 \\
\bottomrule
\end{tabular}
\end{center}

\begin{parts}
	\part[8] Compute $v_j = \delta_j + \alpha p_j$ for each product. Then compute the market shares using:
	$$s_j = \frac{\exp(v_j)}{1 + \sum_k \exp(v_k)}$$

	\vspace{5cm}

	\part[6] Compute the own-price elasticity for each product. Which product has the most inelastic demand?

	\vspace{4cm}

	\part[6] A researcher estimates demand using OLS (regressing $\ln(s_j) - \ln(s_0)$ on price and product characteristics). Explain why this leads to biased estimates of $\alpha$. What is the direction of the bias?

	\vspace{4cm}

	\part[10] Using the log-sum formula, compute consumer surplus per consumer. If product 1 is removed from the market, what is the change in consumer surplus?

\end{parts}

\newpage
\subsection*{3. Price Discrimination (20 points)}
\question A streaming service has two customer segments: ``binge watchers'' (B) and ``casual viewers'' (C). Marginal cost is \$2 per subscriber.

\begin{itemize}
\item Binge watcher demand: $Q_B = 200 - 10P_B$
\item Casual viewer demand: $Q_C = 100 - 10P_C$
\end{itemize}

\begin{parts}
	\part[6] The firm can identify customer type through usage patterns. Find the optimal prices under price discrimination by indicators.

	\vspace{4cm}

	\part[4] Compute total profit under price discrimination.

	\vspace{3cm}

	\part[6] Suppose the firm cannot distinguish customers. What is the optimal uniform price?

	\vspace{4cm}

	\part[4] Which group is better off under price discrimination compared to uniform pricing?

\end{parts}

\newpage
\subsection*{4. Self-Selection (20 points)}
\question An airline offers two fare classes: Business (B) and Economy (E). There are two types of travelers: executives with high willingness to pay and students with low willingness to pay.

\begin{center}
\begin{tabular}{lcc}
\toprule
 & \multicolumn{2}{c}{Willingness to Pay} \\
Consumer Type & Business Class & Economy Class \\
\midrule
Executive (50 travelers) & \$500 & \$200 \\
Student (100 travelers) & \$150 & \$120 \\
\bottomrule
\end{tabular}
\end{center}

Marginal cost is \$50 for economy and \$100 for business class.

\begin{parts}
	\part[4] If the airline could perfectly identify consumer types, what prices would it charge and what is its profit?

	\vspace{3cm}

	\part[4] The airline cannot identify types but offers both fare classes. Write down the incentive compatibility (IC) constraint that ensures executives buy business class.

	\vspace{3cm}

	\part[6] Find the profit-maximizing prices for business and economy class under self-selection. (Hint: Which IC constraint binds?)

	\vspace{5cm}

	\part[6] The airline considers ``damaging'' economy class by adding restrictions (no changes, middle seat only). How would this affect the equilibrium? Explain the economic logic.

\end{parts}

\end{questions}

\end{document}
