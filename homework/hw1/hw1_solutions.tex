% ECN 594: Homework 1 Solutions
\documentclass[11pt]{article}
\usepackage{fullpage}
\usepackage[left=1.0in,top=1.0in,right=1.0in,bottom=1.0in,headheight=3ex,headsep=3ex]{geometry}
\usepackage{graphicx}
\usepackage{float}
\usepackage{adjustbox}
\usepackage{amsmath}
\usepackage{amssymb}
\usepackage{hyperref}
\usepackage{booktabs}

\newcommand{\blankline}{\quad\pagebreak[2]}

\title{ECN 594: Homework 1 Solutions \\ \large{Demand Estimation}}
\author{Solution Key}
\date{}

\linespread{1.3}
\usepackage[T1]{fontenc}
\usepackage{fancyhdr,lastpage}
\pagestyle{fancy}
\lhead{}
\chead{}
\rhead{\footnotesize ECN 594: HW1 Solutions}
\lfoot{}
\cfoot{\small \thepage/\pageref*{LastPage}}
\rfoot{}

\usepackage[dvipsnames]{xcolor}
\definecolor{asumaroon}{rgb}{0.549,0.114,0.251}
\hypersetup{colorlinks,breaklinks,linkcolor=asumaroon,urlcolor=asumaroon,anchorcolor=asumaroon,citecolor=black}

\renewcommand{\theenumi}{\alph{enumi}}

\begin{document}
\maketitle

\subsection*{Question 0: Submission}

Students receive full credit for submitting all required materials to Canvas: a PDF with written answers, Python code, clearly labeled figures, and group member names if applicable.

\subsection*{Question 1: Basic Logit Model}

(a) Expected Sign of $\alpha$

We expect $\alpha$ to be negative. The price coefficient captures how utility changes when price increases, holding other factors constant. Since consumers prefer to pay less rather than more for a product, higher prices reduce utility and thus reduce the probability of purchase. A negative $\alpha$ is consistent with downward-sloping demand.

\vspace{1em}
(b) Berry Inversion

The Berry inversion starts from the logit choice probability. For product $j$, the market share is $s_j = \exp(\delta_j) / (1 + \sum_{k=1}^{J} \exp(\delta_k))$, while the outside option share is $s_0 = 1 / (1 + \sum_{k=1}^{J} \exp(\delta_k))$. Taking the ratio gives $s_j / s_0 = \exp(\delta_j)$, and applying logs yields $\ln(s_j) - \ln(s_0) = \delta_j$. Substituting the expression for mean utility $\delta_j = \beta_0 + \beta_1 \cdot \text{sugar}_j + \alpha \cdot p_j + \xi_j$, we obtain the estimating equation:
\begin{align*}
\boxed{\ln(s_j) - \ln(s_0) = \beta_0 + \beta_1 \cdot \text{sugar}_j + \alpha \cdot p_j + \xi_j}
\end{align*}

This is a linear regression where the dependent variable is the log difference between product share and outside good share, and the error term is $\xi_j$.

\vspace{1em}
(c) OLS Estimation

\begin{center}
\begin{tabular}{lc}
\toprule
Parameter & OLS Estimate \\
\midrule
$\hat{\beta}_0$ (constant) & $-2.9665$ \\
$\hat{\beta}_1$ (sugar) & $0.0463$ \\
$\hat{\alpha}$ (price) & $-10.2036$ \\
\bottomrule
\end{tabular}
\end{center}

\vspace{1em}
(d) 2SLS Estimation

\begin{center}
\begin{tabular}{lc}
\toprule
Parameter & 2SLS Estimate \\
\midrule
$\hat{\beta}_0$ (constant) & $-2.8915$ \\
$\hat{\beta}_1$ (sugar) & $0.0483$ \\
$\hat{\alpha}$ (price) & $-10.9449$ \\
\bottomrule
\end{tabular}
\end{center}

\vspace{1em}
(e) Comparison of OLS and 2SLS

The OLS estimate of the price coefficient is less negative than the 2SLS estimate ($-10.20$ vs $-10.94$). This difference arises because prices are likely correlated with unobserved product quality. Firms set higher prices for products that consumers value for reasons not captured by sugar content, such as brand reputation or taste. This positive correlation between price and the error term biases the OLS estimate of $\alpha$ toward zero. The 2SLS estimate, which uses instruments that are correlated with price but uncorrelated with unobserved quality, produces a more negative coefficient indicating that consumers are more price sensitive than OLS suggests.

\vspace{1em}
(f) Own-Price Elasticities

\begin{figure}[H]
\centering
\includegraphics[width=0.8\textwidth]{q1e_elasticities_logit.png}
\caption{Own-price elasticities vs.\ prices for basic logit model (market C01Q1)}
\end{figure}

The figure shows a negative linear relationship between prices and own-price elasticities. This pattern is a limitation of the logit model rather than something learned from the data. The formula $\eta_{jj} = \alpha \cdot p_j \cdot (1-s_j)$ mechanically ties elasticity to price. Since $\alpha < 0$ and $(1-s_j) \approx 1$ for products with small market shares, elasticity becomes roughly proportional to price. In reality, this need not hold: some high-priced products may have relatively inelastic demand (luxury goods with loyal customers), while some low-priced products may face elastic demand (commodities where consumers easily switch). The logit model cannot capture these patterns because the price-elasticity relationship is imposed by functional form assumptions.

\newpage
\subsection*{Question 2: Logit with Demographic Interactions}

(a) Advantage Over Simple Logit

One key advantage is that this model allows for consumer heterogeneity in preferences. In the simple logit model, all consumers have the same price coefficient $\alpha$, which implies everyone responds identically to price changes. With demographic interactions, different consumers can have different price sensitivities depending on their income. This is more realistic because we expect high-income consumers to be less sensitive to price than low-income consumers. As a result, the model generates more flexible substitution patterns at the aggregate level and breaks the tight mechanical relationship between prices and elasticities that we saw in Question 1(f).

\vspace{1em}
(b) Definition of $\delta_{jt}$

The term $\delta_{jt}$ is the mean utility of product $j$ in market $t$, representing the portion of utility that is common to all consumers. In this model, $\delta_{jt} = \beta_0 + \beta_1 \cdot \text{sugar}_{jt} + \alpha_0 \cdot p_{jt} + \xi_{jt}$. It captures the average valuation of the product before accounting for individual-specific variation from demographics or the idiosyncratic logit shock.

\vspace{1em}
(c) Estimation Results

\begin{center}
\begin{tabular}{lc}
\toprule
Parameter & Estimate \\
\midrule
\multicolumn{2}{l}{\textit{Linear parameters}} \\
$\hat{\beta}_0$ (constant) & $-2.6323$ \\
$\hat{\beta}_1$ (sugar) & $-0.0134$ \\
$\hat{\alpha}_0$ (price) & $-10.9489$ \\
\midrule
\multicolumn{2}{l}{\textit{Demographic interactions}} \\
$\hat{\beta}_{1,inc}$ (income $\times$ sugar) & $0.1685$ \\
$\hat{\alpha}_{inc}$ (income $\times$ price) & $-1.2668$ \\
\bottomrule
\end{tabular}
\end{center}

\vspace{1em}
(d) Interpretation of $\alpha_{inc}$

The parameter $\alpha_{inc}$ captures how price sensitivity varies with income. A positive $\alpha_{inc}$ means higher-income consumers are less sensitive to price, so their effective price coefficient $(\alpha_0 + \alpha_{inc} \cdot \text{income})$ is less negative than for lower-income consumers.

\vspace{1em}
(e) Elasticities with Demographics (Bonus)

\begin{figure}[H]
\centering
\includegraphics[width=\textwidth]{q2c_comparison.png}
\caption{Own-price elasticities: basic logit (left) vs.\ logit with demographics (right)}
\end{figure}

The tight linear relationship from the basic logit model is now broken, and the pattern appears scattered. This happens because the model with demographics allows different consumers to have different price coefficients $\alpha_i$. In the basic logit, all consumers share the same $\alpha$, so elasticity mechanically depends on price through the formula. With demographic interactions, aggregate elasticity depends on the distribution of consumer types in each market and how they sort across products. A product attracting primarily high-income consumers (who are less price sensitive) may have a relatively low elasticity even if its price is high. This is more realistic because the model now lets the data reveal the price-elasticity relationship rather than imposing it through functional form.

\end{document}
