% ECN 594: Homework 1 Solutions
\documentclass[11pt]{article}
\usepackage{fullpage}
\usepackage[left=1.0in,top=1.0in,right=1.0in,bottom=1.0in,headheight=3ex,headsep=3ex]{geometry}
\usepackage{graphicx}
\usepackage{float}
\usepackage{adjustbox}
\usepackage{amsmath}
\usepackage{amssymb}
\usepackage{hyperref}
\usepackage{booktabs}

\newcommand{\blankline}{\quad\pagebreak[2]}

\title{ECN 594: Homework 1 Solutions \\ \large{Demand Estimation}}
\author{Solution Key}
\date{}

\linespread{1.3}
\usepackage[T1]{fontenc}
\usepackage{fancyhdr,lastpage}
\pagestyle{fancy}
\lhead{}
\chead{}
\rhead{\footnotesize ECN 594: HW1 Solutions}
\lfoot{}
\cfoot{\small \thepage/\pageref*{LastPage}}
\rfoot{}

\usepackage[dvipsnames]{xcolor}
\definecolor{asumaroon}{rgb}{0.549,0.114,0.251}
\hypersetup{colorlinks,breaklinks,linkcolor=asumaroon,urlcolor=asumaroon,anchorcolor=asumaroon,citecolor=black}

\renewcommand{\theenumi}{\alph{enumi}}

\begin{document}
\maketitle

\subsection*{Question 1: Basic Logit Model}

\textbf{(a) Berry Inversion}

Starting from the logit choice probability:
\begin{align*}
s_j &= \frac{\exp(\delta_j)}{1 + \sum_{k=1}^{J} \exp(\delta_k)} \\
s_0 &= \frac{1}{1 + \sum_{k=1}^{J} \exp(\delta_k)}
\end{align*}

Taking the ratio:
\begin{align*}
\frac{s_j}{s_0} = \exp(\delta_j)
\end{align*}

Taking logs:
\begin{align*}
\ln(s_j) - \ln(s_0) = \delta_j
\end{align*}

Substituting $\delta_j = \beta_0 + \beta_1 \cdot \text{sugar}_j + \alpha \cdot p_j + \xi_j$:
\begin{align*}
\boxed{\ln(s_j) - \ln(s_0) = \beta_0 + \beta_1 \cdot \text{sugar}_j + \alpha \cdot p_j + \xi_j}
\end{align*}

This is a linear regression with dependent variable $\ln(s_j) - \ln(s_0)$ and error term $\xi_j$.

\vspace{1em}
\textbf{(b) OLS Estimation}

\begin{center}
\begin{tabular}{lc}
\toprule
Parameter & OLS Estimate \\
\midrule
$\hat{\beta}_0$ (constant) & $-2.967$ \\
$\hat{\beta}_1$ (sugar) & $0.046$ \\
$\hat{\alpha}$ (price) & $-10.20$ \\
\bottomrule
\end{tabular}
\end{center}

\vspace{1em}
\textbf{(c) OLS Bias}

OLS is likely biased because prices are \textbf{endogenous}:
\begin{itemize}
\item Firms set higher prices for products with higher unobserved quality ($\xi_j$)
\item This creates positive correlation: $\text{Cov}(p_j, \xi_j) > 0$
\item Since $\xi_j$ is in the error term, we have $\text{Cov}(p_j, \text{error}) > 0$
\item This causes \textbf{upward bias} in $\alpha$ (makes it less negative than the true value)
\end{itemize}

\textit{Intuition:} Products consumers like for unobserved reasons (brand, taste, advertising) charge higher prices. OLS interprets high prices with high shares as ``price doesn't hurt demand much'' when really it's unobserved quality driving both.

\vspace{1em}
\textbf{(d) 2SLS Estimation}

\begin{center}
\begin{tabular}{lcc}
\toprule
Parameter & OLS & 2SLS \\
\midrule
$\hat{\beta}_0$ (constant) & $-2.967$ & $-2.891$ \\
$\hat{\beta}_1$ (sugar) & $0.046$ & $0.048$ \\
$\hat{\alpha}$ (price) & $-10.20$ & $-10.94$ \\
\bottomrule
\end{tabular}
\end{center}

As expected, OLS is \textbf{less negative} than 2SLS ($-10.20$ vs $-10.94$), confirming upward bias. The 2SLS estimate suggests consumers are more price sensitive than OLS indicates.

\vspace{1em}
\textbf{(e) Own-Price Elasticities}

See Figure 1. There is a \textbf{negative linear relationship} between prices and elasticities.

This is a \textbf{bug} (limitation) of the logit model. The formula $\eta_{jj} = \alpha \cdot p_j \cdot (1-s_j)$ mechanically ties elasticity to price. Since $\alpha < 0$ and $(1-s_j) \approx 1$ for small shares, elasticity is roughly proportional to price.

This relationship is imposed by functional form, not learned from data. In reality, some high-priced products may have inelastic demand (luxury goods), while some low-priced products may be elastic (commodities).

\newpage
\subsection*{Question 2: Logit with Demographic Interactions}

\textbf{(a) Interpretation of $\alpha_{inc}$}

$\alpha_{inc}$ captures how income affects price sensitivity.

Total price coefficient: $\alpha_i = \alpha_0 + \alpha_{inc} \cdot \text{income}_i$

\textbf{Expected sign: Positive}

Reasoning:
\begin{itemize}
\item $\alpha_0 < 0$ (everyone dislikes higher prices)
\item Higher-income consumers are less price sensitive
\item If $\alpha_{inc} > 0$, high-income consumers have $\alpha_i$ closer to zero (less negative)
\end{itemize}

Example: If $\alpha_0 = -5$ and $\alpha_{inc} = 2$:
\begin{itemize}
\item Low income ($=0$): $\alpha = -5$ (very price sensitive)
\item High income ($=1$): $\alpha = -3$ (less price sensitive)
\end{itemize}

\vspace{1em}
\textbf{(b) Estimation Results}

\begin{center}
\begin{tabular}{lc}
\toprule
Parameter & Estimate \\
\midrule
$\hat{\beta}_0$ (constant) & $-3.80$ \\
$\hat{\alpha}_0$ (price) & $-2.93$ \\
$\hat{\beta}_{0,inc}$ (constant $\times$ income) & $4.87$ \\
$\hat{\alpha}_{inc}$ (price $\times$ income) & $-36.6$ \\
$\hat{\beta}_1$ (sugar $\times$ income) & $0.13$ \\
\bottomrule
\end{tabular}
\end{center}

Note: The large negative $\alpha_{inc}$ is an artifact of how income is scaled in the data.

\vspace{1em}
\textbf{(c) Elasticities with Demographics}

See Figure 2. The tight linear relationship is now \textbf{broken}---the pattern is scattered.

\textbf{Why?}
\begin{itemize}
\item Basic logit: all consumers have same $\alpha$, so elasticity mechanically depends on price
\item With demographics: different consumers have different $\alpha_i$
\item Aggregate elasticity depends on distribution of consumer types and sorting
\item Products attracting high-income (less price-sensitive) consumers may have lower elasticities even at high prices
\end{itemize}

This is more realistic---the model now lets data reveal the price-elasticity relationship.

\vspace{1em}
\textbf{(d) Does Adding Demographics Help with IIA?}

\textbf{Partially yes:}
\begin{itemize}
\item Different consumer types have different substitution patterns
\item At the aggregate level, this creates richer substitution
\end{itemize}

\textbf{Still limited:}
\begin{itemize}
\item For any individual consumer type, IIA still holds
\item A consumer with specific income still has logit substitution patterns
\end{itemize}

\textbf{What would fully address IIA:}
\begin{itemize}
\item Random coefficients (mixed logit / full BLP)
\item Allows idiosyncratic preferences even conditional on demographics
\item Substitution then depends on characteristic similarity, not just shares
\end{itemize}

\newpage
\subsection*{Question 3: Consumer Surplus}

\textbf{(a) Consumer Surplus Formula}

For Type 1 EV errors, expected maximum utility has closed form. The ``inclusive value'' is:
\begin{align*}
IV_t = \ln\left(1 + \sum_{j=1}^{J_t} \exp(\delta_{jt})\right)
\end{align*}

Consumer surplus in monetary units:
\begin{align*}
\boxed{E[CS_t] = \frac{1}{|\alpha|} \ln\left(1 + \sum_{j=1}^{J_t} \exp(\delta_{jt})\right)}
\end{align*}

Notes:
\begin{itemize}
\item CS is relative to having only the outside option
\item The ``1'' inside the log is from outside option ($\delta_0 = 0$)
\item Each product $j$ contributes $\exp(\delta_{jt})$ to the sum
\end{itemize}

\vspace{1em}
\textbf{(b) CS Calculation for C01Q1}

Using 2SLS estimates with $\hat{\alpha} = -10.94$:

\begin{center}
\begin{tabular}{lc}
\toprule
Scenario & Expected CS per Consumer \\
\midrule
All products available & \$0.1547 \\
Without F1B04 & \$0.1536 \\
\midrule
\textbf{Change from removing F1B04} & \textbf{$-\$0.0011$} \\
\bottomrule
\end{tabular}
\end{center}

Consumers lose about \$0.0011 per person when F1B04 is removed.

\vspace{1em}
\textbf{(c) Products with Largest/Smallest Impact}

\textbf{Product whose removal most harms consumers:} Typically the product with highest market share (highest $\delta$).

\textbf{Product whose removal least harms consumers:} Typically the product with lowest market share (lowest $\delta$).

\textbf{Intuition:} From the log-sum formula, products with high $\exp(\delta)$ have larger impact on inclusive value. High $\delta$ means:
\begin{itemize}
\item Lower price (since $\alpha < 0$)
\item Desirable characteristics
\item High unobserved quality
\end{itemize}

This makes sense: removing a popular product that many consumers choose hurts welfare more than removing a niche product few consumers buy.

\textbf{Key relationship:} CS change is roughly proportional to market share, since share $\propto \exp(\delta)$.

\end{document}
