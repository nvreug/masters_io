% ECN 594: Homework 1 - Demand Estimation
\documentclass[11pt]{article}
\usepackage{fullpage}
\usepackage[left=1.0in,top=1.0in,right=1.0in,bottom=1.0in,headheight=3ex,headsep=3ex]{geometry}
\usepackage{graphicx}
\usepackage{float}
\usepackage{adjustbox}
\usepackage{amsmath}
\usepackage{amssymb}
\usepackage{hyperref}

\newcommand{\blankline}{\quad\pagebreak[2]}

\title{ECN 594: Homework 1 \\ \large{Demand Estimation}}
\author{\textbf{Due: See Canvas} \\
\textbf{You may work in groups of up to 2 people.}}
\date{}

\linespread{1.3}
\usepackage[T1]{fontenc}
\usepackage{fancyhdr,lastpage}
\pagestyle{fancy}
\lhead{}
\chead{}
\rhead{\footnotesize ECN 594: Homework 1}
\lfoot{}
\cfoot{\small \thepage/\pageref*{LastPage}}
\rfoot{}

\usepackage[dvipsnames]{xcolor}
\definecolor{asumaroon}{rgb}{0.549,0.114,0.251}
\hypersetup{colorlinks,breaklinks,linkcolor=asumaroon,urlcolor=asumaroon,anchorcolor=asumaroon,citecolor=black}

\renewcommand{\theenumi}{\alph{enumi}}

\begin{document}
\maketitle

\subsection*{Instructions}
\begin{itemize}
\item This assignment is out of 100 points, with 10 bonus points possible. The maximum score is 100/100 (so if you earn 110 points, you receive 100/100).
\item You may work in groups of up to 2 people. If working in a group, please submit one assignment with both names.
\item Submit your solutions as a PDF along with your Python code (either a \texttt{.py} file or Jupyter notebook).
\item Use the \texttt{pyblp} package for estimation. Documentation is available at \url{https://pyblp.readthedocs.io/}.
\item This assignment uses the same cereal dataset as in lecture. The data files are available on Canvas.
\end{itemize}

\subsection*{Question 0: Submission (5 points)}

Submit all materials to Canvas by the deadline:
\begin{itemize}
	\item A PDF containing your written answers to all questions
	\item Your Python code (either a \texttt{.py} file or Jupyter notebook \texttt{.ipynb})
	\item All figures should be embedded in your PDF and clearly labeled
	\item If working in a group, include both names on your submission
\end{itemize}

\subsection*{Background and Data}

In this homework you will estimate demand for breakfast cereals using the methods we covered in class. You will use the \texttt{pyblp} package in Python to estimate logit and logit with demographic interactions models.

There are two datasets:
\begin{itemize}
	\item \textbf{product\_data.csv}: Contains data about market shares, prices, and product characteristics.
	\begin{itemize}
		\item \textit{market\_ids}: Market identifiers (city-quarter, e.g., `C01Q1' = city 1, quarter 1)
		\item \textit{product\_ids}: Product identifiers
		\item \textit{shares}: Market shares
		\item \textit{prices}: Prices
		\item \textit{sugar}: Sugar content (grams)
		\item \textit{demand\_instruments0, ..., demand\_instruments19}: Pre-computed demand instruments
	\end{itemize}
	\item \textbf{agent\_data.csv}: Contains consumer demographic data for each market.
	\begin{itemize}
		\item \textit{market\_ids}: Market identifiers
		\item \textit{weights}: Weight of each consumer draw (= 1/20)
		\item \textit{income}: Draw from the income distribution in each market
	\end{itemize}
\end{itemize}

\subsection*{Question 1: Basic Logit Model (45 points)}

Consider the homogeneous-consumer logit model:
\begin{align*}
	u_{ijt} = \beta_0 + \beta_1 \cdot \text{sugar}_{jt} + \alpha \cdot p_{jt} + \xi_{jt} + \epsilon_{ijt}
\end{align*}
where:
\begin{itemize}
	\item $u_{ijt}$: Utility of consumer $i$ for product $j$ in market $t$
	\item $\text{sugar}_{jt}$: Sugar content of product $j$ in market $t$
	\item $p_{jt}$: Price of product $j$ in market $t$
	\item $\xi_{jt}$: Unobserved product quality
	\item $\epsilon_{ijt}$: i.i.d. Type 1 Extreme Value error
\end{itemize}

The utility of the outside option (not purchasing any cereal) is normalized to $u_{i0t} = 0 + \epsilon_{i0t}$.

\begin{enumerate}
	\item \textbf{(5 points)} Before running any regressions, what sign do you expect $\alpha$ (the price coefficient) to have? Explain your reasoning.

	\item \textbf{(5 points)} Using the Berry inversion, write down the linear regression equation that you would estimate. Clearly show the dependent variable on the left-hand side.

	\item \textbf{(5 points)} Estimate the model using OLS. Report the coefficients $\hat{\beta}_0$, $\hat{\beta}_1$, and $\hat{\alpha}$. (No need to report standard errors.)

	\item \textbf{(10 points)} Estimate the model using 2SLS, instrumenting for price using the provided instruments (\textit{demand\_instruments0, ..., demand\_instruments19}). Report the coefficients. (No need to report standard errors.)

	\item \textbf{(10 points)} Compare $\hat{\alpha}$ from 2SLS to your OLS estimate. Which is more negative? Briefly explain why they differ. (2 sentences max.)

	\item \textbf{(10 points)} Using your 2SLS estimates, compute the own-price elasticity for each product in market `C01Q1'. Create a scatterplot with prices on the x-axis and own-price elasticities on the y-axis. What pattern do you observe? Is this a feature or a bug of the logit model?
\end{enumerate}

\subsection*{Question 2: Logit with Demographic Interactions (50 points)}

Now consider a model that allows preferences to vary with observed consumer demographics:
\begin{align*}
	u_{ijt} = \beta_{0} + (\beta_1 + \beta_{1,inc} \cdot \text{income}_i) \cdot \text{sugar}_{jt} + (\alpha_0 + \alpha_{inc} \cdot \text{income}_i) \cdot p_{jt} + \xi_{jt} + \epsilon_{ijt}
\end{align*}

This model has five parameters:
\begin{itemize}
	\item \textbf{Linear parameters:} $\beta_0$ (constant), $\beta_1$ (sugar), $\alpha_0$ (price)
	\item \textbf{Demographic interactions:} $\beta_{1,inc}$ (income $\times$ sugar), $\alpha_{inc}$ (income $\times$ price)
\end{itemize}

Note: This model does \textit{not} include random coefficients---all heterogeneity comes from observed demographics. Again, the utility of the outside option (not purchasing any cereal) is normalized to $u_{i0t} = 0 + \epsilon_{i0t}$.

\begin{enumerate}
	\item \textbf{(10 points)} Explain one advantage that this model has compared to the simple logit model in Question 1.

	\item \textbf{(10 points)} In the context of this model, what is $\delta_{jt}$? Write out the expression for $\delta_{jt}$.

	\item \textbf{(20 points)} Estimate this model using \texttt{pyblp}. Report all five parameter estimates. (No need to report standard errors.)

	\item \textbf{(10 points)} Interpret the parameter $\alpha_{inc}$. (2 sentences max.)

	\item \textbf{(Bonus: 10 points)} Using your estimates, compute own-price elasticities for each product in market `C01Q1'. Create a scatterplot as in Question 1(f). How does the relationship between prices and elasticities differ from the basic logit? Explain why this happens.
\end{enumerate}

\subsection*{Submission Checklist}
\begin{itemize}
	\item[$\square$] PDF with answers to all questions
	\item[$\square$] Python code (\texttt{.py} or \texttt{.ipynb})
	\item[$\square$] All figures clearly labeled
	\item[$\square$] Both group members' names (if applicable)
\end{itemize}

\end{document}
