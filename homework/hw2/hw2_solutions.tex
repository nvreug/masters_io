% ECN 594: Homework 2 Solutions
\documentclass[11pt]{article}
\usepackage{fullpage}
\usepackage[left=1.0in,top=1.0in,right=1.0in,bottom=1.0in,headheight=3ex,headsep=3ex]{geometry}
\usepackage{graphicx}
\usepackage{float}
\usepackage{adjustbox}
\usepackage{amsmath}
\usepackage{amssymb}
\usepackage{hyperref}
\usepackage{booktabs}

\title{ECN 594: Homework 2 Solutions \\ \large{Competition and Merger Simulation}}
\author{Solution Key}
\date{}

\linespread{1.3}
\usepackage[T1]{fontenc}
\usepackage{fancyhdr,lastpage}
\pagestyle{fancy}
\lhead{}
\chead{}
\rhead{\footnotesize ECN 594: HW2 Solutions}
\lfoot{}
\cfoot{\small \thepage/\pageref*{LastPage}}
\rfoot{}

\usepackage[dvipsnames]{xcolor}
\definecolor{asumaroon}{rgb}{0.549,0.114,0.251}
\hypersetup{colorlinks,breaklinks,linkcolor=asumaroon,urlcolor=asumaroon,anchorcolor=asumaroon,citecolor=black}

\renewcommand{\theenumi}{\alph{enumi}}

\begin{document}
\maketitle

\section*{Part A: Oligopoly Theory}

\subsection*{Question 1: Cournot Competition}

\textbf{(a) 3-firm Cournot equilibrium}

With $N$ identical firms, demand $P = a - Q$, and marginal cost $c$:

Firm $j$'s profit: $\pi_j = (P - c)q_j = (a - Q - c)q_j = (a - q_j - \sum_{k \neq j} q_k - c)q_j$

FOC: $\frac{\partial \pi_j}{\partial q_j} = a - 2q_j - \sum_{k \neq j} q_k - c = 0$

By symmetry, $q^* = q_j$ for all $j$, so:
\begin{align*}
a - 2q^* - (N-1)q^* - c &= 0 \\
q^* &= \frac{a - c}{N + 1} = \frac{100 - 10}{4} = 22.5
\end{align*}

\begin{center}
\begin{tabular}{lc}
\toprule
Variable & Value \\
\midrule
Per-firm quantity $q^*$ & 22.5 \\
Total quantity $Q^*$ & 67.5 \\
Price $P^*$ & \$32.50 \\
Per-firm profit $\pi^*$ & \$506.25 \\
\bottomrule
\end{tabular}
\end{center}

\vspace{1em}
\textbf{(b) Lerner Index verification}

Direct calculation:
$$L = \frac{P - MC}{P} = \frac{32.5 - 10}{32.5} = 0.692$$

Using formula $L = \frac{s_j}{|\varepsilon|}$:
\begin{itemize}
\item Market share: $s_j = \frac{1}{3} = 0.333$
\item Price elasticity: $\varepsilon = \frac{dQ}{dP} \cdot \frac{P}{Q} = -1 \cdot \frac{32.5}{67.5} = -0.481$
\item $L = \frac{0.333}{0.481} = 0.692$ \checkmark
\end{itemize}

\vspace{1em}
\textbf{(c) 2-firm case and welfare comparison}

With $N = 2$: $q^* = \frac{90}{3} = 30$, $Q^* = 60$, $P^* = \$40$, $\pi^* = \$900$

\begin{center}
\begin{tabular}{lccc}
\toprule
 & 3 firms & 2 firms & Change \\
\midrule
Consumer Surplus & \$2,278.13 & \$1,800.00 & $-\$478.13$ \\
Producer Surplus & \$1,518.75 & \$1,800.00 & $+\$281.25$ \\
Total Welfare & \$3,796.88 & \$3,600.00 & $-\$196.88$ \\
\bottomrule
\end{tabular}
\end{center}

Fewer firms $\Rightarrow$ higher price $\Rightarrow$ lower CS, higher PS, lower TW (deadweight loss).

\subsection*{Question 2: Bertrand Competition}

\textbf{(a)} With homogeneous products: $P^* = c = \$10$, $\pi^* = \$0$

\textbf{(b)} Cournot: $P = \$32.50$, $\pi = \$506.25$. Bertrand: $P = \$10$, $\pi = \$0$.

Bertrand is more aggressive---firms undercut until $P = MC$. Cournot has higher prices because quantity commitment creates strategic substitutability.

\textit{Bertrand realistic:} Easy price adjustment, no capacity constraints, high transparency.

\textit{Cournot realistic:} Capacity committed in advance, output hard to adjust.

\subsection*{Question 3: Collusion}

\textbf{(a)} Monopoly: $Q_m = 45$, $P_m = \$55$, $\pi_m = \$2,025$

Per-firm under collusion: $q = 15$, $\pi = \$675$

\textbf{(b)} Optimal deviation given others produce $q = 15$ each:

Best response: $q_{dev} = \frac{a - c - Q_{others}}{2} = \frac{100 - 10 - 30}{2} = 30$

$Q_{dev} = 60$, $P_{dev} = \$40$, $\pi_{dev} = \$900$

\textbf{(c)} Critical discount factor:
$$\delta^* = \frac{\pi_{dev} - \pi_{coll}}{\pi_{dev} - \pi_{punish}} = \frac{900 - 675}{900 - 506.25} = \frac{225}{393.75} = 0.571$$

Or using formula: $\delta^* = \frac{(N+1)^2}{N^2 + (N+1)^2} = \frac{16}{9+16} = 0.64$

Collusion sustainable if $\delta \geq 0.57$--0.64 (depending on exact specification).

\newpage
\section*{Part B: Merger Simulation}

\subsection*{Question 4: Pre-Merger Equilibrium}

\textbf{(a) Pre-merger equilibrium prices:}

Solving the FOC system $s_j + (p_j - c_j) \frac{\partial s_j}{\partial p_j} = 0$ with initial guess $p = c + 0.5$:
\begin{center}
\begin{tabular}{lcc}
\toprule
Product & Equilibrium Price & Markup \\
\midrule
1 & \$1.51 & \$0.51 \\
2 & \$1.71 & \$0.51 \\
3 & \$2.01 & \$0.51 \\
4 & \$1.61 & \$0.51 \\
\bottomrule
\end{tabular}
\end{center}

\textbf{(b) Market shares:}
\begin{center}
\begin{tabular}{lcc}
\toprule
Product & Share & Percentage \\
\midrule
1 & 0.0268 & 2.68\% \\
2 & 0.0221 & 2.21\% \\
3 & 0.0135 & 1.35\% \\
4 & 0.0243 & 2.43\% \\
Outside & 0.9133 & 91.33\% \\
\bottomrule
\end{tabular}
\end{center}

\textbf{(c) Own-price elasticities:}

Using $\eta_{jj} = \alpha p_j (1 - s_j)$:
\begin{center}
\begin{tabular}{lcc}
\toprule
Product & Elasticity & Status \\
\midrule
1 & $-2.95$ & Elastic \\
2 & $-3.35$ & Elastic \\
3 & $-3.96$ & Elastic \\
4 & $-3.15$ & Elastic \\
\bottomrule
\end{tabular}
\end{center}

All products have $|\eta| > 1$, confirming elastic demand.

\textbf{(d) FOC verification for product 1:}

Actual markup: $p_1 - c_1 = 1.51 - 1.0 = \$0.51$

FOC markup: $\frac{1}{|\alpha|(1-s_1)} = \frac{1}{2 \times 0.973} = \$0.51$

The FOC is satisfied (difference $< \$0.001$).

\textbf{(e) HHI:}

Using inside shares: HHI $\approx 2,634$ (highly concentrated, $>2,500$).

\subsection*{Question 5: Post-Merger Prices}

\textbf{(a) Ownership matrices:}

Pre-merger: $\mathcal{O} = I_4$ (identity matrix)

Post-merger:
$$\mathcal{O} = \begin{pmatrix} 1 & 1 & 0 & 0 \\ 1 & 1 & 0 & 0 \\ 0 & 0 & 1 & 0 \\ 0 & 0 & 0 & 1 \end{pmatrix}$$

\textbf{(b) Intuition:} The merged firm internalizes substitution between products 1 and 2. Pre-merger, raising $p_1$ meant losing customers to product 2 (competitor). Post-merger, those customers are ``recaptured''---the merged firm keeps them. This makes demand less elastic, leading to higher optimal prices.

\textbf{(c) and (d) Post-merger equilibrium:}

\begin{center}
\begin{tabular}{lccc}
\toprule
Product & Pre-merger & Post-merger & Change \\
\midrule
1 & \$1.51 & \$1.53 & $+0.75\%$ \\
2 & \$1.71 & \$1.73 & $+0.81\%$ \\
3 & \$2.01 & \$2.01 & $+0.00\%$ \\
4 & \$1.61 & \$1.61 & $+0.00\%$ \\
\bottomrule
\end{tabular}
\end{center}

Merging firms raise prices; non-merging firms see negligible changes (logit with small shares leads to limited strategic interaction).

\subsection*{Question 6: Welfare Analysis}

\textbf{(a) Consumer surplus:}
\begin{center}
\begin{tabular}{lcc}
\toprule
 & Pre-merger & Post-merger \\
\midrule
CS & \$45.37 & \$44.77 \\
Change & \multicolumn{2}{c}{$-\$0.60$ ($-1.3\%$)} \\
\bottomrule
\end{tabular}
\end{center}

\textbf{(b) Producer profits:}

Merged firm: Pre $\$13.78 + \$11.29 = \$25.07$ $\to$ Post $\$25.08$ (+0.04\%)

Non-merging firms see negligible changes (Firm 3: $+\$0.01$, Firm 4: $+\$0.01$).

Total PS: Pre $\$44.39$ $\to$ Post $\$44.42$ ($+\$0.03$)

\textbf{(c) Total welfare:}

TW falls by $\$0.57$ ($-0.63\%$). The merger is \textbf{welfare-reducing}---consumer harm exceeds producer gain.

\textbf{(d) Efficiency defense:}

With 10\% cost reductions on products 1 and 2, welfare change becomes $+\$10.44$ ($+11.6\%$). The efficiency defense \textbf{succeeds}---sufficient cost savings offset anticompetitive price increases.

\textit{Policy implication:} Merging parties must demonstrate credible efficiency gains to offset consumer harm.

\end{document}
